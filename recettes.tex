\documentclass[a4paper,twoside,openright]{report}
\usepackage[print]{autiwa}

\setcounter{secnumdepth}{2}
\usepackage{minitoc}
\setcounter{minitocdepth}{1}

\pagestyle{fancy}
\fancyhf{}
\fancyhead[RO,LE]{\footnotesize\textbf{\thepage}}
\fancyhead[RE]{\footnotesize\leftmark}
\fancyhead[LO]{\footnotesize\leftmark}

\makeatletter

\def\cleardoublepage{\clearpage\if@twoside \ifodd\c@page\else
\hbox{} 
\thispagestyle{empty}
\newpage
\if@twocolumn\hbox{}\newpage\fi\fi\fi} 

% macros qui servent à la mise en page, elles ne doivent pas être utilisées directement
\newcommand{\preparationTopSep}{\vspace{.2cm} \hrule height0.25pt width\hsize \vspace{1em}}
\newcommand{\statistiqueTopSep}{\vspace{.3cm} \hrule height1pt width\hsize \nobreak \vskip\parskip \vspace{.3cm}}
\newcommand{\ingredientsTopSep}{\vspace{.4cm} \hrule height0.75pt width\hsize\vspace*{1\p@}\hrule height0.25pt width\hsize \vspace{1em}}
\newcommand{\partStyle}{\bfseries \large}

% Custom column definition with paragraph centered horizontally and vertically
\newcolumntype{M}[1]{>{\centering\arraybackslash}m{#1}}

% redéfinition d'une nouvelle commande de section qui n'a pas de numérotation, est centrée, avec mise en page modifiée. C'est pour afficher le titre des recettes et qu'elles apparaissent dans le menu. J'avais fait avec addcontentline mais la référence n'était pas bonne et pointait sur la page précédente parfois. Bref, pas propre.
\newcommand\nomRecette{\@startsection {section}{6}{\z@}{0ex}{2.3ex }%
	{\reset@font\Huge\bfseries\centering}}

\usepackage{bbding}%to add \FiveStar and \FiveStarOpen
\newcommand{\note}[1]
{
  \ifthenelse{\equal{#1}{0}}{non testé}{}
  \ifthenelse{\equal{#1}{1}}{\FiveStar\FiveStarOpen\FiveStarOpen\FiveStarOpen\FiveStarOpen}{}
  \ifthenelse{\equal{#1}{2}}{\FiveStar\FiveStar\FiveStarOpen\FiveStarOpen\FiveStarOpen}{}
  \ifthenelse{\equal{#1}{3}}{\FiveStar\FiveStar\FiveStar\FiveStarOpen\FiveStarOpen}{}
  \ifthenelse{\equal{#1}{4}}{\FiveStar\FiveStar\FiveStar\FiveStar\FiveStarOpen}{}
  \ifthenelse{\equal{#1}{5}}{\FiveStar\FiveStar\FiveStar\FiveStar\FiveStar}{}
}

\newcommand{\ingredient}[1][]{\ifthenelse{\equal{#1}{}}{\item }{\vspace{1ex}\hrule\vspace{1ex}\item[\textcurrency]\textbf{#1}}}

\newcommand{\etape}[1][]{\ifthenelse{\equal{#1}{}}{\item }{\item[\textbf{#1}]}}

% Environnement qui crée une nouvelle recette. Ingrédients, étape, cuisson etc doivent être contenus dans un environnement recette.
\newenvironment{recette}[4]{\newpage \nomRecette{#1}\statistiqueTopSep\note{#2}\ifthenelse{\equal{#4}{}}%
{\hfill\includegraphics[width=1em]{figures/logo_minuterie.pdf} \; \begin{itshape}\textbf{Préparation \& Cuisson :} #3\end{itshape}}%if none
{\hfill\includegraphics[width=1em]{figures/logo_minuterie.pdf} \; \begin{itshape}\textbf{Préparation :} #3 \hfill \includegraphics[width=1em]{figures/logo_minuterie.pdf} \;\textbf{Cuisson :} #4\end{itshape}}}{}

% Environnement pour la mise en page des ingrédients. Un argument optionnel de l'environnement permet de spécifier pour combien de personnes sont les doses. Les entrées de cet environnement doivent être ``\ingredient nom de l'ingrédient''. Si on veut séparer les ingrédients, pour dénoter deux sortes de choses, ingrédient pour une sauce et une pâte par exemple, il faut faire une ligne du style \ingredient[pour la sauce] afin de séparer.
\newenvironment{ingredients}[1][]{\ingredientsTopSep{\partStyle Ingrédients\ifthenelse{\equal{#1}{}}{}{ (#1)}}\begin{multicols}{2}\begin{itemize}\renewcommand{\labelitemi}{$\bullet$}}{\end{itemize}\end{multicols}}

% Environnement pour mettre en page la préparation du plat. Les entrées sont de la forme ``\etape texte de l'étape``. Si on veut séparer les étapes, pour dénoter un groupement d'étape par exemple, il faut faire une ligne du style \etape[pour la sauce] afin de séparer.
\newenvironment{preparation}{\preparationTopSep{\partStyle Préparation }\vspace{0.5em}\begin{enumerate}}{\end{enumerate}}

% À utiliser si on souhaite faire une mise en page un peu plus évoluée, avec deux listes par exemples.
\newenvironment{preparation*}{\preparationTopSep{\partStyle Préparation }\vspace{0.5em}\par}{}

% Environnement pour mettre en page la cuisson du plat
\newenvironment{cuisson}{\bigskip{\bfseries \large Cuisson }\par}{}
\makeatother

%TODO Faire une commande pour que dans l'index on voit la liste des recettes en fonction de la note qu'elles ont. 

\pdfsuppresswarningpagegroup=1

\title{Petit livre de (G)astronomie}
\author{Autiwa}
\makeindex
\begin{document}
\begin{titlepage}
\begin{center}
~
\vfill
% Upper part of the page
\begin{figure}[t]
\centering
\includegraphics[width=0.15\textwidth]{figures/logo-autiwa.pdf}
\end{figure}

% Title
\HRule \\[0.4cm]
{ \huge \bfseries \makeatletter\@title\makeatother}\\[0.4cm]

\HRule \\[0.75cm]
{\large \today}\\[0.75cm]
\makeatletter
\@author
\makeatother
\vfill
\begin{center}
\includegraphics[width=0.65\textwidth]{figures/couverture.pdf}
\end{center}
\vfill
~
% Bottom of the page


\end{center}
\end{titlepage}

\cleardoublepage
\null
\vfill
\noindent
Crédits : Le titre ainsi que l'illustration de la première page, je le dois à Claire, Fanny, Marianne, Clément, Romuald, David 
et Axel. La pluri-disciplinarité, entre cuisine et astronomie, que je représente modestement ne pourra jamais être aussi bien 
illustrée que par ce titre et ce dessin et je vous remercie une fois de plus ici pour cela.
\hfill
\cleardoublepage

\dominitoc
\tableofcontents
\cleardoublepage
\chapter{Plat Principal}
\minitoc

% Beginning of group where section is deactivated
% This is only to get the good structure of the document 
% since ``section'' is in fact embedded in the 'recette' environment.
% This group allow us to deactivate sections ONLY in the given file and 
% not for the entire document.
{\renewcommand{\section}[1]{}

\section{Bibimbap}
\begin{recette}{Bibimbap}{4}{1h30}{}\index{bibimbap}\index{coréen}
\begin{ingredients}
\ingredient[Marinade]
\ingredient 300g de viande (boeuf ou canard, sans os c'est plus simple)
\ingredient 12.5cl de sauce soja
\ingredient 2 cuillères à soupe d'huile de tournesol
\ingredient 1 cuillère à soupe de vinaigre
\ingredient 1 cuillère à soupe de miel
\ingredient 1 gousse d'ail
\ingredient 20g de sucre en poudre
\ingredient[Plat]
\ingredient 500g de riz % 4 doses du cuiseur, c'est 650g)
\ingredient 1 courgette (je ne me souviens pas si les légumes sont crus ou revenus avant, je crois crus)
\ingredient 1 carotte
\ingredient 4 oeufs
\ingredient 1 oignon
\ingredient 8 feuilles de salade
\ingredient 200g de champignons (4 champignons)
\end{ingredients}

\begin{preparation}
\etape Emincez la viande en lamelles très fines, 5mm d'épais maximum et 1cm de large environ, pour une longueur de 3cm environ
\etape Faites mariner la viande finement émincée la nuit précédente (sinon au moins 30 minutes)
\etape Egouttez la viande de la marinade
\etape Emincez les légumes, normalement rapés (courgette, carotte, oignon, champignons)
\etape Faites cuire le riz (cuiseur à riz, ou dans 75cl d'eau salée jusqu'à évaporation totale, peut-être moins d'ailleurs, pour que le riz soit un peu sec)
\etape Faites revenir les légumes 
\etape Dans la cocotte, rajoutez alors le riz, la viande cru, les œufs entiers et crus ainsi que le reste de marinade. 
\etape Faites revenir le tout à feu vif. Les oeufs doivent colorer les autres ingrédients. Faites cuire jusqu'à ce que ce soit homogène, environ 2/3 minutes
\end{preparation}
\end{recette}

\section{Blanquette de Veau}
\begin{recette}{Blanquette de Veau}{5}{1h}{4h}\index{blanquette de veau}\index{veau}
\begin{ingredients}
\ingredient 1,2 kg d'épaule ou de tendron de veau coupé en morceaux
\ingredient 100g de lardons
\ingredient 2 carottes
\ingredient 4 oignons
\ingredient 1 branche de celeri (nouveau)
\ingredient 2 gousse d'ail
\ingredient 25cl de vin blanc sec
\ingredient 25cl de bouillon de volaille (cube + eau)
\ingredient 20cl de crème fraîche
\ingredient 1 bouquet de persil, céleri, sel, poivre, farine, quelques gouttes de citron
% ajouter poireau et navet ntait pas une bnne idee.
\end{ingredients}

\begin{preparation}
\etape Pelez carotte, ail, oignons. Coupez les carottes en petits dés, émincez l'oignon et mettez les dans un récipient commun;

\etape Mettez l'ail en bouilli à l'aide d'une fourchette et mettez le dans un bol avec 25cl d'eau et le cube de bouillon. 
Faites chauffez 1 minute au micro onde.

\etape Dans de l'eau bouillante non salée, plongez les morceaux de viande une minutes afin de les faire blanchir, puis égouttez et réservez-les. (cette étape permet en particulier de bien ``cautériser'' la viande pour que le jus n'en parte pas.

\etape Faites revenir les lardons, réservez les à part de la viande. 

\etape faites revenir les légumes (sauf céleri) tout en profitant de l'eau qu'ils rendent pour déglacer les sucs de la viande

\etape Ajoutez les lardons, puis saupoudrez une à deux cuillères à soupe de farine. Mélanger, puis versez le bouillon et le vin 
blanc afin d'obtenir une sauce épaisse et un peu blanche. 

\etape Ajoutez alors le céleri coupé en 3 ou 4 morceaux, sel et poivre et la viande. Ne pas rajouter d'eau, même si ça ne recouvre pas complètement, la sauce va augmenter au cours de la cuisson. Laissez mijoter (85-96°C) 4h à feu très doux (Si la viande est 
très dure, c'est 
que vous ne l'avez pas fait assez cuire). 

\etape Juste avant de servir, rajoutez la crème fraîche, un peu de persil et quelques gouttes de citron
\end{preparation}
\end{recette}

\section{Bœuf Bourguignon}
\begin{recette}{Bœuf Bourguignon}{4}{15 min.+12h+45min.}{6h}\index{bœuf bourguignon}\index{bœuf}\index{daube}
\begin{ingredients}
\ingredient 1.5kg de viande de bœuf (de préférence un peu grasse)
\ingredient 100g de lardons fumés (plutôt lardon qu'allumette vu que ça va cuire longtemps)
\ingredient 75cl de vin rouge corsé
\ingredient 5cl de cognac
\ingredient 3 gousses d'ail
\ingredient 2 oignons
\ingredient 4 échalotes
\ingredient 4 carottes
\ingredient 100g d'olives noires
\ingredient 20cl de bouillon de volaille (adapter pour que la viande soit recouverte quand ça mijote)
\ingredient 1 cuillère à soupe de coulis de tomate
\ingredient une cuillère à soupe de farine
\ingredient huile d'olive, sel, sucre, poivre, un clou de girofle, céleri, herbe de provence, bouquet garni
\end{ingredients}

\begin{preparation}
\etape Disposez la viande dans un plat ou saladier (je met dans une marmite pour ma part), en étalant en une couche la moins 
haute possible (afin que tout baigne dans le vin). 
\etape Émincez oignons et échalotes. Coupez les carottes en petits cubes (de 0.5cm de coté environ). Hachez l'ail. 
\etape Ajoutez sur la viande oignons, échalotes, carotte et bouquet garni. 
\etape Dans un autre saladier, versez le vin et le cognac. Ajoutez-y ail, épices (clou de girofle, céleri et herbe de 
Provence. Salez et poivrez. Mélangez puis ajoutez cette préparation sur la viande.
\etape Faites ensuite mariner cette préparation environ 12h au réfrigérateur (en couvrant le récipient par un couvercle ou du
film étirable). 
\etape Sortez la marinade du frigo. Égouttez la viande puis les légumes de la marinade dans des récipients séparés
\etape Pendant ce temps, faites revenir les lardons dans le récipient qui servira à mijoter (une cocotte ou marmite).
Réservez-les dès qu'ils sont un peu blanchis, il ne faut pas qu'ils soient cuits, juste qu'il rendent le gras
\etape Complétez le gras des lardons avec de l'huile d'olive si besoin et saisissez la viande à feu vif, juste pour la colorer
un peu, très légèrement. Réservez la viande. 
\etape Faites alors revenir les légumes de la marinade, afin qu'ils rendent l'alcool qu'ils ont absorbé, et qu'ils rissolent un
peu. 
\etape Saupoudrez une cuillère à soupe de farine sur les légumes, puis mélangez la bien. Ajoutez alors le bouillon de bœuf,
mélangez. 
\etape Incorporez ensuite la cuillère à soupe de tomate, une pincée de sucre, le vin de la marinade et les olives. 
\etape Mélangez la préparation et ajoutez enfin les lardons et la viande. 
\end{preparation}

\begin{cuisson}
%2/9 n'est pas assez fort j'ai l'impression, mais j'attends de voir ce que ça donne au long terme avant de changer la recette et mettre 3/9
% avec de la viande pas grasse, j'avais fait cuire 6h 2/9, puis 3h à 3/9 puis 3h à 30/9 plusieurs jours après. La viande pas grasse initialement sèche était mieux au bout de tout ce temps. Donc si c'est sec et dur, faut continuer à cuire.
Couvrez la marmite et faites mijoter (85-96°C) à feu doux (2-3/9) pendant environ 6h. Si la viande n'est pas grasse, il faudra peut-être plus longtemps. Gouttez la viande. Il faut qu'elle commence à perde sa tenue et à s'émietter.
\end{cuisson}
\end{recette}

\section{Brandade de poisson}
\begin{recette}{Brandade de poisson}{2}{10 minutes+1 nuit+1h30}{}\index{brandade de poisson}\index{morue}
\begin{ingredients}
\ingredient 1kg de poisson surgelés (filets de colin ou autre, le moins cher)
\ingredient gros sel (quantité à définir
\ingredient 2 gousses d'ail
\ingredient 50g de persil frais
\ingredient 50cl de crème fraiche épaisse
\ingredient un peu de lait (ou l'eau de cuisson du poisson à défaut)
\ingredient 2.5kg de pomme terre
\ingredient laurier, persil, beurre, huile d'olive, jus de citron
\end{ingredients}

\begin{remarque}
C'est bien sur la brandade de morue à la base, mais la lotte coûtant moins cher que la morue, j'ai adapté la recette. Le 
principe est de faire avec n'importe quel poisson.
\end{remarque}


\begin{preparation}
\etape La veille, mettez le poisson au sel (saumure avec 150g de sel / L) (15h au sel au frigo dans mon cas)
\etape Le lendemain, émincez l'ail très finement 
\etape Faire cuire le poisson dans une casserole d'eau froide au départ, avec deux feuilles de laurier. Portez à ébullition et 
laisser 
cuire à petite ébullition pendant 10 minutes.
\etape En parallèle, faites cuire les pommes de terre dans une casserole d'eau pendant 20 minutes (ou à l'auto-cuiseur 16 
minutes).
\etape Égouttez le poisson, enlevez l'arrête centrale et émiettez-le dans une casserole. Ajoutez de l'huile puis laissez alors rissoler à feu moyen en 
remuant de temps en temps.
\etape Hors du feu, ajoutez la crème fraîche, l'ail et le persil et tourner à nouveau pour bien mélanger les ingrédients.
\etape Écrasez les pommes de terre à la fourchette et mettre la purée ainsi formée dans la casserole.
 Si le mélange est trop sec et ne forme 
pas une purée, ajouter un peu de lait
\etape Placez dans un plat à gratin, ajouter quelques fines lamelles de beurre dessus et faire gratiner pendant 5-10 minutes à 
200°C (j'ai fait préchauffer 10 minutes, puis mis en grill porte fermée pendant 10 minutes)
\end{preparation}
\end{recette}

\section{Butter Chicken}
\begin{recette}{Butter Chicken}{}{1 nuit+45 min.}{1h}\index{poulet au beurre}\index{butter chicken}
\begin{ingredients}
\ingredient 1kg de haut de cuisse de poulet

\ingredient[Pour la marinade]
\ingredient 2 gousses d'ail (ou 1 cas de pâte d'ail)
\ingredient 2cm de gingembre
\ingredient 1/2 cuillère à café de paprika
\ingredient 10cl de crème fraiche
\ingredient 1 cuillère à soupe de garam masala
\ingredient 1/2 cuillère à café de curcuma (ou pas, c'est le truc qui donne un goût de terre)
\ingredient 1/2 cas de sel

\ingredient[Pour la sauce]
\ingredient 40g de beurre
\ingredient 24cl de crème fraiche (le reste du pot)
\ingredient 1 cuillère à soupe de jus de citron
\ingredient 2 gousses d'ail (ou 1 cas de pâte d'ail et dans ce cas pas de citron)
\ingredient 2cm de gingembre
\ingredient un peu de cardamone (normalement c'est un truc de graine un peu écrasé). Peut-être 1/2 cuillère à café)
\ingredient 2 cuillère à soupe de coriandre
\ingredient 1 cuillère à café de garam masala
\ingredient 1 cuillère à café de paprika
\ingredient 50g de cajou moulue
\ingredient 70g de concentré de tomate
\end{ingredients}

\begin{preparation}
\etape Mixez les ingrédients pour la marinade
\etape Enlevez la peau du poulet et coupez les haut de cuisse en deux le long de l'os puis mélangez à la marinade
\etape Laissez mijoter pendant au moins une nuit.
\etape Enlevez le poulet de la marinade en secouant sans chercher à tout enlever. Gardez la marinade pour plus tard
\etape Faites revenir les morceaux de poulet dans un peu d'huile jusqu'à ce que ça commence à dorer/accrocher (je déglace et réserve la marinade qui accroche pour pouvoir continuer à saisir le poulet) puis réservez
\etape Faites chauffer les épices, ail et gingembre dans le beurre pendant 1 à 2 minutes, puis ajoutez les noix de cajou, le concentré de tomate, le reste de marinade et un peu de bouillon
\etape Faites mijoter 20 minutes
\etape Mixez la sauce
\etape Ajoutez le poulet déjà saisi et faites mijoter 10 minutes
\etape Ajoutez la crème, laissez mijoter 1-2 minutes
\end{preparation}

\begin{cuisson}
Laisser mijoter 30 minutes max à feu doux, parfois moins si le poulet a cuit 20 minutes saisi dans la poële, 10 minutes dans la sauce suffisent. 

Du riz en accompagnement.
\end{cuisson}
\end{recette}

\section{Canard madère}
\begin{recette}{Canard madère}{4}{45min}{4h}\index{canard madère}\index{madère}\index{canard}
% (Excellent)

\begin{ingredients}
\ingredient 4 cuisses de canard
\ingredient 20g de beurre
\ingredient $2$ oignons
\ingredient $1$ carotte
\ingredient $2$ gousses d'ail
\ingredient 150g d'olives noires
\ingredient 50 cl de madère
\ingredient 2 cuillère à soupe rase de farine
\ingredient sel, poivre du moulin
\end{ingredients}

\begin{preparation}
\etape Écrasez les gousses d'ails, émincez l'oignon et coupez la carotte en petits cubes (en fine lamelle que vous coupez en 
tranche)
\etape Mettez l'ail dans le récipient qui recevra la viande saisie
\etape Faites fondre le beurre dans une sauteuse et faites revenir les cuisses à feu vif. Pas besoin que la viande soit cuite à l'intérieur, c'est juste pour faire dorer.
\etape Réservez les morceaux puis faites revenir l'oignon et les dés de carotte.
\etape Ajoutez alors la farine, remuez jusqu'à l'incorporer autour des légumes. Ajoutez enfin le madère et les olives noires. Ajoutez un peu de poivre, mélangez et rajoutez enfin les morceaux de canard.
\end{preparation}

\begin{cuisson}
Faites cuire pendant 4h à feu doux (3/10).
\end{cuisson}


\begin{remarque}
Mon avis est que cette sauce va très bien avec du riz.
\end{remarque}

\end{recette}

\section{Canard aux pruneaux}
\begin{recette}{Canard aux pruneaux}{3}{}{}\index{canard aux pruneaux}\index{pruneaux}\index{canard}
% (recette que j'ai inventé)
\begin{ingredients}
\ingredient morceaux de canards (8 manchons par exemple)
\ingredient 3 échalotes
\ingredient 150g de champignons
\ingredient 25cl de bouillon de volaille
\ingredient une cuillère à café de fond de veau
\ingredient 10cl de cognac
\ingredient 20cl de vin blanc
\ingredient pruneaux
\ingredient huile, beurre, sel, poivre
\end{ingredients}

\begin{preparation}
\etape Faites revenir les morceaux de canard à feu vif dans une sauteuse avec moitié beurre moitié huile d'olive. Une fois bien 
doré, réservez les.
\etape déglacez avec le cognac, et mettez les échalotes et les champignons dans la sauteuse. Couvrez et laissez mijoter jusqu'à 
ce que ce soit cuit (en remuant de temps en temps)
\etape rajoutez le bouillon de volaille, le vin blanc, le fond de veau et les pruneaux. Remuez, puis rajoutez les morceaux de 
canard.
\etape Laissez mijoter 30 minutes environ (ou plus longtemps si les morceaux sont plus gros et plus longs à cuire).
\end{preparation}

\end{recette}

\section{Canard laqué}
\begin{recette}{Canard laqué}{3}{10 min + 6h}{2h}\index{canard laqué}\index{canard}\index{sauce soja}\index{miel}

\begin{ingredients}
\ingredient 4 cuisses de canard (ou morceaux de canard)
\ingredient 6 pincées de sel
\ingredient 5g de poudre aux cinq-épices (ou 5 baies)
\ingredient 50g de miel
\ingredient 15cl de sauce soja
\ingredient 5cl de vinaigre blanc
\ingredient 40g de fécule de maïs (ou farine)
\ingredient 2 gousses d'ail écrasées et finement hachées
\ingredient 10 g de levure de boulanger (ou 20g de levure fraiche)
\end{ingredients}

\begin{preparation}
\etape Plongez les morceaux de canard dans de l'eau bouillante 30 secondes puis lavez et essuyez l'intérieur et l'extérieur avec 
des serviettes en papier.
\etape En utilisant un poinçon, faites de multiples trous dans la peau et les muscles des morceaux.
\etape Dans un bol, déposez un sac de congélation de taille moyenne. Versez tous les ingrédients de la laque, puis entortillez 
la poche pour la fermer et remuez jusqu'à obtenir un mélange homogène (c'est le miel le plus difficile à mélanger)
\etape Ajoutez alors les morceaux de canard, et fermez le sac de congélation pour de bon.
\etape Laissez mariner le canard au moins 6 heures, au réfrigérateur de préférence, en le retournant et l'arrosant de laque de 
temps en temps.
\end{preparation}

\begin{cuisson}
Passez la laque quelques minutes dans une casserole afin de l'épaissir. pas trop longtemps sinon ça fera de la gelée. 

Ajoutez un peu d'eau dans le lèche frite afin que le jus de crame pas. 

Faites ensuite cuire les morceaux de canard au four, à 200°C pendant une heure. En cours de cuisson, arrosez les morceaux de 
laque de temps en temps, quand la viande a un peu perdu son enrobage. 

(Sur une autre recette, c'est 200°C pendant une heure enrobé de papier aluminium, puis 40 minutes à 150°C sans le papier, et en arrosant toutes les 10 minutes de la laque

Servez chaud ou froid.
\end{cuisson}
\end{recette}

\section{Canard à l'orange}
\begin{recette}{Canard à l'orange}{3}{30 min}{}\index{canard à l'orange}\index{orange}\index{vinaigre}

\begin{ingredients}
\ingredient 40 à 50cl de jus d'orange
\ingredient 2 cuillères à soupe de gelée de groseille
\ingredient 2 cuillères à soupe de fécule de pomme de terre
\ingredient 15cl de bouillon de volaille
\ingredient 30g de sucre
\ingredient 15cl de vinaigre de xérès
\end{ingredients}

\begin{preparation}
\etape Préparez le jus d'orange
\etape Faites chauffer le sucre et le vinaigre et laissez réduire jusqu'à la formation d'un caramel et la dissipation des odeurs 
de vinaigre.
\begin{attention}
Lors de la disparition complète des odeurs de vinaigre, le caramel va commencer à prendre, il faut donc que ça aille vite à ce 
moment là, afin de ne pas se retrouver avec un vrai caramel très épais. 
\end{attention}

\etape Une fois pris, rajoutez de suite le jus d'orange, les morceaux de carotte et le citron. Laissez cuire 10 à 15 minutes
\etape Pendant ce temps, préparez à peu près 3 cuillères à soupe de fécule de pomme de terre dans 15cl de bouillon.
\etape Mélangez les deux et laissez cuire environ 10 minutes en remuant tout le temps. 
\etape Ajoutez 2 cuillères à soupe de gelée de groseille, et en rajouter si c'est trop acide.
\end{preparation}
\end{recette}

\section{Cassoulet à la moi}
\begin{recette}{Cassoulet à la moi}{0}{10h}{5h}\index{cassoulet}
\begin{ingredients}[7 pers.]
\item 1 kg de mouton
\item 500 g de haricots blancs secs
\item 2 tomates
\item 1 carotte
\item 3 gousses d'ail
\item 1 oignon piqué de 2 clous de girofle
\item 2 oignons
\item 1 cuillère à soupe de farine
\item 50 cl de bouillon de légumes
\item sel, poivre, thym, laurier, huile
\end{ingredients}

\begin{preparation}
\etape La veille, faites tremper les haricots dans un grand volume d'eau froide.
\etape Le lendemain, égouttez les haricots. Mettez-les dans une grande casserole avec les tomates coupées en quartiers, 
l'oignon, 1 branche de thym et une feuille de laurier. Couvrez d'eau froide et portez à ébullition.
\etape Baissez le feu et laissez cuire 1 h à feu moyen, salez et poivrez à mi-cuisson.
\etape Pendant ce temps, découpez la viande en morceaux. Faites chauffer une cocotte avec le beurre et l'huile. Faites-y dorer 
les morceaux de viande sur toutes les faces.
\etape Retirez-les de la cocotte et réservez à part.
\etape Émincez l'oignon et coupez la carotte en tout petits cubes
\etape Faites revenir oignon et carotte jusqu'à ce qu'ils soient légèrement dorés. 
\etape Saupoudrez de farine, remuez à la cuillère en bois et laissez blondir.
\etape Versez le bouillon, salez, poivrez et mélangez. Remettez la viande dans la cocotte.
\etape Pelez et coupez la carotte en rondelles et l'ail en morceaux. Ajoutez-les dans la cocotte ainsi que le thym et le 
laurier. Portez à ébullition, puis couvrez et laissez mijoter 1 h à feu moyen.
\etape Égouttez les haricots, ils doivent être encore un peu fermes. Ajoutez-les dans la cocotte avec la viande.
\etape Poursuivez la cuisson pendant 1 h.
\etape Servez bien chaud, avec la sauce de cuisson réduite à part.
\end{preparation}
\end{recette}


\section{Chatrou (poulpe)}
\begin{recette}{Chatrou (poulpe)}{4}{1h30}{}\index{chatrou}\index{poulpe}
\begin{ingredients}
\ingredient 5-10cl d'huile tournesol (il y avait un fond de 2-3mm dans l'autocuiseur de jacqueline, quantité à déterminer)
\ingredient 2kg de chatrou
\ingredient 4-5 cives
\ingredient 8 gousses d'ail
\ingredient 1 oignon
\ingredient 1 piment végétarien
\ingredient poivre, 1 clou de girofle
\ingredient 20ml de jus de citron (jus de 1/2 citron)
\ingredient 200ml de coulis de tomate (100ml par kg de chatrou)
\end{ingredients}

\begin{preparation}
\etape Égouttez et émincez le poulpe dans la marmite (jetez cette eau là, vous n'en aurez pas besoin)
\etape Préparez un bol (pour la fin de cuisson) et deux saladiers (un pour le bouillon et un pour le chatrou). 
\etape Émincez les cives, le piment végétarien et mettez dans le bol du futur bouillon avec le clou de girofle
\etape Emincez l'oignon dans le 2e saladier pour l'instant vide.
\etape Hachez 8 gousses d'ail. Mettez les dans le bol avec le jus de citron. 
\etape Faites revenir le poulpe à feu vif (7/9) jusqu'à ce que le jus rendu soit à ébullition.
\etape Une fois à ébullition, couvrez, baissez à feu moyen (5/9) et laissez cuire pendant 15 minutes. 
\etape Avec une passoire, récupérez le chatrou et mettez dans le saladier avec l'oignon émincé. Avec une louche, prélevez deux louches de jus et mettez dans le bol d'ail. Versez alors le jus restant dans un saladier préparé tout à l'heure avec le clou de girofle.
\etape Faites chauffer l'huile (il en faut pas mal)
\etape Faites revenir le chatrou avec l'oignon pendant 10 minutes environ. Quand ça commence à accrocher (7-8 minutes) écourtez la cuisson plutôt que de cramer le fond
\etape Ajoutez le jus du saladier et le coulis de tomate, puis grattez le fond avec une spatule (ne pas chercher à épaissir la sauce)
\etape Faites mijoter 1h à feu moyen (4/9) et à couvert
\etape Ajoutez alors le jus du bol (+ail) et laissez à couvert et hors du feu pendant 10 minutes
\end{preparation}
\end{recette}


\section{Cochon roussi à la moi}
\begin{recette}{Cochon roussi à la moi}{}{30 min.+1 nuit+1h}{2h}
\begin{ingredients}
\ingredient[Marinade]
\ingredient 3 gousses d'ail
\ingredient 2 oignon
\ingredient 200ml d'eau
\ingredient 8g d'huile
\ingredient 2 cac de colombo (10g)
\ingredient 1/3 de cac de muscade (0.5g)
\ingredient 1/3 de cac de canelle (1g)
\ingredient 1g de poivre
\ingredient 2g de thym
\ingredient[cuisson]
\ingredient 1 roti dans l'échine de 1.5kg
\ingredient 1 patate
\ingredient 1 cive (ou les tiges d'une botte de 5 oignons)
\ingredient 10g de jus de citron (normalement 25, i.e 1/2 citron)
\ingredient 450ml d'eau
\ingredient 9g de sel
\end{ingredients}
% j'ai fait une version sans faire la marinade parce que j'avais pas le temps, et qui était très bonne aussi. C'était même plus simple à cuisiner parce que je n'avais pas à enlever la marinade des morceaux.

\begin{preparation}
\etape Coupez le roti en cube grossier (des 1/4 de cylindres, puis des tranches de 4-5cm de large)
\etape Mixez l'oignon et l'ail
\etape Ajoutez l'eau, l'huile et le jus de citron et les épices
\etape Mettez la viande à mariner pendant la nuit
\etape Le lendemain, séparez la viande de la marinade
\etape Déposez le porc dans la marmite/cocotte avec la pomme de terre coupée en morceaux grossiers et saisissez les morceaux (avec une cocotte en fonte le bon moment pour les tourner c'est quand ils ne collent plus au fond).
\etape Réservez la viande. Faites alors rissoler les cives
\etape Ajoutez deux cuillères à soupe de farine, puis ajoutez la marinade, l'eau, le sel et le jus de citron
\end{preparation}

\begin{cuisson}
Faites mijoter pendant 4h environ à feu doux (3/9) et à couvert.
\end{cuisson}
\end{recette}

\section{Colombo de poulet}
\begin{recette}{Colombo de poulet}{4}{1h}{1h}\index{poulet}\index{colombo}
\begin{ingredients}
\ingredient 1.5kg de poulet
\ingredient 4 oignons
\ingredient [optionnel] 4 piments végétariens
\ingredient 5 gousses d'ail
\ingredient 15ml de jus de citron (1/2 citron)
\ingredient 50cl de bouillon de poulet
\ingredient 5g de sel
\ingredient 40g de poudre de colombo
\ingredient 2 cuillère à soupe de maizena
\end{ingredients}


\begin{preparation}
\etape Emincez l'oignon
\etape Faites chauffer l'eau avec le bouillon cube dans un bol 2 minutes au micro onde puis mélangez. 
\etape Emincez finement le piment végétarien, mixez l'ail et mettez le tout à infuser dans le bouillon
\etape Pesez la poudre de colombo et le sel dans un bol
\etape Saisissez dans du beurre ou de l'huile les morceaux de poulet à feu vif puis réservez-les.
\etape Faites revenir l'oignon.
\etape Une fois revenus, rajoutez la poudre de colombo et le sel et mélangez. 
\etape Ajoutez enfin le bouillon et le jus de citron puis mélangez. Ajoutez alors le poulet.
\end{preparation}

\begin{cuisson}
Laissez alors mijoter à couvert et à feu doux (3/9) 45 minutes environ.
\end{cuisson}
\end{recette}

\section{Court-bouillon de poisson}
\begin{recette}{Court-bouillon de poisson}{}{1h}{2h}
\begin{ingredients}
\ingredient morceaux de poisson
\ingredient 1 cive (200g)
\ingredient 1 oignon
\ingredient huile de tournesol
\ingredient graines de roucous (environ 15)
\ingredient 4cl de jus de citron (1 citron)
\ingredient 5 gousses d'ail
\ingredient 20cl d'eau
\ingredient un peu de mélange 4 épices, persil, thym, clou de girofle
\ingredient sel, poivre
\end{ingredients}
% la dernière fois, 50g de jus de citron jaune en bouteille, c'était trop acide

\begin{preparation}
\etape [facultatif] Mettez les têtes de poissons dans 50cl d'eau, avec 5g de sel, laurier et faites cuire pendant 30 minutes puis filtrez pour avoir du bouillon au lieu de l'eau pour la suite de la cuisson
\etape Faites chauffer l'huile avec les graines de roucous dans la marmite. Une fois coloré (\~5 minutes), jetez les graines
\etape Faites revenir le poisson pendant 5 minutes à feu vif (7/9)
\etape Réservez le poisson et faites revenir oignons et cives émincées
\etape Ajoutez une seule cuillère à soupe de farine. Ajoutez le jus de citron, l'eau et l'ail écrasé puis mélangez. 
\etape Portez à ébullition
\etape Ajoutez le poisson, mettez à feu doux (3/9) et à couvert pendant 10 minutes
\end{preparation}
% Jacqueline  prépare un mélange avec un demi citron, deux gousses d'ail et un peu d'huile à verser en fin de cuisson mais j'ai remarqué qu'elle ne versait pas toute la marinade. Donc j'ai un peu modifié la recette
% les temps de cuissons ne sont pas sûrs, car elle fait au pif et une fois couvert, elle n'a pas vraiment fait cuire, elle a coupé le feu, c'est tout
\end{recette}

\section{Couscous à la moi}
\begin{recette}{Couscous à la moi}{4}{}{}\index{couscous à la moi}\index{poulet}\index{agneau}
\begin{ingredients}[8 pers.]
\ingredient 800g de semoule fine
\ingredient 1kg d'agneau
\ingredient (facultatif) 1kg poulet / 4 merguez
\ingredient $2$ oignons (130g)
\ingredient $1$ carotte (130g)
\ingredient $1$ navet (130g)
\ingredient $1$ aubergine (130g) (plus petite possible)
\ingredient 1 poireau (130g)  (plus petit possible)
\ingredient 1 branche de céleri (130g)
\ingredient[falcultatif] 20g de piquillos ou piment végétarien antillais
\ingredient $4$ gousses d'ail
\ingredient $50\unit{g}$ de concentré de tomate en boite % 50g de tomates séchées la dernière fois
\ingredient 1 cuillère à soupe rase de sucre
\ingredient $100\unit{g}$ de raisins secs
\ingredient 1 bouillon cube
\ingredient 1 cuillère à soupe de jus de citron (pour remplacer le citron confit)
\ingredient épices : \begin{itemize}
		\item une cuillère à soupe bombée de Ras-el-hanout
		\item une demi cuillerée à café de curry
		\item 1 bouquet garni
		\end{itemize}
\ingredient sel
\end{ingredients}

\begin{preparation}
\etape Coupez en petits cubes aubergine, carotte, poireau, navets et céleri et réservez dans un saladier.
\etape Dans un bol, faites chauffer le bouillon cube et de l'eau 2 minutes au micro-onde. Ajoutez le concentré de tomate, les épices, le jus de citron et l'ail mixé puis mélangez à la fourchette. 
\etape Dans la marmite, faites revenir l'agneau coupé en gros morceaux (4x4cm) dans un peu d'huile d'olive d'un seul coté.
\etape Mouillez avec les 2L d'eau puis portez à ébullition pendant 30 minutes tout en écumant de temps en temps. 
\etape Ajoutez alors les raisins, le contenu du saladier et du bol puis laissez mijoter pendant 2h30 environ à feu moyen 4/9
\etape Si vous voulez mettre du poulet ou des merguez, rajoutez-les 1h avant la fin de cuisson sinon ça sera trop cuit
\etape Dans le saladier déjà sale des légumes, versez 30g d'huile, une demi cuillère à café de ras el-hanout et 5g de sel. Mélangez le tout avant de mettre la semoule (sinon ça va faire des boulettes d'épices).
\etape Ajoutez la semoule et mélangez bien le tout. 
\etape Ajoutez alors de l'eau jusqu'à recouvrir la semoule, secouez le saladier pour égaliser la surface et laissez reposer 30 minutes avant d'égrener à la fourchette.
\end{preparation}
\end{recette}


\section{Curry de poisson}
\begin{recette}{Curry de poisson}{}{30min.}{}
\begin{ingredients}
\ingredient 1kg de morceaux de poisson
\ingredient 1 oignon
\ingredient 2cac de curry
\ingredient 45ml de jus de citron (1 citron)
\ingredient 1 morceau de gingembre (50g)
\ingredient 30cl d'eau + 1 bouillon cube
\ingredient 20cl de crème fraiche
\ingredient sel, poivre
\end{ingredients}

\begin{preparation}
\etape Emincez l'oignon, mixez le gingembre et réservez dans un récipient
\etape Coupez le poisson en cube grossiers (j'ai fait des tranches d'environ 1cm dans des filets de 4-5cm de large
\etape Faites revenir l'oignon et le gingembre sans chercher à faire bien dorer
\etape Ajoutez le curry, le bouillon et le sel puis faites cuire 5 minutes à feu doux
\etape Ajoutez la crème fraiche, portez à ébullition puis rajoutez le poisson
\etape Faites alors cuire 7 minutes à couvert et à feu moyen-vif (7/9)
\etape En fin de cuisson, ajoutez le jus de citron et éventuellement de la coriandre. 
\end{preparation}
\end{recette}

\section{Croque monsieur}
\begin{recette}{Croque monsieur}{3}{}{}\index{croque monsieur}
\begin{ingredients}
\ingredient 24 tranches de pain de mie
\ingredient 120g de beurre (5g par tranche)
\ingredient fromage en tranche
\begin{remarque}
J'achète un morceau d'emmental de 500g, et je fais des tranches. Avec deux tranches sur la largeur je fais une surface de pain 
de mie.
\end{remarque}

\ingredient 4 tranches de jambon blanc
\begin{remarque}
C'est aussi très bon si on remplace le jambon par du saumon ou de la charcuterie diverse.
\end{remarque}

\ingredient poivre
\end{ingredients}

\begin{preparation}
\etape Faites fondre la moitié du beurre et mélangez ensuite le reste du beurre en petit cube pour obtenir du beurre très mou. Si c'est trop fondu, mettez le bol de beurre dans un bain marie d'eau froide pendant 15 minutes.
\etape Beurrez un coté du pain de mie
\etape disposez le coté beurré à l'extérieur (il sera en contact avec la partie chaude)
\etape disposez une couche de fromage, une couche de jambon, une pincée de poivre, puis une autre couche de fromage et enfin une 
tranche de pain de mie, coté beurré à l'extérieur
\etape faites cuire dans un appareil pour les croque-monsieurs (ou au four le cas échéant)
\end{preparation}
\end{recette}

\section{Escalopes à la milanaise}
\begin{recette}{Escalopes à la milanaise}{3}{}{}\index{escalopes à la milanaises}\index{escalopes}\index{veau}
\begin{ingredients}
\ingredient 2 escalopes de veau
\ingredient 30g de parmesan râpé
\ingredient 30g de chapelure
\ingredient 1 œuf
\end{ingredients}

\begin{preparation}
\etape Prenez deux assiettes. Dans l'une d'elle, on mélange chapelure et parmesan. Dans l'autre on bat l'œuf
\etape On trempe les deux faces des escalopes d'abord dans l'œuf puis dans le mélange chapelure/parmesan
\etape Faites ensuite cuire les escalopes dans un peu de matière grasse.
\end{preparation}
\end{recette}


\section{Garbure}
\begin{recette}{Garbure}{3}{30 min}{5 heures}\index{garbure}\index{confit}
\begin{ingredients}
\ingredient 1 chou vert, coupé en fines lanières
\ingredient 400 g de poitrine nature coupée en gros dés
\begin{remarque}
Une carcasse, des cous, talon de jambon coupé en dés et couenne conviennent très bien
\end{remarque}

\ingredient 200 g de haricots lingots (à faire tremper la veille dans l’eau) 
\ingredient 6 cuisses de confit de canard
\ingredient 4 pommes de terre, épluchées et coupées en quartier
\ingredient 4 gousses d’ail, entières
\ingredient 2 poireaux, taillés en rondelles
\ingredient 2 navets, coupés en quartier
\ingredient 2 carottes, coupées en rondelles
\ingredient 2 oignons, émincés
\ingredient 12 grains de poivre, thym, laurier
\end{ingredients}


\begin{preparation}
\etape Mettez la viande, le poivre, thym laurier.
\etape Rajoutez les légumes et les clous de girofle piqué dans un oignon. 
\etape si vous utilisez du confit, rajoutez le.
\etape recouvrir la viande puis portez à ébullition
\end{preparation}

\begin{cuisson}
Faites mijoter 4h et à couvert. Rajoutez le chou petit à petit (il prend trop de place cru, il faut attende qu'il cuise un peu pour pouvoir en rajouter). 

J'ai fait cuire 7h parce que j'ai dû rajouter le chou petit à petit, j'avais pas la place de tout mettre d'un coup.
\end{cuisson}
\end{recette}


\section{Gigot de 7h}
\begin{recette}{Gigot de 7h}{3}{45 min}{7 heures}\index{gigot de 7h}\index{gigot}\index{agneau}
% source: https://lescolisduboucher.com/recettes/agneau/gigot-de-7-heures-a-la-cuillere-d-alain-ducasse
\begin{ingredients}
\ingredient[Pate morte]
\ingredient 300g de farine
\ingredient 20cl d'eau
\ingredient un pincée de sel
\ingredient[Gigot]
\ingredient un gigot ou épaule d'agneau
\ingredient 4 ou 5 gousses d'ail
\ingredient 2 gros oignons
\ingredient 1 carotte
\ingredient 20cl de vin blanc sec
\ingredient 25cl de bouillon
\ingredient sel, poivre, thym, laurier
\end{ingredients}


\begin{preparation}
\etape Préparez la pâte morte et laissez reposer.
\etape Emincez carottes et oignons.
\etape Salez et poivre abondamment le gigot des deux cotés.
\etape Dans la cocotte préalablement huilée, saisissez le gigot de tous les cotés jusqu'à avoir une jolie croûte dorée.
\begin{remarque}
Suivant la taille de votre cocotte, il vous faudra peut-être couper l'os situé à l'extrémité, vers la souris du gigot, ne jetez 
pas ce bout d'os, mettez-le au fond de la cocotte, cela apportera encore plus de goût.
\end{remarque}
\etape Réservez le gigot puis faites fondre l'oignon et la carotte dans les sucs pendant 5 minute environ
\etape Déglacez avec le vin blanc. Ajoutez le bouillon et les herbes. 
\etape Ajoutez le gigot préalablement salé et poivré sur les légumes. Ajoutez enfin les gousses d'ail
\etape Roulez la pâte morte en un boudin de 2cm de diamètre environ, puis posez le sur le tour de la cocotte. Posez ensuite le couvercle par dessus pour le sceller (je déconseille de mettre sur boudin sur le couvercle d'abord, il risque de tomber en retournant le couvercle).
\end{preparation}

\begin{cuisson}
Faites cuire le gigot dans la cocotte fermée pendant 7 heures à 120°C. 

Je conseille de servir ce plat avec des pommes de terre au four (\refsec{sec:pomme-de-terre-four}). Il est aussi possible 
d'épaissir un peu la sauce à la fin de la cuisson du gigot. 
\end{cuisson}
\end{recette}

\section{Gratin Dauphinois}
\begin{recette}{Gratin Dauphinois}{3}{}{}\index{gratin dauphinois}\index{pomme de terre}
\begin{ingredients}
\ingredient $800$ g de pommes de terre
\ingredient $25$ cl de lait entier
\ingredient $30$ cl de crème fraîche
\ingredient sel
\ingredient poivre
\ingredient noix de muscade
\ingredient $1$ grosse noix de beurre
\ingredient $3$ gousses d'ail
\end{ingredients}

\begin{preparation}
\etape Laver, éplucher et émincer les pommes de terre en tranches de $3$ mm environ.\footnote{Ne pas les laver après la coupe.}
\etape Les disposer dans une casserole avec $25$ cl de lait (entier si possible), une grosse noix de beurre, sel, poivre et 
muscade.
\etape Porter à ébullition puis baisser le feu légèrement et poursuivre la cuisson une dizaine de minutes.\footnote{Remuer de 
temps en temps avec une spatule pour éviter que la préparation attache.}
\etape Quand les pommes de terres s'enrobent d'une sorte de crème, verser à ce moment $30$ cl de crème.
\etape Laisser cuire à petit feu pendant une dizaine de minutes environ.
\etape Retirer du feu, ajouter l'ail.
\etape Disposer délicatement les pommes de terre dans un plat à gratin.
\etape Aplanir la surface et laisser refroidir pour que les goûts se mélangent.
\end{preparation}

\begin{cuisson}
Enfourner à $180\degres$ et laisser cuire entre $20$ et $30$ minutes. Servir dans le plat de cuisson.
\end{cuisson}
\end{recette}

\section{Hachis Parmentier}
\begin{recette}{Hachis Parmentier}{3}{2h}{30 min.}\index{hachis parmentier}\index{pomme de terre}
\begin{ingredients}[1 plat à gratin]
\ingredient 2.5kg de pommes de terre
\ingredient 600g de steak hachés
\ingredient 2 oignons
\ingredient 1 carotte
\ingredient 60cl de lait
\ingredient 100g de fromage rapé
\ingredient sel
\ingredient noix de muscade
\ingredient 40g de beurre
\end{ingredients}

\begin{preparation}
\etape Emincez les oignons et carottes.
\etape Épluchez les pommes de terre et coupez-les en gros cubes. 
\etape Faites chauffer une grande casserole remplie d’eau salée. Aux premiers bouillons, plongez-y les morceaux de pommes de terre et faites-les cuire pendant 25 min jusqu’à ce qu’elles soient fondantes.
\etape Égouttez les pommes de terre puis passez-les encore chaudes au moulin à légumes. 
\etape Mélangez-les avec le lait préalablement chauffé et 30 g de beurre. Salez et poivrez puis ajoutez la noix de muscade moulue et mélangez bien.
\etape Faites revenir le steak hachés puis réservez
\etape Dans la graisse du steak hachés, faites revenir les légumes
\etape Mélangez alors le tout avec la purée. Ajoutez enfin du fromage rapé
\end{preparation}

\begin{cuisson}
Préchauffez le four à 210°C. Déposez dans un plat à gratin et faites gratiner le hachis parmentier de bœuf au four pendant 30 min.
\end{cuisson}
\end{recette}

\section{Lapin confit}
\begin{recette}{Lapin confit}{4}{}{}\index{lapin}\index{confit}
\begin{ingredients}
\ingredient morceaux de lapin
\ingredient 2L d'huile de tournesol
\ingredient 200g de sel
\ingredient poivre, thym, laurier, romarin
\end{ingredients}

\begin{preparation}
\etape Préparez une assiette avec un peu de gros sel. Posez les morceaux de chaque coté dans le sel pour que quelques grains restent collés (ne pas en mettre plus pour les petits morceaux et ne pas chercher à saler dans les petits interstices). Déposez alors les morceaux dans un plat, puis mettez un peu d'épices et une pincée de sucre
\etape recommencez pour chaque morceaux. Laissez ainsi entre une nuit et 24h maxi. 
\etape Le lendemain, rincez et essuyez les morceaux
\end{preparation}

\begin{cuisson}
Faites chauffer l'huile dans une cocotte en fonte (ou équivalent) au four à 170°C. Mettez le thym, romarin et laurier dans l'huile pour la parfumer si vous en avez mis dans le sel la veille (rincez les aromates à l'eau avant). Faites cuire les morceaux pendant 5 minutes environ. 

Baissez alors le four à 130°C et faites cuire une à deux heures. Le lapin sera toujours blanc. Il faut le passer au four environ 20 minutes à 180°C pour qu'il devienne croustillant et consommable.
\end{cuisson}
\end{recette}

\section{Lapin à la tomate}
\begin{recette}{Lapin à la tomate}{5}{1h}{1h}\index{lapin à la tomate}\index{lapin}\index{poulet}
\begin{ingredients}[4 pers.]
\ingredient un lapin
\ingredient un oignon
\ingredient 1 ou 2 carottes
\ingredient 100 ou 200g de lardons fumés
\ingredient 150 à 200g de champignons
\ingredient un cube de volaille et 20cl d'eau
\ingredient 20cl de vin blanc (un verre)
\ingredient une cuillère à soupe rase de farine
\ingredient une boîte de coulis de tomate (entre 200 et 500g, la quantité exacte importe peu)
\ingredient sel, poivre, herbes de Provence
\end{ingredients}

\begin{preparation}
\etape Faites bien dorer les morceaux de lapin dans du beurre (et un peu d'huile) ; en plusieurs fois s'il n'y a pas de place 
dans la cocotte (attention, ça éclabousse!).
\etape Réservez les morceaux de lapin dans une assiette.
\etape Faites revenir l'oignon émincé et les carottes coupés en petits morceaux (rajoutez un peu d'huile si besoin). Réservez.
\begin{remarque}
Pendant ce temps, je met le bouillon cube et l'eau dans un bol que je fais chauffer au micro-onde, puis je mélange avec une 
fourchette quand c'est chaud.
\end{remarque}
\etape Rajoutez ensuite les lardons, faites revenir. Réservez les lardons et conservez la graisse. 
\etape Ajoutez alors les champignons, faites les revenir, puis ajoutez les lardons, oignon et carotte.
\etape Saupoudrez alors le tout avec la farine et mélangez. 
\etape Mouillez ensuite avec un verre de vin blanc, le coulis de tomate et le bouillon préalablement préparé. Ajoutez les herbes 
de Provence et mélangez.
\etape Rajoutez les morceaux de lapin dans la cocotte et remuez-les un peu dans la sauce.
\end{preparation}

\begin{cuisson}
Couvrez et laissez cuire à feu très doux 1h en mélangeant de temps en temps. Ajoutez sel et poivre en fin de cuisson.
\begin{remarque}
Les lardons salent déjà pas mal la sauce, je ne la resale quasiment jamais. Par contre je poivre avant de faire mijoter une 
heure.
\end{remarque}
\end{cuisson}
\end{recette}

\section{Lapin en gibelotte}
\begin{recette}{Lapin en gibelotte}{4}{}{}\index{paupiettes}\index{lapin}\index{gibelote}\index{poulet}
\begin{ingredients}
\ingredient un lapin
\ingredient 100 g de champignons
\ingredient deux ou trois oignons
\ingredient 25 cl de vin blanc sec
\ingredient 25 cl de bouillon (1 bouillon cube de volaille)
\ingredient $100\unit{g}$ de lardons
\ingredient 1 cuillère à soupe rase de farine
\ingredient 2 cuillères à café de fond de veau
\ingredient sel, poivre (un sachet d'arômes).
\end{ingredients}

\begin{preparation}
\etape Faire revenir les lardons (réservez), puis les champignons (réservez), et enfin les oignons (réservez).
\etape Découper le lapin et faire dorer les morceaux dans de l'huile d'olive (penser à laisser un peu plus longtemps les cuisses 
qui ont plus de viande)
\etape Réserver les morceaux
\etape Dans les sucs, mettez une cuillère à soupe rase de farine. Laissez roussir, puis diluez avec un peu du vin blanc.
\etape Ajoutez alors le reste de vin blanc, le bouillon, 2 cuillères à soupe de fond de veau, les lardons, oignons et 
champignons. Remuez pour diluer le fond de veau.
\etape Arômatisez selon votre gout.
\end{preparation}

\begin{cuisson}
Faire cuire à feu doux pendant 1h30.

\begin{remarque}
C'est aussi excellent avec des paupiettes de veau.

Dans ce cas, à la fin de la cuisson, stockez séparément les paupiettes et la sauce, pour pouvoir dégraisser la sauce une fois 
froide.
\end{remarque}
\end{cuisson}
\end{recette}

\section{Lasagnes}
\begin{recette}{Lasagnes}{4}{1h}{5h+24h+1h}\index{lasagne}\index{bœuf}
\begin{ingredients}
\ingredient[bolognaise]
\ingredient voir page~\pageref{sec:bolognaise}
\ingredient[Béchamel]
\ingredient 500ml de lait
\ingredient 30g de beurre
\ingredient 30g de farine 
\ingredient Noix de muscade
\ingredient[lasagnes]
\ingredient 400g de mozarella
\ingredient 200g de gruyère rapé
\ingredient 600g ($\sim 2$ paquets) de pâtes fraiches pour lasagne
\end{ingredients}

\begin{remarque}
La recette donne chez moi 6 couches de pâtes (donc 5 couches de bolognaise) avec 2 paquets de 250g de pâtes fraiches (chaque paquet a 6 pâtes, et je prend 2 pâtes par couche). 
\end{remarque}

\begin{preparation}
\etape La veille, préparez la bolognaise qui mijotera 3h environ

\etape Le lendemain, préparez un roux. Faites fondre le beurre puis incorporez-y la farine. Laissez épaissir à feux doux sans colorer (roux blanc).
\etape Réservez puis portez du lait à ébullition.
\etape Hors du feu, incorporez le lait au roux petit à petit pour ne pas former de grumeaux. 
\etape Faites alors chauffer puis tournez jusqu'à ce que ça épaississe (il ne faut plus que ça ressemble à du lait ou de la crème). Je le met assez fort et je remue rapidement jusqu'à ce que ce soit bon. Il faut 2-3 minutes à peine. 
\etape Couvrez la béchamel d'un film plastique en attendant de vous en servir pour éviter qu'une croute se forme.
\etape Préchauffez le four à 180°C
\etape Préparez un plat à gratin pour le montage. Déposez au fond une couche de pâte, puis une couche de sauce par dessus ($\sim$ 2 louches). 
Disposez de la mozarella. Continuez jusqu'à épuisement de la garniture. 
\etape Finissez par une couche de pâte, la béchamel, puis le fromage râpé. 
\end{preparation}

\begin{cuisson}
Enfournez 30 minutes à 180°C. Vous pouvez terminer par 5 minutes en grill afin de faire dorer le dessus.

\end{cuisson}
\end{recette}

\section{Lasagnes au saumon}
\begin{recette}{Lasagnes au saumon}{4}{}{}\index{lasagne}\index{saumon}
\begin{ingredients}
\ingredient 500g de pavé de saumon
\ingredient 3 oignons
\ingredient 50cl de crème fraiche
\ingredient 15cl de crème fraiche liquide
\ingredient 400g de mozarella
\ingredient 200g de gruyère râpé
\ingredient 450g ($\sim 2$ paquets) de pâtes fraiches pour lasagne
\end{ingredients}

\begin{preparation}
\etape Couper le saumon en petits cubes
\etape Émincez l'oignon
\etape Saisir les morceaux de saumon dans un peu d'huile, puis réservez.
\etape Faites revenir l'oignon (rajoutez un peu d'huile au besoin)
\etape Remettez alors le saumon. Salez, poivrez et ajoutez la crème fraiche
\etape Préchauffez le four à 180°C
\etape Préparez un plat à gratin pour le montage. Déposez au fond une couche de pâte, puis une couche de sauce par dessus. 
Disposez des tranches de mozarella et du gruyère râpé. Continuez jusqu'à épuisement de la garniture. 
\etape Finissez par une couche de pâte, déposez les derniers morceaux de mozarella, la crème fraiche liquide, puis le fromage 
râpé. 
\end{preparation}

\begin{cuisson}
Enfournez 30 minutes à 180°C. Vous pouvez terminer par 5 minutes en grill afin de faire dorer le dessus.
\end{cuisson}
\end{recette}

\section{Légumes farcis}
\begin{recette}{Légumes farcis}{4}{}{}\index{aubergine}\index{farce}\index{poivron}
\begin{ingredients}
\ingredient Aubergines ou autres légumes à farcir (comptez 6 demi aubergines farcies pour 1kg de saucisse et un plat à gratin normal.
\ingredient[Farce]
\ingredient 750g de boule de campagne (pour récupérer la mie)
\ingredient 1kg de chair à saucisse
\ingredient 3 oignons
\ingredient 2 gousses d'ail
\ingredient persil
\ingredient sel, poivre, noix de muscade
\ingredient 5cl de cognac
\ingredient [option] 1 oeuf (je ne le met pas)
\end{ingredients}

\begin{preparation}
\etape Préchauffez le four à 200°C
\etape Creusez les légumes
\etape mixez oignon, ail et persil
\etape Mixez la mie du pain
\etape Mélangez tous les ingrédients et farcissez les légumes avec
\end{preparation}

\begin{cuisson}
Enfournez 45 minutes à 200°C (les patates il faut compter 1h, les champignons en théorie 30 minutes)
\end{cuisson}
\end{recette}

\section{Lentilles}
\begin{recette}{Lentilles}{3}{}{}\index{lentilles}
\begin{ingredients}
\ingredient 500g de lentilles
\ingredient un oignon
\ingredient une demi carotte
\ingredient deux gousses d'ail
\ingredient 6 saucisses / saucisse de morteaux, autre morceaux fumés
\ingredient poitrine demi-sel
\ingredient cube de bouillon de volaille
\ingredient laurier sauce, herbes de Provence, sel, poivre
\end{ingredients}

\begin{preparation}
\etape Découpez finement la carotte et oignons en mirepoix (petits cubes) et mixez l'ail.
\etape Mettez le bouillon cube dans un bol d'eau au micro onde 2 minutes puis mélangez. Ajoutez-y l'ail.
\etape Faites des entailles verticales dans la couenne pour la séparer en 4 demi cylindres (afin que le gras s'échappe mieux)
\etape Ajoutez l'huile dans une sauteuse et passez brièvement le petit salé du coté de la couenne.
\etape Réservez le petit salé et colorer les saucisses sur toutes les faces.
\etape Retirez les saucisses de la cocotte.
\etape Ajoutez les légumes dans la cocotte et les faire revenir doucement à l'huile.
\etape Ajoutez les lentilles. Mouillez à hauteur avec de l'eau et le cube de bouillon de volaille, rajoutez les viandes et 
laissez cuire 1h à feu doux.
\end{preparation}
\end{recette}

\section{Makis saumon/avocat}
\begin{recette}{Makis saumon/avocat}{}{1h30}{}
\begin{ingredients}[5 rouleaux, pour 2 personnes]
\ingredient 315g de riz % (315g de riz la fois où j'ai fait, i.e 2 dose du cuiseur à riz)
\ingredient 50g de vinaigre de riz (4 cuillères à soupe)
\ingredient 30g de sucre (2 cuillères à soupe)
\ingredient 8g de sel (1/2 cuillère à soupe)
\ingredient 1 avocat
\ingredient 100g de saumon
\ingredient 5-6 feuilles de Nori (ou feuille de riz)
\end{ingredients}

\begin{preparation}
\etape Faites cuire le riz
\etape Dans une casserole, mettre le vinaigre, le sucre et le sel. Chauffez doucement jusqu'à dilution du sucre.
\etape Ensuite versez dans le riz cuit, et mélangez
\etape Laissez refroidir pendant 1h minimum.
\etape Préparez le saumon et l'avocat. 1cm de large maximum, et entre 3 et 5mm d'épaisseur pour les lamelles. 
\etape Etalez du riz sur environ 9cm de large et la plus petite épaisseur possible (environ 5mm en pratique), un peu au milieu de la feuille de nori. 
\begin{center}
\includegraphics[width=7cm]{figures/maki.pdf}
\end{center}
\etape Humidifiez les deux derniers centimètres de la feuille afin de faire la jointure.
\etape Réservez au frais jusqu'au moment du repas
\etape Préparez un bol d'eau froid avec un sopalin imbibé. Découpez délicatement sans appuyer afin de ne pas écraser. Mouillez/nettoyez la lame périodiquement afin qu'elle n'accroche pas pendant la découpe des makis.  
\end{preparation}
\end{recette}

\section{Nouilles japonaises}
\begin{recette}{Nouilles japonaises}{5}{4h}{1h}\index{nouille}\index{poulet}\index{Champignon noir}
\begin{ingredients}
\ingredient 450g de nouilles somen (fines)
\ingredient 2 cuisses entières de poulets (ou 4 pilons ou 4 cuisses)
\ingredient 15g de champignons noirs déshydratés
\ingredient 2 oignons
\ingredient 12.5cl ou 150g de sauce pour sauté (attention à la teneur en sel de la sauce 
\ingredient 1 demi poivron rouge
\ingredient [optionnel] 1 cac d'huile de sésame
\end{ingredients}

\begin{preparation}
\item Faites tremper les champignons noirs dans de l'eau la veille (ou dans de l'eau chaude $\sim$ 30 minutes si vous n'avez pas eu le 
temps)
\begin{remarque}
Enlevez le pieds des champignon noir si présent, cette partie est un peu dure
\end{remarque}

\item Émincez le plus finement possible l'oignon et le poivron
\item Émincez le champignon noir en coupant grossièrement en petits dés (entre 0.5 et 0.75cm de coté)
\item Enlevez le maximum de viande des morceaux de poulets et coupez les en dés grossiers. 
\item Faites chauffer l'eau pour les pâtes et saler un peu.
\item Faites revenir les os, puis mettez les dans l'eau pour les nouilles. Réutilisez l'eau des champignons si vous l'avez gardé
\item Faites revenir les morceaux de poulet et laissez le cuire (il faut 
qu'il soit cuit pour la suite) puis réservez le
\item Faites revenir oignon, poivron en rajoutant un peu d'huile. Rajoutez les champignons à mi cuisson
\item Laissez bien dorer. Rajoutez un peu d'eau en fin de cuisson au besoin pour déglacer les sucs
\item Plongez les pates dans l'eau bouillante 2min30s, égouttez puis faites revenir pendant 2-3 minutes les pâtes avec les 
légumes et le poulet en rajoutant la sauce.
\end{preparation}
\end{recette}

\section{Paëlla (à la moi)}
\begin{recette}{Paëlla (à la moi)}{2}{4h}{30 min}\index{paëlla}\index{fruits de mer}\index{chorizo}
\begin{ingredients}
\ingredient[Paella]
\ingredient 1kg de riz étuvé
\ingredient 1800g de bouillon
\ingredient 20g de sel (attention, le sel doit être mis dans le bouillon, mais la quantité dépend de la quantité de riz)
\ingredient 1kg de crevettes cuites
\ingredient 1kg d'anneaux de calmars décongelés (le moins cher en fait)
\ingredient 500g-1kg de moules
\ingredient 500g de coques (facultatif mais mieux)
\ingredient 200g de chorizo doux
\ingredient 1kg de pilons de poulet
\ingredient 140g de petits pois (les petites boites)
\ingredient 5g de paprika (2.5 cac)
\ingredient 1 dose de safran
\ingredient[fumet de poisson]
\ingredient les têtes et écailles des crevettes et autres coquilles
\ingredient 20g de beurre
\ingredient 1 oignon
\ingredient 1 petit poireau
\ingredient Thym, laurier
\ingredient sel (voir la quantité de riz)
\ingredient 75cl de vin blanc
\end{ingredients}

\begin{preparation}
\etape Durant la recettes, vous aurez besoin des récipients suivants:
\begin{itemize}
\item un tupperware si vous préparez la veille, sinon un saladier, dans lequel stocker chair de crevettes, moule et fruits à coque
\item un saladier dans lequel stocker les déchets de crevettes, moule et fruits à coque (une fois le bouillon fait, vous n'en aurez plus besoin)
\item un grand récipient dans lequel stocker le bouillon et tous les liquides au fur et à mesure
\item Une assiette ou n'importe quoi pour stocker le chorizo
\item un saladier pour stocker le poulet cuit
\end{itemize}
\etape Ces étapes peuvent être faites la veille pour gagner du temps
\etape Coupez grossièrement l'oignon et le poireau et réservez dans un grand saladier.
\etape Pelez les crevettes et réservez d'une part la chair dans un tupperware et d'autre part les déchets dans le saladier avec poireau et oignon.
\etape Faites cuire les fruits à coques et les moules dans du vin blanc et sortez au fur et à mesure quand ils sont ouverts.
\etape Réservez le jus de cuisson pour plus tard. 
\etape Réservez la chair des moules et des fruits à coques avec la chair des crevettes, et les coquilles avec les déchets des crevettes.
\etape Faites cuire les déchets de crustacés 3 minutes dans le beurre avec l'oignon et le poireau
\etape Ajoutez alors thym, laurier, le sel. Ajoutez le jus de cuisson des fruits à coque et compléter pour avoir environ 1L de bouillon
\etape Portez à ébullition. Une fois que ça bout, baissez le feu, couvrez et maintenez à petite ébullittion pendant 30 minutes environ. 
\etape Filtrez et réservez le bouillon pour plus tard
\etape Ensuite, ou le lendemain:
\etape Faites cuire la moitié du chorizo, réservez, puis faites cuire le poulet dans le gras du chorizo. Rajoutez un peu d'eau (20cl maxi), couvrez et faites cuire 30 minutes à feu doux et réservez. Récupérez l'eau et mettez dans le bouillon
\etape Pendant ce temps Emincez les anneaux d'encornets: je coupe les petits anneaux en deux, et les gros anneaux en 4. Puis mettez dans une passoire pour les égoutter (pensez à récupérer le jus)
\etape Faites chauffer la marmite à feu très vif. Faites cuire les encornets pendant 5 minutes en remuant. Réservez avec les fruits de mers
\etape Faites cuire le reste de chorizo émincé puis réservez avec la première moitié que vous avez déjà fait cuire
\etape Faites alors cuire le riz, le paprika et le safran dans le gras du chorizo pendant 5 à 10 minutes
\etape Ajoutez alors le bouillon et complétez avec de l'eau pour avoir la quantité nécessaire (adaptez en fonction de la quantité de riz, voir plus haut). 
\etape Faites bouillir, couvrez, et laissez cuire à feu moyen pendant 20 minutes. 
\etape Eteignez sans ouvrir et laisser reposer pendant 20 minutes
\etape Ajoutez alors le poulet et les fruits de mers et réchauffez 5 minutes puis servez.
\end{preparation}
\end{recette}

\section{Pâtes à l'ail}
\begin{recette}{Pâtes à l'ail}{3}{30 min.}{}\index{pâtes}\index{ail}
\begin{ingredients}[Pour 500g de pâtes]
\ingredient 20g d'ail en poudre (pas le même  goût avec l'ail frais
\ingredient 60g de parmesan
\ingredient 20 cl de crème liquide légère 
\ingredient sel, poivre
\end{ingredients}


\begin{preparation}
\etape Faites chauffer le parmesan et l'ail en poudre dans la crème pour mélanger puis éteignez le feu, salez, poivrez et laissez reposer à couvert.
\etape Une fois les pâtes cuites, rajoutez dans le plat le contenu de la casserole. Remuez jusqu'à ce que ça ait la consistance qui vous convienne (avec la chaleur des pâtes, ça va épaissir un peu).
\end{preparation}
\end{recette}

\section{Pâtes à la bolognaise (à la moi)}
\begin{recette}{Pâtes à la bolognaise (à la moi)}{4}{1h}{3h}\index{pâtes}\index{pâtes à la bolognaise}\index{bolognaise}
\begin{ingredients}
\ingredient 500g de viande hachée
\ingredient 250g de chair à saucisse
\ingredient 100g de lardons fumés
\ingredient deux oignons
\ingredient 1 carotte
\ingredient 1 petite branche de céleri (ne pas en mettre trop sinon c'est pas bon)
\ingredient 1kg de purée de tomate
\ingredient 25cl de vin rouge
\ingredient 25cl de lait
\ingredient sel, poivre, céleri, laurier (deux feuilles), sucre ou lait (pour adoucir le coulis)
\end{ingredients}
% La dernière fois, j'ai fait 1.5kg de coulis, 40cl de vin, 40cl de lait, et 800g de haché (en conservant les autres proportions)

\begin{preparation}
\etape Coupez la carotte en petits dés et émincez l'oignon.
\etape Faites revenir la viande (steak haché, lardon et chair à saucisse)
\etape Réservez la viande dans le récipient qui servira à mijoter. 
\etape Retirer le surplus de gras et faire revenir les légumes
\etape Ajoutez la viande aux légumes, une cuillère à soupe de farine.
\etape Déglacez les sucs à l'aide du vin dans la poêle qui a servi à saisir. 
\etape Versez le vin, le lait, le coulis de tomate puis mélangez. Ajoutez alors la branche de céleri
\end{preparation}


\begin{cuisson}
Couvrir et laisser cuire à feu très doux (3/9, pas moins, il faut une petite ébullition quand même) pendant 3h. Ajoutez sel et poivre en fin de cuisson.
\end{cuisson}
\end{recette}\label{sec:bolognaise}

\section{Pâtes à la carbonara}
\begin{recette}{Pâtes à la carbonara}{4}{}{}\index{pâtes}\index{pâtes à la carbonara}\index{carbonara}
\begin{ingredients}[Pour 500g de pâtes]
\ingredient 100g de lardons
\ingredient 2 oignons
\ingredient 50 à 100g de parmesan
\ingredient 20 cl de crème liquide légère 
\ingredient herbes aromatiques (herbes de Provence, romarin,\dots)
\ingredient sel (1/2 cac), poivre
\end{ingredients}
% la recette italienne originale:
% Pour 500g de pâtes
% 300g de guanciale/parmesan (essayer avec 200?)
% 300g de pancetta (essayer avec 150g?)
% 1 oeuf entier
% 4 jaunes d'oeufs

\begin{preparation}
\etape Faites cuire les lardons
\etape Une fois cuits, réservez-les dans un saladier puis faites rissoler l'oignon afin qu'il soit bien doré.
\etape Dans le saladier, rajoutez l'oignon. Déglacez alors la poêle avec un peu d'eau et versez dans le saladier.
\etape Ajoutez la crème liquide, les herbes aromatique, le sel et le poivre, puis laissez infuser jusqu'à ce que les pâtes soient cuite
\etape Une fois les pâtes cuites, rajoutez dans le plat le contenu de la casserole et le parmesan. Remuez jusqu'à ce que ça ait la consistance qui vous convienne (avec la chaleur des pâtes, ça va épaissir un peu).
\begin{remarque}
S'il y a vraiment trop de liquide, rallumez un peu le feu sous la marmite tout en remuant jusqu'à ce que la consistance vous convienne.
\end{remarque}
\end{preparation}

\end{recette}

\section{Pâtes au boudin}
\begin{recette}{Pâtes au boudin}{2}{10 min}{}\index{pâtes}\index{boudin}
\begin{ingredients}
\ingredient 200 g Boudin à l'oignon
\ingredient 500g de pâtes
\end{ingredients}

\begin{preparation}
\etape Enlevez la peau au boudin, et coupez le en morceaux grossiers
\etape Mettez ces morceaux dans une poêle à feu moyen
\etape Avec une spatule en bois, remuez et écrasez un peu les morceaux pour qu'ils se désolidarisent et finissent par former une 
sorte de bouilli peu agréable à l'œil.
\etape Ajoutez ça aux pâtes et remuez pour que ça soit homogène.
\end{preparation}
\end{recette}

\section{Pâtes au hareng}
\begin{recette}{Pâtes au hareng}{4}{1h}{}\index{pâtes}\index{hareng fumé}
\begin{ingredients}
\ingredient 25cl de crème fraîche liquide + 5cl d'eau (sinon c'est trop épais)
\ingredient 1 fenouil
\ingredient 200g de hareng fumé
\ingredient 100g de parmesan
\ingredient sel, poivre, aneth, un peu de jus de citron
\end{ingredients}

\begin{remarque}
Il est possible de préparer la sauce avant, puis faire cuire les pâtes plus tard et d'ajouter la sauce et le hareng fumé sur les pâtes chaudes
\end{remarque}


\begin{preparation}
\etape Émincez le fenouil (comme un oignon). Coupez le hareng fumé en lamelles puis en morceaux
\etape Faites revenir le fenouil émincé dans un peu d'huile pendant 10 minutes environ.
\etape Rajoutez alors un peu d'eau (5cl environ), couvrez et laissez mijoter à feu doux (3/9) pendant 10 minutes (c'est pour attendrir le fenouil qui a tendance à être encore dur à ce stade).
\etape Enlevez le couvercle, et laissez évaporer l'eau résiduelle s'il en reste encore beaucoup
\etape Réservez dans un saladier. 
\etape Ajoutez alors un peu d'eau pour déglacer (5cl) dans la poële et versez dans le saladier (faites réduire et ne versez qu'un peu avant que ça n'accroche de nouveau). 
\etape Ajoutez alors la crème liquide, un peu de citron, du poivre et le hareng dans le saladier. 
\etape Laissez infuser jusqu'à ce que les pâtes soient cuites (possibilité de préparer la suite de la recette plus tard, pratique pour prévoir le repas un peu à l'avance)
\etape Égouttez les pâtes, ajoutez alors le contenu du saladier et le parmesan dans les pâtes. 
\etape Remuez jusqu'à avoir la consistance voulue puis servez les.
\end{preparation}
\end{recette}

\section{Pâtes au pesto}
\begin{recette}{Pâtes au pesto}{2}{10 min}{}\index{pâtes}\index{pesto}\index{basilic}\index{pignon de pin}
\begin{ingredients}
\ingredient 2 gousses d'ail
\ingredient 50g de parmesan râpé
\ingredient 25g de pignon de pin
\ingredient 50g de feuilles de basilic frais
\ingredient 5cl d'huile d'olive
\ingredient 500g de pâtes
\end{ingredients}

\begin{preparation}
\etape Faites griller les pignons de pin à la poêle sur feu moyen. Surveillez les afin de ne pas les faire cramer. Dès que 
c'est roussi, arrêtez.
\etape dans un mixer, ajoutez l'ail, le basilic, les pignons et le parmesan. Mixez jusqu'à obtenir une pâte lisse et homogène 
(des coups de spatules seront nécessaires pour mixer les morceaux récalcitrants). 
\etape Dans un petit saladier, incorporez l'huile d'olive à la pâte ainsi obtenue. 
\begin{remarque}
Je rajoute l'huile dans le mixer, je remixe, et je met le tout dans un saladier. Ensuite, pour nettoyer le mixer, je met un peu 
de savon et d'eau, ça facilite le lavage
\end{remarque}
\etape incorporez cette sauce dans les pâtes une fois qu'elles seront cuites. 

\end{preparation}
\end{recette}

\section{Pâtes au roquefort}
\begin{recette}{Pâtes au roquefort}{4}{}{}\index{pâtes}\index{pâtes au roquefort}\index{roquefort}\index{champignon}
\begin{ingredients}[Pour 500g de pâtes]
\ingredient 250g de champignons
\ingredient 20cl de crème liquide
\ingredient 100g de roquefort
\end{ingredients}

\begin{preparation}
\etape Ecrasez le roquefort à la fourchette pour qu'il fonde plus facilement
\etape Dans une casserole, faites chauffer la crème liquide et le roquefort à feu doux (3/9). Eteignez dès que le roquefort est fondu (il ne faut pas trop qu'il cuise).
\etape Faites revenir les champignons émincés et faites les bien dorer et versez dans la crème de roquefort
\etape Ajoutez un peu d'eau dans la poële où vous avez fait revenir les champignons pour récupérer les sucs qui ont attaché, puis faites réduire et ajoutez au récipient de crème de roquefort. Couvrez et réservez jusqu'à la cuisson des pâtes. 
\etape une fois les pâtes cuites et égouttées, versez la préparation et remuez. 
\etape Servez
\end{preparation}
\end{recette}

\section{Pâtes au saumon (à la moi)}
\begin{recette}{Pâtes au saumon (à la moi)}{4}{1h}{}\index{pâtes}\index{saumon}
\begin{ingredients}
\ingredient 25cl de crème fraîche liquide + 5cl d'eau (sinon c'est trop épais)
\ingredient 1 fenouil
\ingredient 200g de saumon fumé
\ingredient 100g de parmesan
\ingredient sel, poivre, aneth, un peu de jus de citron
\end{ingredients}

\begin{remarque}
Il est possible de préparer la sauce avant, puis faire cuire les pâtes plus tard et d'ajouter la sauce et le saumon fumé sur les pâtes chaudes
\end{remarque}


\begin{preparation}
\etape Émincez le fenouil (comme un oignon). Coupez le saumon fumé en lamelles puis en morceaux
\etape Faites revenir le fenouil émincé dans un peu d'huile pendant 10 minutes environ.
\etape Réservez dans un saladier. 
\etape Ajoutez alors un peu d'eau pour déglacer (5cl) dans la poële et versez dans le saladier. 
\etape Ajoutez alors la crème liquide, un peu de citron, du poivre et le saumon dans le saladier. 
\etape Laissez infuser jusqu'à ce que les pâtes soient cuites (possibilité de préparer la suite de la recette plus tard, pratique pour prévoir le repas un peu à l'avance)
\etape Égouttez les pâtes, ajoutez alors le contenu du saladier et le parmesan dans les pâtes. 
\etape Remuez jusqu'à avoir la consistance voulue puis servez les.
\end{preparation}
\end{recette}


\section{Paupiettes de veau à la Normande}
\begin{recette}{Paupiettes de veau à la Normande}{4}{1h30}{}\index{paupiettes}\index{veau}\index{pommeau de normandie}
\begin{ingredients}
\ingredient 8 paupiettes de veau
\ingredient 200g de champignons
\ingredient 4 oignons
\ingredient 1 pomme assez ferme
\ingredient 200g de lardons
\ingredient 25cl de pommeau de Normandie
\ingredient 25cl de bouillon de Volaille
\ingredient sel, poivre, sucre, fond de veau, farine
\end{ingredients}

\begin{preparation}
\etape Faites revenir les paupiettes de veau à feu vif, d'abord sur la partie gras, puis sur les autres faces. 
\etape Réservez les paupiettes et mettez les champignons.
\etape Une fois les champignons revenus, réservez les dans un récipient différent des paupiettes
\etape Faites cuire les lardons pour rajouter un peu de gras, réservez-les avec les champignons
\etape mettez les oignons à revenir avec une pincée de sucre. 
\etape Pendant ce temps, coupez les pommes en petits cubes de $5\unit{mm^3}$ environ. 
\etape Une fois les oignons cuits, rajoutez les champignons et les lardons, puis ajouter en saupoudrant, une cuillère à soupe 
rase de farine. Remuez jusqu'à ce que ce soit homogène puis ajoutez le bouillon, une cuillère à soupe de fond de veau et le 
pommeau de Normandie. 
\etape Rajoutez alors les pommes et les paupiettes.
\end{preparation}

\begin{cuisson}
Faites alors cuire pendant une heure environ à feu doux et à couvert en remuant de temps en temps.
\end{cuisson}


\end{recette}

\section{Petits pois}
\begin{recette}{Petits pois}{3}{30 minutes}{2h}\index{petits pois}
\begin{ingredients}
\ingredient 1kg de petits pois surgelés (pas en conserve)
\ingredient $4$ ou $5$ oignons
\ingredient 2 carottes
\ingredient un peu de salade
\ingredient 200g de lardons fumés
\ingredient 6 saucisses
\ingredient (facultatif) un jarret demi sel
\ingredient 10g de beurre
\ingredient une cuillère à soupe de farine
\ingredient 30cl de bouillon de volaille
\end{ingredients}

\begin{preparation}
\etape Émincez les oignons, coupez les carottes en petits cubes (d'abord en julienne, puis en petits morceaux)
\etape Préparez le bouillon en mettant au micro-onde un bouillon cube dans un bol d'eau.
\etape Faites revenir les lardons puis réservez-les. Faites dorer les saucisses puis réservez-les aussi.
\etape Faites revenir les oignons et les carottes. Rajoutez un peu d'huile au besoin.
\etape Rajoutez alors les lardons, une cuillère à soupe de farine, et mélangez.
\etape Rajoutez le bouillon, les petits pois, les saucisses, la salade, mélangez et faites mijoter à feu moyen (5/9) pendant 2h.
\end{preparation}

\end{recette}


\section{Pizza ventrèche-roquefort}
\begin{recette}{Pizza ventrèche-roquefort}{3}{10 min + 20 min}{20 min}\index{pizza}\index{crème}
\begin{ingredients}
\ingredient[Pour la sauce tomate]
\ingredient Petite boite de concentré de tomate
\ingredient huile d'olive
\ingredient Origan, Poivre

\ingredient[Pour la pizza]
\ingredient Pâte à pizza
\ingredient 50g de ventrèche fine
\ingredient 50g de roquefort
\ingredient 150g de fromage râpé
\end{ingredients}

\begin{preparation}
\etape Préchauffez le four à la température maximale (moi c'est 275°C)
\etape Sortez la pâte du frigo 10 minutes avant afin qu'elle soit à température ambiante ou faites une pâte vous même.
\etape Étalez là dans un plat pour aller au four directement avec le papier sulfurisé fourni.
\etape Mélangez le concentré de tomate, de l'origan, un peu de poivre et de l'huile d'olive.
\etape Étalez le avec une cuillère à café. Il ne dois pas y en avoir beaucoup (pas besoin que la couche de coulis soit 
uniforme).
\etape saupoudrez abondamment de gruyère râpé. Il faut généralement au moins un paquet de 200g.
\etape Alignez les tranches fines de ventrèche, quadrillez si vous en avez assez. Émiettez le roquefort par dessus et enfournez.
\end{preparation}

\begin{cuisson}
Faites cuire la pizza environ 10 minutes à 275\degres C. Surveillez bien entendu, ça peut être aussi court que 4-5 minutes si 
le four est extrêmement chaud.
\end{cuisson}
\end{recette}

\section{Pizza 4 fromages}
\begin{recette}{Pizza 4 fromages}{3}{}{20 min}\index{pizza}

\begin{ingredients}
\ingredient[Pour la sauce tomate]
\ingredient Petite boite de concentré de tomate
\ingredient huile d'olive
\ingredient Origan, Poivre

\ingredient[Pour la pizza]
\ingredient Pâte à pizza
\ingredient 5 tranches de fromage de chèvre (buche)
\ingredient 4 tranches de mozarella
\ingredient 75g de roquefort
\ingredient Fromage râpé (environ 200g)
\end{ingredients}

\begin{preparation}
\etape Faites préchauffer le four à 280\degres C.
\etape Sortez la pâte du frigo et étalez là dans un plat pour aller au four.
\etape Mélangez le concentré de tomate, de l'origan, un peu de poivre et de l'huile d'olive.
\etape Étalez le avec une cuillère à café. Il ne doit pas y en avoir beaucoup (pas besoin que la couche de coulis soit 
uniforme).
\etape saupoudrez abondamment de gruyère râpé. Il faut généralement au moins un paquet de 200g. 
\etape Répartissez les tranches de chèvre. 
\begin{figure}[htb]
\centering
\includegraphics[width=0.7\textwidth]{figures/pizza_4_fromages.pdf}
\caption{Comment répartir roquefort, chèvre et mozzarella (le rapé est uniformément réparti sur la base tomate avant d'y déposer les trois autres fromages).}
\end{figure}
\end{preparation}

\begin{cuisson}
Faites cuire la pizza environ 10 minutes à 280\degres C.
\end{cuisson}
\end{recette}

\section{Pizza chèvre/miel}
\begin{recette}{Pizza chèvre/miel}{3}{}{20 min}\index{pizza}\index{miel}

\begin{ingredients}
\ingredient[Pour la sauce tomate]
\ingredient Petite boite de concentré de tomate
\ingredient huile d'olive
\ingredient Origan, Poivre

\ingredient[Pour la pizza]
\ingredient Pâte à pizza
\ingredient Fromage de chèvre (buche)
\ingredient Fromage râpé (environ 200g)
\ingredient Miel
\end{ingredients}

\begin{preparation}
\etape Faites préchauffer le four à 280\degres C.
\etape Sortez la pâte du frigo et étalez là dans un plat pour aller au four.
\etape Mélangez le concentré de tomate, de l'origan, un peu de poivre et de l'huile d'olive.
\etape Étalez le avec une cuillère à café. Il ne dois pas y en avoir beaucoup (pas besoin que la couche de coulis soit 
uniforme).
\etape saupoudrez abondamment de gruyère râpé. Il faut généralement au moins un paquet de 200g. 
\etape Répartissez les tranches de chèvre. 
\etape Trempez le manche d'une cuillère à café dans le pot de miel et déposez un mince filet de miel un peu partout.
\end{preparation}

\begin{cuisson}
Faites cuire la pizza environ 10 minutes à 280\degres C.
\end{cuisson}
\end{recette}

\section{Pizza au saumon}
\begin{recette}{Pizza au saumon}{3}{10 min + 20 min}{20 min}\index{pizza}\index{crème}\index{saumon}
\begin{ingredients}
\ingredient Pâte à pizza
\ingredient 10cl de crème fraîche liquide (ça peut marcher avec 20 aussi puisqu'on fait réduire)
\ingredient 100g de saumon fumé
\ingredient \~ 150-200g de fromage râpé
\ingredient Un minuscule fenouil
\ingredient poivre, herbes
\end{ingredients}

\begin{preparation}
\etape Dans une casserole à feu moyen-vif, mettez la crème liquide, sel, poivre, et épices (pour ma part, céleri et mélange 
pour poisson). Une pincée de farine. Remuez sans arrêt jusqu'à ce que ça commence à accrocher (typiquement, quand vous remuez, 
il y a une pellicule qui reste contre le fond. Éteignez le feu et laissez refroidir.
\etape Pendant ce temps là, coupez les tranches de saumon fumé en carré grossier de 5cm de coté environ (c'est uniquement pour pouvoir mieux répartir sur la pizza, c'est pas obligatoire).
\etape Émincez le plus finement possible le fenouil. Pour ma part, sur un petit fenouil, j'émince uniquement les tiges et je 
n'utilise pas le bulbe. 
\etape Sur la pâte à pizza (elle même sur un papier cuisson), étalez la crème. Mettez alors la pâte sur la grille du four.
\etape Mettez alors le fromage rapé. Puis saupoudrez le fenouil sur toute la surface.
\end{preparation}

\begin{cuisson}
Faites cuire la pizza environ 10-15 minutes à 280\degres C (oui, 280). Une fois sortie du four, rajouter le saumon fumé. 
\end{cuisson}
\end{recette}

\section{Poisson pané}
\begin{recette}{Poisson pané}{4}{30 minutes}{15 min.}\index{poisson}
\begin{ingredients}
\ingredient 4 filets de poisson au choix (merlu, cabillaud, dorade...)
\ingredient 3 œufs et 3 cas de lait
\ingredient 8 c. à soupe de farine
\ingredient 8 c. à soupe de chapelure
\ingredient 2 noix de beurre
\ingredient sel, poivre du moulin
\end{ingredients}

\begin{preparation}
\etape Dans une assiette plate, versez la farine et dans une autre la chapelure. Dans une assiette creuse, battez les œufs avec le lait pour en faire une omelette.
\etape Salez et poivrez à votre convenance les filets de poisson des deux côtés. Coupez-les en gros morceaux. Farinez-les puis imprégnez-les des deux côtés tour à tour d'œufs battus puis de chapelure.
\etape [optionnel] Pour un peu plus de panure, refaire la procédure oeuf / chapelure pour une deuxième couche
\etape Faites fondre du beurre dans une poêle et faites frire votre poisson pané jusqu'à ce que la panure soit bien dorée (environ 3-4 minutes de chaque coté). Servez immédiatement.
\end{preparation}
\end{recette}

\section{Poisson sauce au vin blanc}
% https://afcmlacuisine.fr/cabillaud-sauce-vin-blanc
\begin{recette}{Poisson sauce au vin blanc}{4}{45 minutes}{15 min.}\index{poisson}
\begin{ingredients}
\ingredient 1kg de poisson (j'ai fait avec deux maquereaux la dernière fois)
\ingredient 4 échalottes
\ingredient 20cl de crème fraiche
\ingredient 25cl de vin blanc
\ingredient 20cl de bouillon
\ingredient thym, laurier
\end{ingredients}

\begin{preparation}
\etape Préparez de la saumure (comptez 200g de sel pour 1L d'eau)
\etape Mettez-y le poisson environ 20 minutes
\etape Pendant ce temps, faites revenir les échalottes émincés dans un peu d'huile
\etape Ajoutez une cuillère à soupe de Maizena, puis ajoutez le vin blanc
\etape Faites réduire 2 minutes à ébullition
\etape Ajoutez la crème et le bouillon
\etape Rincez le poisson, ajoutez le dans la sauce, puis faites cuire à couvert pendant 15 minutes à feu moyen (4/9)
\end{preparation}
\end{recette}

\section{Porc au caramel}
\begin{recette}{Porc au caramel}{3}{1h30}{40min}\index{porc}\index{caramel}
\begin{ingredients}
\ingredient[Pour la sauce]
\ingredient un roti dans l'échine (1.5kg environ)
\ingredient 2 oignons
\ingredient 2 gousses d'ail
\ingredient Un morceau de gingembre frais ($\sim 50$ g)
\ingredient 12.5cl de sauce soja

\ingredient[Pour le caramel]
\ingredient 150g de sucre
\ingredient 5cl d'eau
\end{ingredients}

\begin{preparation}
\etape couper le porc en fins morceaux (je fais des tranches d'un demi centimètre environ, puis des lamelles d'1cm de large, et
je coupe ces lamelles en morceaux d'1 cm de long).
\etape pelez puis mixez l'ail, le gingembre et l'oignon que vous réservez dans un grand saladier (qui contiendra aussi le porc)
\etape Faire revenir les morceaux de porc à feu très vif pendant 3 minutes environ (pour 2kg je fais environ 4 fournées). Il
faut juste faire blanchir la viande et légèrement dorer. Réservez les morceaux cuits dans le saladier contenant gingembre et
oignon
\etape Dans la sauteuse, ajoutez enfin la totalité du porc et des légumes mixés. 
\etape Saupoudrez une cuillère à soupe de farine puis mélangez. 
\etape Ajoutez la sauce soja et laissez le tout réchauffer à feu doux pendant qu'on s'occupe du caramel
\etape Dans une casserole, versez le sucre et le fond d'eau. Mettez à feu vif et attendez que le caramel prenne une coloration
brune.
\etape Nappez alors le caramel obtenu sur le porc dans la sauteuse qui va ainsi durcir. Ne mélangez pas, laissez ainsi.
\end{preparation}

\begin{cuisson}
Couvrez et laissez mijoter à feux doux pendant 40 minutes environ.
\end{cuisson}
\end{recette}

\section{Poulet à la moutarde}
\begin{recette}{Poulet à la moutarde}{5}{30 min}{50 min}\index{poulet à la moutarde}\index{moutarde}\index{poulet}
\begin{ingredients}[6 pers.]
\ingredient 1kg de morceaux de poulet
\ingredient 8 échalotes
\ingredient 25cl de bouillon de volaille
\ingredient 4 cuillères à soupe \textbf{bombées} de moutarde à l'ancienne (il n'en faut pas moins, sinon il y a trop de crème)
\ingredient 50 cl de crème fraiche épaisse
\ingredient sel, poivre, romarin
\end{ingredients}

\begin{preparation}
\etape Verser un fond d'huile dans une marmite, puis saisir à feu vif les morceaux de poulet, d'abord coté peau, avant de les réserver.
\etape Faites revenir dans le gras les échalotes pelées et émincées, en remuant jusqu'à ce qu'elles soient dorées.
\etape Mouillez avec le bouillon, saler, poivrer. Mettez le romarin. Remettre les morceaux de viande, la peau contre le fond, couvrir et laissez mijoter (4/10) pendant 30 min.
\etape Enlevez les morceaux de viande afin de pouvoir bien remuer puis ajoutez la moutarde et la crème. Mélangez 
jusqu'à ce que la crème et la moutarde fassent une mixture homogène (plus de grumeau de moutarde ou de crème). 
\etape Rectifiez l'assaisonnement, remettez les morceaux de viande et laissez mijoter à couvert et à feu un peu plus fort (5/10) pendant 20 minutes environ.
\end{preparation}
\end{recette}

\section{Poulet au curry}
\begin{recette}{Poulet au curry}{4}{45min}{1h}\index{poulet au curry}\index{poulet}\index{curry}
% C'était excellent, mais cette fois là j'avais utilisé moyennement de l'huile (et graisse de canard). peut-être que c'est la quantité de graisse qui a fait que c'était super bon (pas sûr)

\begin{ingredients}
\ingredient 4 cuisses de poulet
\ingredient une pomme mixée
\ingredient 2 oignons
\ingredient une gousse d'ail
\ingredient une boite (500g) de coulis de tomate
\ingredient 25cl de bouillon de volaille
\ingredient 20cl de crème fraiche
\ingredient 2 cuillères à soupe bombées de curry
\ingredient un peu de jus de citron
\ingredient un peu de muscade râpée, un peu de cannelle, 1/4 de cac de sel, poivre
\end{ingredients}

\begin{preparation}
\etape Émincez les oignons, et mixez la pomme.
\etape Saisissez les morceaux de poulet dans une sauteuse
\etape Réservez les morceaux avec la pomme mixée puis faites blondir l'oignon dans les sucs.
\etape Pendant ce temps, préparez le bouillon de volaille, et écrasez la gousse d'ail (à la fourchette) pour l'incorporer au 
bouillon.
\etape Ajoutez alors le curry, la cannelle, la muscade avec les oignons. Mélangez, puis ajoutez le bouillon et la tomate. Ajoutez enfin les morceaux de poulet et la pomme mixée.
\end{preparation}

\begin{cuisson}
Faites cuire pendant 45 minutes environ, à couvert et à feu doux (3/10). Ajoutez alors la crème fraiche et un peu de jus 
de citron, puis laissez mijoter encore 15 minutes environ. 

Servez avec un riz bazmati.
\end{cuisson}
\end{recette}


\section{Poulet Chasseur}
\begin{recette}{Poulet Chasseur}{4}{1h}{1h}\index{poulet}\index{poulet chasseur}
\begin{ingredients}
\ingredient 8 morceaux de poulet
\ingredient $250\unit{g}$ de champignons
\ingredient 3 échalotes
\ingredient $4\unit{cl}$ de Cognac
\ingredient $4\unit{cl}$ de vin blanc
\ingredient Un bol de bouillon de volaille
\ingredient farine, beurre, huile, sel, poivre
\ingredient estragon, cerfeuil
\end{ingredients}


\begin{preparation}
\etape Découpez et dégraissez les morceaux de poulet. 
\etape Épluchez, lavez et émincez les champignons. Épluchez et ciselez les échalotes.
\etape Saisissez dans du beurre ou de l'huile les morceaux de poulet à feu vif puis réservez-les.
\etape Faites revenir les échalotes, réservez-les dans une assiette (pas avec les morceaux de poulet).
\etape Faites revenir les champignons dans la sauteuse. 
\etape Rajoutez alors les échalotes. Ajoutez le cognac, sortez la sauteuse de feu et faites flamber. 
\etape Ajoutez une cuillère à soupe rase de farine, mélangez.
\etape Ajoutez le vin blanc, le bouillon de volaille, le cerfeuil et l'estragon. 
\end{preparation}

\begin{cuisson}
Laissez alors mijoter à couvert et à feu doux une heure environ. À la fin, contrôlez l'assaisonnement et la liaison.
\end{cuisson}
\end{recette}

\section{Poulet gascon}
\begin{recette}{Poulet gascon}{4}{45min}{1h}\index{poulet gascon}\index{poulet}\index{floc de gascogne}
% (Excellent)

\begin{ingredients}
\ingredient 4 cuisses de poulet
\ingredient $200$ g de champignons
\ingredient 200g de lardons
\ingredient $2$ fenouils
\ingredient $2$ gousses d'ail
\ingredient $15$ cl de floc de Gascogne
\ingredient 15cl de bouillon de volaille
\ingredient 1 cuillère à soupe rase de farine
\ingredient sel, poivre du moulin
\end{ingredients}

\begin{preparation}
\etape Émincez finement le fenouil et les champignons et placez les dans des récipients séparés.
\etape Faites chauffer le bouillon cube et l'eau au micro-onde. Écrasez les gousses d'ail à l'aide d'une fourchette et ajoutez 
les dans le bouillon.
\etape Faites fondre le beurre dans une sauteuse et faites revenir les cuisses à feu vif. Pas besoin que la viande soit cuite à 
l'intérieur, c'est juste pour faire dorer.
\etape Réservez les morceaux puis faites revenir le fenouil et réservez le quand il est revenu.
\etape Faites revenir les lardons puis réservez les
\etape Faites enfin revenir les champignons.
\etape Remettez alors dans la sauteuse fenouil et lardon puis mélangez.
\etape Ajoutez alors la farine, remuez jusqu'à l'incorporer autour des légumes.
\etape Ajoutez le bouillon de volaille afin d'homogénéiser la farine entourant les légumes et le bouillon.
\etape Ajoutez enfin le floc de Gascogne, l'ail écrasé et les morceaux de poulets.
\end{preparation}

\begin{cuisson}
Faites cuire pendant une heure environ à feu doux et à couvert.
\end{cuisson}
\end{recette}

%https://bistroguru.com/recette/recette-poulet-korma/
%https://www.mesinspirationsculinaires.com/article-poulet-korma.html
% ingredient for the korma sauce at safeway: Water, Sugar, Desiccated Coconut, Cream, Coconut Paste, Onion, Canola Oil, Food Starch Modified, Contains 2% or Less of Tomato Paste, Heavy Cream, Ginger, Garlic, Spices (Including Turmeric), Salt, Lactic Acid, Lemon, Juice Concentrate, Dried Cilantro Leaf. 
\section{Poulet Korma}
\begin{recette}{Poulet Korma}{3}{1h30}{}
\begin{ingredients}
\ingredient[etape 1]
\ingredient 4 oignons moyens hachés
\ingredient 2 c-a-c bombées de garam massala % aux us j'ai mis 2 cac)
\ingredient 2 c-a-c bombées coriandre en poudre
\ingredient 1 cac rase de paprika
\ingredient 2 c-a-soupe huile végétale
\ingredient 30g de beurre
\ingredient[etape 2]
\ingredient 1kg de poulet coupé en morceau
\ingredient 70g de concentré de tomate (2 cas)
\ingredient 4 gousses ail écrasées (ou 40g de pâte d'ail)
\ingredient 50g de gingembre frais râpé % aux us j'ai mis l'équivalent de 2-3 gousses d'ail
\ingredient 400 ml bouillon de poulet
\ingredient[etape 3]
%\ingredient 400 ml lait de coco
\ingredient 1 yaourt nature (ou 20cl de creme fraiche) % aux us j'ai mis 450g de creme fraiche
\ingredient 60g d'amande moulu
\ingredient 5g de sel
\ingredient poivre
\end{ingredients}

\begin{preparation}
\etape Emincez les oignons, mixez l'air et le gingembre et réservez dans un bol.
\etape Faites revenir l'oignon dans l'huile, le beurre et les épices pendant 10 minutes. 
\etape Ajoutez le poulet, le bouillon, le concentré de tomate, l'ail et le gingembre mixé et faites cuire à couvert et à feux moyen (5/9) pendant 30 minutes.
\etape Ajoutez alors le lait de coco et les amandes moulues. Faites alors cuire 30 minutes de plus à feu très doux (3/9) et à couvert (attention à ne pas faire bouillir la sauce) Rectifier l'assaisonnement.
\etape Servir chaud accompagné de riz.
\end{preparation}
\end{recette}

\section{Purée de pois cassé}
\begin{recette}{Purée de pois cassé}{3}{45min}{}\index{purée de pois cassé}\index{pois cassé}\index{saucisse de morteau}
% (Excellent)

\begin{ingredients}
\ingredient 500g de pois cassé
\ingredient 3 pommes de terre
\ingredient 1 carotte
\ingredient 1 oignon
\ingredient 1 saucisse de morteau
\ingredient 1 bouillon cube
\ingredient sel, poivre, céleri
\end{ingredients}

\begin{preparation}
\etape Mettez les pois cassé, les légumes et la saucisse dans une marmite. Ajoutez de l'eau froide (attention, les pois cassés gonflent, il faut donc un peu d'eau sinon ça va accrocher). Salez, poivrez, mettez les 
épices (céleri par exemple), ajoutez le bouillon cube. 
\etape Faites chauffer à feu vif jusqu'à ébullition
\etape faites cuire 45 minutes à feux doux.
\etape réservez la saucisse et mixez (si vous avez beaucoup d'eau, enlevez en, puis rajoutez en pour avoir la consistance 
voulue.
\etape Piquez la saucisse tant qu'elle est chaude et avant de la servir pour enlever le gras qu'elle contient. 
\end{preparation}

\end{recette}


\section{Raclette}
\begin{recette}{Raclette}{4}{20 minutes}{}\index{raclette}
\begin{ingredients}[Par personne]
\ingredient 250g de fromage
\ingredient 600g de pommes de terre (570g)
\ingredient 200g de charcuterie
\ingredient petits oignons au vinaigre
\begin{remarque}
 Ces doses sont pour des gros mangeurs, mais en fonction des familles, il faut compter ça direct pour tout le monde.
\end{remarque}

\end{ingredients}

\begin{preparation}
\etape Je n'enlève pas la croute du fromage
\etape Mettez à tremper la plaque en marbre dès que la raclette est terminée pour que ça se nettoie plus facilement
\end{preparation}
\end{recette}

\section{Ratatouille confite et son roti de porc}
\begin{recette}{Ratatouille confite et son roti de porc}{4}{2h}{10h}\index{ratatouille}\index{roti}\index{porc}
\begin{ingredients}
\ingredient 200g de lardons % indispensable pour le gout
\ingredient 1.5kg de rôti de porc (échine)
\ingredient 4 gros oignons
\ingredient 2 belles aubergines
\ingredient 2 poivrons verts
\ingredient 1 poivron rouge
\ingredient 4 courgettes
\ingredient 10 tomates
\ingredient 140g de concentré de tomates
\ingredient 5 gousses d'ail
\ingredient 4 morceaux de sucre
\ingredient huile d'olive, sel, poivre, herbes de provence, laurier
\end{ingredients}

\begin{preparation}
\etape Peler les tomates (voir \refsec{sec:peler_tomate}), puis les écraser à la main dans une marmite. Ajoutez les 5 gousses 
d'ail écrasées, le sucre, le laurier et les herbes. 
\etape Faire réduire jusqu'à la consistance d'une purée, il faut que le liquide ait quasiment disparu. Normalement, en faisant 
réduire à feu moyen à moyen-vif pendant que vous faites revenir les autres légumes ça sera prêt. 
\begin{remarque}
Les autres légumes rendront de l'eau, ce n'est pas grave si la purée est bien réduite.
\end{remarque}
\etape Pendant ce temps, préparez les légumes : 
\etape Émincer les oignons épluchés et les faire fondre à feu doux dans une poêle avec du poivre. Réservez dans un saladier
\etape Émincer les poivrons et les faire fondre avec du sel jusqu'à ce qu'ils soient mous. Les mettre dans le saladier
\etape Laver les courgettes, les couper en petits cubes et les faire dorer à la poêle. Les mettre dans le saladier.
\etape Laver les aubergines, les couper en petits cubes et les faire dorer avec du poivre. Les réserver avec oignons, poivrons 
et courgettes.
\etape Dans la purée de tomate, rajouter le concentré de tomate, les lardons crus et les légumes revenus (oignon, poivron, 
courgette et aubergine).
\end{preparation}

\begin{cuisson}
Laisser mijoter à couvert pendant 7 à 10h environ à feu très doux (au minimum). Si la ratatouille commence à accrocher parce 
qu'il n'y a pas assez de jus, rajoutez un bol d'eau environ, remuez et remettez à mijoter.

2h avant de servir, rajoutez le rôti dans la ratatouille confite déjà chaude, et laissez mijoter au minimum jusqu'au début du 
repas. Tournez le rôti à mi-cuisson.

Le confit de ratatouille est prêt quand il change de couleur et devient foncé. 

%TODO pour cuire le roti, il faut environ 4h dans la ratatouille, après (ou à ce moment là) il a commencé à cramer. Il faut 
laisser la ficelle sinon il risque de se désagréger.
\end{cuisson}
\end{recette}

\section{Risotto aux champignons}
\begin{recette}{Risotto aux champignons}{3}{1h}{}\index{risotto}
\begin{ingredients}
\ingredient 500g de riz (cuisson longue 20 min)
\ingredient 200g de champignons
\ingredient 2 oignons
\ingredient 10cl de vin blanc (pas trop, ou pas du tout de vin blanc, sinon ce n'est pas évaporé au bout des 20 minutes de 
cuisson)
\ingredient 25cl de crème fraiche liquide
\ingredient 65cl de bouillon de volaille (à 500g de riz correspond 1L de liquide (vin + crème fraiche liquide + bouillon)
\ingredient sel, poivre, herbes de provence
\end{ingredients}

\begin{preparation}
\etape Faites blondir l'oignon émincé dans une sauteuse
\etape Réservez-le et faites revenir les champignons
\etape Réservez les champignons, ajoutez de l'huile et faites-y rissolez le riz jusqu'à ce qu'il devienne translucide.
\etape Ajoutez l'oignon, les champignons et mélangez.
\etape Ajoutez le vin blanc et remuez quelques instants 
\etape Ajoutez alors le bouillon de volaille et la crème fraiche liquide. 
\etape Ajoutez les herbes et rectifiez l'assaisonnement.
\etape Laissez cuire 25 minutes à feu moyen et à couvert, le temps que le riz soit cuit. 
\begin{remarque}
Il faut plus de temps que le temps normal indiqué pour le riz
\end{remarque}
\end{preparation}
\end{recette}


\section{Rougail saucisse}
\begin{recette}{Rougail saucisse}{3}{30 min.}{}
\begin{ingredients}
\ingredient 1kg de saucisses fumées
\ingredient 1kg de coulis de tomates
\ingredient 25cl de bouillon
\ingredient 8 oignons
\ingredient 10 gousses d'ail
\ingredient 15g de gingembre frais % normalement c'est 1/2 cac de gingembre moulu
% ne pas mettre de rhum vieux. 
\ingredient 1 cac de curcuma en poudre (1.5g)
\ingredient 1 cac de garam massala (1g)
\ingredient 8g de sel
\end{ingredients}


\begin{preparation}
\etape Emincez l'oignon
\etape Coupez les saucisses en rondelle
\etape Faites chauffer le bouillon cube dans 50cl d'eau au micro onde pendant 2 minutes, puis mélangez à la fourchette. 
\etape Mixez l'ail et le gingembre et mettez à macérer dans le bouillon
\etape Faites revenir les saucisses puis réservez les.
\etape Faites revenir l'oignon émincé. Ajoutez une cuillère à soupe de maïzena, le sel et les épices, mélangez.
\etape Ajoutez le bouillon avec ail et gingembre, déglacez les sucs et mélangez. 
\etape Ajoutez enfin le coulis de tomate et les saucisses.
\etape Portez à ébullition puis faites mijoter à feu doux (3/9) pendant 1h environ. 
\end{preparation}
\end{recette}


\section{Salade bigourdane}
\begin{recette}{Salade bigourdane}{5}{30 min}{}\index{salade bigourdane}\index{salade landaise}
\begin{ingredients}
\ingredient salade verte
\ingredient 10 à 15 noix
\ingredient 100g de lardons
\ingredient 100g de gésiers de canard confits
\begin{remarque}
Le plus pratique, ce sont les paquets de gésiers au rayon lardons ou canard. Les gésiers de volailes sont émincés et absolument 
pas gras. Beaucoup moins embêtant que des gésiers entiers, conservés dans la graisse.
\end{remarque}
\ingredient 50g de gruyère non râpé
\ingredient 1/4 de baguette de pain frais
\ingredient une gousse d'ail (ou ail semoule)
\ingredient vinaigrette (voir \refsec{sec:vinaigrette} ou \refsec{sec:vinaigrette-moutarde})

\end{ingredients}

\begin{preparation}
\etape Préparez les noix, que vous laissez dans un saladier. Ajoutez la salade, la vinaigrette et remuez.
\etape Si vous avez une gousse d'ail fraiche, frottez le pain avec. Puis coupez le pain en deux dans le sens de la longueur, et 
faites 3 à 4 lamelles dans chacun des deux morceaux. Coupez ensuite ces lamettes en petits cube d'environ un centimètre de long.
\etape Coupez le gruyère en cubes d'un demi centimètre de coté environ.
\etape Recoupez les gésiers pour faire des cubes d'un peu moins d'1 cm de coté.
\etape Faites revenir les lardons à la poêle. 
\etape Réservez-les puis faites revenir les gésiers, et réservez-les avec les lardons.
\etape Faites alors dorer le pain dans la graisse ainsi rendue. Si vous avez de l'ail semoule à la place de la gousse d'ail, 
ajoutez le à ce moment là dans la poêle afin de mélanger au pain. Il faut que le pain soit légèrement croustillant au bord, mais 
moelleux à l'intérieur. Il faut donc le surveiller, le tourner de temps en temps, et ne pas mettre à feu trop vif. Le pain ne va 
pas forcément dorer, il se peut que vous vous retrouviez avec des biscottes si vous cherchez absolument à ce qu'il colore. 
\etape Une fois le pain presque prêt, ajoutez les lardons et gésiers, remuez avec de faire réchauffer le tout, et de re-graisser 
le pain afin qu'il finisse de dorer. 
\etape Une fois chaud et prêt, éteignez le feu. Ajoutez alors le fromage en cube dans la poêle en dehors du feu, puis versez 
immédiatement sur la salade, et mangez de suite. Ainsi, le fromage sera légèrement fondant, sans faire de filaments pour autant.
\end{preparation}
\end{recette}

\section{Salade de pomme de terre aux harengs}
\begin{recette}{Salade de pomme de terre aux harengs}{5}{30 min}{}\index{hareng}
\begin{ingredients}
\ingredient 2.5kg de pomme de terre
\ingredient 300-500 de harengs
\ingredient 3 oignons
\ingredient 20 cornichons
\ingredient 8 oeufs (à faire dur)
\ingredient 240g de vinaigrette
\end{ingredients}

\begin{preparation}
\etape Faites mariner harengs, oignons et vinaigrette pendant 2 jours
\etape Faites cuire les pommes de terre dans l'eau. Mettez les dans l'eau froide salée. Une fois à ébullition, comptez 15 minutes (piquez au couteau pour vérifier)
\etape récupérez de l'eau bouillante pour faire les oeufs dur (9 minutes dans de l'eau à ébullition, puis plongez dans l'eau froide)
\etape Plongez les pommes de terre dans l'eau froide puis pelez les.
\etape coupez les cornichons et les pommes de terre. Mélangez au reste.
\end{preparation}
\end{recette}

\section{Saumon au champagne}
\begin{recette}{Saumon au champagne}{4}{20 min.}{45 min.}\index{saumon}\index{poisson}\index{champagne}
\begin{ingredients}
\ingredient 1kg de saumon
\ingredient 1 oignon
\ingredient 1 carotte
\ingredient 40g de beurre (ou huile d'olive)
\ingredient 50cl de crème fraîche
\ingredient 3 brins de thym frais
\ingredient 75 cl de mousseux brut premier prix
\ingredient sel et poivre du moulin 
\end{ingredients}

\begin{preparation}
\etape Préchauffez le four à 180°C
\etape Mixez oignon et carotte avec un peu de champagne.
\etape Dans une grande casserole, mélangez au reste de champagne, le beurre, thym, sel et poivre puis portez à ébullition (pour que le temps de cuisson du poisson soit plus facile à calculer)
\etape Dans un plat allant au four, mettez le saumon, versez la préparation au champagne bouillante. 
\etape Couvrez et faites cuire au four pendant 20 minutes environ
\etape Réservez alors le saumon au chaud
\etape Dans une casserole, mettez deux cuillères à soupe de farine et 20g de beurre
\etape Une fois homogène, ajoutez la sauce et faites réduire à 1/4 du volume initial
\etape Ajoutez alors la crème hors du feu puis servez
\end{preparation}
\end{recette}

\section{Saumon laqué miel/soja}
\begin{recette}{Saumon laqué miel/soja}{4}{20 min.}{45 min.}\index{saumon}\index{poisson}\index{sauce soja}
\begin{ingredients}
\ingredient 1kg de saumon
\ingredient 15cl de sauce soja (10 cas, ou 150g)
\ingredient 80g de miel (4 cas)
\ingredient 2 gousses d'ail (en poudre ou frais)
% enlevé la farine, parce que ça fait deux fois que la sauce est forte et je sais pas si c'est la sauce soja ou la farine. 
%\ingredient 1 cuillère à soupe de farine (testé et sauce était forte, mais sait pas si c'est dû à la farine)
\end{ingredients}

\begin{preparation}
\etape Mixez toute la préparation afin que ce soit homogène
\etape Laissez le saumon mariner au moins une heure dans un plat allant au four, recouvert de papier alu (sinon, la veille au soir)
\end{preparation}

\begin{cuisson}
Préchauffez le four à 180°C. Faites cuire dans le plat recouvert de papier alu pendant 16 minutes environ à chaleur tournante. (le saumon est froid puisqu'il était au frigo).
\begin{remarque}
Attention, sans chaleur tournante, j'ai dû faire 26 minutes à 200°C.
\end{remarque}

\end{cuisson}
\end{recette}

\section{Soupe à l'oignon}
\begin{recette}{Soupe à l'oignon}{3}{1h}{1h}\index{oignon}
\begin{ingredients}
\ingredient 2 kg d'oignons
\ingredient 25cl de vin blanc
\ingredient 2L d'eau
\ingredient 50g de beurre et 2 cas d'huile
\ingredient 1 cube de bouillon de volaille
\ingredient 2 feuilles de laurier et une branche de thym, sel, poivre
\end{ingredients}

\begin{preparation}
\etape Pelez les oignons et émincez les (au robot de préférence vu la quantité)
\etape Faites les revenir dans la marmite avec le beurre et l'huile pendant 30 minutes environ pour qu'ils soient bien dorés
\etape Ajoutez 2 cuillères à soupe de farine bombées, mélangez bien puis ajoutez l'eau, le vin blanc et le bouillon de volaille. Salez, ajoutez le thym et le laurier
\end{preparation}

\begin{cuisson}
Laissez cuire 1h à feu moyen
\end{cuisson}
\end{recette}

\section{Tagliatelles aux Noix de St Jacques}
\begin{recette}{Tagliatelles aux Noix de St Jacques}{4}{}{}\index{pâtes}\index{tagliatelle}\index{noix st Jacques}
\begin{ingredients}
\ingredient 500g de noix St Jacques
\ingredient 5 gousses d'ail
\ingredient 6 champignons
\ingredient 15cl de vin blanc
\ingredient 25cl de crème liquide
\ingredient 4 cuillères à soupe rase de sauce tomate (ne surtout pas en mettre plus)
\ingredient sel, poivre, persil
\end{ingredients}

\begin{preparation}
\etape Faire revenir les noix St Jacques, les gousses d'ail et le persil finement hachés avec une noix de beurre pendant 2 à 3 
minutes.
\etape Rajouter les champignons et laisser cuire quelques minutes. N'attendez pas que les champignons soient cuits, laissez 
simplement fondre un peu puis ajoutez le vin blanc et laissez réduire jusqu'à ce que l'odeur de vin disparaisse presque 
complètement.
\etape Ajoutez alors la crème liquide et la sauce tomate.
\begin{remarque}
Il ne faut surtout pas mettre plus de tomate que les 4 cuillères à soupe. On peut rajouter un soupçon de sucre pour corriger un 
peu la tomate ou le vin blanc.
\end{remarque}
\etape Laissez mijoter 5 minutes puis servir sur une assiette les tagliatelles cuites puis disposez les noix en sauce par dessus
\end{preparation}
\end{recette}

\section{Tajine d'agneau aux pruneaux}
\begin{recette}{Tajine d'agneau aux pruneaux}{5}{2h}{5h}\index{tajine}\index{agneau}\index{pruneaux}
\begin{ingredients}
\ingredient 250g de pruneaux
\ingredient 1kg d’oignon (à peu près)
\ingredient 1,5kg d’épaule d’agneau (je prends un gigot)
\ingredient 2 gousses d'ail
\ingredient 75cl de bouillon de volaille (ou d'eau)
\ingredient une cuillère à café de cannelle
\ingredient un petit morceau de gingembre frais (équivalent en volume des gousses d'ail)
\ingredient une dosette de safran (0.1g)
\ingredient quelques grains de coriandre ($\sim 10$)
\ingredient 1 cac rase de sel, poivre
\end{ingredients}

\begin{preparation}
\etape Saisissez les morceaux de viande dans un peu d'huile ou de graisse puis réservez-les dans une cocotte. 
\etape Faites alors revenir les oignons émincés dans l'huile d'olive et profitez-en pour récupérer les sucs de la viande. 
Quand ils sont dorés, mettez-les dans la cocotte.
\etape Salez et poivrez. Ajoutez l'ail et le gingembre écrasé, 
la cannelle, le safran et les grains de coriandre.
\etape Rajoutez le bouillon  et les pruneaux.
\begin{remarque}
Servez accompagné de semoule de blé.
\end{remarque}
\end{preparation}

\begin{cuisson}
Portez à ébullition, puis faites mijoter (85-96°C) à feux doux (2-3/9) et à couvert pendant 5 heures environ.
\end{cuisson}
\end{recette}

\section{Tartiflette}
\begin{recette}{Tartiflette}{3}{1h}{30 min}\index{tartiflette}\index{roblochon}

\begin{ingredients}
\ingredient $1,5\unit{kg}$ de pommes de terre à chair ferme
\ingredient $200\unit{g}$ de lardons
\ingredient 2 oignons
\ingredient $1$ reblochon fermier
\ingredient 20-25cl de crème fraiche
\ingredient 5cl de vin blanc sec (facultatif)
\end{ingredients}

\begin{preparation}
\etape Éplucher les pommes de terre. Faites les cuire à l'autocuiseur 15 minutes (à partir du moment où ça siffle).
\etape Au terme de la cuisson, égoutter et laisser tiédir. (ne pas rafraichir !!!)
\etape Faites revenir les lardons quelques minutes puis réservez les
\etape Émincez l'oignon et faites le suer à la poêle dans la graisse des lardons.
\etape Coupez en cubes grossiers les pomme de terre. Mélangez ces cubes avec les oignons, les lardons, la crème et le petit 
verre de vin blanc sec et étalez ça dans le plat à gratin. 
\etape Découpez le reblochon en deux dans le sens de l'épaisseur (pour plus de facilité, on peut le découper alors qu'il est 
encore dans l'emballage, le fromage se tient mieux) et le déposer sur vos pommes de terre, croûte vers le bas.
\end{preparation}

\begin{cuisson}
Enfourner à four très chaud ($220-250\degres C$). Jusqu'à ce que le reblochon fonde et gratine en surface.
\end{cuisson}
\end{recette}

\section{Tourin à l'ail}
\begin{recette}{Tourin à l'ail}{3}{}{}\index{tourin à l'ail}\index{ail}
\begin{ingredients}
\ingredient 20 gousses d'ail
\ingredient 2 gros oignons
\ingredient 2 cuil. à soupe de farine
\ingredient 3 oeufs
\ingredient 1 cuil. à soupe de vinaigre
\ingredient huile d'olive, sel, poivre
\end{ingredients}

\begin{preparation}
\etape Faire bouillir 2 l d'eau avec 20 gousses d'ail épluchées.
\etape Dans un faitout, faire revenir deux gros oignons émincés dans de l'huile d'olive jusqu'à ce qu'ils deviennent 
translucides, sans les faire brunir.
\etape Ajouter deux cuil. à soupe de farine, mélanger et mouiller avec les 2 l d'eau et l'ail.
\etape Faire bouillir, ajouter une bonne pincée de sel et deux pincées de poivre.
\etape Couvrir et laisser mijoter à feux doux pendant une petite heure.
\etape Pendant ce temps, casser trois oeufs en séparant les blancs des jaunes.
\etape Ajouter dans les jaunes une cuil. à soupe de vinaigre de vin rouge, et diluer avec un peu de bouillon. Réserver.
\etape Au bout de 1 h de cuisson, ajouter les blancs d'oeuf au bouillon en agitant continuellement avec une cuillère en bois : 
ils formeront de longs filaments blancs.
\begin{remarque}
Pour cela, il faut commencer à remuer le bouillon pour lui donner une rotation relativement importante, puis verser lentement le 
blanc d'œuf tout en continuant de remuer.
\end{remarque}
\etape Hors du feu, ajouter les jaunes en les mélangeant d'un mouvement large et ferme.
\etape Remettre à feu très doux, sans laisser bouillir, une dizaine de minutes.
\end{preparation}

\begin{remarque}
Au moment de servir le tourin, vous pouvez disposer dans chaque assiette une tartine de pain de campagne arrosée d'huile 
d'olive, et assaisonner de poivre suivant votre goût : pour le tourin, soyez avare de sel et prodigue de poivre !
\end{remarque}
\end{recette}

\section{Velouté de champignons}
\begin{recette}{Velouté de champignons}{0}{1h}{}\index{champignons}\index{velouté de champignons}
\begin{ingredients}
\ingredient 500g de champignons
\ingredient 20cl de crème fraiche
\ingredient 1L de bouillon de volaille
\ingredient 3 échalottes
\ingredient 2 gousses d'ail
\ingredient 1 cuillère à soupe rase de farine
\ingredient sel, poivre, céleri, persil
\end{ingredients}

\begin{preparation}
\etape Faire suer l'échalotte dans un peu de graisse
\etape Ajoutez les champignons émincés et faites les revenir un peu
\etape Ajoutez alors une cuillère à soupe de farine puis mélangez
\etape Ajoutez alors le bouillon, l'ail émincé, sel, poivre, céleri et persil
\etape Couvrez et laissez cuire à feu doux pendant 45 minutes
\etape Ajoutez la crème fraiche, puis mixez le tout
\end{preparation}
\end{recette}

\section{Velouté de butternut aux noix}
\begin{recette}{Velouté de butternut aux noix}{4}{1h}{}\index{velouté}\index{courgette}\index{velouté de courgettes}
\begin{ingredients}
\ingredient Une courge
\ingredient bouillon de volaille (un cube + un bol de bouillon)
\ingredient 1 échalotte (adapter en fonction de la quantité de courge)
\ingredient 2 noix
\ingredient sel, poivre, curry
\end{ingredients}

\begin{preparation}
\etape Pelez et épépinez la courge, puis coupez la en petits cubes. Émincez finement l'échalotte
\etape Faites suer la courge et l'échalotte dans un peu d'huile d'olive
\etape Ajoutez alors le bouillon, sel et poivre. Rajoutez de l'eau jusqu'à recouvrir les morceaux de courge. 
\etape Laissez cuire 
une vingtaine de minutes (feu moyen, avec un couvercle) jusqu'à ce que la courge soit moelleuse.
\etape Séparez les légumes du bouillon. 
\etape Ajoutez une cuillère à café rase de curry deux noix aux légumes. Puis mixez le tout. Durant le processus, rajoutez du 
bouillon jusqu'à avoir la fluidité voulue.
\end{preparation}
\end{recette}

\section{Velouté de potimarron}
\begin{recette}{Velouté de potimarron}{4}{30 min.}{45 min.}\index{velouté}\index{potimarron}\index{velouté de potimarron}
\begin{ingredients}
\ingredient Un potimarron (pas obligé de l'éplucher)
\ingredient bouillon de volaille (un cube + un bol de bouillon)
\ingredient 2-4 oignons
\ingredient 20cl de crème liquide
\ingredient 1 clou de girofle (piqué dans un oignon)
\ingredient 1 cuillère à soupe de curry
\ingredient 1 cac rase de sel, poivre, noix de muscade
\end{ingredients}

\begin{preparation}
\etape Pelez et épépinez la courge, puis coupez la en petits cubes. Coupez l'oignon en morceaux grossiers, gardez un oignon à part pour les clous de girofles.
\etape Faites suer la courge et l'oignon (sauf celui pour le clou de girofle) dans un peu d'huile d'olive avec le curry et la noix de muscade
\etape Ajoutez alors le bouillon, sel et poivre. Rajoutez l'eau jusqu'à recouvrir les morceaux de courge. Ajoutez l'oignon piqué du clou de girofle.
\etape Laissez cuire 45 minutes (feu moyen 6/9, avec un couvercle) jusqu'à ce que la courge soit moelleuse.
\etape Enlevez la girofle de l'oignon (et jetez là), rajoutez la crème liquide puis mixez le tout
\end{preparation}
\end{recette}

\section{Velouté de courgette}
\begin{recette}{Velouté de courgette}{4}{1h}{}\index{velouté}\index{courgette}\index{velouté de courgettes}
\begin{ingredients}
\ingredient 1kg de courgettes
\ingredient bouillon de volaille
\ingredient 125g de boursin (ou équivalent ail et fines herbes)
\ingredient sel, poivre
\end{ingredients}

\begin{preparation}
\etape Coupez les courgettes en morceaux 
\etape Dans la marmite, ajoutez de l'eau jusqu'à couvrir les courgettes et faites bouillir pendant 45 minutes dans le bouillon
de volaille. 
\etape Égouttez alors mixez les courgettes ainsi cuites avec le fromage
\etape Ajoutez sel, poivre. Vous pouvez aussi rajouter un soupçon de céleri.
\end{preparation}
\end{recette}

}% End of the ``group'' where section is deactivated


\chapter{Recettes à tester}
\minitoc

% Beginning of group where section is deactivated
% This is only to get the good structure of the document 
% since ``section'' is in fact embedded in the 'recette' environment.
% This group allow us to deactivate sections ONLY in the given file and 
% not for the entire document.
{\renewcommand{\section}[1]{}

\section{Chatrou (poulpe)}
\begin{recette}{Chatrou (poulpe)}{4}{1h30}{}\index{chatrou}\index{poulpe}
\begin{ingredients}
\ingredient 5-10cl d'huile tournesol+roucou (il y avait un fond de 2-3mm dans l'autocuiseur de jacqueline, quantité à déterminer)
\ingredient 2kg de chatrou
\ingredient 4-5 cives
\ingredient 8 gousses d'ail
\ingredient 1 oignon
\ingredient 1 piment végétarien
\ingredient sel, poivre, 1 clou de girofle, 1 feuille de bois d'inde
\ingredient 20ml de jus de citron (jus de 1/2 citron)
\end{ingredients}

\begin{preparation}
\etape Égouttez et émincez le poulpe
\etape Émincez les cives, le piment végétarien.
\etape Faites revenir le poulpe pendant 15 minutes. Mettez le jus dans un saladier et mettez le poulpe à part.
\etape Prélevez un peu de jus dans un bol (pour la fin) et ajoutez-y le jus de citron. 
\etape Hachez l'ail finement et réservez 5 gousses dans un saladier et 3 gousses dans un bol
\etape Faites chauffer l'huile (il en faut pas mal)
\etape Faites revenir le chatrou avec les cives, l'oignon, le piment émincé, un clou de girofle et la feuille de bois d'inde pendant 10 minutes environ
\etape Faites mijoter 30 minutes avec le jus du saladier et à couvert
\etape Ajoutez alors le jus du bol (+ail) et laissez à couvert et hors du feu pendant 10 minutes

\end{preparation}
\end{recette}


\section{Gratin pays (Guadeloupe)}
\begin{recette}{Gratin pays (Guadeloupe)}{0}{1h30}{}\index{patate douce}\index{igname}\index{banane jaune}\index{banane plantain}
\begin{ingredients}
\ingredient 2 bananes jaunes (plantain)
\ingredient 750g de patate douce (blanche) (la orange est trop sucrée pour ce gratin)
\ingredient 750g d'igname (1 igname)
\ingredient 2 gousses d'ail
\ingredient 25cl de crème fraiche %(au resto, on m'a dit "du lait")
\ingredient 100g de fromage rapé
\ingredient sel, poivre, noix de muscade
\end{ingredients}

\begin{preparation}
\etape Mettez la patate douce et l'igname dans l'eau froide salée et portez à ébullition
\etape Une fois à ébullition, rajoutez les bananes entières avec la peau. Faites cuire les bananes 10 minutes et le reste 20 minutes. Les morceaux doivent être fondant, continuer la cuisson au besoin
\etape Ne mettez pas toute la banane car il ne faut pas que ce soit trop sucré. 
\etape Préchauffez le four à 180°C
\etape Ecrasez le tout, ajoutez de la muscade, l'ail écrasé, la crème fraiche et étalez dans un plat à gratin
\etape Saupoudrez de fromage rapé puis enfournez pendant 30 minutes.
\end{preparation}
\end{recette}

\section{Pulled Pork}
\begin{recette}{Pulled Pork}{0}{1h30}{}
\begin{ingredients}
\ingredient roti de porc dans l'échine (ou autre morceau de porc
\ingredient sauce barbecue
\ingredient 75cl de cidre
\end{ingredients}

\begin{preparation}
\etape badigeonnez la viande de sauce barbecue
\etape Faites cuire 20 minutes au four à 275°C pour saisir
\etape Ajoutez le cidre
\etape Faites alors cuire 6-7h à 130°C dans un récipient hermétiquement fermé.
\end{preparation}
\end{recette}

\section{Riz indien}
\begin{recette}{Riz indien}{0}{1h30}{}
\begin{ingredients}
\ingredient 500g de riz
\ingredient 1 carotte à raper
\ingredient petite boite de petit pois
\ingredient 75g d'oignon frit
\end{ingredients}

\begin{preparation}
\etape Rapez la carotte
\etape Faites cuire le riz au cuiseur à riz en mettant la carotte rapée et les petits pois dès le début en plus de l'eau et du sel
\etape Une fois cuit, rajoutez l'oignon frit
\end{preparation}
\end{recette}

%https://bistroguru.com/recette/recette-poulet-korma/
%https://www.mesinspirationsculinaires.com/article-poulet-korma.html
% ingredient for the korma sauce at safeway: Water, Sugar, Desiccated Coconut, Cream, Coconut Paste, Onion, Canola Oil, Food Starch Modified, Contains 2% or Less of Tomato Paste, Heavy Cream, Ginger, Garlic, Spices (Including Turmeric), Salt, Lactic Acid, Lemon, Juice Concentrate, Dried Cilantro Leaf. 
\section{Poulet Korma}
\begin{recette}{Poulet Korma}{0}{1h30}{}
\begin{ingredients}
\ingredient 1 yaourt nature (ou 20cl de creme fraiche) % aux us j'ai mis 450g de creme fraiche
\ingredient 4 gousses ail écrasées
\ingredient 50g de gingembre frais râpé % aux us j'ai mis l'équivalent de 2-3 gousses d'ail
\ingredient 2 c-a-c garam massala % aux us j'ai mis 2 cac)
\ingredient 1kg de poulet coupé en morceau
\ingredient 70g de concentré de tomate (2 cas)
\ingredient 4 oignons moyens hachés
\ingredient 2 c-a-soupe huile végétale
\ingredient 30g de beurre
\ingredient 2 c-a-c coriandre en poudre
\ingredient coriandre ciselée
\ingredient 400 ml lait de coco
\ingredient 400 ml bouillon de poulet
\ingredient 60g d'amande moulu
\ingredient sel
\ingredient poivre
\end{ingredients}

\begin{preparation}
\etape Emincez les oignons, mixez l'air et le gingembre et réservez dans un bol.
\etape Faites revenir l'oignon dans l'huile, le beurre et les épices pendant 10 minutes. 
\etape Ajoutez le poulet, le bouillon, le concentré de tomate, l'ail et le gingembre mixé et faites cuire à couvert pendant 30 minutes.
\etape Ajoutez alors le lait de coco et les amandes moulues. Faites alors cuire 30 minutes de plus à feu très doux et à couvert (attention à ne pas faire bouillir la sauce) Rectifier l'assaisonnement.
\etape Servir chaud accompagné de riz.
\end{preparation}
\end{recette}

\section{Pâtes à l'ail}
\begin{recette}{Pâtes à l'ail}{3}{30 min.}{}\index{pâtes}\index{ail}
\begin{ingredients}[Pour 500g de pâtes]
\ingredient 125g d'ail frais
\ingredient 60g de parmesan
\ingredient 20 cl de crème liquide légère 
\ingredient sel
\end{ingredients}


\begin{preparation}
\etape Faites cuire les gousses d'ail en chemise (sans enlever la peau) pendant 20 minutes au four à 200°C sans préchauffage. 
\etape Ecrasez l'ail en purée et mélangez à la crème liquide et au sel, puis laissez infuser jusqu'à ce que les pâtes soient cuite
\etape Une fois les pâtes cuites, rajoutez dans le plat le contenu de la casserole et le parmesan. Remuez jusqu'à ce que ça ait la consistance qui vous convienne (avec la chaleur des pâtes, ça va épaissir un peu).
\begin{remarque}
S'il y a vraiment trop de liquide, rallumez un peu le feu sous la marmite tout en remuant jusqu'à ce que la consistance vous convienne.
\end{remarque}
\end{preparation}
\end{recette}

\section{Colombo de poulet}
\begin{recette}{Colombo de poulet}{4}{1h}{1h}\index{poulet}\index{colombo}
\begin{ingredients}
\ingredient 1.5kg de poulet
\ingredient 4 oignons
\ingredient [optionnel] 4 piments végétariens
\ingredient 5 gousses d'ail
\ingredient 15ml de jus de citron (1/2 citron)
\ingredient 50cl de bouillon de poulet
\ingredient 5g de sel
\ingredient 40g de poudre de colombo
\ingredient 2 cuillère à soupe de maizena
\end{ingredients}


\begin{preparation}
\etape Emincez l'oignon
\etape Faites chauffer l'eau avec le bouillon cube
\etape Emincez finement le piment végétarien, mixez l'ail et mettez le tout à infuser dans le bouillon
\etape Pesez la poudre de colombo et le sel dans un bol
\etape Saisissez dans du beurre ou de l'huile les morceaux de poulet à feu vif puis réservez-les.
\etape Faites revenir l'oignon.
\etape Une fois revenus, rajoutez la poudre de colombo et le sel et mélangez. 
\etape Ajoutez enfin le bouillon et le jus de citron puis mélangez. Ajoutez alors le poulet.
\end{preparation}

\begin{cuisson}
Laissez alors mijoter à couvert et à feu doux (4/9) 45 minutes environ.
\end{cuisson}
\end{recette}

}% End of the ``group'' where section is deactivated


% https://www.temps-de-cuisson.info/cuisson/
\chapter{Viandes (cuisson)}
\minitoc

% Beginning of group where section is deactivated
% This is only to get the good structure of the document 
% since ``section'' is in fact embedded in the 'recette' environment.
% This group allow us to deactivate sections ONLY in the given file and 
% not for the entire document.
{\renewcommand{\section}[1]{}


\section{Canard entier au four}
\begin{recette}{Canard entier au four}{4}{20 minutes}{1h30}\index{canard au four}\index{canard}
\begin{ingredients}
\ingredient un canard entier
\ingredient pain dur
\ingredient sel, poivre du moulin
\end{ingredients}

\begin{preparation}
\etape Enlevez ce que vous pouvez de l'intérieur du canard. Si l'estomac est explosé, vous pouvez passer un peu d'eau pour 
éliminer les morceaux.
\etape Garnissez de pain l'intérieur du canard, puis fermez avec la peau qui dépasse peut-être, en la coinçant avec des piques 
en bois.
\etape Salez et poivrez le canard.
\end{preparation}

\begin{cuisson}
Faites cuire à 220°C pendant 1h30 environ (20 minutes par 500g)

Dans mon cas, j'ai fait avec un canard gras de 3.3kg, comme il avait beaucoup de gras, j'ai diminué un peu le poids du canard 
pour le calcul du temps de cuisson, puisque ce gras fond rapidement. In fine, il a cuit 1h30 à peu près.
\end{cuisson}
\end{recette}


\section{Cote de boeuf au four}
\begin{recette}{Cote de boeuf au four}{4}{2h}{30 min}\index{cote de boeuf}
\begin{ingredients}
\ingredient 1 cote de boeuf (elle était épaisse de 3cm dans mon cas)
\end{ingredients}

\begin{preparation}
\etape Sortez la viande 2h avant cuisson pour qu'elle soit à température ambiante
\etape Juste avant cuisson, badigeonnez les deux faces d'huile et salez (il faut saler avant pour que le sel pénètre dans la viande mais il faut saler au dernier moment pour que la viande ne s'assèche pas
\begin{remarque}
Ne piquez surtout pas la viande avant ou pendant la cuisson
\end{remarque}
\etape Préchauffez le four à 200°C
\etape Saisissez la viande à feu très vif 1min30s de chaque coté (réaction de maillard). 
\etape Faites cuire au four. L'idéal est d'avoir une sonde de température (cf tableau des températures en fin de livre). Comptez 50°C à coeur pour une viande saignante, ou 45°C pour une viande bleue (environ 10 minutes)
\etape Une fois cuite, recouvrez de papier aluminium et laisser reposer 10 minutes avant de manger. (la viande continue de cuire dans le papier alu, raison pour laquelle il ne faut pas encore être à température avant) 
\end{preparation}
\end{recette}

\section{Cuisse de dinde au four}
\begin{recette}{Cuisse de dinde au four}{3}{20min+12h}{1h30}\index{dinde}
\begin{ingredients}
\ingredient Une cuisse de dinde ($\sim$ 2kg)
\ingredient sel caraïbe ou sel normal
\ingredient 2-5 gousses d'ail
\ingredient 2 oignons
\ingredient 20cl d'eau
\end{ingredients}

\begin{preparation}
\etape Piquer la viande (faut y aller franchement)
\etape Saler abondamment la viande. Laissez mariner une nuit
\etape Le lendemain, mixer ail et oignon et mettre dans l'eau
\end{preparation}

\begin{cuisson}
Badigeonnez la dinde d'une couche de marinade (ne pas tout mettre) puis enfournez la (sans le papier aluminium) à 200°C, sans préchauffage. 

Le temps de cuisson total est 1h30. 45 minutes avant la fin, grattez la marinade de la cuisse pour la faire tomber dans le plat, ajoutez le reste de marinade dedans, puis couvrez de papier aluminium pour la fin de cuisson.
\end{cuisson}
\end{recette}

\section{Cuisse de poulet bi-cuisson}
\begin{recette}{Cuisse de poulet bi-cuisson}{0}{10 min + 6h}{}\index{langouste}
\begin{ingredients}
\ingredient Cuisses ou pilons de poulet
\end{ingredients}

\begin{preparation}
\etape Faites cuire le poulet 30 minutes au four à 180°C. Ainsi il est presque cuit, et beaucoup plus facile et rapide à faire au barbecue. C'est notamment pratique quand il y en a une grande quantité à faire. 
\etape Réservez le jus de cuisson du poulet pour vous en servir de sauce au moment du repas.
\etape Faites alors cuire au dernier moment au barbecue à feu vif pour saisir et faire dorer, pas besoin de laisser longtemps, il est déjà quasiment cuit
\end{preparation}
\end{recette}

% https://www.temps-de-cuisson.info/cuisson/viandes/gigot-dagneau-au-four/
\section{Gigot d'agneau}
\begin{recette}{Gigot d'agneau}{3}{1 nuit}{2h}\index{gigot}\index{agneau}
\begin{ingredients}
\ingredient 1 gigot d'agneau
\ingredient 8 gousses d'ail (facultatif)
\ingredient huile
\ingredient épices (thym, romarin, origan, poivre)
\end{ingredients}


\begin{preparation}
\etape Mettez le gigot dans un plat allant au four.
\etape Badigeonnez le gigot d'huile, puis mettez les épices. 
\etape Laissez mariner au moins une nuit

\end{preparation}

\begin{cuisson}
Comptez 15 minutes de cuisson par 450g à 180°C pour une cuisson rosée (déduisez les 15 premières minutes à 240°C du total). 

\begin{remarque}
En résumé, pour un gigot de 2.230kg, faites cuire à 240°C pendant 15 minutes et à 180°C pendant 59 minutes (74-15).
\end{remarque}

\begin{enumerate}
 \item Faites préchauffer le four 10 minutes à 240°C
 \item Faites cuire 15 minutes à 240°C pour saisir le gigot
 \item Faites cuire à 180°C 15 minutes/450g (soit environ 65 minutes pour un gigot de 2kg)
 \begin{remarque}
  40 minutes avant la fin de la cuisson, mettez les gousses d'ail en chemise dans le plat.
 \end{remarque}
\etape Laissez reposer la viande 20 minutes en dehors du four, couvert de papier aluminium, un torchon par dessus, pour garder la chaleur
\end{enumerate}

\end{cuisson}
\end{recette}

\section{Magret de canard à la poele}
\begin{recette}{Magret de canard à la poele}{4}{20 min}{1h}\index{poulet}
\begin{ingredients}
\ingredient Magret de canard
\ingredient gros sel
\end{ingredients}

\begin{preparation}
\etape Sortez le magret du frigo 1h avant. Coupez en deux dans la longueur, perpendiculairement au gras
\etape Coté gras, entaillez dans la longueur sur la moitié de l'épaisseur à peu près, sans traverser
\etape Entaillez le gras du magret (quadrillez) puis frottez au gros sel des deux cotés
\end{preparation}

\begin{cuisson}
\begin{itemize}
\item Faites chauffer la poele à feu vif (7/9)
\item Faites cuire le magret coté peau à feu vif (6/9) 2 minutes de chaque coté (en général, s'il est joli, c'est qu'il est cuit)
\item Laissez alors le magret reposer enveloppé dans du papier aluminium pendant 5 minutes
\end{itemize}
\end{cuisson}
\end{recette}

\section{Pilons de poulet marinés au four}
\begin{recette}{Pilons de poulet marinés au four}{4}{20 min}{1h}\index{poulet}
\begin{ingredients}
\ingredient 1 kg de pilon
\ingredient 2 oignons
\ingredient 2 gousses d'ail
\ingredient une cuillère à café de piment végétarien
\ingredient 2 cuillères à soupe d'épice pour poulet (surtout pas de sel supplémentaire parce que c'est déjà salé)
\ingredient 5cl de citron
\ingredient 20cl d'eau
\end{ingredients}

\begin{preparation}
\etape Mixez ail et oignon, mélangez avec le reste de la marinade
\etape Piquez le poulet un peu partout pour que la marinade pénètre bien
\etape Laissez le poulet mariner au frais pendant une nuit environ.
\end{preparation}

\begin{cuisson}
Faites préchauffer le four à 200°C environ 10 minutes

Séparez le poulet de la marinade et réservez cette dernière. Enfournez le poulet seul pendant 40 minutes à 200°C

Puis rajoutez la marinade et faites cuire 10 minutes à 180°C

\end{cuisson}
\end{recette}

\section{Pintade entière au four}
\begin{recette}{Pintade entière au four}{4}{15 min}{2h}\index{pintade}
\begin{ingredients}
\ingredient 1 pintade
\ingredient un peu de beurre
\ingredient du pain
\ingredient ail
\end{ingredients}

\begin{preparation}
\etape Préchauffez le four à 180°C pendant 10 à 15 minutes
\etape Badigeonnez la pintade d'un peu de beurre (ou huile à défaut).
\etape ajoutez de l'ail, du pain dans la pintade jusqu'à remplir
\etape disposez la pintade à l'envers sur le plat, pour que les blancs soient vers le bas et les protéger afin qu'ils ne sèchent pas.
\end{preparation}

\begin{cuisson}
Comptez 25 min. / 500g à 180°C (donc 1h40 pour une pintade de 2kg). 40 minutes avant la fin de cuisson, j'ai rajouté environ 10cl de bouillon de volaille et 20 minutes avant la fin, prendre directement du jus de cuisson et en remettre sur la pintade pour ne pas qu'elle seche. 

\end{cuisson}
\end{recette}

\section{Poulet entier}
\begin{recette}{Poulet entier}{4}{15 min}{2h}\index{poulet}
\begin{ingredients}
\ingredient 1 poulet
\end{ingredients}

\begin{preparation}
\etape Préchauffez le four à 180°C pendant 10 à 15 minutes
\etape Badigeonnez le poulet d'un peu d'huile. Salez, poivrez
\etape ajoutez de l'ail, du pain (olive selon votre goût), puis fermez la peau du ventre avec une pique ou deux pour maintenir en place
\end{preparation}

\begin{cuisson}
Comptez 25 min. / 500g à 180°C (donc 1h15 pour un poulet de 1.5kg). On peut faire les 10 dernières minutes à 200°C pour faire dorer la peau

% source : https://www.papillesetpupilles.fr/2019/10/comment-cuire-un-poulet-au-four.html/
% Enfournez à four chaud : C’est la façon classique que je pratiquais avant. Enfournez dans un four préchauffé à 180°C / 190°C et pour le temps de cuisson, comptez 25 minutes par 500 g. Pour 1 poulet d’1,5 Kg cela fait donc 1h15 (C’est la règle mais perso j’ai tendance à cuire moins :p ).
% Enfournez à four froid : Mettez votre poulet dans le four et allumez-le à 150°C. Pour le temps de cuisson, comptez 1 heure par kilo de poulet. Les 10 dernières minutes, montez le four à 200°C pour, cela fera croustiller la peau. Le fait de cuire doucement permet d’éviter que les graisses du poulet ne fondent et s’échappent de la volaille. Elles restent à l’intérieur et gardent le poulet moelleux. Là aussi c’est la règle et je cuis en général un peu moins.

Laissez reposer la viande couvert de papier aluminium et d'un torchon pendant 20 à 30 minutes.

\end{cuisson}
\end{recette}

\section{Saucisse de toulouse au four}
\begin{recette}{Saucisse de toulouse au four}{4}{15 min}{30 min}\index{saucisse}
\begin{ingredients}
\ingredient Saucisse de toulouse
\end{ingredients}

\begin{preparation}
\etape Préchauffez le four en grill à 275°C
\etape Ne piquez pas la saucisse, ne la sortez pas du frigo à l'avance non plus
\end{preparation}

\begin{cuisson}
Sortez la saucisse du frigo et enfournez immédiatement. 

\begin{itemize}
 \item Pour de la saucisse fine: 5 puis 2min30s de l'autre coté, à 275°C
 \item Pour de la grosse saucisse: 5 minutes à 275°C de chaque coté
\end{itemize}

% Avec la grosse saucisse, voici les essais:
% 30 minutes à 180°C (pas mal mais un chouillat cuit)
% 20 minutes à 220°C (pas bon mais difficile a savoir si trop cuit ou pas)
% 25 minutes à 220°C (pas bon mais difficile a savoir si trop cuit ou pas)
% au barbecue, pas percé, c'était hyper bon mais il faut pas trop cuire. Donc je dirais qu'il faut feu très vif, et pas trop longtemps. la saucisse n'était pas bien grillée et c'était de la grosse

% en grill à 275°C, saucisse dans un plat à gratin (pour avoir moins de vaisselle)
% 5 minutes de chaque coté, sans sortir la saucisse à l'avance et avec un petit préchauffage du four, puis cuisson avec porte entr-ouverte, et saucisse quasi en haut, sous la grille. 

Je n'ouvre même pas la porte pour les tourner à mi cuisson. 
\end{cuisson}
\end{recette}

\section{Saucisse de toulouse à la poele}
\begin{recette}{Saucisse de toulouse à la poele}{4}{15 min}{30 min}\index{saucisse}
\begin{ingredients}
\ingredient Saucisse de toulouse
\end{ingredients}

\begin{preparation}
\etape Préchauffez la poele a feu moyen/vif (6/9)
\etape Faites cuire la grosse saucisse 5 minutes de chaque coté, avec le couvercle pour que ça cuise aussi à coeur.
\end{preparation}

\begin{cuisson}
Préchauffez la poele a feu moyen/vif (6/9).

Faites cuire la grosse saucisse 5 minutes de chaque coté, avec le couvercle pour que ça cuise aussi à coeur.

Enveloppez la saucisse dans du papier aluminium et laissez reposer 5 minutes.
\end{cuisson}
\end{recette}

\section{Steak haché à la poele}
\begin{recette}{Steak haché à la poele}{}{5 min}{}\index{Steak haché}
\begin{ingredients}
\ingredient Steak haché
\end{ingredients}

\begin{cuisson}
Préchauffez la poele a feu moyen/vif (7/9).

\begin{itemize}
\item Déposez le steak d'un coté, salez le dessus, et faites cuire sans couvercle à feu vif 7/9
\item Retournez le steak et salez le dessus et faites cuire sans couvercle
\end{itemize}

\begin{itemize}
\item bleu: 1min30s de chaque coté
\item saignant: 2min de chaque coté
\item à point: 2min30s de chaque coté
\end{itemize}

\begin{remarque}
Pour les burgers, quand vous retournez les steaks, ajoutez le fromage sur le coté déjà cuit, puis couvrez avec le couvercle (et faites cuite la même durée que prévu)
\end{remarque}

\end{cuisson}
\end{recette}

\section{Travers de porc au four}
\begin{recette}{Travers de porc au four}{5}{15 min+24h}{5h+12h+1h}\index{travers de porc}
\begin{ingredients}[2 pers.]
\ingredient 1kg de travers de porc (Comptez 500g par personne et 250g par enfant)
\end{ingredients}

\begin{preparation}
\etape Faites mariner les travers de porc pendant 24 à 48h. Je conseille la marinade \refsec{sec:travers_texane}. Idéalement, faites mariner dans le plat qui ira au four (et recouvrez de papier aluminium)
\etape Le lendemain, mettez le plat au four (avec toute la marinade), recouvert de papier aluminium. 
\end{preparation}

\begin{cuisson}
\begin{itemize}
\item Faites cuire à couvert (laissez le papier aluminium mis la veille) pendant 4h (5h si vous doublez les doses) à 130°C (Cette étape peut être faite la veille et la dernière cuisson le lendemain). La viande doit se rétracter de l'os, si ce n'est pas le cas, remettez encore un peu.
\item Sortez le plat du four et videz le jus dans un saladier pour le réserver.
\item Remettez alors les travers à cuire pendant 1h à 180°C. Surveillez la cuisson. rajoutez du jus de temps en temps si ça sèche trop (je le fais une seule fois à mi cuisson).
\end{itemize}

% initialement, j'ai fait cuire 4h à 110°C, puis 1h à 180°C, mais c'était trop. 
% 4h a 135 degres, c'etait moins bon, pas assez cuit.

\end{cuisson}
\end{recette}

}

\chapter{Grillades}
\minitoc

% Beginning of group where section is deactivated
% This is only to get the good structure of the document 
% since ``section'' is in fact embedded in the 'recette' environment.
% This group allow us to deactivate sections ONLY in the given file and 
% not for the entire document.
{\renewcommand{\section}[1]{}

\section{Optimiser une marinade}
\begin{itemize}
\item Couper les aromates et légumes en tout petits morceaux pour augmenter les surfaces d'échanges. Pour certains, on peut même mixer. 
\item Chauffer un peu la marinade en amont afin d'aider à diffuser les arômes
\item Eviter de trop saler, c'est l'eau du produit qui risque de sortir et non les arômes qui vont rentrer. 
\item L'alcool déshydrate, mieux vaut le flamber rapidement pour éliminer une grande partie de l'éthanol
\item 1/4 de cuillère à café  de bicarbonate de sodium par litre dans la marinade attendrit les viandes et favorise les échanges. 
\end{itemize}

\section{Langoustes au barbecue}
\begin{recette}{Langoustes au barbecue}{0}{10 min + 6h}{}\index{langouste}
\begin{ingredients}
\ingredient langoustes
\end{ingredients}

\begin{preparation}
\etape Coupez les langoustes dans le sens de la longueur. (j'ai utilisé un hachoir de boucher et un marteau)
\etape Faites cuire 2 minutes coté chair, Laissez plus longtemps si la chair n'est pas encore blanchie
\etape Retournez et faites cuire 5 minutes coté carapace.
\end{preparation}
\end{recette}



\section{Marinade Aigre-douce}
\begin{recette}{Marinade Aigre-douce}{0}{10 min + 6h}{}\index{marinade}\index{miel}\index{vinaigre}
\begin{ingredients}
\ingredient 2 cuillères à soupe d'huile d'olive
\ingredient 2 cuillères à soupe de vinaigre (de vin ou balsamique, au choix)
\ingredient 2 cuillères à soupe de miel liquide
\ingredient 1 cuillère à soupe de moutarde
\ingredient 1 gousse d'ail hachée
\ingredient une pincée de sel, et un peu de poivre
\end{ingredients}

\begin{preparation}
\etape Mettez les ingrédients dans une poche de congélation
\etape Fermez la poche de manière grossière (en entortillant l'ouverture par exemple) puis secouez jusqu'à ce que la marinade 
soit homogène
\etape mettez la viande dans la poche, de préférence du porc qui va très bien avec, et laissez reposer au frigo quelques heures, 
une nuit typiquement
\etape Il ne reste plus qu'à ouvrir la poche et faire griller les morceaux marinés comme de la viande normale.
\end{preparation}
\end{recette}

\section{Marinades au vin blanc}
\begin{recette}{Marinades au vin blanc}{0}{5 min + 6h}{}\index{marinade}\index{moutarde}
\begin{ingredients}
\ingredient $\sfrac{1}{2}$ litre de vin blanc sec
\ingredient $1$ verre d'eau
\ingredient $1$ grosse cuillère à soupe de moutarde forte
\ingredient $1$ grosse cuillère à soupe de moutarde à l'ancienne
\ingredient $1$ grosse cuillère à soupe de thym
\end{ingredients}

\begin{preparation}
\etape Mettez les ingrédients dans une poche de congélation
\etape Fermez la poche de manière grossière (en entortillant l'ouverture par exemple) puis secouez jusqu'à ce que la marinade 
soit homogène
\etape mettez la viande dans la poche, de préférence du porc qui va très bien avec, et laissez reposer au frigo quelques heures, 
une nuit typiquement
\etape Il ne reste plus qu'à ouvrir la poche et faire griller les morceaux marinés comme de la viande normale.
\end{preparation}

\begin{remarque}
On peut ajouter du poivre et des oignons à cette marinade \dots  et aussi d'autres herbes\dots
\end{remarque}
\end{recette}

\section{Marinade de Porc au paprika}
\begin{recette}{Marinade de Porc au paprika}{3}{5 min + 6h}{}\index{marinade}\index{paprika}\index{porc}
\begin{ingredients}
\ingredient 2 cuillères à soupe d'huile d'olive
\ingredient 1 cuillère à café de paprika
\ingredient herbe de provence, poivre
\end{ingredients}

\begin{preparation}
\etape Mettez les ingrédients dans une poche de congélation
\etape Fermez la poche de manière grossière (en entortillant l'ouverture par exemple) puis secouez jusqu'à ce que la marinade 
soit homogène
\etape mettez la viande dans la poche, de préférence du porc qui va très bien avec, et laissez reposer au frigo quelques heures, 
une nuit typiquement
\etape Il ne reste plus qu'à ouvrir la poche et faire griller les morceaux marinés comme de la viande normale.
\end{preparation}

\end{recette}

\section{Marinade Tandoori}
\begin{recette}{Marinade Tandoori}{3}{5 min + 6h}{}\index{marinade}\index{tandoori}
\begin{ingredients}
\ingredient $2$ cuillères à soupe de Tandoori
\ingredient $2$ cuillères à soupe de jus de citron (ou de vinaigre)
\ingredient $2$ cuillères à soupe d'huile d'olive
\ingredient $1$ ou $2$ yahourt nature
\end{ingredients}

\begin{preparation}
\etape Mettez les ingrédients dans une poche de congélation
\etape Fermez la poche de manière grossière (en entortillant l'ouverture par exemple) puis secouez jusqu'à ce que la marinade 
soit homogène
\etape mettez la viande dans la poche, de préférence du porc qui va très bien avec, et laissez reposer au frigo quelques heures, 
une nuit typiquement
\etape Il ne reste plus qu'à ouvrir la poche et faire griller les morceaux marinés comme de la viande normale.
\end{preparation}

\begin{remarque}
Vous pouvez également mariner le poulet dans un peu de yaourt additionné de citron vert, de l'ail écrasé, et du \textbf{curry}. 
Ces marinades doivent imprégner assez longtemps le poulet.
\end{remarque}

\end{recette}

\section{Marinade texane}\label{sec:travers_texane}
\begin{recette}{Marinade texane}{4}{15 min + 6h}{}\index{marinade}\index{sauce soja}
\begin{remarque}
Normalement, c'est fait avec des coustilles (travers de porc). La viande doit mariner au moins 6 heures, l'idéal étant plus de 24h (j'ai fait 36h la dernière fois).
\end{remarque}
\begin{ingredients}
\ingredient 1 oignon
\ingredient 3 gousses d'ail
\ingredient 2 cuillères à soupe de miel
\ingredient 3 cuillères à soupe de sauce soja
\ingredient 1 cuillère à soupe d'huile d'olive
\ingredient 1 cuillère à soupe de Ketchup
\ingredient 2 cuillères à soupe de vinaigre
\ingredient 1 cuillère à café de sel
\ingredient Thym, laurier (ne pas mixer, mettre à part)
\end{ingredients}

\begin{preparation}
\item Je met l'oignon et les gousses d'ails coupées grossièrement dans un mixeur avec le vinaigre et l'huile (ceci permet de 
mieux couper les morceaux)
\item Je prépare dans un bol le reste de la marinade, puis j'inclus le contenu du mixeur
\item Versez un peu d'eau dans le mixeur pour récupérer un maximum de marinade.
\item Je mélange puis étale la mixture sur la viande que je laisse mariner quelques heures (environ une nuit).
\item Ajoutez alors thym et laurier sur la viande avant de fermer avec du papier aluminium
\end{preparation}

\begin{remarque}
La marinade est un peu épaisse, et il faut l'étaler et non la verser (vu que j'en fais pas beaucoup dans ces cas là, je met la 
viande dans un plat et étale à la cuillère sur chaque face).
\end{remarque}
\end{recette}


}% End of the ``group'' where section is deactivated


\chapter{Accompagnements}\label{sec:accompagnement}
\minitoc

% Beginning of group where section is deactivated
% This is only to get the good structure of the document 
% since ``section'' is in fact embedded in the 'recette' environment.
% This group allow us to deactivate sections ONLY in the given file and 
% not for the entire document.
{\renewcommand{\section}[1]{}

\section{Aubergines à la poële}
\begin{recette}{Aubergines à la poële}{4}{45 minutes}{}\index{aubergines}
\begin{ingredients}
\ingredient 1kg d'aubergines
\ingredient 2 cuillères à soupe d'huile d'olive
\end{ingredients}

\begin{preparation}
\etape Pelez et coupez en cube les aubergines (environ $1\unit{cm}^3$).
\etape Faites chauffer l'huile puis mettez les aubergines en remuant pour bien répartir l'huile. 
\etape Remuez régulièrement jusqu'à ce que les aubergines soit dorées et moelleuses.
\end{preparation}

Le goût peut faire penser à des champignons, c'est très fin et je n'ai pas encore trouvé comment utiliser ce goût.

\begin{remarque}
La quantité d'huile est un point crucial. Trop, et ça sera trop gras, mais pas assez, et ça cramera au lieu de dorer. Notez 
qu'au début, tant que les aubergines ne sont pas réduites, il n'est pas bon de rajouter de l'huile tant que ça absorbe ou tant 
qu'on voit que ça ne dore pas. En effet, ce qui compte c'est à la fin. Car quand les aubergines vont réduire, l'huile qu'elles 
ont absorbé va ressortir et ça ne sert donc à rien d'en mettre trop. 

C'est quelque chose qui se voit à l'œil à force de faire la recette.
\end{remarque}
\end{recette}

\section{Banane jaune (plantain)}
\begin{recette}{Banane jaune (plantain)}{3}{20 min}{1h}\index{banane}
\begin{ingredients}
\ingredient Comptez deux bananes par personne si c'est le seul accompagnement.
\end{ingredients}

\begin{preparation}
\etape Coupez les deux extrémités de la banane, puis fendez la peau sur toute la longueur sans l'enlever
\etape Faites cuire la banane avec la peau jusqu'à ce que la peau soit uniformément noire, puis continuez la cuisson jusqu'à ce que l'intérieur soit moelleux (contrôlez avec un couteau)
\end{preparation}
\end{recette}

\section{Carotte à la crème}
\begin{recette}{Carotte à la crème}{3}{20 min}{1h}\index{carottes}\index{crème}
\begin{ingredients}
\ingredient 1kg de carottes
\ingredient 2 échalotes
\ingredient 2 oignons
\ingredient 1 gousse d'ail
\ingredient 20g de beurre
\ingredient 20cl de bouillon (eau + bouillon cube de boeuf ou de volaille)
\ingredient 20 cl de crème fraîche épaisse
\ingredient sel, poivre, cerfeuil (frais ou pas)
\end{ingredients}

\begin{preparation}
\etape Éplucher et couper en rondelles les carottes. Peler et émincer les échalotes, les oignons et l’ail. Hacher le cerfeuil.

\etape Dans une cocotte, faire suer dans le beurre les oignons, les échalotes et l’ail pendant 4 minutes sur feu doux.

\etape Ajouter les rondelles de carottes et prolonger la cuisson 5 minutes en remuant de temps en temps. Il faut que les 
carottes commencent à rendre un peu de jus.

\etape Verser le bouillon, saler et poivrer. Couvrir et laisser cuire sur feu moyen pendant 25 minutes (ajouter de l'eau s'il 
n'en reste plus, les carottes doivent tout absorber).

\begin{remarque}
Les carottes doivent être cuites avant d'ajouter la crème. Une fois la crème ajoutée, ça cuit beaucoup moins vite.
\end{remarque}

\etape Ajouter la crème fraîche et le cerfeuil, bien mélanger. Goûter et rectifier l’assaisonnement si nécessaire. Laisser cuire 
encore 10 minutes sur feu doux.

\etape Servir aussitôt pour accompagner un rôti de porc ou des escalopes .
\end{preparation}
\end{recette}

\section{Fondue de poireaux}
\begin{recette}{Fondue de poireaux}{3}{20 min}{1h}\index{carottes}\index{crème}
\begin{ingredients}[3 personnes]
\ingredient 1kg de poireaux
\ingredient 20cl de crème liquide
\ingredient 1 cac de moutarde
\ingredient 2 cuillères à soupe de jus de citron
\ingredient sel
\end{ingredients}

\begin{preparation}
\etape émincez les poireaux
\etape Faites fondre 20g de beurre puis faites revenir les poireaux 2/3 minutes environ
\etape laissez cuire à feu doux (4/9) pendant 25 minutes
\etape Rajoutez la crème fraiche épaisse, le citron, la moutarde et le sel.
\etape Laissez cuire encore 10 minutes de plus à feux doux (4/9) et à couvert
\begin{remarque}
La fondue de poireaux accompagne très bien des poissons.
\end{remarque}
\end{preparation}
\end{recette}

\section{Frites}
\begin{recette}{Frites}{4}{}{}\index{pomme de terre}
\begin{ingredients}
\ingredient Pommes de terres (Environ 300-350g par personne)
\end{ingredients}

\begin{preparation}
\etape Pelez et coupez les pommes de terre en lamelles, toutes de taille similaire (5-7mm de coté par ex)
\etape Vous pouvez les laver si vous voulez, la communautée des fritophiles est divisée sur le sujet. Mais il faut que les frites soient sèches au moment de les plonger dans l'huile bouillante.
\end{preparation}

\begin{cuisson}
La taille maximum d'une fournée dépend de la quantité d'huile. Comptez 200g de frites fraiches ou 100g de frites surgelés par litre d'huile. 

Faites précuire les frites pendant 8 minutes à 160$\degres$ C. Épongez avec du papier absorbant pour enlever l'excès d'huile. 

Vous pouvez ensuite attendre une ou deux heures si vous le souhaitez avant la dernière cuisson, pratique pour timer le repas au mieux. 

Faites cuire 2m30 à 190$\degres$ C. Enlevez l'excès d'huile puis salez et servez. 
\end{cuisson}
\end{recette}


\section{Frites de patate douce}
\begin{recette}{Frites de patate douce}{}{30 min.}{45min.}\index{patate douce}
\begin{ingredients}
\ingredient 1.5kg de patate douce
\ingredient 50g de farine
\ingredient 9g de levure chimique
\ingredient 38g d'huile
\ingredient 5g de sel
\ingredient 2.5g d'épices (paprika, cumin)
\end{ingredients}
%\ingredient 1kg de patate douce
%\ingredient 35g de farine
%\ingredient 6g de levure chimique (1 cac)
%\ingredient 25g d'huile (2 cas)
%\ingredient 3g de sel
%\ingredient 1 cac d'épices (paprika, cumin)

\begin{preparation}
\etape Préchauffez le four à 220°C
\etape Pelez, lavez puis coupez les patates douces en frites.
\etape Dans un bol, mélangez la farine, le sel et les épices
\etape Mélangez les frites avec l'huile. Ajoutez alors la préparation farine/épice puis mélangez.
\etape Étalez les frites sur une seule couche sur une grande plaque.
\end{preparation}

\begin{cuisson}
Faites cuire 45 minutes à 200°C.
\end{cuisson}
\end{recette}

\section{Frites de carottes au four}
\begin{recette}{Frites de carottes au four}{4}{}{}\index{carotte}
\begin{ingredients}
\ingredient Carottes
\ingredient huile de tournesol
\end{ingredients}

\begin{preparation}
\etape Pelez et coupez les carottes en lamelles, toutes de taille similaire (5-7mm de coté par ex)
\etape Trempez les morceaux dans de l'huile (dans un sac congélation par exemple, pour limiter la quantité d'huile utilisée)
\etape Disposez les morceaux de carotte dans un plat allant au four
\end{preparation}

\begin{cuisson}
Faites cuire environ 1h à 150°C puis 20 minutes à 180°C
\end{cuisson}
\end{recette}

\section{Haricots verts}
\begin{recette}{Haricots verts}{}{}{}\index{haricots verts}
\begin{ingredients}
\ingredient 1kg de haricots verts surgelés
\ingredient bouillon cube
\ingredient ail en poudre, persil, sel
\end{ingredients}

\begin{preparation}
\etape Mettez les haricots dans une marmite, remplissez d'eau, salez, mettez le bouillon cube et faites bouillir
\etape Faites cuire 10 minutes à partir de l'ébullition
\etape Égouttez puis faites revenir 2 minutes dans la marmite avec un peu d'huile, l'ail et le persil afin d'assécher les haricots
\etape Versez les haricots ailleurs puis au besoin, rajoutez un peu d'eau en fin de cuisson pour récupérer ce qui a accroché plus facilement, mais pas trop d'eau
\end{preparation}
\end{recette}

\section{Poëlée forestière}
\begin{recette}{Poëlée forestière}{4}{}{}\index{pomme de terre}
\begin{ingredients}
\ingredient $500\unit{g}$ de pommes de terre coupées en dé
\ingredient $4$ ou $5$ oignons
\ingredient $200\unit{g}$ de lardons
\end{ingredients}

\begin{preparation}
\etape Faites cuire les lardons
\etape Une fois cuits, sortez les et faites revenir les oignons à feux doux dans la graisse des lardons en en rajoutant au 
besoin. Tournez les de temps en temps jusqu'à ce qu'ils soit dorés.
\etape sortez les et mettez les avec les lardons. Maintenant mettez les pommes de terre, surgelés ou coupées préalablement, à 
cuire à feux doux jusqu'à ce qu'elles soit cuites, et dorées. Il est important de les laisser cuire à feux doux, et de ne pas 
changer augmenter le feu pendant la cuisson.
\etape Une fois les pommes de terres cuites, ajoutez les oignons et les lardons, remuez de sorte à obtenir un mélange homogène 
et laissez le temps que les oignons et lardons se réchauffent, remuez et servez.
\end{preparation}
\end{recette}

\section{Pommes de terre au four}
\begin{recette}{Pommes de terre au four}{4}{}{}\label{sec:pomme-de-terre-four}\index{pomme de terre}
\begin{ingredients}
\ingredient environ 1-1.25 kg de pomme de terre pour 2 personnes
\ingredient 2 cac de sel rase pour 1kg de pomme de terre 
\ingredient huile d'olive
\ingredient un sac de congélation
\end{ingredients}

\begin{preparation}
\etape Quelques heures avant (pour que ce soit bien égoutté), pelez les pommes de terre. Coupez les en grosses frites de taille aussi uniforme que possible (pas trop épaisses (3-4 morceaux par demi pomme de terre)
\etape Lavez et laissez égoutter les pommes de terre jusqu'à ce qu'il soit temps de cuisiner, mais pendant 1 h environ. Essuyez au sopalin si vraiment c'est le rush et pour aller plus vite.
\etape Dans le sac de congélation mettez les pommes de terre et un peu d'huile d'olive, puis mélangez en la secouant.
\etape Disposez les pommes de terre dans un lèche frite, salez (sel et épices si vous voulez)
\end{preparation}

\begin{cuisson}
Faites préchauffer le four à 200\degres C 10 minutes environ, puis enfournez-les :
\begin{itemize}
\item 40 minutes pour des potatoes assez grosses
\item 35 minutes si elles sont taillées assez fines façon frites
\end{itemize}
en les disposant dans un grand plat à tarte. Puis laissez reposer 10-20 minutes four éteint. 
\begin{remarque}
Si vous ne pouvez pas faire reposer au four, laissez 20 minutes dehors couvert de papier aluminium et d'um torchon épais pour garder la chaleur, puis remettez au four éteint le temps de couper la viande qui était au four.
\end{remarque}

% Le four à 90°C est trop chaud et les patates trop cuites si vous faites ça au lieu de faire le four éteint. Il faut sans doute baisser encore pour que ça marche. 



%200°C pdt 40 minutes, c'est la version de tatie. Peut-être un peu plus de temps ou de température pour ajuster. 
\end{cuisson}
\end{recette}


\section{Pommes sarladaise au four}
\begin{recette}{Pommes sarladaise au four}{4}{}{}\index{pomme de terre}
\begin{ingredients}
\ingredient 2.5kg de pomme de terre
\ingredient sel, poivre, graisse de canard
\ingredient 1/2 oignon
\ingredient 1 gousse d'ail
\ingredient un sac de congélation
\end{ingredients}

\begin{preparation}
\etape Mixez l'ail et l'oignon et versez dans un grand saladier
\etape Ajoutez environ une louche de graisse de canard (pas plus)
\etape Coupez les pommes de terre en potatoes (grosses frites)
\etape Salez abondamment et poivrez puis mélangez
\etape Mélangez le tout
\end{preparation}

\begin{cuisson}
Faites préchauffer le four à 200\degres C 10 minutes environ, puis enfournez les 45 minutes à une heure en les disposant dans un grand plat 
à tarte.

Si les pommes de terre sont fermes ou farineuses, faites cuire un peu plus jusqu'à ce qu'elles soient moelleuse (1h puis laisser dans le four éteint pendant 30 minutes, c'est à peu près ce que j'ai fait la dernière fois)
\end{cuisson}
\end{recette}

\section{Pommes de terre vinaigrette}
\begin{recette}{Pommes de terre vinaigrette}{2}{}{}\index{pomme de terre}
\begin{ingredients}
\ingredient 2 grosses pommes de terre par personne
\ingredient 1 échalote hachée
\ingredient 2 cuillères à soupe de moutarde
\ingredient 4 cuillères à soupe d'huile d'olive
\ingredient 2 cuillères à soupe de vinaigre de vin
\ingredient ciboulette
\ingredient sel, poivre
\end{ingredients}

\begin{preparation}
\etape Faire bouillir une grande casserole d'eau. Eplucher les pommes de terre et les couper en morceaux. Jeter les pommes de 
terre dans l'eau bouillante et les faire pendant au moins 25mn (plus selon la taille des morceaux). Bien vérifier que les 
morceaux soient cuits au centre.

\etape Égoutter et faire refroidir les pommes de terre. Préparer la vinaigrette en mélangeant l'échalote hachée, la moutarde, 
l'huile d'olive et le vinaigre. Saler, poivrer.

\etape Dans un saladier, mélanger la sauce et les pommes de terre et rectifier l'assaisonnement si nécessaire.

\begin{remarque}
Mettre au frais si vous préférez la salade de pommes de terre froide que chaude.
\end{remarque}

\etape Au moment de servir parsemer de ciboulette.
\end{preparation}
\end{recette}

\section{Riz}
\begin{recette}{Riz}{2}{}{}\index{riz}\label{sec:riz}
\begin{ingredients}
\ingredient 600g de riz
\ingredient 720g d'eau (800g si à la casserole et non au cuiseur ; autre manière de faire, c'est même volume d'eau que de riz)
% historiquement je faisais 800g d'eau, puis en voyant une vidéo, j'ai tenté 750 et c'était mieux. Maintenant j'ai vu deux autres sources qui estiment à 720 donc il faut que j'essaie. Je soupçonne qu'il faut une marge fixe d'eau pour compter l'évaporation pendant cuisson et donc plus on fait de riz et moins on a besoin de mettre d'eau car toute la proportion d'eau n'est pas destinée au riz (en gros, c'est linéaire mais il y a une abscisse à l'origine
\ingredient 10g de sel
\end{ingredients}

\begin{preparation}
\etape Mesurez le volume de riz et faites le tremper dans de l'eau froide pendant 30 minutes
\etape égouttez le, puis rajoutez le sel et le volume d'eau pour la cuisson
\etape Portez à ébullition (sans forcément couvrir, ça prend entre 2 et 5 minutes en fonction de votre feu).
\etape Faites alors cuire pendant 15 minutes à couvert et à feu moyen (4/9)
\etape Eteignez le feu et laissez reposer hors du feu, toujours couvert sans ouvrir, pendant 10-15 minutes (pour absorption complete de l'eau)
\end{preparation}
\end{recette}

\section{Semoule}
\begin{recette}{Semoule}{2}{10 min}{}\index{semoule}
\begin{ingredients}
\ingredient semoule de blé dur (compter 60g par personne si accompagnement, sinon 100g)
\ingredient eau (un peu moins du volume de semoule. Par exemple pour 275ml de semoule, compter 250ml d'eau)
\ingredient huile d'olive, sel
\end{ingredients}

\begin{preparation}
\etape À l'aide d'un verre doseur, choisissez une quantité de semoule (60g par exemple)
\etape Ajoutez un filet d'huile à la semoule et mélangez bien afin que l'huile soit répartie autour des grains
\etape Faites bouillir une quantité d'eau égale au volume de la semoule que vous souhaitez faire cuire (par exemple, pour 100g 
de semoule, c'est environ $12.5\unit{cl}$. 
\begin{remarque}
Je fais chauffer l'eau au micro onde pour ma part. C'est rapide et même si je ne vois pas les bulles, l'eau est quand même bien 
chaude. 
\end{remarque}
\etape Ajoutez alors l'eau avec la semoule, et remuez bien avec une fourchette afin que ce soit homogène. Laissez à couvert 
(dans un bol avec une assiette par exemple) pendant 5 à 10 minutes.
\etape Égrénez enfin la semoule avec une fourchette afin de bien séparer les grains.
\end{preparation}
\end{recette}


}% End of the ``group'' where section is deactivated


\chapter{Sauces}
\minitoc

% Beginning of group where section is deactivated
% This is only to get the good structure of the document 
% since ``section'' is in fact embedded in the 'recette' environment.
% This group allow us to deactivate sections ONLY in the given file and 
% not for the entire document.
{\renewcommand{\section}[1]{}

\section{Beurre de morue}
\begin{recette}{Beurre de morue}{3}{10 min}{}\index{beurre de morue}
\begin{ingredients}
\ingredient 200g de miettes de morue dessalée
\ingredient 1/2 oignon
\ingredient 2 gousses d'ail
\ingredient un peu de jus de citron
\ingredient huile de tournesol
\ingredient idée épices: persil, échalotte, piment végétarien, piment
\end{ingredients}


\begin{preparation}
\etape Mixer la morue et l'oignon séparément
\etape Dans le mixer, mettre un peu de morue, d'oignon et d'huile et de jus de citron et l'ail, puis mixer de nouveau
\etape Rajouter de la morue et de l'huile au fur et à mesure. 
\etape ajouter du citron et des épices
\begin{remarque}
Le principe est sensiblement le même que pour la mayonnaise, raison pour laquelle il faut ajouter un peu d'huile quand on rajoute la morue
\end{remarque}

\end{preparation}
\end{recette}

\section{Beurre persillé}\label{sec:beurre_persil}
\begin{recette}{Beurre persillé}{3}{10 min}{}\index{beurre persillé}
\begin{ingredients}
\ingredient 300g de beurre doux (si demi sel, enlevez 2g de sel pour 100g)
\ingredient 40g d'échalottes
\ingredient 40g de mie de pain
\ingredient 20g d'ail (4 gousses)
\ingredient 40g de persil plat
\ingredient 9g de sel
\end{ingredients}


\begin{preparation}
\etape Sortez le beurre à l'avance pour qu'il soit mou
\etape Mixez finement ail et persil et échalottes
\etape Mélangez au robot à la feuille le beurre pour le mettre en pommade, puis ajoutez le sel, l'ail et le persil
\etape Si besoin, remixez le tout pour que l'ail et le persil soient vraiment fin (au hachoir c'était pas assez fin mixé seul)
\etape Vous pouvez ensuite congeler sous forme de boudin pour couper la quantité voulue sans avoir à tout décongeler d'un coup.
\end{preparation}
\end{recette}

\section{Guacamole}
\begin{recette}{Guacamole}{3}{10 min}{1h}\index{guacamole}
\begin{ingredients}
\ingredient 1 avocat
\ingredient 1 citron
\ingredient 1 gousse d'ail
\ingredient 2 cuillères à café de mélange d'épice pour Guacamole (voir \refsec{sec:epice_guacamole})
\end{ingredients}

\begin{preparation}
\etape Coupez l'avocat en morceaux et citronnez légèrement
\etape Ajoutez dans le bol du mixer l'ail et le mélange d'épice. 
\etape Réservez au frais 1h
\end{preparation}
\end{recette}

\section{Mayonnaise}
\begin{recette}{Mayonnaise}{3}{10 min}{}\label{sec:mayonnaise}\index{mayonnaise}
\begin{ingredients}
\ingredient 1 jaune d'oeuf
\ingredient 33 cl d'huile de tournesol
\ingredient 1 cuillère à café de moutarde
\ingredient 1/2 cac de vinaigre balsamique
\ingredient 2 petites gousses d'ail mixées
\ingredient sel
\end{ingredients}

\begin{remarque}
\begin{itemize}
\item Si la mayonnaise ne monte pas, c'est très certainement parce qu'il y a trop d'huile. Afin de la récupérer, prenez un nouveau récipient, mettez une petite partie de la mayonnaise ratée et ajoutez une cuillère à soupe d'eau bouillante. Mixez le tout puis rajouter petit à petit la mayonnaise ratée. 
\item le jaune d'oeuf doit être de préférence à température ambiante ;
\item il faut fouetter sans cesse pour que l'émulsion ne retombe pas ;
\item enfin les \og agréments\fg s'ajoutent toujours à la fin.
\item Pour la conserver : couvrir d'un film alimentaire \og au contact\fg, il doit toucher la surface de la mayonnaise. La mayonnaise maison se conserve 1 à 2 jours maximum.
\end{itemize}
\end{remarque}

\begin{preparation}
\etape Mélanger le jaune d'oeuf et la moutarde, sel et poivre. 
\etape Mettez le robot en vitesse 4 avec le fouet. Mettez l'huile goutte par goutte, en attendant pour que l'émulsion se fasse
\etape Augmentez ensuite à vitesse 6 et en petit filet.  Il ne faut pas ajouter l'huile trop vite et en trop grande quantité 
sous peine de noyer la mayonnaise et de casser l'émulsion. 
\etape Enfin incorporer le vinaigre, l'ail mixé et ajuster l'assaisonnement.
\end{preparation}
\end{recette}

\section{Sauce maison}
\begin{recette}{Sauce maison}{2}{}{}\index{floc de gascogne}\index{sauce}
\begin{ingredients}
\ingredient 150g de champignons de Paris
\ingredient 100g de lardons
\ingredient 2 oignons
\ingredient 25cl de bouillon de volaille
\ingredient 20cl de floc de gascogne
\ingredient 1 cuillère à soupe de farine
\ingredient sel, poivre, herbes de provence
\end{ingredients}

\begin{preparation}
\etape Faites revenir les lardons
\etape Réservez-les puis faites revenir les lardons
\etape Réservez l'oignon et faites revenir les champignons en rajoutant de l'huile au besoin.
\begin{remarque}
Ne couvrez pas pendant la cuisson des champignons afin d'éviter qu'il y ait trop de jus dans la sauteuse
\end{remarque}
\etape Rajoutez les oignons et lardons, réchauffez un peu la mixture et ajoutez la farine en saupoudrant. Mélangez bien le tout avant de rajouter le floc puis le bouillon de volaille.
\etape Mélangez bien le tout, couvrez et laissez cuire à feu doux jusqu'à ce que la sauce soit bien liée. Typiquement, c'était bon au bout de 5 à 10 minutes pour moi, mais ça dépend de la quantité de sauce que vous voulez. Si vous faites cuire longtemps, vous risquez de devoir mélanger de temps en temps pour ne pas que ça accroche.
\end{preparation}
\end{recette}

\section{Sauce makis}
\begin{recette}{Sauce makis}{2}{}{}\index{makis}\index{sauce}\index{sauce soja}\index{vinaigre de riz}
% sauce soja sucrée: Pour 100mL, 57g de sucre, 10g de sel.
\begin{ingredients}
\ingredient 80g de sauce soja
\ingredient 34g de vinaigre de riz
\ingredient 20g d'eau
% dans la recette originale c'était 125g de soja pour 40g de sucre
\ingredient 80g de sucre 
\ingredient 5g de farine (1 cuillère à café)
\end{ingredients}

\begin{preparation}
\etape Mélangez les ingrédients pour diluer la farine et faites cuire à feu moyen (3/9) pendant 6 minutes
\end{preparation}
\end{recette}

\section{Sauce marchand de vin}
\begin{recette}{Sauce marchand de vin}{3}{1h}{}\index{sauce marchand de vin}\index{marchand de vin}\index{vin}
\begin{ingredients}
\ingredient 4 échalottes
\ingredient 25 cl de vin rouge
\ingredient 50g de beurre
\ingredient 1 cuillère à soupe de farine
\ingredient 12.5 cl de bouillon (ou fond de viande)
\ingredient sel, poivre
\end{ingredients}

\begin{preparation}
\etape Pelez les échalottes et ciselez-les finement
\etape Faites les blondir dans une petite sauteuse ou une casserole avec un peu de beurre et d'huile, juste le temps qu'elles deviennent translucide
\etape Une fois raisonnablement dorées, saupoudrez la farine et incorporez la aux échalottes.
\etape Ajoutez le vin rouge, mélangez jusqu'à ce que la préparation soit homogène et faites réduire de moitié environ (il faut que la majeure partie du vin et de l'alcool soit évaporée) ce qui prendra environ 15 à 20 minutes à feux doux/moyen. 
\etape Ajoutez alors le beurre, par noisettes, et mélangez bien, sur feux doux. 
\etape Versez enfin le fond de viande, poivrez et laissez mijoter 10 minutes environ, sur feux doux et à couvert. 
\end{preparation}

\begin{remarque}
Vers la fin, faites cuire une viande, pourquoi pas au barbecue, une pièce de bœuf, entrecote de préférence, et mangez ça avec des pommes de terre rissolées ou sarladaises. En tout fin de cuisson il est possible de rajouter un peu de persil haché.
\end{remarque}
\end{recette}

\section{Sauce aux cèpes}
\begin{recette}{Sauce aux cèpes}{3}{30 min}{}\index{cepes}
\begin{ingredients}
\ingredient 300g de cèpes
\ingredient 2 échalottes
\ingredient 20cl de crème fraiche (ou liquide)
\ingredient 1 cuillère à soupe de cognac
\end{ingredients}


\begin{preparation}
\etape Nettoyez les cèpes, détachez les têtes des queues. Coupez les têtes en tranches d'1cm environ
\etape Émincez les échalottes et les queues de cèpes finement
\etape Faites chauffer de l'huile dans une poêle. Mettez-y les tranches de cèpes et faites les rissoler sur feu vif (7/9) en les retournant. Continuez la cuisson sur feu moins vif (5/9) 5 ou 6 minutes puis terminez à feu ardent pour raffermir les tranches.
\etape Égouttez à fond l'huile, assaisonnez de sel et poivre et réservez dans une casserole couverte.
\etape Remettez un peu d'huile et le beurre. Dès qu'ils fument ajoutez le hachis de cèpes et d'échalote, sautez sur feu vif jusqu'à rissolage. Versez alors dans la casserole
\etape Déglacez la poële avec un peu d'eau pour récupérer les sucs puis versez dans la casserole
\etape Ajoutez alors la crème dans la casserole et 1 cuillère à soupe de cognac. Couvrez et faites cuire 5 à 6 minutes à feu moyen (4/9)
\end{preparation}
\end{recette}

\section{Tapenade}
\begin{recette}{Tapenade}{4}{20 min.}{}\index{Tapenade}
\begin{ingredients}
\ingredient 160g olive noire
\ingredient 5 filets d'anchois
\ingredient 5-8 câpres
\ingredient 1 gousse d'ail
\ingredient 10cl d'huile d'olive
\end{ingredients}

\begin{preparation}
\etape Mixez le tout
\end{preparation}
\end{recette}

\section{Vinaigrette classique}
\begin{recette}{Vinaigrette classique}{3}{5 min}{}\label{sec:vinaigrette}\index{vinaigrette}
\begin{ingredients}
\ingredient 60g (4 cas) d'huile
\ingredient 30g (2 cas) de vinaigre balsamique (ou de xeres)
\ingredient 2.5g (0.5 cac) de sel
\ingredient une bonne pincée de poivre (ou 2 ou 3 tours de moulin)
\ingredient Éventuellement des herbes
\end{ingredients}

\begin{remarque}
Avec un ratio de 3 cas d'huile et d'1 cas de vinaigre balsamique, les proportions sont idéales pour que l'émulsion soit 
``solide'', mais maintenant je met moins d'huile
\end{remarque}


\begin{preparation}
\etape Dans un pot en verre (à confiture par ex) ou un petit tupperware, mettez le sel et le poivre. Ajoutez-y le vinaigre et secouez jusqu'à ce que le sel soit dissous.
\etape Ajoutez alors les herbes de provence et tous les condiments que vous souhaitez ajouter. 
\etape Incorporez l'huile ensuite puis secouez jusqu'à ce que la vinaigrette soit homogène.
\end{preparation}

\begin{remarque}
On peut faire une vinaigrette à l'ail en mettant une gousse d'ail avec le sel et le poivre.

On peut remplacer le vinaigre par du jus de citron.
\end{remarque}
\end{recette}

\section{Vinaigrette à la moutarde}
\begin{recette}{Vinaigrette à la moutarde}{3}{5 min}{}\label{sec:vinaigrette-moutarde}\index{vinaigrette}\index{moutarde}
\begin{ingredients}
\ingredient 6 cuillères à soupe d'huile
\ingredient 2 cuillères à soupe de vinaigre balsamique
\ingredient 1 demi-cuillère à café de moutarde
\ingredient 1 demi-cuillère à café rase de sel
\end{ingredients}

\begin{preparation}
\etape Dans un pot en verre (à confiture par ex) ou un petit tupperware, mettez le sel et le poivre. Ajoutez-y le vinaigre et secouez jusqu'à ce que le sel soit dissous.
\etape Ajoutez alors la moutarde, les herbes de provence et tous les condiments que vous souhaitez ajouter. 
\etape Ajoutez ensuite l'huile puis secouez jusqu'à ce que la moutarde soit diluée et que la vinaigrette soit homogène.
\end{preparation}
\end{recette}


\section{Vinaigrette à l'ail}
\begin{recette}{Vinaigrette à l'ail}{3}{5 min}{}\label{sec:vinaigrette-ail}\index{vinaigrette}\index{ail}
\begin{ingredients}
\ingredient 40g d'ail ($\sim 6$ gousses)
\ingredient 200g d'huile d'olive
\ingredient 100g de vinaigre
% Jacqueline met du vinaigre de citron mais je n'aime pas
%\ingredient 60g de moutarde  % dose etchebest
% Avant une cuillère à soupe bombée de moutarde mais je trouve que c'est trop
\ingredient 5g de sel
%\begin{remarque}
%Les doses ne sont pas bonnes parce que Jacqueline a ensuite rajouté des trucs un peu au pif, mais ça donne une base. Les doses d'huile et de vinaigre doivent être beaucoup plus importantes que ça je pense
%\end{remarque}
%vinaigrette jacqueline:
%\ingredient 6 gousses d'ail
%\ingredient 270 mL d'huile de tournesol
%\ingredient 100 mL de vinaigre de citron
%\ingredient 30g de moutarde  % dose etchebest c'est 60g de moutarde pour 300g d'huile
% Avant une cuillère à soupe bombée de moutarde mais je trouve que c'est trop
%\ingredient 1 cuillère à café rase de sel
\end{ingredients}

\begin{preparation}
\etape Mettez tout dans le bol d'un mixeur, puis mixez le tout jusqu'à ce que ce soit homogène et que l'ail soit bien mixé.
\etape A conserver au frais ensuite.
\end{preparation}
\end{recette}

\section{Vinaigrette au soja}
\begin{recette}{Vinaigrette au soja}{3}{5 min}{}\index{vinaigrette}\index{sauce soja}
% J'ai fait cette vinaigrette à partie de la vinaigrette huile de sésame et sauce soja de Maille. 
% J'ai considéré que le sel venait exclusivement de la sauce soja, raison pour laquelle j'ai réduit la quantité. J'ai déduit l'apport de sucre par la sauce et j'ai complété par du sucre en poudre. J'ai considéré, en me basant sur la recette, que les doses originales de la vinaigrette, c'était 75% huile et 25% de vinaigre. J'ai complété par de l'eau pour faire 100mL. 
% Les pourcentages ont été fait en millilitres et donnent:
% Oil: 41 mL 35of oil and 6 ml of sesame oil
% Soy: 12 mL
% Vinegar: 25 mL
% Water: 22 mL
% , puis converti en poids à partir des densité suivantes :
% Oil: 0.9 g/mL
% Soy: 1.12 g/mL
% Vinegar: 1.01 g/mL
% Water: 1 g/mL
% Et enfin, les masses suivantes:
% Oil: 36.9 g
% Soy: 13.440000000000001 g
% Vinegar: 25.25 g
% Water: 22.0 g
\begin{ingredients}[40cL de vinaigrette]
\ingredient 148g d'huile de tournesol
\ingredient 52g de sauce soja
\ingredient 100g de vinaigre de vin
\ingredient 88g d'eau
\ingredient 20g de sucre
\ingredient 10g de graines de sésame torréfiées
\end{ingredients}

\begin{preparation}
\etape Pesez et mélangez la totalité des ingrédients
\end{preparation}
\end{recette}

}% End of the ``group'' where section is deactivated

\chapter{Mélange d'épices}
\minitoc

% Beginning of group where section is deactivated
% This is only to get the good structure of the document 
% since ``section'' is in fact embedded in the 'recette' environment.
% This group allow us to deactivate sections ONLY in the given file and 
% not for the entire document.
{\renewcommand{\section}[1]{}

\section{Garam Masala}
\begin{recette}{Garam Masala}{3}{15 min}{}\label{sec:garam_masala}\index{garam masala}
\begin{ingredients}
\ingredient 5g de cumin
\ingredient 2.5g de coriandre
\ingredient 7.5g de cardamone
\ingredient 7.5g de poivre noir
\ingredient 5g de canelle
\ingredient 2.5g de clous de girofle
\ingredient 2.5g de muscade
\end{ingredients}

\begin{preparation}
\etape Pesez tout et mixez dans un moulin à café. Dans mon cas, j'ai mixé la muscade, le poivre, les graines de coriandre et la 
muscade
\end{preparation}
\end{recette}

\section{Sel Caraïbe}
\begin{recette}{Sel Caraïbe}{}{15 min}{}
\begin{ingredients}
\ingredient 200g de sel
\ingredient 43g de paprika
\ingredient 5g d'herbe de provence
\ingredient 21g de poudre d'oignon (frit? peut-être meilleur avec celui là)
\ingredient poudre de tomate séché (j'ai pas trouvé ni utilisé, mais je pense que c'est un ingrédient)
\ingredient 20g d'ail en poudre
\ingredient 4g de sucre
\ingredient 20g de graine de moutarde mixé
\ingredient 10g de curcuma
\ingredient 6g de celeri
\end{ingredients}
% sel, sucre, paprika, oignon, tomate, piment, herbe de provence

\begin{preparation}
\etape Pesez tout et mélangez
\end{preparation}
\end{recette}

\section{Épices pour Guacamole}
\begin{recette}{Épices pour Guacamole}{3}{15 min}{}\label{sec:epice_guacamole}\index{guacamole}
\begin{ingredients}
\ingredient 5g de paprika
\ingredient 5g de coriandre sèche
\ingredient 5g de persil séché
\ingredient 5g de muscade
\ingredient 5g de cumin
\ingredient 2.5g de sel

\end{ingredients}

\begin{preparation}
\etape Pesez tout et mixez dans un moulin à café. 
\etape Comptez 1 à 2 cuillères à café de ce mélange avec un avocat (et une gousse d'ail mixée)
\end{preparation}
\end{recette}

\section{Piment confit}
\begin{recette}{Piment confit}{3}{15 min}{}\index{piment fort}
\begin{ingredients}
\ingredient 5 piments forts
\ingredient un petit oignon
\ingredient une petite carotte
\ingredient 2 clous de girofles
\end{ingredients}

\begin{preparation}
\etape Emincez finement l'oignon et coupez la carotte en petits dés
\etape En prenant vos précautions (gant latex notamment), lavez les piments à l'eau froide, enlevez la tige et les pépins puis émincez finement
\etape Mélangez alors le tout et ajouter de l'huile de tournesol
\etape Utilisez l'huile pour pimenter vos préparations. A conserver au frais.
\end{preparation}
\end{recette}

\section{Pate d'ail}
\begin{recette}{Pate d'ail}{3}{15 min}{}\index{ail}
\begin{ingredients}
\ingredient 100g de gousse d'ail pelées
\ingredient 6.5g de sel
\ingredient 3g de jus de citron
\end{ingredients}

\begin{preparation}
\etape J'ai piqué les doses sur un pot de pulpe d'ail. Est indiqué que ça se conserve un mois au frigo après ouverture. 
\end{preparation}
\end{recette}

\section{Purée de piment fort}
\begin{recette}{Purée de piment fort}{3}{15 min}{}\index{piment fort}
\begin{ingredients}
\ingredient 100g de piment fort Guadeloupéen entier (habanero)
\ingredient 2 oignons
\ingredient 10cl d'huile
\ingredient 2 cuillères à soupe de vinaigre
\ingredient 1 cuillère à soupe de moutarde
\end{ingredients}

\begin{preparation}
\etape Enlevez la queue des piments
\begin{remarque}
Manipulez le piment avec précaution, utilisez des gants en latex si nécessaire et ne vous grattez pas (en particulier les yeux)
\end{remarque}
\etape Mixez les piments entier avec les oignons et le reste. 
\etape A conserver au frais. Cela se garde pendant un an sans problème. Utilisez une pointe de couteau de purée de piment par assiette au maximum, c'est extrêmement fort.
\end{preparation}
\end{recette}

}% End of the ``group'' where section is deactivated
\chapter{Entrée}
\minitoc

% Beginning of group where section is deactivated
% This is only to get the good structure of the document 
% since ``section'' is in fact embedded in the 'recette' environment.
% This group allow us to deactivate sections ONLY in the given file and 
% not for the entire document.
{\renewcommand{\section}[1]{}


\section{Coca (tarte aux poivrons)}
\begin{recette}{Coca (tarte aux poivrons)}{4}{20 min.}{30 min.}\index{poivron}\index{pâte feuilletée}
\begin{ingredients}
\ingredient Une pâte feuilleté
\ingredient 1kg de poivrons
\ingredient 3 oignons
\ingredient 100g de fromage rapé
\end{ingredients}

\begin{preparation}
\etape Emincez oignons et poivrons séparémment
\etape Faites revenir les oignons jusqu'à ce qu'ils commencent à dorer
\etape Préchauffez le four à 200°C
\etape Ajoutez alors les poivrons et faites revenir jusqu'à ce que ça commence à accrocher (sur une poële antiadhésive)
\etape étalez la préparation sur la pâte feuilletée.
\etape Ajoutez du fromage rapé par dessus
\end{preparation}

\begin{cuisson}
Faites cuire 30 minutes à 200°C
\end{cuisson}
\end{recette}

\section{Gratin de moules}
\begin{recette}{Gratin de moules}{4}{1h}{}\index{beurre persillé}\index{moule}
\begin{ingredients}[2 personnes]
\ingredient 300g de moules (prévoir plutôt des grosses moules)
\ingredient 50g de fromage rapé
\ingredient 100g de beurre persillé (voir \refsec{sec:beurre_persil})
\end{ingredients}

\begin{preparation}
\etape Dans une casserole, portez un fond d'eau (ou de vin blanc sec) à ébullition
\etape Ajoutez les moules, couvrez et laissez cuire à feu vif pendant 2 minutes pour ouvrir les moules
\etape Ouvrez les moules et ne gardez qu'un seul coté de coquille, avec la moule dedans de sorte à pouvoir les mettre à plat dans le plat ensuite

\etape Mettez le four à chauffer en grill à 270°C
\etape Répartissez le beurre persillé dans les coquilles de moule et disposez les dans un plat allant au four (un plat à escargot est le mieux, pour maintenir les moules en place)
\etape Ajoutez du fromage rapé
\end{preparation}
\begin{cuisson}
Enfournez les moules 4-5 minutes en grill.
\end{cuisson}
\end{recette}

\section{Œufs mimosa}
\begin{recette}{Œufs mimosa}{3}{1h}{20 min}\index{œufs}\index{mimosa}
\begin{ingredients}
\ingredient 6 œufs
\ingredient une boite de thon
\ingredient mayonnaise (maison \refsec{sec:mayonnaise} ou pas)
\ingredient un peu de vinaigre
\end{ingredients}

\begin{preparation}
\etape Portez de l'eau à ébullition dans une casserole
\etape Ajoutez un peu de vinaigre (pour que le blanc coagule si jamais il y a une coquille qui se casse) puis posez délicatement au fond de la casserole les œufs, un par un, à l'aide d'une cuillère à soupe (ou équivalent)
\etape Laissez cuire les œufs pendant 10 minutes dans l'eau bouillante
\etape sortez les œufs et passez les sous l'eau froide (ça permet de décoller la pellicule sous la coquille plus facilement
\etape Mélangez la mayonnaise et le thon dans un bol. 
\etape Pelez les œufs durs et enlevez le jaune
\etape Écrasez les jaunes à la fourchette, mettez en la moitié avec la mayonnaise après avoir préalablement salé et poivré. Puis garnissez le blanc avec le mélange thon/mayonnaise
\etape saupoudrez le jaune d'œuf sur les œufs ainsi préparés (à l'aide d'une rape par exemple).
\end{preparation}
\end{recette}

\section{Maki californiens}
\begin{recette}{Maki californiens}{4}{1h}{}\index{maki}\index{saumon}\index{avocat}
\begin{ingredients}[2 personnes]
\ingredient 1 avocat
\ingredient 4 feuilles de nori
\ingredient 100g saumon frais (ou fumé)
\ingredient 300g de Riz (collant)
\ingredient Vinaigre de riz (ou vinaigre normal)
\ingredient 2 cuillères à soupe de sucre
\ingredient Sauce soja (pour la dégustation)
\end{ingredients}

\begin{preparation}
\etape Rincez le riz jusqu'à ce que l'eau de rinçage ne soit plus trouble
\etape Faites dissoudre le sucre dans une petite tasse de vinaigre.
\etape Plongez le riz dans 1.5 fois son volume d'eau, puis laissez cuire 15 minutes à partir de l'ebullition
\etape Laissez refroidir le riz quelques instants et incorporez-y le vinaigre. 
\etape Coupez en lamelle le saumon et les avocats
\etape Etalez un papier célophane (ou mieux, une natte de bambou si vous avez. Préparez un bol d'eau froide pour vous 
rincer les mains
\etape Posez la feuille d'algue dans le sens de la largeur. Recouvrez-la d'une couche de riz. Laissez un ou deux centimètres 
(pas moins !) non recouvert sur la largeur supérieure afin de pouvoir ensuite refermer aisément le rouleau.
\etape Placez dans le sens de la largueur une bande de saumon et d'avocat, aux 2/3 de la longueur par rapport à la bande de 
libre que vous avez réservés. 
\etape Enroulez l'algue en vous servant du célophane ou du rouleau de bambou en partant de la largeur NON libre. Humidifiez la 
bande laissée libre pour une fermeture plus facile. Placez au frais.
\etape Faites de même pour les autres rouleaux. 
\etape Avant de servir, découpez les rouleaux en lamelles de 2 à 3 cm, sans appuyer, et avec un couteau très tranchant.
\etape Dégustez enfin en trempant les tranches dans de la sauce soja si vous en avez. 
\end{preparation}
\end{recette}


\section{Patés à la viande}
\begin{recette}{Patés à la viande}{4}{20 min.}{30 min.}\index{chair à saucisse}\index{pâte feuilletée}
\begin{ingredients}[~14 patés par pâte]
\ingredient 2 pâtes feuilletés
\ingredient 300g de chair à saucisse
\ingredient 1 oignon
\ingredient cives
\ingredient 2 gousses d'ail
\ingredient sel, poivre
\end{ingredients}

\begin{preparation}
\etape Mélanger chair à saucisse avec oignon, ail et cive mixés. Salez et poivrez
\etape Faites revenir la farce à la poêle.
\etape Couper des cercles de pâte feuilletée. Pincer les côté, remplir de farce puis sceller en demi lune. 
\begin{remarque}
Une fois découpé les ronds sur une pâte, mettez là en boule, farinez là, et étalez là au rouleau pour refaire des ronds. Une fois fait, réservez les chutes de pâtes au frigo et refaite la même chose une fois que vous aurez écoulé toutes les pâtes. Si vous faites des boules trop grosses vous n'arriverez pas à faire une pâte fine au final.
\end{remarque}
\end{preparation}

\begin{cuisson}
Enfourner à 200 degrés pendant 13-15 minutes
\end{cuisson}
\end{recette}


\section{Roulés au Roquefort}
\begin{recette}{Roulés au Roquefort}{3}{20 min. + 1h + 5min. + 3h}{20 min.}\index{bleu}\index{roquefort}\index{roulés au roquefort}\index{pâte feuilletée}
\begin{ingredients}[15 petits roulés]
\ingredient Une pâte feuilleté (ne pas la sortir à l'avance)
\ingredient[variante roquefort]
\ingredient 100g de Roquefort
\ingredient 12.5cl de crème fraîche épaisse
\ingredient[variante pesto]
\ingredient pas testé, mais pesto et du fromage je crois, c'était super bon mais j'ai pas les doses)
\end{ingredients}

\begin{preparation}
\etape Mélangez les ingrédients dans une casserole jusqu'à obtention d'une préparation homogène et pas trop liquide.
\etape Laissez refroidir 1h puis étalez la préparation sur la pâte feuilletée.
\etape Roulez la pâte assez serrée pour obtenir un rouleau de 5cm de diamètre.
\etape Placez le rouleau au congélateur, au moins deux heures (sans le papier autour pour qu'il ne colle pas à la pâte). Il faut que ce soit bien congelé, afin que ça ne s'écrase pas quand vous coupez(je l'ai laissée une nuit).
\etape Une fois sorti du congélateur (laissez réchauffer quelques minutes pour que ça ne casse pas au besoin), coupez des rondelles d'un peu moins d'1cm d'épaisseur. Placez-les sur une plaque en les espaçant un peu (si les tranches sont fines, ça ne gonfle pas énormément).
\end{preparation}

\begin{cuisson}
Sortez le rouleau de pâte du congélateur puis coupez des tranches comprises entre 5mm et 1cm (plutôt 5mm). Si c'est bien congelé, et le couteau bien aiguisé, ça doit se couper sans trop d'effort et surtout sans s'écraser, pour laisser des tranches parfaites.

Cuire dans un four préchauffé à 220°C pendant 15 à 20 minutes
\end{cuisson}
\end{recette}

\section{Mini-pizzas}
\begin{recette}{Mini-pizzas}{4}{10 min.}{20 min.}\index{pâte feuilletée}
\begin{ingredients}
\ingredient Une pâte feuilleté
\ingredient 200g de sauce bolognaise
\ingredient 100g de fromage rapé
\end{ingredients}

\begin{preparation}
\etape Préchauffez le four à 200°C
\etape Disposez la pate feuilletée à plat (pas dans un plat à tarte donc)
\etape Piquez la pâte avec une fouchette. 
\etape étalez la bolognaise sur la pâte feuilletée en laissant une marge d'un peu moins d'un centimètre sur les bords.
\etape Ajoutez du fromage rapé par dessus
\end{preparation}

\begin{cuisson}
Faites cuire 18-20 minutes à 200°C. 
Une fois froid, découpez en petit carrés de 5cm de coté environ.
\end{cuisson}
\end{recette}

\section{Tarte aux champignons}
\begin{recette}{Tarte aux champignons}{4}{20 min.}{30 min.}\index{champignons}\index{pâte feuilletée}
\begin{ingredients}
\ingredient Une pâte feuilleté
\ingredient 250g de champignons
\ingredient 20cl de crème fraîche épaisse
\ingredient 200g de lardons
\ingredient 100g de fromage rapé
\end{ingredients}

\begin{preparation}
\etape Préchauffez le four à 200°C
\etape Faites revenir les lardons puis les champignons. 
\etape Hors du feu, ajoutez la crème fraiche
\etape étalez la préparation sur la pâte feuilletée.
\etape Ajoutez du fromage rapé par dessus
\end{preparation}

\begin{cuisson}
Faites cuire 30 minutes à 200°C
\end{cuisson}
\end{recette}

\section{Tarte aux poireaux}
\begin{recette}{Tarte aux poireaux}{4}{20 min.}{30 min.}\index{poireaux}\index{pâte feuilletée}
\begin{ingredients}
\ingredient Une pâte feuilleté
\ingredient 500g de poireaux
\ingredient 20cl de crème fraîche épaisse
\ingredient 100g de fromage rapé
\end{ingredients}

\begin{preparation}
\etape Préchauffez le four à 200°C
\etape émincez les poireaux
\etape Faites fondre 20g de beurre puis faites revenir les poireaux 2/3 minutes environ
\etape laissez cuire à feu doux (4/9) pendant 25 minutes
\etape Rajoutez la crème fraiche épaisse, le citron et le sel.
\etape étalez la préparation sur la pâte feuilletée.
\etape Ajoutez du fromage rapé par dessus
\end{preparation}

\begin{cuisson}
Faites cuire 30 minutes à 200°C
\end{cuisson}
\end{recette}

}% End of the ``group'' where section is deactivated
\chapter{Charcuterie}
\minitoc

% Beginning of group where section is deactivated
% This is only to get the good structure of the document 
% since ``section'' is in fact embedded in the 'recette' environment.
% This group allow us to deactivate sections ONLY in the given file and 
% not for the entire document.
{\renewcommand{\section}[1]{}

% https://www.tompress.com/A-10003889-la-saumure-conseils-de-preparation-et-d-utilisation.aspx
\section{Saumure}
\begin{recette}{Saumure}{}{20 min}{}\index{sel}
\begin{ingredients}
\ingredient 1L d'eau
\ingredient 357g de sel (maximum à 25°C, mais généralement, 200g suffisent)
\end{ingredients}

\begin{preparation}
\etape Faire une saumure permet un meilleur contact du sel avec la pièce à saler. C'est valable pour toutes les recettes dans le sel qui vont suivre. 
\end{preparation}
\end{recette}

\section{Jarret demi sel}
\begin{recette}{Jarret demi sel}{}{20 min}{3 jours}\index{porc}
\begin{ingredients}
\ingredient 1 jarret de porc frais
\ingredient Gros sel de guérande (500g pour un jarret?)
\end{ingredients}

\begin{preparation}
\etape Salez copieusement le jarret, disposez un lit de sel dans un plat, le magret coté chair, puis recouvrez de sel jusqu'à hauteur de graisse incluse. 
\etape Recouvrez d'un torchon et placez le 3 jours au réfrigérateur
\etape Au bout de 3 jours, sortez le sel, rincez, puis cuisinez rapidement (pour des lentilles par exemple).
\end{preparation}
\end{recette}

\section{Magret séché}
\begin{recette}{Magret séché}{4}{24h}{7 jours}\index{magret}\index{canard}\index{magret séché}
\begin{ingredients}
\ingredient 1 magret frais de canard
\ingredient 300g de gros sel de guérande
\ingredient 8g de sucre
\ingredient [optionnel] 10g cas de thé fumé (pour donner un gout de fumé) % Au depart j'ai mis le thé dans le sel et c'etait tres bon mais tres subtil.  Je vais essayer de le mettre apres pour voir.
\end{ingredients}

\begin{preparation}
\etape Mélangez le sel et le sucre. 
\etape Disposez un peu de ce mélange dans un moule à cake ou tout autre plat similaire. Déposez le magret coté chair (peau en haut) puis versez le reste du mélange jusqu'à hauteur de graisse incluse. 
\etape Recouvrez d'un torchon et placez le 24h au réfrigérateur
\etape Le lendemain, sortir tout le sel de la surface du magret. 
\etape Rincez sous l'eau fraiche puis épongez avec du papier absorbant.
\etape Poivrez le magret, mettez du thé fumé si vous le souhaitez, et enveloppez-le dans un torchon propre et placez le au réfrigérateur pendant 3 semaines environ.
\etape Vous pouvez ensuite servir ce magret séché comme de la charcuterie, en tranche très fine.
\end{preparation}
\end{recette}

\section{Saumon fumé}
\begin{recette}{Saumon fumé}{4}{24h}{7 jours}\index{saumon}\index{saumon fumé}
\begin{ingredients}
\ingredient 1 filet de saumon frais (300-500g)
\ingredient 25cl de Gros sel de guérande
\ingredient 12cl de sucre
\ingredient 5 cuillères à soupe de thé fumé grand lapsang souchong
\end{ingredients}

\begin{important}
 Ne pas laisser dans le sel plus de 24h et ne pas enlever la peau AVANT de saler, sinon le saumon fumé sera trop salé. 
\end{important}


\begin{preparation}
\etape Lavez et essuyez avec soin le filet de saumon
\etape Posez le filet dans une boite plate (tupperware). Placez la peau contre la plaque. 
\etape Mélangez le sucre, le sel et le thé. Répartir le mélange sur la face du saumon de manière homogène. 
\etape Couvrir d'un film étirable avant de poser le couvercle. 
\etape Gardez 24h au réfrigérateur. 
\etape Rincez le filet sous l'eau fraiche pour enlever les traces de sel et de thé. 
\etape Vous pouvez ensuite enlever la peau à l'aide d'un couteau, exactement comme vous enlèveriez le gras d'un magret. 
\etape À consommer ensuite dans les 15 jours.
\end{preparation}
\end{recette}

}% End of the ``group'' where section is deactivated

\chapter{Dessert}
\minitoc

% Beginning of group where section is deactivated
% This is only to get the good structure of the document 
% since ``section'' is in fact embedded in the 'recette' environment.
% This group allow us to deactivate sections ONLY in the given file and 
% not for the entire document.
{\renewcommand{\section}[1]{}

\section{Banana bread}
\begin{recette}{Banana bread}{0}{30min.}{1h}\index{banane}
\begin{ingredients}
\ingredient 2 cuillères à soupe de lait
\ingredient 2 bananes mures (noires ou quasiment)
\ingredient 160g de sucre
\ingredient 250g de farine
\ingredient 85g de beurre fondu tiède (ou mou)
\ingredient 2 oeufs
\ingredient 4g de bicarbonate (1/2 cuillère à café)
\ingredient 11g de levure chimique (1 sachet)
\ingredient 1 pincée de sel
\ingredient une cuillère à soupe de rhum
\end{ingredients}

\begin{preparation}
\etape Préchauffez le four à 165°C
\etape Réduisez les bananes en purée avec une fourchette
\etape Faites blanchir les oeufs et le sucre
\etape Pendant ce temps, préparez un saladier avec levure, bicarbonate, sel et farine
\etape Ajoutez alors le lait, le beurre fondu tiédi aux oeufs
\etape Ajoutez enfin la farine, puis la purée de banane
\etape Disposez dans un moule à cake préalablement fariné
\end{preparation}

\begin{cuisson}
Faites cuire au four pendant 55 minutes à 165°C. Contrôlez 
que c'est cuit en plantant le couteau, qui doit ressortir sans pate sur la lame.
\end{cuisson}
\end{recette}

%https://tangerinezest.com/banane-flambee-au-rhum/
%https://www.youtube.com/watch?v=ddkCaxSQRo8
%j'essaie de refaire une recette de restaurant ou la banane avait une sorte de sauce onctueuse. J'ai testé plusieurs recettes, dont une avec de la crème liquide en pensant que c'était l'ingrédient manquant mais je trouvais qu'il y avait trop de beurre et la sauce était pas crémeuse. La photo de cette recette ressemble le plus à ce qu'on a mangé, je pense que la clé est de bien faire cuire les bananes pour que ce soient elles qui fassent la sauce. 
\section{Banane flambée}
\begin{recette}{Banane flambée}{0}{30min.}{1h}\index{banane}
\begin{ingredients}
\ingredient 4 bananes mures (noires ou quasiment)
\ingredient 25g de beurre
\ingredient 50g de sucre
\ingredient 2 cuillère à soupe de rhum
\end{ingredients}

\begin{preparation}
\etape Dans une poêle, faites fondre du beurre à feu moyen. Épluchez et coupez les bananes en 2 dans le sens de la longueur puis déposez-les dans la poêle.
\etape Faites dorer sur chaque face. Ne pas passer à l'étape suivante si les bananes n'ont pas fait une sorte de jus
\etape Ajoutez le sucre et retournez de nouveau les bananes après 2 minutes. Augmentez un peu votre feu et laisser légèrement caraméliser.
\etape Versez le rhum sur le dessus des bananes et faites flamber.
\end{preparation}

\begin{cuisson}
Faites cuire au four pendant 55 minutes à 165°C. Contrôlez 
que c'est cuit en plantant le couteau, qui doit ressortir sans pate sur la lame.
\end{cuisson}
\end{recette}

\section{Banoffee pie}
\begin{recette}{Banoffee pie}{0}{}{}\index{banane}\index{confiture de lait}\index{banoffee pie}
\begin{ingredients}
\ingredient $70$ g de beurre
\ingredient $250$ g de gâteaux types palets breton (2 paquets)
\ingredient $30$ g de sucre
\ingredient $400$ g de lait concentré sucré (une boite moyenne)
\ingredient un peu de lait
\end{ingredients}

\begin{remarque}
\og Banoffee\fg est la contraction de banana et toffee (caramel).
\end{remarque}

\begin{preparation}
\etape Piler les gâteaux pour faire de la chapelure avec des morceaux moyens.

\begin{remarque}
Pour ma part, je tape dans le paquet de gâteaux sans même l'ouvrir pour économiser un torchon et mettre directement le concassé 
dans le plat. Ceci marche très bien pour des palets bretons par exemple.
\end{remarque}

\etape Le mélanger au sucre et y ajouter le beurre fondu\footnote{on peut faire fondre le beurre au micro onde}. Rajoutez un peu 
de lait (un fond de verre) pour lier.

\etape Ensuite, on étale le mélange au fond d'un plat et on met au frigo une demi journée (que ce soit froid et que ça durcisse 
en fait).

\etape Pour le dessus, on met à cuire au bain marie pendant 3h une boite de lait concentré sucré qui va devenir un peu 
caramélisé.

\etape On dispose des fruits au dessus de ce caramel, tranche de bananes ou selon le gout.

\etape On recouvre de chantilly.
\end{preparation}
\end{recette}

\section{Brownie au Chocolat}
\begin{recette}{Brownie au Chocolat}{3}{15 min}{35 min}\index{brownie}\index{chocolat}
\begin{ingredients}
\ingredient 3 œufs
\ingredient 100 g de sucre
\ingredient 120 g de farine
\ingredient 350 g de chocolat à dessert
\ingredient 150 g de beurre
\ingredient 11g (1 paquet) de levure
\ingredient \textbf{Option: } morceaux de noix
\end{ingredients}

\begin{preparation}
\etape Faites préchauffer le four à 150°C
\etape Faire fondre le chocolat et le beurre au bain marie (je met un saladier au dessus d'une casserole remplie d'eau chaude).
\etape Battre au fouet les œufs et le sucre
\etape Incorporer la farine et la levure
\etape Ajouter ensuite le beurre et le chocolat fondus.
\begin{remarque}
On peut alors rajouter des éclats de noix juste avant de mettre dans le plat
\end{remarque}

\end{preparation}

\begin{cuisson}
%avant c'était 35 minutes a 150 sans préchauffage mais la derniere fois c'etait trop cuit
Faites cuire au four pendant 25 minutes à 150°C. Contrôlez 
que c'est cuit en plantant le couteau, qui doit ressortir sans pate sur la lame.
Laisser tiédir avant de consommer. 
\end{cuisson}
\end{recette}

\section{Broyé du Poitou à la confiture}
\begin{recette}{Broyé du Poitou à la confiture}{3}{15 min}{15 min}\index{broyé du poitou}\index{confiture}\index{gâteau basque}
\begin{ingredients}
\ingredient 1 œuf
\ingredient 125 g de sucre
\ingredient 250 g de farine
\ingredient 125 g de beurre
\ingredient 11g (1 paquet) de levure
\ingredient 8g (1 sachet) de sucre vanillé
\ingredient 300g de confiture
\end{ingredients}

\begin{preparation}
\etape Préchauffer le four à 210°C
\etape Mélangez la totalité des ingrédients jusqu'à obtenir une boule de pâte style pâte brisée
\etape Séparez en deux boules, l'une légèrement plus grande que l'autre (celle du dessus a besoin d'un peu plus de pâte pour ne 
pas s'embêter.
\begin{remarque}
Pour étaler les pâtes, utilisez deux papier cuisson pour que ça n'accroche pas au rouleau à patisserie.\end{remarque}
\etape  Étalez la première pâte (la plus petite) puis conservez uniquement le papier cuisson du bas. Posez la main à plat sur la pâte, puis renversez là pour la déposer sur le moule, puis enlevez délicatement le papier cuisson pour ne pas casser la pâte.
\etape Ajoutez alors la totalité de la confiture. Même si ça peut sembler beaucoup, la pâte va gonfler. Ne mettez pas de confiture sur environ 1cm au bord du plat pour pouvoir joindre les deux pâtes
\etape Préparez alors la deuxième pâte et déposez là sur le moule de la même manière que pour la première pâte
\etape Pincez les deux pâtes sur les bords pour les sceller

\end{preparation}

\begin{cuisson}
Enfournez alors la préparation pendant 15 minutes à 210°C, le temps que le dessus soit très légèrement roussi.
\end{cuisson}
\end{recette}

\section{Cake}
\begin{recette}{Cake}{3}{20 min}{40 min}\index{cake}
\begin{ingredients}
\ingredient 3 œufs
\ingredient 150g beurre mou
\ingredient 150g de sucre en poudre
\ingredient 200g de farine
\ingredient 100g de raisins secs
\ingredient 120g de fruits confits
\ingredient cerises confites
\ingredient 5 cl de rhum (3 cuillères à soupe)
\ingredient 0.5 sachet de levure chimique (avec 1 complet, c'est un peu trop gonglé aéré pour un cake)
\ingredient 1 pincée de sel
\end{ingredients}

\begin{preparation}
\etape Versez les raisins et le rhum dans une casserole, chauffez sur feu doux. Hors du feu, flamber, couvrir d'un couvercle et 
laisser tiédir.
\etape Coupez les fruits confits en petits dés, et fariner les légèrement.
\etape Préchauffez le four à 210°C. Beurrez et farinez le moule à cake.
\etape Séparez le blanc du jaune de deux œufs et conservez blanc ET jaune.
\etape Montez les blancs en neige avec une pincée de sel, puis réservez.
\etape Mélangez le beurre et le sucre. 
\etape Quand le mélange est bien crémeux, ajoutez 1 œuf entier et les 2 jaunes d'œufs, en 
mélangeant vivement.
\etape Incorporez la farine et la levure.
\etape Ajoutez les fruits confits, (sauf les cerises), et les raisins avec le rhum de macération.
\etape Incorporez les blancs en neige à la pâte, sans craindre de les faire retomber.
\etape Versez la moitié de la pâte dans le moule, disposer les cerises confites, et verser le reste de la pâte.

\end{preparation}

\begin{cuisson}
Enfourner pour 40 minutes de cuisson environ à 180°C. 
%Dans la rotissoire, j'ai mis 40 minutes à 190°C, puis 10 et 10 minutes, mais il faut retourner le moule car la cuisson n'était pas uniforme. 
\end{cuisson}
\end{recette}


\section{Cake au chocolat}
\begin{recette}{Cake au chocolat}{0}{20 min}{40 min}\index{cake}
\begin{ingredients}
\ingredient 60g cacao en poudre
\ingredient 200g beurre
\ingredient 200g sucre en poudre
\ingredient 200g farine
\ingredient 4 oeuf
\ingredient 5g levure chimique
\ingredient 1 pincée de sel
\end{ingredients}

\begin{preparation}
\etape Découpez les fruits confits en très petits cubes et faites-les macérer dans le cointreau.
\etape Allumez le four th.7 (210°C).
\etape Tamisez ensemble le cacao, la farine et la levure.
\etape Dans le robot mélangeur, travaillez le beurre et le sucre en une pommade blanche et mousseuse.
\etape Ajoutez les oeufs un à un en mélangeant soigneusement avant d'ajouter le suivant.
\etape Incorporez la farine/cacao au mélange et continuez de travailler pour obtenir une pâte bien liée.
\etape Ajoutez les fruits confits macérés puis mélangez de nouveau.
\etape Beurrez et farinez un moule à cake de 25 cm de long.
\etape Remplissez de pâte au 3/4 de la hauteur car le cake doit bomber à la cuisson.
\etape Laissez cuire 35 min.
\etape Piquez le centre avec la lame d'un couteau pointu qui doit ressortir sèche. Sinon, baissez la chaleur du four th.6 (180°C) et laissez cuire quelques minutes de plus.
\etape Sortez le gâteau du four et attendez 5 min avant de le démouler puis laissez-le refroidir complètement sur une grille.
\etape Vous pouvez le servir accompagné d'une crème anglaise.


\end{preparation}

\begin{cuisson}
https://www.cuisineaz.com/recettes/cake-au-cacao-et-fruits-confits-4306.aspx
\end{cuisson}
\end{recette}

\section{Carrot Cake}
\begin{recette}{Carrot Cake}{3}{20 min}{35 min}\index{carrot cake}\index{carotte}
\begin{ingredients}
\ingredient[Pour le gâteau]
\ingredient 200g de carotte
\ingredient 3 œufs
\ingredient 100g de farine
\ingredient 100g de sucre
\ingredient Un sachet de levure chimique
\ingredient 1 cuillère à café de cannelle moulue
\ingredient Extrait de vanille
\ingredient Quelques noix
\ingredient[Pour le glaçage]
\ingredient 1 blanc d'œuf (rajoutez le jaune dans le gâteau)
\ingredient 200g de sucre glace (le sucre normal ne fonctionne pas bien, trop gros grains)
\end{ingredients}

\begin{preparation}
\etape Préchauffer le four à 180°C (thermostat 6).
\etape Mixez les carottes et les noix afin d'obtenir de tout petits morceaux.
\etape Mélanger les oeufs, la farine, la levure, le sucre, les épices, finir avec les carottes et les noix.
\etape Bien mélanger.
\etape Versez dans un moule à gâteau.
\end{preparation}

\begin{cuisson}
Faites cuire 35 minutes à 180°C (thermostat 6). La lame d'un couteau plantée au centre du gâteau doit ressortir sèche.

Battez avec un batteur électrique le blanc d'œuf et le sucre du glaçage à vitesse maximum jusqu'à obtenir une pâte blanche (la 
pâte va 
sécher et durcir sur le gâteau, pas besoin de battre plus d'une minute, ça se forme très rapidement).

Mettre au frais quelques heures.
\end{cuisson}
\end{recette}

\section{Cannelés}
\begin{recette}{Cannelés}{0}{15 min+une nuit}{1h15}\index{cannelés}
\begin{ingredients}[30 gros (un peu moins de 2L de pâte)]
\ingredient 900g de lait 
\ingredient 44+10g de beurre salé ou [normal + pincée de sel](les 10g sont pour beurrer les moules)
\ingredient 410g de sucre
\ingredient 247g de farine T45
\ingredient 3 œufs + 2 jaunes + 0.5 blanc (la moitié d'un blanc d'un oeuf)
\ingredient 3 cuillère à soupe de rhum
\ingredient vanille
\end{ingredients}
% http://www.lacuisinedebernard.com/2010/01/voila-mes-declicieux-caneles-ils-vont.html
% 
% \begin{ingredients}[55 petits cannelés ou 17 gros (un peu plus d'1L de pâte)]
% \ingredient 50cl de lait (510g)
% \ingredient 25g de beurre (+15g pour beurrer les moules)
% \ingredient 232g de sucre
% \ingredient 140g de farine T45
% \ingredient 2 œufs + 1 jaune
% \ingredient 7.5cl de rhum (une cuillère à soupe)
% \ingredient vanille
% \end{ingredients}

\begin{preparation}
\etape Mettre à chauffer le lait la vanille et le beurre puis portez à ébullition (mélangez le tout au fouet quand c'est chaud. Il est possible que du beurre doux se mélange très mal par rapport au beurre salé)
puis faites refroidir 30 minutes avant de passer à la suite.
\etape Dans un saladier faire blanchir le sucre (au moins 5 minutes à vitesse 2) avec les œufs en utilisant le fouet (c'est mieux que la feuille), puis ajoutez la farine à la cuillère à soupe, petit à petit.
\etape Verser le lait tiédis à la louche pour éviter les grumeaux
\etape ajoutez enfin le rhum
\etape Laissez refroidir puis mettez au réfrigérateur une nuit.
\end{preparation}

\begin{cuisson}
Sortez la pâte du frigo au moins une heure avant pour qu'elle soit à température ambiante (cuisson ratée la dernière fois parce que la pâte était trop froide au bout d'une heure). 

\begin{remarque}
 On peut aussi mettre la pâte au bain marie dans de l'eau tiède dans l'évier. Une demi heure à ce traitement permet de réchauffer beaucoup plus vite. La cuisson était beaucoup mieux quand j'ai fait ça.
\end{remarque}


Ne pas trop secouer la pâte le jour J pour ne pas faire rentrer d'air dedans ((et éviter de trop faire gonfler les cannelés), remuez doucement avec une cuillère en bois.

Préchauffez le four à 275°C. 

Préparez les moules. Faire fondre un peu de beurre et tapissez les parois des moules de beurre. Le mieux c'est au 
doigt, sinon ça sera pas bien badigeonné et ça manquera de dorure à la cuisson (c'est très important pour bien les saisir).

Ré-homogénéisez la pâte avec une spatule (pas au fouet). Versez la dans les moules en les remplissant presque jusqu'au bord (ça gonfle pendant la cuisson, mais ça rediminue ensuite). 

Les temps de cuisson : 
\begin{itemize}
\item Pour des petits cannelés : 
Laissez cuire au max (300°C si vous avez, sinon 275°C à chaleur tournante) pendant 5 minutes. 

Puis baissez la température à 200°C sans la chaleur tournante pour 50 minutes de plus. 

\item Pour des grands cannelés : 
Laissez cuire au max (chaleur tournante) pendant 15 minutes, puis 36 minutes (ref=35) à 200°C (chaleur tournante aussi)
\end{itemize}

Laissez refroidir 30 minutes dans les moules avant de démouler (sinon ils se déforment et s'affaissent).
\end{cuisson}
\begin{remarque}
Les jours suivants, vous pouvez les conserver au frigo et les réchauffer 3 minutes à 220°C
\end{remarque}
\end{recette}



\section{Charlotte Aux Fraises}
\begin{recette}{Charlotte Aux Fraises}{0}{}{}\index{charlotte aux fraises}\index{fraise}
\begin{ingredients}[6 pers.]
\ingredient $500$ g de fraises (goûteuses et bien tendres), équeutées et coupées en morceaux
\ingredient $100$ g de fraises (les plus belles), pour la décoration
\ingredient $25$ cl de lait
\ingredient $35$ cl de crème fraîche liquide
\ingredient $1$ gousse de vanille
\ingredient 4 œufs
\ingredient $6$ cuil. à soupe de sucre
\ingredient $1$ cuil. à café de maïzena (ou de fécule)
\ingredient $4$ feuilles de gélatine (soit 8 g), mises à tremper dans de l'eau froide
\ingredient $1$ boîte de biscuits à la cuillère
\ingredient Chantilly en bombe pour la décoration
\end{ingredients}

\begin{preparation}
\etape[Commencez par confectionner la garniture de la charlotte]
\etape Portez à ébullition le lait et la gousse fendue, couvrez et laissez infuser.
\etape Fouettez les jaunes d'œufs avec 3 cuil. à soupe de sucre pour obtenir une mousse blanche.
\etape Incorporez la maïzena, puis le lait chaud.
\etape Remettez ensuite le tout dans la casserole et faites épaissir, sur feu doux, sans laisser bouillir.
\etape Retirez du feu, ajoutez la gélatine essorée, en mélangeant pour qu'elle se dissolve parfaitement, puis laissez légèrement 
tiédir.
\etape Fouettez la crème fraîche en chantilly, en ajoutant 3 cuil. à soupe de sucre en cours d'opération.
\etape Incorporez cette chantilly dans la crème à la vanille tièdie.

\etape[Enfin, procédez au montage de la charlotte]
\etape Tapissez un moule à charlotte de film plastique (ou d'aluminium).
\etape Mélangez les morceaux de fraises à la moitié de la crème de garniture.
\etape Versez la moitié de la garniture sans fraises dans le moule, couvrez d'une couche de biscuits, mettez la garniture aux 
fraises, puis des biscuits, le reste de garniture nature et terminez par une couche de biscuits.
\etape Posez une assiette sur la charlotte aux fraises et tassez légèrement.
\etape Laissez reposer la charlotte au réfrigérateur pendant au moins 4 heures.
\etape Au moment de servir, démoulez la charlotte aux fraises et décorez-la avec les fraises restantes et de la crème chantilly.
\end{preparation}
\end{recette}

\section{Christmas pudding à la moi}
\begin{recette}{Christmas pudding à la moi}{4}{1h+1 nuit}{8h}\index{christmas pudding}\index{noel}
\begin{ingredients}[4 personnes]
\ingredient 113g de beurre
\ingredient 55g de farine
\ingredient 4g de levure chimique
\ingredient 112g de biscotte broyée en chapelure
\ingredient 0.5g de noix de muscade
\ingredient 1g de cannelle
\ingredient 225g de sucre
\ingredient 350g de raisin sec (avant 400)
\ingredient 100g d'orange confite (en petits dés ou mixée, moi je mixe)
\ingredient 25g de poudre d'amande
\ingredient 60g de pomme mixée (~1/2)
\ingredient 2 oeufs
\ingredient 25g de rhum
\ingredient 75ml de vin rouge
\ingredient 75ml de guiness
\end{ingredients}

\begin{preparation}
\etape Mettez le beurre fondu, la farine, la levure, les biscottes mixées, le sucre et les épices (cannelle, noix de muscade) dans un bol et mélangez le tout à l'aide de la feuille. Ajoutez alors les fruits secs, la poudre d'amande, la pomme mixée et l'orange confite.

\etape Dans un autre bol, mélangez les oeufs, le rhum, le vin rouge et la guiness. Fouettez pour mélanger les oeufs puis ajoutez dans le bol du robot.

\etape Beurrez et farinez le récipient qui servira à cuire le gâteau. versez la préparation. Couvrez avec le couvercle du tupperware et laissez reposer pendant toute une nuit (ou 24h dans mon cas).
\end{preparation}

\begin{cuisson}
Le lendemain, ajoutez du papier cuisson et fermez avec une ficelle pour bien isoler. Coupez le surplus de papier une fois scellé.

Dans une grand marmite, mettez le tupperware et ajoutez de l'eau chaude jusqu'à la limite du tupperware (il ne faut pas le noyer). Couvrez avec un couvercle, idéalement en verre pour voir sans avoir à ouvrir. Portez à ébullition (environ 30 minutes à feu moyen/fort -- 5/9), puis baissez le feu jusqu'à juste maintenir l'ébullition mais pas plus (4/9). Faites alors cuire 8h en tout (en comptant les 30 premières minutes). Rajoutez de l'eau au besoin, mais je n'ai pas eu besoin d'en rajouter (avec le couvercle j'ai perdu très peu d'eau et j'en avais mis pas mal initialement).

Dégustez de préférence un peu tiède (20s au micro onde). Je n'ai gardé le gâteau qu'une semaine au frigo avant de le déguster mais c'est censé pouvoir se garder. De manière générale, ne pas manger si ça sens bizarre ou s'il y a de la moisissure mais sinon c'est bon, et ça se garde environ un an normalement, dans un endroit frais et sec.
\end{cuisson}
\end{recette}

\section{Cookies}
\begin{recette}{Cookies}{0}{20 min.}{10 min.}\index{cookies}
\begin{ingredients}
\ingredient 85 g de sucre de canne
\ingredient 85 g de beurre mou
\ingredient 1 œuf
\ingredient 150 g de farine
\ingredient 1 pincée de sel, extrait de vanille (ou 1 sachet de sucre vanillé)
\ingredient 1/2 sachet de levure
\ingredient 100g pépites de chocolat (je concasse une demi plaque de chocolat à dessert)
\end{ingredients}

\begin{preparation}
\etape Mélangez le beurre mou, le sucre, l'oeuf entier et la vanille
\etape Ajoutez petit à petit le mélange levure/farine
\etape Ajoutez enfin les pépites de chocolat. Finissez de pétrir à la main sinon vous ne vous en sortirez pas
\etape Préchauffer le four à 180°C
\etape Faire des petites boules que vous aplatirez ensuite avec la main
\end{preparation}

\begin{cuisson}
Faites cuire 10 minutes à 180°C . Sortez et laissez refroidir les cookies SANS LES TOUCHER. Ils ne semblent 
pas cuits mais c'est normal, ils vont finir de cuire hors du feu en refroidissant.

\begin{remarque}
Ne pas faire deux grilles en même temps, même avec chaleur tournante ça ne cuit pas bien. 
\end{remarque}

\end{cuisson}
\end{recette}

\section{Crème brûlée}
\begin{recette}{Crème brûlée}{3}{20 min}{2h}\index{crème brûlée}\index{crème catalane}
\begin{ingredients}[4 pers.]
\ingredient 5 jaunes d'œufs (vous pouvez réserver les blancs d'œufs pour faire des meringues \refsec{sec:meringues})
\ingredient 100g de sucre semoule
\ingredient 40cl de crème liquide entière (10cl par crème brûlée)
\ingredient 40g de sucre en poudre (il faut que les grains soient fins ; pour le caramel)
\ingredient extrait de vanille
\end{ingredients}

\begin{preparation}
\etape Mélanger dans un bol les 100g de sucre et les jaunes d'œufs au fouet.
\etape Verser la crème et l'extrait de vanille dans une casserole. Faites chauffer afin de bien mélanger et faire légèrement 
réduire. 
\etape Incorporez alors le sucre et les œufs dans la crème en prenant bien soin de faire refroidir un peu la crème (si la crème 
est à 100°C, les œufs vont cuire et vous préparerez une omelette bizarre.
\etape Répartir dans les plats de service (ramequin, assiette catalane).
\end{preparation}

\begin{cuisson}
Faites cuire au four à 100°C pendant 3 heures, sans préchauffage. La crème doit être un peu tremblotante 
(légèrement frémissante), mais pas plus. 

Laisser refroidir au frigo. 

Avant de servir, saupoudrez de sucre en poudre et le brûler avec un petit chalumeau de cuisine en décrivant des cercles avec le 
chalumeau jusqu'à former un caramel. 

\end{cuisson}
\end{recette}

\section{Crème chantilly}
\begin{recette}{Crème chantilly}{3}{20 min}{}\index{crème chantilly}\index{chantilly}
\begin{ingredients}[4 pers.]
\ingredient 50cl de crème liquide entière (C'est la matière grasse qui fait que la chantilly prend)
\ingredient extrait de vanille
\ingredient 50g de sucre
\end{ingredients}

\begin{preparation}
\etape Mélangez le sucre et la crème et l'extrait de vanille.
\etape Mettez au frais la crème, le saladier et le fouet du batteur (2h avant à peu près)
\etape 
Battre la crème à l'aide d'un fouet électrique. Changer de vitesse (du plus lent au plus rapide) progressivement, toutes les 30 
secondes environ.
\begin{remarque}
Ne pas fouetter trop vite (j'ai été jusqu'à 8/10), attendre que la crème prenne avant d'augmenter la vitesse
\end{remarque}
\end{preparation}
\end{recette}

\section{Crêpes}
\begin{recette}{Crêpes}{3}{10 min + 1h}{1h}\index{crêpes}
\begin{ingredients}[$\sim$ 24 grandes crêpes]
\ingredient 500g farine
\ingredient 6 œufs
\ingredient 1L de lait
\ingredient 30 g de sucre vanillé (4 sachets)
\ingredient 1 pincée sel
\ingredient Rhum et extrait d'orange pour parfumer.
\end{ingredients}


\begin{preparation}
\etape Dans un saladier, verser la farine et le sel.
\etape Y faire un puit et mettre les œufs et le 1/3 du lait. Mélanger le tout sans précautions
\etape Laissez reposer pendant 5 minutes (ça permet de bien humidifier les éventuels grumeaux)
\etape Mélangez de nouveau puis rajoutez le reste de lait
\etape Rajouter une cuillère à soupe de rhum et une cuillère à café d'extrait d'orange.
\etape Laisser reposer la pâte une heure au frais.
\end{preparation}

\begin{remarque}
Pour des crêpes plus légères, mettre moitié de lait et moitié d'eau
\end{remarque}

\begin{cuisson}
Faire cuire les crêpes dans une poêle très chaude légèrement huilée. 

\begin{enumerate}
 \item Faites chauffer la ou les poêles (idéalement deux) à feu moyen/vif, légèrement huilée
 \item Versez la pâte en faisant tourner la poêle pour étaler
 \item videz le surplus de pâtes dans le saladier (pour que ce soit plus fin)
 \item Retournez la crêpe quand le coté commence à roussir
 \item Surveillez et sortez la crêpe dès que des points roux apparaissent
 \begin{remarque}
  Si vous videz le surplus, la pâte sera très fine, elle craquèlera beaucoup plus vite quand vous la retournez
 \end{remarque}

\end{enumerate}
\end{cuisson}
\end{recette}

\section{Croustade aux pommes}
\begin{recette}{Croustade aux pommes}{3}{1h30}{15 min.}\index{croustade}\index{pomme}
\begin{ingredients}
\ingredient 12 feuilles de filo
\ingredient 4 pommes
\ingredient 100 g de beurre
\ingredient 50 g de sucre en poudre (+ du sucre pour saupoudrer dessus)
\ingredient 8 cuillères à soupe d'armagnac
\ingredient 1 demi cuillère à café de cannelle en poudre
\end{ingredients}


\begin{preparation}
\etape Clarifiez le beurre : faites-le fondre à feu doux dans une petite casserole, ôtez à l'aide d'une cuillère la fine 
pellicule blanche qui s'est formée à la surface, puis versez délicatement le beurre fondu dans un bol, pourque le petit lait 
déposé au fond reste dans la casserole. Le beurre ainsi obtenu ne contient plus du tout d'humidité, ce qui l'empêchera de 
noircir pendant la cuisson du feuilletage.
\etape Préparez les fruits : Épluchez les pommes, coupez-les en fines lamelles en éliminant le cœur et les pépins. 
\etape Faites-les 
sauter dans une poële à feu moyen dans 1 cuillère à soupe de beurre. Mettez les 50g de sucre et de la cannelle. 
Faites-les 
caraméliser 10 mn en secouant le manche de la poêle.
\etape Préchauffez le four à thermostat 6 (180°C).
\etape Avec un pinceau souple, badigeonnez de beurre clarifié le fond d'un moule à manqué. 
\etape Beurrez de la même façon 3 feuilles de filo. Disposez-les afin qu'une de leur extrémité dépasse à peine et l'autre 
dépasse beaucoup et faites des rotations afin que ces gros bouts qui dépassent finissent par dépasser de tous les cotés.  
Répartissez 1/3 des pommes, sucrez. Recommencez l'opération 2 fois. Rabattez les bords des feuilles de filo sur la dernière 
couche de fruits.
\begin{remarque}
On peut faire une couche de plus si on ne met pas de feuille sur le dessus (et le recouvrement sert de dessus en fait, surtout 
si on a bien respecté afin que ça dépasse pas mal de chaque coté.
\end{remarque}

\end{preparation}

\begin{cuisson}
Badigeonnez de beurre, poudrez de sucre et enfournez pour 15 mn.
Démoulez délicatement la croustade en la retournant sur une assiette, entaillez la surface avec un couteau et arrosez 
d'armagnac.
Servez aussitôt.
\end{cuisson}
\end{recette}

\section{Crumble aux pommes}
\begin{recette}{Crumble aux pommes}{0}{}{30 min}\index{crumble}\index{pommes}
\begin{ingredients}
\ingredient 5 pommes
\ingredient des framboises (ou du jus de citron, faute de framboises)
\ingredient $150$g de cassonnade
\ingredient $150$g de farine
\ingredient $125$g de beurre ramolli (pas fondu)
\ingredient une cuillère à soupe de cannelle
\end{ingredients}

\begin{preparation}
\etape Coupez les pommes en dés et disposez-les au fond du moule.
\etape Dans un saladier, mettez la farine, le beurre, le sucre et la canelle et malaxez le tout avec les mains. Mélangez jusqu'à 
obtenir quelque chose d'homogène et de friable.
\etape Répartissez le mélange sur les pommes sans tasser
\end{preparation}

\begin{cuisson}
Enfourner une demi-heure à $180$\degres C. Servir chaud ou tiède dans le plat de cuisson.
\end{cuisson}
\end{recette}

\section{Devil Food cake}
\begin{recette}{Devil Food cake}{3}{20 min}{40 min}\index{chocolat}
\begin{ingredients}
\ingredient[Gâteau]
\ingredient 170g de beurre mou
\ingredient 240g de sucre en poudre
\ingredient 80g de cacao amer en poudre
\ingredient 30cl de lait
\ingredient 3 œufs entiers
\ingredient 200g de farine 
\ingredient 11g de levure chimique
\ingredient 4g de sel
\ingredient[Glaçage]
\ingredient 120g de chocolat noir pâtissier
\ingredient 80g de chocolat lait pâtissier
\ingredient 30g de beurre doux
\ingredient 20cl de crème fraîche épaisse 30\% (\textbf{entière})
\end{ingredients}

\begin{preparation}
\etape Préchauffer le four à 180°c. 
\etape Mélangez les oeufs et le sucre jusqu'à obtenir une consistance crémeuse. 
\etape Dans une casserole, faites fondre le beurre puis ajoutez le chocolat en poudre
\etape Ajoutez la farine, la levure et le lait au mélange oeuf/sucre. 
\etape Ajoutez alors le mélange beurre/chocolat. 
\etape Versez la préparation dans un moule beurré et fariné. 
\end{preparation}

\begin{cuisson}
Faites cuire pendant 37 minutes à 180°C. Laisser tiédir dans les moules puis démouler sur une grille et attendre le refroidissement complet avant montage.

Glaçage chocolat : au bain-marie, faire fondre les deux chocolats, ajouter le beurre en parcelles. Faire chauffer dans une casserole la crème fraîche jusqu’à ébullition. Verser ensuite, hors du feu, la crème sur le chocolat et beurre fondus. Bien mélanger et laisser refroidir (sans toutefois que ce soit complètement froid).

Montage : répartir un peu moins de la moitié du glaçage sur le premier gâteau. Déposer le deuxième gâteau et y verser le reste du glaçage. Lisser à la spatule les côtés et le dessus du gâteau. Prêt à déguster…
\end{cuisson}
\end{recette}

\section{Far breton}
\begin{recette}{Far breton}{3}{15 min}{1h}\index{far breton}\index{flan patissier}\index{pommes}\index{pruneaux}
\begin{ingredients}
\ingredient 220g de farine
\ingredient 130g de sucre
\ingredient un sachet de sucre vanillé
\ingredient 75cl de lait
\ingredient 5 œufs
\ingredient 20g de beurre
\ingredient (Facultatif) 500g de pruneaux ou pomme, traditionnellement non
\end{ingredients}

\begin{preparation}
\etape Préchauffez le four à 180°C (thermostat 6)
\etape Dans un saladier, mélangez le sucre, la farine et le sucre vanillé
\etape Ajoutez les œufs en prenant soin de bien mélanger le tout à chaque fois
\etape Versez le lait et ajoutez le beurre au préalablement fondu puis mélangez jusqu'à obtenir une pâte homogène
\etape Ajoutez vos pruneaux si vous souhaitez obtenir un far aux pruneaux (pensez à les dénoyauter). Mais vous pouvez 
évidemment le déguster nature, ou avec des pommes.
\etape Beurrez le fond du moule et versez-y la pâte. Une autre manière de faire est de mouiller le plat, puis de saupoudrer de 
la farine.
\end{preparation}

\begin{cuisson}
Enfourner une heure environ à $180$\degres C
\end{cuisson}
\end{recette}

\section{Financier}
\begin{recette}{Financier}{3}{}{1h}\index{financier}\index{blanc d'œufs}
\begin{ingredients}
\ingredient 2 blancs d'oeufs
\ingredient 75g de sucre glace
\ingredient 70g de beurre fondu
\ingredient 25g de farine
\ingredient 40g de poudre d'amande
\ingredient vanille
\end{ingredients}

\begin{preparation}
\etape Faire préchauffer le four à $190\degres C$
\etape Battez les blancs et le sucre 
\etape Ajoutez le beurre fondu, la farine, la poudre d'amande et la vanille puis mélangez
\etape Répartissez  la pâte dans le moule à financiers posé sur la grille froide du four
\end{preparation}

\begin{cuisson}
Faites cuire 12 à 14 minutes dans le four préchauffé à 190°C
\end{cuisson}
\end{recette}

\section{Flan}
\begin{recette}{Flan}{3}{}{1h}\index{flan}
\begin{ingredients}[8 personnes]
\ingredient[Flan]
\ingredient 2 litre de lait
\ingredient 160g de sucre
\ingredient 10 œufs
\ingredient[Caramel]
\ingredient 200g de sucre % 100g initialement, j'ai testé 200 et c'était trop (partie solidifiée apres cuisson)
\end{ingredients}

\begin{preparation}
\etape Ne pas faire préchauffer le four.
\etape Dans une casserole, mettez le sucre pour le caramel. Vous pouvez ajouter une goutte d'eau pour que le caramel se fasse plus 
vite. Puis versez au fond du plat pour le flan.
\etape Dans la même casserole, faites chauffer le lait avec le sucre sans aller jusqu'à ébullition (pour ne pas cuire les oeufs ensuite), puis mélangez avec les oeufs (Optionnel: passez au tamis pour enlever les morceaux d'oeufs qui ne sont pas bien mélangés)
\etape Versez alors la préparation avec le lait.
\end{preparation}

\begin{cuisson}
Mettre le moule dans un autre récipient plus grand contenant de l'eau froide, et faites cuire au bain marie pendant 2h à 160°C, à four froid au départ (la surface doit 
être roussie).

\begin{remarque}
Il faut la croute sur le dessus, sinon c'est pas cuit. En fonction de la quantité il faudra plus ou moins longtemps, 2h c'est pour 2L de lait
\end{remarque}
\end{cuisson}
\end{recette}

\section{Flan coco}
\begin{recette}{Flan coco}{3}{20 min.}{1h}\index{flan}
\begin{ingredients}
\ingredient[Flan]
\ingredient 500ml de lait entier
\ingredient 400g de lait de coco
\ingredient 20cl de crème de coco
\ingredient 80g de sucre (si pas assez sucré, 200g la prochaine fois)
\ingredient 7 œufs
\ingredient[Caramel]
\ingredient 100g de sucre
\end{ingredients}

\begin{preparation}
\etape Faire préchauffer le four à $180\degres C$
\etape Dans une casserole, faites brunir 100g de sucre. Vous pouvez ajouter une goutte d'eau pour que le caramel se fasse plus 
vite.
\etape Pendant ce temps, mélangez les œufs, le lait et l sucre dans un récipient. Ajoutez une gousse de vanille fendue.
\etape Versez le caramel au fond du moule, puis ajoutez la préparation avec le lait.
\end{preparation}

\begin{cuisson}
Mettre le moule dans un autre récipient plus grand contenant de l'eau, et faites cuire au bain marie pendant 1h (la surface doit 
être roussie).
\end{cuisson}
\end{recette}

\section{Flan patissier}
\begin{recette}{Flan patissier}{3}{}{1h}\index{flan}
\begin{ingredients}[8 personnes]
\ingredient[Pâte]
\ingredient 430 grammes farine T45
\ingredient 200 grammes cassonade sucre blond ou roux de canne
\ingredient 200 grammes beurre doux froid
\ingredient 2 pincée levure chimique
\ingredient 2 pincée sel
\ingredient 2 oeuf
\ingredient[Flan]
\ingredient 5 oeufs
\ingredient 220 grammes sucre en poudre sucre blanc, fin ou extra fin
\ingredient 100 grammes maïzena
\ingredient 1 gousse de vanille
\ingredient 250 grammes crème liquide entière 25 cl
\ingredient 1 litre lait entier 1000g
\ingredient 1 cuillère à café arôme vanille ou extrait naturel de vanille
\end{ingredients}

\begin{preparation}
\etape Dans le bol du robot pâtissier, mettre la farine, le sucre blond ou roux, le beurre froid coupé en dés, la pincée de levure chimique et de sel et l'oeuf. Avec la feuille (fouet plat) en vitesse lente (vitesse 2), mélanger jusqu'à ce que la pâte s'agglomère autour de la feuille et se détache du bol.
\etape Prendre la pâte, faire une boule, l'aplatir et l'emballer dans du film alimentaire. Réserver au réfrigérateur minimum 30 minutes.
\etape Beurrer et fariner un cercle de 24 cm de diamètre et 6 cm de hauteur et le déposer sur une plaque perforée tapissée d'une feuille silicone (ou papier sulfurisé).
\etape Etaler la pâte sur un plan fariné en un cercle d'environ 35 cm de diamètre.
\etape Et foncer le cercle. Bien marquer les bords. Découper le surplus de pâte avec un petit couteau. Placer le tout a réfrigérateur.
\etape Préchauffer le four à 180°c, chaleur statique (zones de chauffe en haut et en bas).
\etape Dans une grande casserole en inox, faire bouillir le lait avec 120g de sucre et la gousse de vanille fendue et grattée.
\etape Pendant ce temps, dans un grand saladier fouetter les oeufs avec 100g de sucre restant, l'arôme vanille et la maïzena. Rajouter enfin la crème liquide et bien mélanger.
\etape Incorporer peu à peu le lait chaud et continuant de fouetter (retirer la gousse de vanille).
\etape Remettre le mélange dans la casserole, chauffer tout en fouettant jusqu'à ce que la crème soit bien épaissie (le fouet doit laisser des marques dans la crème). Retirer du feu.
\etape Verser l'appareil à flan dans le fond de tarte, lisser la surface à l'aide d'une spatule.
\end{preparation}

\begin{cuisson}
\begin{enumerate}
\item Enfourner au milieu du four (niveau 3) pour 50 minutes.
\item A la fin de cuisson, sortir la plaque du four. La poser sur une grille et laisser refroidir 15 minutes (le flan va s'affaisser, c'est normal).
\item Après 15 minutes de refroidissement, faire glisser la plaque de silicone avec le flan (en laissant le cercle) sur une volette. Puis retirer délicatement la feuille silicone (s'aider d'une longue spatule pour décoller le flan) et garder le flan sur la volette 1 heure (ainsi la base va sécher).
\item Faire glisser ensuite le flan (avec le cercle) sur le plat de service en s'aidant encore de la spatule pour décoller le flan. Retirer le cercle.
\item Laisser encore 30 minutes à température ambiante.
\item Placer au réfrigérateur minimum 3 heures (ou jusqu'au lendemain), sans couvrir.
\end{enumerate}

\end{cuisson}
\end{recette}

\section{Forêt noire à la moi}
\begin{recette}{Forêt noire à la moi}{3}{20 min.}{1h}\index{confiture}
%source: https://gabriellaboiteasucre.fr/2020/05/25/la-foret-noir-revisitee/
\begin{ingredients}
\ingredient[Moly cake]
\ingredient 193g de farine
\ingredient 193g de sucre
\ingredient 20cl de crème liquide
\ingredient 3 oeufs
\ingredient un sachet de levure chimique (11g)
\ingredient 30g de cacao non sucré (Van Houten ou autre)

\ingredient[crème chocolat (ou confiture)]
\ingredient 105g de chocolat
\ingredient 2 jaune d'oeufs
\ingredient 25g de sucre
\ingredient 125 de lait
\ingredient 125g de crème liquide

\ingredient[Chantilly Mascarpone]
\ingredient 20cl de crème liquide
\ingredient 120g de mascarpone
\ingredient 20g de sucre glace

\end{ingredients}

\begin{preparation}
\etape Faire préchauffer le four à $160\degres C$
\etape Beurrez et farinez le moule
\etape Fouettez les oeufs et le sucre jusqu'à ce que la préparation soit jaune clair et ait triplée de volume.
\etape Ajoutez la farine, le cacao et la levure chimique préalablement tamisés et remuez avec une spatule
\etape Faites monter la crème liquide en texture mousseuse et ajoutez la à la préparation
\etape Versez dans votre moule et au four pendant 45 minutes à 160°C
\etape [facultatif] Préparez la crème chocolat. 
\etape Faites fondre le chocolat
\etape Battre les oeufs et le sucre puis ajoutez le lait et la crème liquide
\etape Mettez le tout dans une casserole et laissez chauffer jusqu'à ce que la préparation épaississe (ou atteigne 82°C)
\etape Versez la crème en 2 fois sur le chocolat et mélangez. Laissez alors densifier au frais.
\etape Préparez alors la crème chantilly. Mélangez la crème liquide, le mascarpone et le sucre glace, fouettez comme pour la chantilly à vitesse pas trop importante pour éviter de faire du beurre.
\etape Pour le montage, coupez le gâteau en deux en comptant 2/3 de la hauteur pour le bas (parce que le haut a aussi l'épaisseur du bombé.
Imbibez l'intérieur avec le sirop, pochez des boules de chantilly sur le pourtour de la première couche puis étalez la confiture à l'intérieur.
\etape étalez de la chantilly sur le dessous du 2e étage, puis mettez sur le premier étage (ainsi, la confiture va un peu imbiber le gâteau du dessous sans détremper la crème chantilly)
\etape étalez enfin le reste de chantilly sur le dessus du gâteau et pochez quelques boules pour décorer si vous en avez le courage.

\end{preparation}

\begin{cuisson}

\end{cuisson}
\end{recette}

\section{Galette des rois frangipane}
\begin{recette}{Galette des rois frangipane}{4}{}{1h}\index{galette des rois frangipane}\index{frangipane}
\begin{ingredients}
\ingredient 2 pâtes feuilletées
\ingredient 1 jaune d’œuf
\ingredient[Pour la crème pâtissière]
\ingredient 220ml de lait
\ingredient 1 jaune d’œuf
\ingredient 1 pincée de sel
\ingredient 60 g de sucre
\ingredient 20 g de farine
\ingredient 20 g de maïzena
\ingredient[Pour la crème d’amande]
\ingredient 80 g de beurre mou
\ingredient 100 g de sucre
\ingredient 2 œufs
\ingredient 120 g de poudre d’amande
\ingredient arôme d’amande amère
\ingredient arôme vanille
\end{ingredients}

\begin{preparation}
\etape[Préparation de la crème pâtissière]
\etape Faites chauffer le lait à feu doux.
\etape Pendant ce temps, mélangez le jaune d’œuf avec le sucre. Ajoutez la farine et maïzena et mélangez bien de nouveau jusqu’à obtenir un appareil lisse.
\etape Versez une louche de lait chaud dans votre mélange œufs/sucre/farine pour commencer à le délayer en remuant doucement pour homogénéiser l’appareil.
\etape Puis versez le restant de lait sans cesser de remuer doucement.
\etape Versez le tout dans la casserole et remettez sur feu doux, sans cesser de remuer jusqu’à ce vous obteniez une crème épaisse. Cela peut prendre un petit moment, le tout est d’être patient et de ne pas cuire trop fort et de remuer sans cesse pour éviter que les œufs coagulent.
\etape Finissez en versant la crème pâtissière dans un cul de poule et laissez refroidir.
\etape[Réalisation de la crème d’amande]
\etape Préchauffez votre four à 200°C.
\etape Mélangez la poudre d’amande avec les œufs, ajoutez le sucre et le beurre mou.
\etape Mélangez la crème d’amande avec la crème pâtissière jusqu’à ce que le mélange soit homogène.
\etape Versez l’extrait d’amande amère et l’arome vanille (1 à 2 c. à café selon la force de l’arôme) mélangez bien
\etape[Montage de la galette à la frangipane]
\etape Étalez une pâte feuilletée sur une plaque à pâtisserie recouverte d’une feuille de papier cuisson. Versez en son centre la frangipane, laissez environ 2 à 3 cm d’espace entre la frangipane et les bords de la pâte feuilletée (n’oubliez pas de poser la fève).
\etape Soudez les bords des deux pâtes feuilletées en appuyant les dents d’une fourchette (on appelle cela « chiqueter »), cela évitera à la crème d’amande de tenter de sortir pendant la cuisson, et ça fait aussi très jolie.
\etape Faites un tout petit trou avec la pointe d’un couteau dans la pâte (au centre par exemple) afin d’éviter que la pâte ne gonfle de trop et badigeonnez la galette du jaune d’œuf avec un pinceau.

\end{preparation}

\begin{cuisson}
Baissez votre four à 180°C et enfournez sur une plaque à pâtisserie pendant 30 à 40 ou jusqu’à ce que votre galette soit bien dorée et gonflée.
\end{cuisson}
\end{recette}

\section{Gâteau À l'ananas}
\begin{recette}{Gâteau À l'ananas}{4}{30 min.}{1h}\index{gâteau à l'ananas}\index{ananas}
\begin{ingredients}
\ingredient[gâteau]
\ingredient $150$ g de sucre
\ingredient $150$ g de beurre fondu
\ingredient $150$ g de farine
\ingredient 1 pincée de sel
\ingredient $1$ paquet de levure (11g)
\ingredient $4$ œufs
\ingredient $1$ boite de tranches d'ananas (garder le jus pour le sirop)
\ingredient[caramel]
\ingredient 200g de sucre
\ingredient 5cl d'eau
\ingredient[sirop $\sim 25$ cl]
\ingredient jus de la boite d'ananas
\ingredient 3cl de rhum
\ingredient eau pour compléter jusqu'à 25cl
\end{ingredients}

\begin{preparation}
\etape Préchauffez le four à 150°C
\etape Égouttez les tranches et réservez le jus pour plus tard.
\etape Beurrez le moule. 
\etape Faites chauffer le sucre et l'eau du caramel à feu vif. Dès que la préparation bruni, versez la dans le moule. 
\etape Déposez alors les tranches d'ananas dans le caramel.
\etape Mélangez le sucre, les œufs, la farine, la levure, la pincée de sel et ajoutez en dernier le beurre.
\etape Versez et étalez ce mélange au dessus des ananas.
\end{preparation}

\begin{cuisson}
Faites cuire une heure à 150°C.

Démoulez chaud et ajoutez le jus d'ananas mélangé à du rhum (en tout 22cl environ) sur le dessus pour imbiber le gâteau.
\end{cuisson}
\end{recette}

\section{Gâteau à la broche à la moi}
\begin{recette}{Gâteau à la broche à la moi}{4}{45 min.}{1h}\index{gâteau à la broche}
\begin{ingredients}
\ingredient 300g de farine,
\ingredient 6 œufs
\ingredient 250g de beurre
\ingredient 250g de sucre
\ingredient 1 pincée de sel
\ingredient 2 cuillerées de pastis
\ingredient 1 cuillerée à café de fleur d’oranger
\end{ingredients}

\begin{preparation}
\etape Dans une casserole, versez 50g de sucre et le beurre et mettez à fondre à feu doux, arrêter quand le mélange est liquide et homogène
\etape Dans une saladier, versez 250g de sucre, les blancs d'œufs et une pincée de sel puis battez les blancs en neige
\etape Dans le saladier final, versez les jaunes d'oeufs. Incorporez le beurre + sucre, le pastis et la fleur d'oranger
\etape Incorporez la farine dans le saladier. 
\etape Ajoutez enfin les blancs en neige dans la patte sans les casser. 
\end{preparation}

\begin{cuisson}
Beurrez et farinez un plat à gratin. Mettez le four en mode grill avec la porte légèrement entrouverte. Positionnez le plat à gratin à mi hauteur. 

Versez une louche complète de pâte dans le plat, puis étalez là à l'aide d'une spatule en silicone. Disposez la pâte dans les 
4 coins car c'est le plus difficile d'accès. Faites cuire 5 minutes 
environ, jusqu'à ce que la pâte dore, puis recommencez jusqu'à épuisement de la pâte. 
\end{cuisson}
\end{recette}

% gateau de dominique (mais c'était une grosse quantité, j'ai donc divisé les doses par deux)
\section{Gâteau au beurre}
\begin{recette}{Gâteau au beurre}{4}{30 min}{30 min}\index{gâteau au beurre}

\begin{ingredients}
\ingredient 125g beurre 
\ingredient 200g de sucre
\ingredient 3 œufs
\ingredient 22.5 ml de jus de citron (1/2 citron)
\ingredient 300g farine
\ingredient Lait (à définir)
\ingredient 1 sachet de levure 
\ingredient Essences
\end{ingredients}

\begin{preparation}
\etape Mélangez œufs + sucre et faire mousser 
\etape Rajouter essences + citron pressé.
\etape Rajouter beurre ramolli
\etape Rajouter 150g farine + levure
\etape Encore 150g farine + 1 peu de lait pour aider à mélanger 
\end{preparation}

\begin{cuisson}
1h au four à 180°C
\end{cuisson}
\end{recette}

% la recette est pour un monsieur cuisine ou un truc du style, j'ai donc adapté, et il faut repréciser la recette une fois que je l'aurai testé une fois.
\section{Gâteau au chocolat (mélina)}
\begin{recette}{Gâteau au chocolat (mélina)}{0}{30 min}{30 min}

\begin{ingredients}
\ingredient 5 oeufs
\ingredient 1 pincée de sel
\ingredient 150g de beurre
\ingredient 200g de sucre
\ingredient 250g de chocolat 70%
\ingredient 60g de farine T45
\end{ingredients}

\begin{preparation}
\etape Préchauffer le four à 200°C. Beurrer le moule
\etape Séparez les blancs des jaunes d'oeufs
\etape montez les blancs en neige avec une pincée de sel
\etape Transférer les blancs en neige dans un récipient puis les placer au réfrigérateur
\etape Faites fondre le beurre et mélangez avec les jaunes et le sucre
\etape Faites fondre le chocolat puis mélangez avec le reste
\etape Ajoutez la farine
\etape Ajoutez enfin les blancs en neige et mélangez au batteur très rapidement (18s)
\etape Versez dans le moule
\end{preparation}

\begin{cuisson}
Faites cuire environ 20 minutes à 200°C

vérifiez la cuisson à l'aide d'un cure dent, prolonger la cuisson de quelques minutes si nécessaire. 
Retirer le gâteau du four et laisser refroidir quelques minutes dans le moule. Détacher le gâteau de la paroi du moule à l'aide d'un petit couteau, puis retirer le fond et sortir le gâteau. Servir chaud ou froid. 

\begin{remarque}
Plus le temps de cuisson est long, plus le gâteau est ferme. Pour obtenir un gâteau moelleux, ne pas la prolonger trop longtemps. 
\end{remarque}
\end{cuisson}
\end{recette}

\section{Gâteau Au Yahourt}
\begin{recette}{Gâteau Au Yahourt}{4}{30 min}{30 min}\index{gâteau au yahourt}\index{yahourt}

\begin{ingredients}
\ingredient 125g de yahourt nature (1 yahourt ; Faute de yahourt on peut remplacer avec 125g de creme liquide)
\ingredient 175g de farine (2 pots de yahourt)
\ingredient 250g de sucre (2 pots de yahourt)
\ingredient 50g de beurre
\ingredient $1$ paquet de levure chimique
\ingredient $2$ pommes
\ingredient $2$ œufs
\ingredient $1$ citron ou orange rapé
\ingredient ricard (pour parfumer)
\end{ingredients}

\begin{remarque}
On peut remplacer les pommes par des poires.
\end{remarque}

\begin{preparation}
\etape Préchauffer le four.
\etape Mélanger tous les ingrédients
\etape Ajouter les pommes coupées en tranches
\etape Beurrer le moule, fariner, puis verser la préparation.
\end{preparation}

\begin{cuisson}
Mettre au four 30 minutes, thermostat 5 (175\degres C)
\end{cuisson}
\end{recette}

\section{Gâteau Aux noix}
\begin{recette}{Gâteau Aux noix}{3}{15 min}{50 min}\index{gâteau aux noix}\index{noix}

\begin{ingredients}
\ingredient 100g de cerneaux de noix pilés (environ 250g de noix entières)
\ingredient 250g de sucre
\ingredient 250g de farine
\ingredient 4 œufs
\ingredient 100g de beurre (ou 100 mL d'huile de noix)
\ingredient 25cl de vin blanc
\ingredient un sachet de levure chimique
\ingredient une pincée de sel
\end{ingredients}

\begin{preparation}
\etape Préchauffez le four à 180°C
\etape Faites fondre le beurre
\etape Dans un saladier, délayez œufs et sucre
\etape Ajoutez le vin blanc et une pincée de sel, les cerneaux de noix passés au mixer, le beurre fondu, la levure et la 
farine. 
\etape Mélangez jusqu'à obtenir une mixture homogène. 
\end{preparation}

\begin{cuisson}
Mettre au four 50 minutes à 180°C.
\end{cuisson}
\end{recette}


\section{Gâteau basque à la moi}
\begin{recette}{Gâteau basque à la moi}{4}{1h}{15 min}\index{Gâteau basque à la moi}\index{crème patissière}\index{gâteau basque}
\begin{ingredients}
\ingredient[pâte]
\ingredient 1 œuf
\ingredient 125 g de sucre
\ingredient 250 g de farine
\ingredient 125 g de beurre fondu
\ingredient 1 paquet de levure
\ingredient 1 sachet de sucre vanillé
\ingredient[crème]

\ingredient 500mL (500g) de lait entier
\ingredient 50g de farine (ou 30g de farine et 50g de poudre d'amande)
\ingredient 120g de sucre
\ingredient 2 jaunes d'œufs et un oeuf entier
\ingredient 2 cuillère à soupe de rhum
\ingredient Vanille
\end{ingredients}

\begin{preparation}
\etape Préchauffer le four à 210°C
\etape Mélangez la totalité des ingrédients jusqu'à obtenir une boule de pâte style pâte brisée d'environ 570g
\etape Séparez en deux boules, l'une légèrement plus grande (55\%) que l'autre (45\%, soit environ 255g) (celle du dessus a besoin d'un peu plus de pâte pour ne 
pas s'embêter.
\begin{remarque}
Pour étaler les pâtes, utilisez deux papier cuisson pour que ça n'accroche pas au rouleau à patisserie.\end{remarque}
\etape  Étalez la première pâte (la plus petite) puis conservez uniquement le papier cuisson du bas. Posez la main à plat sur la pâte, puis renversez là pour la déposer sur le moule, puis enlevez délicatement le papier cuisson pour ne pas casser la pâte.
\etape Dans une casserole, mettre le lait à bouillir, avec la gousse de vanille. 
\begin{remarque}
versez un fond d'eau dans la casserole, puis videz l’excédent avant de faire chauffer le lait. Cela évite au lait d'accrocher.
\end{remarque}
\etape Dans un cul-de-poule, blanchir (battre ensemble) le sucre, les jaunes et l'oeuf entier jusqu'à obtenir un mélange mousseux.
\etape Incorporez la farine petit à petit sans cesser de remuer.
\etape Mettez le lait à bouillir, avec la gousse de vanille. Une fois que le lait bout, retirez la gousse de vanille et versez la moitié du lait progressivement dans votre mélange, tout en remuant.
\etape Puis versez ce mélange sur le reste de lait dans la casserole. Remettre la casserole à feu moyen, et remuez sans cesse, jusqu'à ce que la crème s'épaississe (Typiquement quand la mousse du dessus disparait complètement).
\etape Sortez la crème du feu et ajoutez le rhum ambré.
\etape Au besoin, réservez la crème dans un saladier, la filmer en attendant de finir la pâte.
\etape Étalez la crème (pas forcément tout) en faisant attention à ne pas en mettre sur 1cm environ au bord (ça va s'étaler en posant la pâte au dessus)
\etape Préparez alors la deuxième pâte et déposez là sur le moule de la même manière que pour la première pâte
\etape Pincez les deux pâtes sur les bords pour les sceller.

\end{preparation}

\begin{cuisson}
Enfournez alors la préparation pendant 15 minutes à 210°C, le temps que le dessus soit très légèrement roussi.
\end{cuisson}
\end{recette}


\section{Gâteau mangue-passion à la moi}
\begin{recette}{Gâteau mangue-passion à la moi}{4}{30 min.}{1h}\index{gâteau à l'ananas}\index{mangue}\index{fruit de la passion}\index{maracudja}
\begin{ingredients}
\ingredient[gâteau]
\ingredient $150$ g de sucre
\ingredient $150$ g de beurre fondu
\ingredient $150$ g de farine
\ingredient 1 pincée de sel
\ingredient $1$ paquet de levure (11g)
\ingredient $4$ œufs
\ingredient 4 fruits de la passion
\ingredient 2 grosses mangues pas trop mûres (un peu mou mais pas trop)
\ingredient[caramel]
\ingredient 200g de sucre (idéalement blanc pour mieux voir le caramel brunir)
\ingredient 5cl de jus de fruit de la passion (cf plus haut)
\ingredient[sirop $\sim 22$ cl]
\ingredient Reste de jus de fruit de la passion
\ingredient 3cl de rhum
\ingredient eau pour compléter jusqu'à 22cl
\end{ingredients}

\begin{preparation}
\etape Extraire la pulpe de maracudja, puis mettez là dans une passoire (au dessus d'un saladier), puis appuyer avec une spatule silicone ou cuillère pour casser la membrane autour des pépins, faire tomber la pulpe et filter les pépins. 
\etape Dans une casserole, mettez le sucre du caramel puis 50g du jus que vous venez d'extraire. Réservez le reste du jus de maracudja pour le sirop. 
\etape Coupez les mangues le long du noyau pour en faire 3 tranches dont le centre est le noyau. Enlevez la peau, et coupez les deux tranches externes en lamelles de 3-4mm.
\etape Préchauffez le four à 150°C
\etape Beurrez le moule. 
\etape Faites chauffer le sucre et le jus pour faire le caramel. Dès que la préparation bruni, versez la dans le moule. 
\etape Déposez alors les tranches de mangues sur le caramel, la partie épaisse à l'extérieur du moule, et la pointe fine au centre. 
\etape Mélangez le sucre, les œufs, la farine, la levure, la pincée de sel et ajoutez en dernier le beurre.
\etape Versez et étalez ce mélange au dessus des fruits.
\end{preparation}

\begin{cuisson}
Faites cuire une heure à 150°C.

Pendant que le gâteau cuit, ajoutez 3cl de rhum vieux au jus de maracudja. Ajoutez environ 50g de sucre, et enfin, complétez pour avoir 22cl de liquide. 

Démoulez chaud et ajoutez le sirop préparé précédemment sur le dessus pour imbiber le gâteau, à la cuillère à soupe pour y aller doucement. 
\end{cuisson}
\end{recette}

%source https://www.femina.fr/article/cyril-lignac-devoile-sa-recette-de-la-buche-de-noel-au-chocolat-et-creme-vanille
\section{Gateau Roulé}
\begin{recette}{Gateau Roulé}{4}{30 min.}{7 min.}\index{gâteau roulé}\index{roulé à la confiture}\index{confiture}
\begin{ingredients}[1 grande plaque 35x42cm]% ce n'est pas la plaque téfal, c'est celle des états-unis
\ingredient 6 œufs
\ingredient 85+135 g de sucre
\ingredient 85 g de farine
\ingredient confiture
\ingredient beurre ou huile pour beurrer le moule
\end{ingredients}

\begin{preparation}
\etape Préchauffer le four à 210°C et beurrer/huiler puis fariner le moule (un moule relativement grand, et rectangulaire de 
préférence.
\etape Séparez les blancs des jaunes de 3 oeufs. 
\etape Montez les  blancs avec 85g de sucre pour qu'ils forment un bec d'oiseau.
\etape Montez les jaunes avec les oeufs entier et les 135g de sucre (ça va monter un peu aussi, c'est bien)
\etape Mélanger la farine aux jaunes.
\etape Mélangez les deux appareils ensemble délicatement. \og Entourez\fg la pâte pour ne pas chasser l'air contenu dans les blancs\footnote{En gros, il faut faire le tour du 
saladier, le dessous, avec des mouvements amples, sans chercher à exploser l'aglomérat de blanc}.
\etape Versez dans le moule beurré puis étalez légèrement pour homogénéiser le biscuit à l'aide d'une spatule coudée
\end{preparation}

\begin{cuisson}
Cuisez pendant 7 minutes à 210°C.

À la sortie du four, retournez le moule sur un papier cuisson contre une table ou un plan de travail pour ne pas perdre d'humidité et laissez refroidir ainsi à l'envers.

Une fois froid, démoulez, puis badigeonnez-le de confiture et roulez.

\begin{remarque}
Le gâteau est meilleur au bout de 48h ry atteint son optimum au bout de 72h (la confiture imbibe alors la génoise.
\end{remarque}
\end{cuisson}
\end{recette}

%source: https://www.gastronomie-wallonne.be/gastro/desserts/gaufre_bruxelles.html
\section{Gauffre de Bruxelles}
\begin{recette}{Gauffre de Bruxelles}{0}{}{}\index{gauffre}
\begin{ingredients}[12 gauffres]
\ingredient 250 g de farine
\ingredient 375 ml de lait
\ingredient 10 g de sucre
\ingredient 100 g de beurre fondu
\ingredient 7.5g (1 sachet) de levure de boulanger sèche (ou 15 de fraiche)
\ingredient 3 oeufs
\ingredient 1 pincée de sel
\ingredient 1/4 de bâton vanille
\end{ingredients}

\begin{preparation}
\etape Délayez la levure dans un peu de lait tiède avec le sucre.
\etape Séparez les blancs des jaunes d'œufs. Battez les blancs en neige bien ferme à l'aide d'un batteur électrique ou d'un fouet.
\etape Dans un saladier, mettez la farine tamisée, faites-y un puits et versez le reste de lait tiède, battez énergiquement le tout. Ajoutez les jaunes, le beurre fondu, la levure puis incorporez délicatement les blancs en neige.
\etape Laissez reposer 1h à couvert (45 min si avec levure fraiche).
\etape Faites chauffer le gaufrier et passez à la cuisson de vos gaufres de Bruxelles. Comptez environ 1:50 de chaque coté
\end{preparation}
\end{recette}

\section{Madeleines}
\begin{recette}{Madeleines}{4}{30min.+1 nuit}{30min.}\index{madeleine}
\begin{ingredients}[36 madeleines]
\ingredient 200 g de beurre
\ingredient 200 g de farine
\ingredient 3 oeufs
\ingredient 130 g de sucre
\ingredient 2 cuillères à soupe de miel (40g)
\ingredient 60 mL de lait
\ingredient 10 g de levure chimique
\ingredient 1 sachet de sucre vanillé
\end{ingredients}

\begin{preparation}
\etape Fondre le beurre dans une petite casserole sur feu doux. Bouillonner, comptez encore 1 minute. Surveiller la cuisson. Prendre une couleur noisette. Réservez.
\etape Faites blanchir les oeufs et le sucre
\etape Ajoutez le miel, le lait et le sucre vanillé. Versez la levure chimique et la farine dans la préparation.
\etape Mélangez puis incorporez le beurre noisette tiède. Laissez refroidir cette pâte et conservez la pâte toute la nuit au réfrigérateur soit dans un bol, soit directement dans les moules (ce qui est mieux et plus pratique). Remplissez les moules aux trois quarts (ça gonfle)
\end{preparation}

\begin{cuisson}
Préchauffez le four à 220°C. 

Enfournez à 220°C pendant 4 minutes, puis à 200°C pendant 4min30 minutes. Surveillez la cuisson !

Démoulez dès la sortie du four.

\begin{remarque}
Vous pouvez aussi faire des coques en chocolat facilement en versant l'équivalent d'une cuillère à café de chocolat par moule, puis appuyez la madeleine et laissez là jusqu'à ce que le chocolat refroidisse. Comptez 100g de chocolat pour 18 madeleines environ.
\end{remarque}

\end{cuisson}
\end{recette}


\section{Marbré}
\begin{recette}{Marbré}{4}{30min.+1 nuit}{30min.}\index{madeleine}
\begin{ingredients}
\ingredient 150g de beurre
\ingredient 300g de sucre
\ingredient 2 oeufs
\ingredient 230g de farine
\ingredient 1 sachet de levure chimique
\ingredient 220g de crème liquide
\ingredient 20g de cacao en poudre non sucré
\end{ingredients}

\begin{preparation}
\etape Faire préchauffer le four à 150°C
\etape Mélanger le sucre et le beurre ramolli
\etape rajouter les deux oeufs
\etape rajouter la farine et la levure mélangé
\etape rajouter la crème liquide progressivement
\etape Mettez de coté la moitié de la pâte (environ 470g pour chacun)
\etape Dans le reste de pâte, rajouter les 20g de chocolat
\etape Dans un moule, versez la moitié de chacune des deux pâtes et avoir une couche de pâte blanche/noire, coupé dans la longueur.
\etape Faites alors de même avec la couche supérieure, de sorte que le blanc soit sur le noir, et le noir sur le blanc
\etape Avec le manche d'une cuillère à café, ou n'importe quoi de fin, faites 
\end{preparation}

\begin{cuisson}
Faire cuire à 150°C pendant 1h à 1h15

\end{cuisson}
\end{recette}

\section{Moelleux au chocolat}
\begin{recette}{Moelleux au chocolat}{4}{30min.}{30min.}\index{Moelleux au chocolat}\index{chocolat}
\begin{ingredients}
\ingredient 200 g de chocolat
\ingredient 125 g de beurre
\ingredient 125 g de sucre
\ingredient 4 oeufs
\ingredient 125 g de farine
\ingredient 1 sachet de levure (11g)
\ingredient une pincée de sel
\end{ingredients}

\begin{preparation}
\etape Préchauffer le four à 180°C.
\etape Faire fondre le beurre et le chocolat.
\etape Séparer les blancs des jaunes d'oeufs dans deux saladiers.
\etape Montez les blancs en neige avec une pincée de sel puis réservez. 
\etape Mélanger les jaunes avec le sucre et un peu d'eau jusqu'à ce que le mélange soit mousseux. 
\etape Y ajouter le chocolat/beurre fondu et bien mélanger. Enfin, ajouter la farine et la levure.
\etape Monter les blancs en neige ferme. Les incorporer délicatement au précédent mélange. 
\end{preparation}

\begin{cuisson}
Mettre le tout dans un moule beurré et faire cuire au four 30 minutes. Vérifier la cuisson en piquant une lame de couteau qui doit ressortir sèche. 
\end{cuisson}
\end{recette}

\section{Mousse au chocolat}
\begin{recette}{Mousse au chocolat}{3}{30 min.}{2h}\index{mousse au chocolat}\index{chocolat}\index{blanc d'œufs}
\begin{ingredients}[4 personnes]
\ingredient 150g de chocolat à dessert
\ingredient 6 œufs
\ingredient 2 sachets de sucre vanillé
\ingredient une pincée de sel
\end{ingredients}

\begin{preparation}
\etape Séparez les blancs des jaunes dans deux récipients différents.
\etape Faire fondre le chocolat (au bain marie dans une casserole par exemple)
\etape Mélanger les jaunes d'œufs avec le sucre
\etape Ajoutez une pincée de sel aux blancs, puis battez les en neige (ferme).
\etape Ajoutez le chocolat fondu aux jaunes et sucre.
\etape Incorporez enfin les blancs en neige dans la préparation de chocolat fondu en aérant (faites des mouvements amples avec 
la cuillère pour casser les blancs le moins possible)
\etape Répartissez dans des récipients individuels par exemple, puis mettez au frigo pendant une à deux heures.
\end{preparation}
\end{recette}


\section{Palmier feuilletés}
\begin{recette}{Palmier feuilletés}{3}{30 min.+4h}{}\index{pâte feuilletée}
\begin{ingredients}
\ingredient 1 pâte feuilletée
\ingredient du sucre
\end{ingredients}

\begin{preparation}
\etape Étalez le sucre sur la pâte feuilletée, puis roulez là de part et d'autre. 
\etape Laissez là au congélateur quelques minutes afin de pouvoir couper sans écraser
\etape Préchauffez le four à 220°C
\etape Faites cuire environ 20 minutes
\end{preparation}
\end{recette}

\section{Panna Cotta aux framboises}
\begin{recette}{Panna Cotta aux framboises}{3}{30 min.+4h}{}\index{Panna Cotta}\index{fruits rouges}
\begin{ingredients}
\ingredient[Panna Cotta (5 pers.)]
\ingredient 50 cl de crème fraiche liquide
\ingredient 50g de sucre
\ingredient 2 feuilles de gélatine
\ingredient Vanille (en gousse ou de l'extrait)
\ingredient[Coulis]
\ingredient 500g de framboises (fraiches ou surgelées)
\ingredient 200g de sucre
\ingredient 20 cl d'eau
\ingredient 1 peu de jus de citron
\end{ingredients}

\begin{preparation}
\etape Faire tremper 3 feuilles de gélatine dans de l'eau froide
\etape Mettre la crème, la vanille, le sucre dans une casserole et faire chauffer jusqu'à frémissement.
\etape Quand le mélange commence tout juste à bouillir, retirer la casserole du feu et ajouter la gélatine. Bien remuer pour que 
la gélatine se dissolve complément.
\etape Verser dans des verres, coupelles…
\etape Laisser refroidir
\etape Porter l'eau et le sucre à ébullition
\etape Ajouter le sirop obtenu aux framboises
\etape Bien mixer le tout, et passer au tamis.
\etape Laisser refroidir
\end{preparation}
\end{recette}

\section{Pavlova}
\begin{recette}{Pavlova}{3}{30 min.+4h}{}
\begin{ingredients}
\ingredient[meringue]
\ingredient 4 blancs d'oeufs
\ingredient 200g de sucre
\ingredient[chantilly]
\ingredient 20cl de crème entière très froide
\ingredient 50g de sucre
\ingredient[garniture]
\ingredient 250g de fraises
\end{ingredients}

\begin{preparation}
\etape Montez le sucre et les blancs à haute vitesse au fouet pour faire une meringue
\end{preparation}
\begin{cuisson}
Faites cuire la meringue à 120°C pendant 1h15
\end{cuisson}
\end{recette}

\section{Poires pochées au vin rouge}
\begin{recette}{Poires pochées au vin rouge}{0}{1h30}{}\index{poires au vin}\index{poire}\index{vin}
\begin{ingredients}
\ingredient 4 moyennes poires assez fermes
\ingredient 40cl de vin rouge
\ingredient 200g de sucre
\ingredient 1 cuillère à soupe d'extrait de vanille
\ingredient 1 cuillère à soupe d'extrait d'orange (ou zeste non traité)
\ingredient un peu de canelle
\end{ingredients}

\begin{preparation}
\etape Faites chauffer le vin, le sucre et les arômes. Portez à ébullition
\etape rajoutez les poires coupées en morceaux grossiers (typiquement 8 morceaux par poire) et laissez cuire une heure et demi 
environ jusqu'à ce que le jus devienne un peu plus épais, et les poires moelleuses
\etape Laissez refroidir et dégustez les poires froides avec un peu de chantilly.
\end{preparation}
\end{recette}

\section{Riz au lait}
\begin{recette}{Riz au lait}{0}{1h}{}\index{riz}\index{riz au lait}
\begin{ingredients}[4 pers.]
\ingredient 120g de riz rond
\ingredient 625g de lait (demi-écrémé ça marche, entier aussi)
\ingredient 50g de sucre
\ingredient 1 sachet de sucre vanillé
\ingredient 70g de raisins sec
\ingredient zeste de citron ou d'orange
\end{ingredients}

\begin{preparation}
\etape Portez de l'eau à ébullition et préparez une autre casserole avec le lait et le sucre en attendant.
\etape Une fois l'eau à ébullition, faites-y cuire le riz 3 minutes dedans. Commencez à faire chauffer le lait à feu moyen (5/9) pendant ce temps sans couvrir. 
\etape Au bout des 3 minutes, égouttez le riz
\etape Une fois que le lait bout, mettez-y le riz et les raisins, et faites cuire doucement (2/9) et à couvert pendant 28 minutes. 
\begin{remarque}
Ça doit être très liquide à la fin, le riz va absorber pas mal de liquide en refroidissant. Il est aussi important de ne pas refroidir le riz de manière active (frigo ou bain d'eau froide)
\end{remarque}
\etape Versez alors dans un récipient et laissez à l'air libre refroidir doucement environ 4h (le riz va continuer de cuire et absorber le liquide durant cette période, il est important de ne pas mettre au frigo ou couvrir).
\end{preparation}
\end{recette}

\section{Semoule au lait}
\begin{recette}{Semoule au lait}{0}{1h}{}\index{riz}\index{riz au lait}
\begin{ingredients}[4 pers.]
\ingredient 100g de semoule fine
\ingredient 1L de lait
\ingredient 120g de sucre
\ingredient vanille et canelle
\ingredient zeste de citron ou d'orange
\end{ingredients}

\begin{preparation}
\etape Dans une casserole, porter à ébullition 1 litre de lait avec un peu de canelle, de vanille et le sucre. Une fois à ébullition, ajoutez la semoule en pluie.
\etape Remuer sans cesse pendant 5 minutes afin de laisser épaissir la semoule et éviter tout grumeau.
\etape Éteindre le feu et transférer la semoule au lait dans des ramequins ou petits bols. laisser tiédir et placer les ramequins au réfrigérateur.
\end{preparation}
\end{recette}

\section{Sorbet maison}
\begin{recette}{Sorbet maison}{0}{1h+24h+30min}{}\index{sorbet}
\begin{ingredients}
\ingredient[blanc en neige]
\ingredient 1 blanc d'oeuf
\ingredient une pincée de sel
\ingredient[sirop de sucre]
\ingredient 140g de sucre
\ingredient 80g d'eau
\ingredient[Pour les fruits]
\ingredient 250g de fruits
\ingredient 15cl de jus de fruit (ou le sirop des fruits au sirop à défaut)
\end{ingredients}

\begin{preparation}
\etape Si comme moi vous avez une sorbetière qui ne fait pas le froid, elle doit être placée 18h avant au congélateur. 
\etape La veille, préparez le sirop de sucre en mettant eau et sucre dans une casserole et en portant à ébulltion. 
\etape Mettez alors les fruits, le sirop de sucre et le jus de fruit au frigo. 
\etape Le lendemain, mélangez les fruits, le jus de fruit et le sirop de sucre, puis mixez. 
\etape Mettez le tout dans une bouteille au congélateur pendant 2h15
\begin{remarque}
Le but c'est d'avoir un jus le plus proche de zéro possible pour que la sorbetière n'ait pas à refroidir beaucoup le liquide. Sinon ça échouera.
\end{remarque}

\etape Montez les blancs en neige.
\etape Sortez alors la sorbetière, préparez là.
\etape Sortez le jus de fruit au tout dernier moment, incorporez le à la préparation quasi congelée.
\etape Laissez la sorbetière fonctionner 20 minutes environ.
\end{preparation}
\end{recette}

\section{Tarte chausson aux pommes}
\begin{recette}{Tarte chausson aux pommes}{4}{1h}{20min.}\index{tarte tatin}\index{pommes}\index{tarte}\index{calzone}\index{chausson aux pommes}
\begin{ingredients}
\ingredient 2 pâtes feuilletées
\ingredient 600g de compote de pomme ou poire
\ingredient 1/2 cac de cannelle, 1 cas de rhum
\ingredient un jaune d'oeuf ou un peu de lait pour la dorure
\end{ingredients}

\begin{preparation}
\etape Préchauffez le four à 220°C
\etape Piquez une des pâtes feuilletées puis étalez la au fond du plat
\etape Mélangez la compote, la cannelle et portez à ébullition (c'est pour que la compote ne soit pas trop liquide). Quand la compote commence à être quasiment sèche (à faire un bruit de succion quand vous tournez), arrêtez.
\etape une fois froid, rajoutez le rhum
\etape Versez et étalez la compote sur la pâte 
\etape Piquez la 2e pâte feuilletée et recouvrez la compote avec
\etape Soudez les deux pâtes ensemble en remontant les bords, puis mouillez un peu la soudure à l'aide d'un pinceau
\begin{figure}[htb]
\centering
\includegraphics[width=0.9\textwidth]{figures/tarte_chausson_pomme.pdf}
\caption{Comment sceller les deux pâtes entre elle}
\end{figure}
\etape Dorez enfin la pâte (jaune d'oeuf ou lait)
\end{preparation}



\begin{cuisson}
Mettez au four pendant 25 minutes environ à 220°C (surveillez la cuisson à partir de 20 minutes et arrêtez quand c'est doré un peu partout (et pas juste à quelques endroits).
\end{cuisson}
\end{recette}

\section{Tarte Tatin}
\begin{recette}{Tarte Tatin}{4}{1h+24h}{35min.}\index{tarte tatin}\index{pommes}
\begin{ingredients}
\ingredient 2kg de pommes (royal gala, pink lady, reine des reinettes, canada)
\ingredient 250g de pâte brisée (\refsec{sec:pate_brisee})
\ingredient 300+150g de sucre en poudre
\ingredient 300mL d'eau
\ingredient 40g de beurre
\ingredient cannelle, jus de citron, vanille
\end{ingredients}

\begin{preparation}
\etape Épluchez les pommes et citronnez-les.
\etape Coupez-les en quart dans la hauteur (et videz les)
\etape Faites un caramel avec 150g de sucre et un petit peu d'eau. Une fois coloré, déposez le au fond du plat à tarte. 
\etape dans un récipient relativement large, faites fondre 300g de sucre, 300 mL d'eau, 40g de beurre, la canelle et la 
vanille. 
\etape Une fois à ébullition, déposez la moitié des pommes et faites cuire 10 minutes. 
\etape Sortez les pommes à l'aide d'une écumoire puis faites 
cuire l'autre moitié de la même façon. Pendant la 2e cuisson, disposez les pommes de la première cuisson sur le caramel, le plus serré possible, en utilisant l'écumoire pour ne pas vous brûler
\etape Faites de même avec les pommes de la 2e cuisson
\etape Réservez le jus de cuisson des pommes à coté de la tarte, vous le ferez réduire pendant la cuisson pour en napper la 
tarte tatin à la fin.
\etape Laissez reposer toute une nuit afin que les pommes soient bien froide.
\end{preparation}

\begin{cuisson}
Faites préchauffer le four à 200°C. Faites réduire le jus de cuisson des pommes à feu moyen (pendant toute la cuisson). Si ça 
commence à être onctueux quand vous agitez la casserole (au lieu d'être simplement liquide) c'est que la gelée est prête, 
baissez le feu (ou éteignez s'il reste beaucoup de cuisson), quitte à rallumer juste avant de verser pour re-liquéfier la gelée.

Le lendemain matin, ajoutez la pâte feuilletée sur le dessus du plat. Enfoncez-la à l'intérieur du plat, profitez-en pour 
resserer les pommes au besoin. 

Presser avec la paume des mains pour bien faire adhérer la pâte aux pommes (sans la percer). 

Certains font un trou au milieu pour que la vapeur s'échappe.

Faites cuire à 200°C pendant 35 minutes environ. Recouvrez alors d'un plat de service puis retourner l'ensemble en 
un mouvement rapide mais contrôlé (attention aux projections, ça risque de couler).

Soulevez alors le plat de cuisson encore chaud. Versez alors la gelée de pomme que vous avez fait réduire. 

Vous pouvez le servir tiède avec une boule de glace à la vanille.


\end{cuisson}
\end{recette}

\section{Tarte Tatin à la banane}
\begin{recette}{Tarte Tatin à la banane}{5}{10 min.}{30 min.}\index{tarte tatin}\index{banane}
\begin{ingredients}
\ingredient 150g de sucre
\ingredient 50g de beurre
\ingredient 5cl d'eau
\ingredient 5 bananes
\ingredient 1 pâte feuilletée
\ingredient cannelle, vanille
\end{ingredients}

\begin{preparation}
\etape Préchauffez le four à 200°C
\etape Coupez les bananes en tranches d'1.5cm environ
\etape Dans une casserole, versez le sucre et laissez cuire jusqu'à l'obtention de la couleur caramel. 
\etape Ajoutez alors le beurre et laissez-le fondre. 
\etape Ajoutez les bananes, la vanille et la canelle
\etape Décuire le caramel avec l'eau puis laissez cuire à feu doux jusqu'à ce que le caramel ait bien fondu dans l'eau. Retirez 
alors du feu. 
\etape Disposez la préparation dans un moule à tarte. 
\etape Déposez la pâte feuilletée sur le dessus et enfoncez le surplus de pâte feuilletée sur les cotés afin de constituer un 
bord (en repoussant les bananes par la même occasion).
\etape Faites trois trous au couteau sur le milieu de la pâte. Il faut que le trou soit suffisamment important pour que l'air 
passe. À la fourchette par exemple, il faut étirer le trou avec la fourchette, piquer n'est pas suffisant.
\end{preparation}

\begin{cuisson}
Enfournez 25 minutes à 200°C
\end{cuisson}
\end{recette}

\section{Tiramisu}
\begin{recette}{Tiramisu}{4}{1h+24h}{}\index{tiramisu}\index{café}\index{thé}
\begin{ingredients}
\ingredient 250g de mascarpone
\ingredient 500 g de boudoirs (\~ 24)
\ingredient 3 œufs
\ingredient 50cl de café fort
\ingredient 100g de sucre en poudre
\ingredient 30g de cacao amer/pur (Van Houten)
\ingredient Rhum
\end{ingredients}

\begin{preparation}
\etape Séparez le blanc du jaune d'œufs
\etape Battez les blancs en neige\footnote{Les 
blancs sont prêts quand ils ne tombent pas en retournant le plat.} (avec une pincée de sel).
\etape Réservez les blancs puis à la place, mélangez les jaunes avec le sucre et faites blanchir
\etape Ajoutez le mascarpone au fouet
\etape Mélangez ensuite la préparation du mascarpone avec les blancs en neige délicatement. Enveloppez le tout de mouvement 
circulaires, on longeant les bords et le dessous du récipient avec de ne pas casser les blancs en neige.
\etape Prendre un moule (un plat à gratin ou quelque chose du genre) et saupoudrez le fond de Van Houten
\etape Trempez les boudoirs dans le café fort et le rhum (le mélange doit être froid) puis étalez-les sur le plat.
\begin{remarque}
Les boudoirs ne doivent pas être totalement imbibés, juste l'extérieur, donc ne les attardez pas trop dans le café.
\end{remarque}
\etape Étalez de la crème sur les boudoirs puis saupoudrez de Van Houten
\etape Répétez les deux dernières opérations jusqu'à épuisement des ingrédients (typiquement 2 couches)
\end{preparation}

\begin{remarque}
Préparez le Tiramisu la veille afin de le faire reposer au frigo au moins quelques heures.
\end{remarque}
\end{recette}

\section{Tourte des Pyrénées à la moi}
\begin{recette}{Tourte des Pyrénées à la moi}{3}{20 min}{1h}\index{Tourte des Pyrénées}
\begin{ingredients}
\ingredient 250 g de farine
\ingredient 175 g de beurre fondu
\ingredient 1 sachet de levure chimique
\ingredient 4 oeufs
\ingredient 175 g de sucre
\ingredient 3 cuillère à soupe de rhum vieux
\end{ingredients}
%source: https://www.saintpedebigorre-tourisme.com/tourte-pyrenees/
% modifié pour avoir les doses de tatie

\begin{remarque}
Les transvasages bizarres sont dûs au fait que j'ai un robot et donc je peux pas mélanger tout d'un seul coup dans le même bol.
\end{remarque}


\begin{preparation}
\etape Préchauffez le four à 170°C
\etape Montez les blancs en neige avec une pincée de sel. 
\etape Pendant ce temps, beurrez et farinez le moule à l'aide de la farine que vous utiliserez pour la suite
\etape Une fois les blancs montés, transvasez dans un autre saladier.
\etape Battre les jaunes d'œufs avec le sucre et le sucre vanillé
\etape Ajoutez la levure, le surplus de farine utilisé pour le moule (complétez pour avoir la bonne quantité), le beurre fondu et les parfums puis mélangez bien le tout.
\etape Incorporez les blancs montés en neige.
\etape puis versez la préparation.
\end{preparation}

\begin{cuisson}
Faites cuire 45 minutes à 170°C.
% 50 minutes c'était cuit, mais un peu sec
\end{cuisson}
\end{recette}


}% End of the ``group'' where section is deactivated


\chapter{Pâtes levées}
\minitoc

\newpage
\section{Techniques du pain}
\subsection{Levures}
\subsubsection{Principe}
La croissance des levures se décompose en 3 phases :
\begin{enumerate}
\item lag phase : C'est la première phase lors de l'introduction des levures dans un nouveau milieu. Dans cette phase, les levures sont actives mais ne se dupliquent pas. la durée de cette phase dépend de la population initiale de levures et des conditions environnementales (température, pH, alcool, oxygène, concentration de sel, nutriments,\dots)
\item La phase exponentielle : cette phase de croissance rapide est caractérisée par le temps de doublage (Generation Time)
\item La phase stationaire : la croissance rapide s'arrête, généralement à cause d'une haute densité de cellules
\end{enumerate}

La levure a deux modes de réplication. La respiration d'une part, très efficace qui consomme peu de glucides. Par fermentation 
alcoolique (pas besoin d'oxygène). Beaucoup moins efficace elle consomme beaucoup plus de glucides. 

La respiration permet une 
duplication efficace des levures. C'est la raison pour laquelle on introduit de l'air dans la pâte (rabattage). On crée ainsi 
des bulles d'air qui constitueront des bulles de levures qui vont ensuite croître exponentiellement lors de la cuisson. 

La fermentation alcoolique quant à elle donne du goût à la pâte. Il ne faut donc pas la négliger. 


\subsubsection{Formule}
La levure de boulanger fait partie de la famille \emph{saccharomyces cerevisiae}.

La formule générale est de la forme :
\begin{align}
\log(\mathrm{GT}) &= a + bT + cT^2
\end{align}
où GT, Generation Time est le temps pour doubler les levures.

Pour la souche AB1, la formule est :
\begin{align}
a&= 2.747 & b&= -0.1865 & c&= 0.00413
\end{align}

Ces données sont extraites de "Growth of Saccharomyces cerevisiae and Saccharomyces uvarum in a temperature gradient incubator" (R. M. Walsh et P. A. Martin, 11 octobre 1976, Journal of the Institute of Brewing). La souche AB140 a été écartée car c'est une autre famille. Ceci dit, je ne sais pas si les mesures effectuées sont parfaitement identiques pour les souches de levures de boulanger, et de quel ordre de grandeur sont les barres d'erreur compte tenu de l'incertitude des souches (il y a plusieurs sous variétés dans cette famille là).

L'évolution des levures dans la phase exponentielle s'écrit :
\begin{align}
N(t) &= N_0 e^{\left(\frac{t\ln(2)}{\mathrm{GT}}\right)}
\end{align}

\subsection{Types de farine}

\begin{center}
\begin{tabular}{|l|l|}\hline
Dénomination française & Dénomination italienne\\\hline
T45      &  Type 00   \\\hline
T55      &  Type 0       \\\hline
T65      &  Type 1      \\\hline
T110     &  Type 2           \\\hline
T150     &  Type Integral         \\\hline
\end{tabular}
\end{center}

Source: \url{http://www.boulangerie.net/forums/bnweb/dt/conversion/te.php}

\subsection{Temps de cuisson}
\begin{center}
\begin{tabular}{|l|l|l|l|}\hline
Catégorie de pain & Poids cru (base) & Poids cuit     &  Temps de cuisson \\\hline
Petit pain        & 70 gr            & 52.5 gr env    &  10 à 12 mn       \\\hline
Ficelle           & 175 gr           & 131.25 gr env  &  15 mn            \\\hline
Baguette          & 350 gr           & 262.5 gr env   &  18 à 20 mn       \\\hline
Pain              & 550 gr           & 400 gr env.    &  25 à 30 mn       \\\hline
Boule             & 1 kg             & 750 gr env     &  45 à 1H          \\\hline
\end{tabular}\end{center}

\subsection{Température de cuisson}
\begin{center}
\begin{tabular}{|l|l|}\hline
Catégorie de pain & Température de cuisson \\\hline
Petits pains de 50 à 120g & 270-280°C \\\hline
pains de 250 à 400g & 250-260°C \\\hline
Pains de plus de 600g & 230-240°C \\\hline
\end{tabular}\end{center}

\subsection{Problème et solution}
\begin{itemize}
\item Si la pâte a des bulles en surface à la fin de la cuisson, c'est qu'elle a subit un choc thermique, le four était trop 
chaud. 
\item Si la pâte n'est pas assez dorée à la fin du temps, c'est que le four n'était pas assez chaud (ou pas assez d'humidité)
\item si la pâte n'est pas assez montée, c'est soit un problème de levure, soit que la pâte était trop sèche (pas assez d'eau). 
La pâte doit rester souple (et l'élasticité provient des premières 20 minutes avant la toute première levée (autolyse)
\item Si la pâte colle vraiment trop aux doigts pendant le façonnage des patons, c'est qu'elle est trop liquide, il manque de 
la farine. Notez que lors du façonnage, il faut des mouvements vifs des doigts qui ne doivent pas rester trop en contact, sinon 
même la pâte correcte vous collera aux doigts. Un bon moyen de vérifier ça est de tapoter le dessus de la pâte avec le bout des 
doigts (comme la calebote de benny hill). En faisant ça, la pâte ne doit pas coller. 
\end{itemize}

\begin{itemize}
\item Afin d'être bien aérée, la pâte doit être relativement liquide. Trop compacte et le pain ne gonflera pas bien à la 
cuisson. Ça peut être déroutant au début, mais la pâte DOIT être liquide. Elle réépaissit via le sel, l'autolyse, et le 
rabattage qui est extrêmement important pour la tenue de la pâte. 
\item Il ne faut surtout pas trop faire monter votre pâte. Trop montée et il n'y aura quasiment pas de montée à la cuisson 
alors qu'elle doit bien gonfler à cette étape là 
\item La grigne peut être compliquée à faire avec la pâte liquide. Mettez la farine avant de grigner pour que ça colle moins, 
faites des mouvements vifs du couteau qui entaillent mieux et n'ont pas le temps de coller. Une grigne complètement 
longitudinale peut aussi être plus facile à faire
\item La couleur se fait lors de la cuisson grâce à un peu d'eau jetée dans le four au début. Pour cela, il suffit de mettre un 
lèche frite pendant le préchauffage. Une fois le pain enfourné, et juste avant de refermer la porte du four, jetez un tiers de 
verre d'eau très chaude. Il n'en faut pas trop sinon le pain va coller à la plaque et sera difficile à décoller. 
\item Cuisez à chaleur tournante parce qu'à cause du lèche frite, la chaleur ne va pas bien se diffuser en dessous si la 
chaleur tournante n'est pas activée, vous aurez donc un pain bien cuit sur le dessus, mais pas assez cuit en dessous. 
\item Rabattage : C'est une étape essentielle pour avoir un pain bien aéré et qui se tient. Il donne de la force à la pâte 
(elle est plus ferme), et ajoute de l'air dans la mie. En effet, il y a deux sortes de bulles dans la mie, les fines dues à la 
levure, et les plus grosses qui elles sont dues au rabattage. Pour les pâtes relativement liquide, il faut plusieurs rabattage 
(j'en fait 3)
\end{itemize}

Rabattage : Prenez la pâte, farinez la table, vos mains et le dessus de la pâte. Étirez jusqu'à avoir une plaque de pâte 
d'environ 50cm par 50cm. Plus c'est étiré (sans se déchirer) mieux c'est. Une fois étiré au maximum, rabattez les coins sur le 
milieu et faites partir le surplus de farine qui est resté collé (super important). Recommencez alors le rabattage des coins 
jusqu'à ne plus pouvoir. Retournez, faites tomber encore le surplus de farine, puis avec le bout des doigts, appuyez légèrement 
sur la boule par le dessous (là où se trouvent les jointures), jusqu'à avoir une belle boule que vous déposerez pour continuer 
la montée. Et ainsi de suite, un rabattage toutes les 20 minutes jusqu'à avoir la consistance voulue. 

Je n'ai pas trouvé d'impact significatif à faire ou ne pas faire les choses suivantes : 
\begin{itemize}
\item Je fais deux tas d'ingrédients, d'un coté farine et sel, mélangés dès le début. De l'autre, l'eau et la levure. Je fais 
un puit du premier et jette le liquide devant, que je recouvre des bords de farine pour que le mélange se fasse plus vite. Je 
n'ai pas noté de différence essentielle à ajouter le sel dès le début. Par contre, jamais le sel dans l'eau. 
\item Je ne fais pas partir l'air, à aucun moment. Je fais une montée au frigo, puis une 2e à l'air libre après avoir fait les 
rabattages, et enfin une montée rapide des pâtons. C'est tout. Et je ne martirise pas la pâte entre temps. 
\item Je ne fais pas plusieurs cuissons à plusieurs températures, une seule température (j'en faisais 2 au début, j'ai pas vu 
de différences au niveau du gout ou de la texture
\item J'ai essayé avec levain, levure fraiche ou de boulanger, je n'ai pas vu de différences. Le levain est beaucoup plus 
compliqué à gérer. On peut faire du polish pour s'en rapprocher. La levure fraiche exige de s'en servir assez souvent. Seule 
chose, diviser par deux les quantités de levure si on prend de la levure sèche par rapport à la levure fraiche. 
\item La levure se comporte très bien si on fait une montée au frigo, et l'autre avantage est qu'une pâte liquide se manipulera 
plus facilement si elle est à basse température, les rabattages seront alors plus faciles à faire.
\end{itemize}


\begin{center}
\begin{tabular}{|l|p{10cm}|}
\hline Défaut & Origine\\\hline
\multirow{4}*{croûte cloquée} & eau trop froide\\
 & manque de pointage (1\iere levée)\\
 & pain retombé à la mise au four\\
 & qualité de la farine\\\hline
\multirow{3}*{manque de volume} & manque d'apprêt (2\ieme levée)\\
 & pâte trop ferme\\
 & pâte trop froide\\\hline
\multirow{3}*{pain plat} & apprêt trop long\\
 & trop d'eau\\
 & pâton maltraité à l'enfournement\\\hline
\multirow{3}*{pain déchiré} & manque de buée\\
 & manque de pointage\\
 & coup de lâme trop rapprochés\\\hline
\multirow{2}*{pain trop acide} & réduire les temps de levée\\
 & changer de levain\\\hline
\multirow{4}*{manque d'acidité, de goût} & allonger les temps de levée\\
 & incorporer des farines plus complètes (T80, T120,\dots)\\
 & Utiliser de l'eau minérale si celle du robinet est trop chlorée (en dernier recours, le chlore limite l'activité du 
levain)\\
 & augmenter la proportion de levain\\\hline
\end{tabular}
\end{center}

% Beginning of group where section is deactivated
% This is only to get the good structure of the document 
% since ``section'' is in fact embedded in the 'recette' environment.
% This group allow us to deactivate sections ONLY in the given file and 
% not for the entire document.
{\renewcommand{\section}[1]{}

\section{Brioche}
\begin{recette}{Brioche}{3}{15h+8h}{35 min}\index{brioche}
%source: https://www.gourmandiseries.fr/brioche-maison-mie-filante-recette/
\begin{ingredients}
\ingredient 250g de farine T55
\ingredient 150g d'œufs ($\sim 3$ œufs) frais (il faut qu'ils soient bien frais, la pâte sera trop liquide sinon)
\ingredient le reste d'oeuf pour la dorure
\ingredient 5g de sel
\ingredient 10g de levure fraiche (ou 5g de levure sèche)
\ingredient 10g d'eau
\ingredient 20g de sucre
\ingredient 125g de beurre mou
\ingredient 14g de fleur d'oranger (1 cas)
\end{ingredients}


\begin{preparation}
\etape mélangez les 3 oeufs, puis pesez 150g et réservez le surplus dans un sac congélation au frigo pour la dorure le lendemain
\etape Mettez la levure avec l'eau, saupoudrez un peu de farine et de sucre puis mélangez. (ne pas mettre trop de farine sinon c'est trop épais)
\etape Mélangez dans le bol du robot la farine, le sel, le sucre et faites un puit. Ajoutez alors le mélange œufs/levure ainsi 
que les aromes si vous le souhaitez
\etape Mélangez grossièrement à la spatule (si vous pouvez mettre le liquide sous la farine au début c'est aussi mieux pour éviter un bloc de farine en dessous), Pétrissez jusqu'à homogénéiser
\etape Pétrissez alors 6 minutes à vitesse 2 (voire 4). Il faut que la pâte s'étire.
\etape Ajoutez le beurre mou (extrêmement mou) et pétrissez de nouveau 4 minutes à vitesse 2
\etape Laissez monter 1h à température ambiante et à couvert
\etape Dégazez la pâte, puis mettez à lever au frigo pendant 1h
\etape Dégazez de nouveau la pâte puis mettez au frigo jusqu'au lendemain
\etape Appuyez au centre de la boule jusqu'à faire un trou. Façonnez ensuite la couronne et disposez la dans un plat à tarte 
fariné.
\etape Dorez au jaune d'œuf
\etape Laissez lever pendant 1h30 de nouveau
\etape Avant d'enfourner, dorez de nouveau puis saupoudrez du sucre de canne (ou du sucre grain) sur le dessus
\end{preparation}


\begin{cuisson}
Préchauffez à 180°C avec un lèche frite en bas. Entaillez le dessus au couteau juste avant d'enfourner.

Versez 1/4 de verre au moment d'enfourner.

Enfournez pendant 2 minutes à 180°C à four ventilé. Puis faites cuire 28 minutes à 160°C

Pour tester la cuisson de la brioche, n’hésitez à la piquer avec un couteau, il doit ressortir sec sans trace de pâte.

Faites refroidir votre brioche sur une grille.
\end{cuisson}
\end{recette}

\section{Korvapuusti (Brioche roulée à la Cannelle)}% brioche finlandaise à la cannelle
\begin{recette}{Korvapuusti (Brioche roulée à la Cannelle)}{3}{3h}{}\index{brioche}
\begin{ingredients}
\ingredient[Brioche]
\ingredient 460g de farine (moitié T45, moitié T55)
\ingredient 221g d'eau
\ingredient 9g de sel
\ingredient 10g de levure sèche
\ingredient 64g de sucre
\ingredient 23g d'œufs (à peu près la moitié d'un oeuf. Mélanger blanc et jaune puis peser)
\ingredient 55g de beurre mou
\ingredient[Farce]
\ingredient 10g de cannelle
\ingredient 50g de sucre
\ingredient 50g de beurre mou
\end{ingredients}


\begin{preparation}
\etape Pesez et disposez la farine dans un bol de robot
\etape Diluez la levure dans l'eau légèrement tiède. Ajoutez une cuillère à soupe de farine et un peu de sel puis laissez reposer 15 minutes
\etape Mélangez le sel et le sucre dans la farine, puis creusez un puit et versez l'oeuf (gardez le reste pour dorer tout à l'heure). 
\etape Préparez les ingrédients pour la farce (sucre, beurre) pour que ça ramolisse.
\etape Pétrissez farine, sel, sucre, oeuf et levure dans l'eau pendant 5 minutes à vitesse 1
\etape Ajoutez le beurre et pétrissez à vitesse 2 pendant 10 minutes
\etape Transvasez dans un saladier couvert et laissez reposer pendant 30 minutes
\etape Dans le bol maintenant récupéré, utilisez la feuille et mélangez sucre, beurre et cannelle pour la farce. Le beurre doit être mou mais pas fondu. Laissez bien mélanger.
\etape Au bout des 30 minutes, étalez la pâte en un rectangle de 25-30cm de large à peu près. Comptez 4mm d'épaisseur. 
\etape Étalez la farce sur toute la pâte puis roulez dans le sens de la largeur (les 30 cm donc). 
\etape Ensuite, coupez des trapèzes de 5cm de long à peu près, dans un sens puis dans l'autre, sur toute la longueur du rouleau (\verb|/_\_/_\...|)
\etape Avec le plat d'une cuillère en bois, appuyez sur l'axe du trapèze pour l'aplatir sur le milieu et joindre les couches sans couper
\etape Disposez les sur une feuille de papier cuisson, sur une plaque allant au four. Couvrez et laisser reposer 1h environ à l'abris des courants d'air
\end{preparation}


\begin{cuisson}
Préchauffez le four 10 minutes à 200°C. 
Avant d'enfourner, vous pouvez dorer les brioches avec le reste d'œuf du début de recette. Je n'ai pas fait car la recette pro disait de ne pas le faire, mais les brioches sont moches et ternes. (ou alors jetez de l'eau chaude au fond du four au moment d'enfourner)
Enfournez environ 10 minutes à 200°C (dans mon cas j'ai dû faire 15 minutes mais mon four est en fin de vie je pense). 
\end{cuisson}
\end{recette}

\section{Gaufres Liégeoises}% recette céline
\begin{recette}{Gaufres Liégeoises}{3}{2h}{40 min}\index{Gauffres}
\begin{ingredients}[12 gaufres]
\ingredient 10 cl de lait tiède
\ingredient 5g de levure de boulanger déshydratée
\ingredient 40 g de cassonade
\ingredient 1/2 gousse de vanille
\ingredient 250 g de farine
\ingredient 100 g de beurre mou
\ingredient 1 oeuf
\ingredient 1 pincée de sel
\ingredient 70 g de sucre perlé
\end{ingredients}


\begin{preparation}
\etape Mettre dans le bol d'un robot : le lait, la levure, la cassonade,
les grains de vanille, la farine, l'oeuf et le sel. Pétrir pendant 5
minutes (on peut le faire à la main également). 
\etape Ajouter le beurre coupé en morceaux et pétrir à nouveau 5 minutes. Laisser lever dans un
endroit tiède pendant 1 heure environ.

\etape Ajouter le sucre perlé à la pâte et bien mélanger. Diviser la pâte
en une douzaine de boules de même taille et laisser reposer 20
minutes.

\end{preparation}


\begin{cuisson}
Huiler les plaques d'un gaufrier. Déposer une boule de pâte sur
chaque empreinte, et cuire 2 à 3 minutes de chaque côté (selon la
puissance de votre gaufrier). Les gaufres doivent être bien dorées.
\end{cuisson}
\end{recette}

% AUTRE RECETTE À TESTER
% http://cuisine.journaldesfemmes.com/recette/1001410-baguettes-maison-et-sans-map
% Ingrédients / pour 8 personnes
% 
%     500 g de farine boulangère
%     300 g d’eau
%     1 cuillère à soupe de sel
%     1 sachet de levure boulangère (déshydratée)
% 
% Réalisation
% 
%     Difficulté
%     Préparation
%     Cuisson
%     Repos
%     Temps Total
% 
%     Facile
%     20 mn
%     30 mn
%     1 h
%     1 h 50 mn
% 
% Préparation Baguettes maison et sans MAP
% 1
% Baguettes maison et sans MAP : Etape 1
% Mettez la farine, l’eau et le sel dans la cuve de votre robot (vous pouvez également faire le tout à la main), et pétrissez (crochet) pendant 3 minutes, jusqu’à ce que vous obteniez une boule. Couvrez votre cuve et laissez reposer 30 minutes.
% 2
% Baguettes maison et sans MAP : Etape 2
% Ajoutez la levure déshydratée, et pétrissez à nouveau à faible vitesse d’abord, puis à moyenne jusqu’à que la boule se réforme à nouveau. Huilez et farinez légèrement un bol et déposez-y votre boule de pâte. Laissez reposer pendant 2 a 3 heures dans un endroit sec (l’idéal pour moi est votre four), la pâte va doubler de volume. Une fois le temps de pousse écoulé, faites deux pâtons de pâte, et sur un plan de travail fariné, dégazez (travailler la pâte avec la paume de la main pour enlever l’air) et façonnez vos baguettes. Disposez-les sur une plaque, faites des petites entailles sur le dessus et laissez reposer de nouveau 15 a 20 minutes le temps de préchauffer le four.
% Pour finir
% Baguettes maison et sans MAP : Etape 3
% La cuisson est une étape importante… Pour avoir un beau résultat et un croquant/craquant, préchauffez votre four à 200° en pensant à mettre de l’eau dans votre lèchefrite. Une fois le préchauffage effectué, remettez de l’eau dans le lèche frite, et enfournez vos baguettes à mi hauteur pendant 25 a 30 minutes.

%%%%%%%%%%%%%%%%%%%%%%%
% Recettes d'olivia
% 
% Hier j'ai mis 500g de farine (dont 50g de complète, le reste de la 55)
% Un sachet de levure , 8g de sel et 315g d'eau
% 
% J'ai fais 5 min de frasage avec presque toute l'eau et toute la farine, 
% Autolyse de 20 min, pendant ce temps j'ai mis la levure dans le reste d'eau. 
% Pétrissage 10min vitesse 2 avec la levure. Sur la fin j'ai rajouté le sel.
% 1ère levée 1h
% Rabat
% 2eme levée 2h30
% Division en 3 patons, repos de 20 min
% Façonnage en baguettes
% Levée 1h
% Cuisson comme je t'ai dit hier !

\section{Naan au fromage}
\begin{recette}{Naan au fromage}{3}{45min+10h+3h}{30 min}\index{naan}\index{levure de boulanger}
\begin{ingredients}[3 naans]
\ingredient 200g de farine de blé T55
\ingredient 1 yaourt (pas light)
\ingredient 20g d'huile de tournesol
\ingredient 4g de sel 
\ingredient 4.5g de levure de boulanger sèche
\ingredient 50 ml d'eau tiède
\ingredient 3g de sucre
\ingredient 6 portions de kiri
\end{ingredients}


\begin{preparation}
\etape Mélangez la levure dans l'eau tiède avec le sucre
\etape Mélangez dans le bol du robot la farine, le sel, l'huile et le yaourt
\etape Ajoutez enfin le mélange de levure
\etape Pétrissez pendant 5 minutes à vitesse 1
\etape Laissez reposer pendant 20 minutes (autolyse)
\etape Pétrissez 10 minutes à vitesse 1, puis 4 minutes à vitesse 2
\etape Formez une boule et laissez reposer 3 heures à couvert
\etape Divisez en 3 pâtons de 140g environ
\etape Sur un plan de travail, étalez les patons au rouleau en galette d'un centimètre d'épais environ. 
\etape Mettez deux portions de kiri au centre puis refermée les galettes. 
\etape Étalez de nouveau en n'appuyant pas trop fort pour ne pas faire ressortir le fromage. Il faut que le naan soit relativement fin
\end{preparation}


\begin{cuisson}
Faites cuire à la poële pendant environ 5 minutes (j'ai fait dans une poële inox sans revêtement), à feu vif (7/9) en tournant régulièrement. 
\end{cuisson}
\end{recette}

\section{Pain rapide}
\begin{recette}{Pain rapide}{3}{45min+10h+3h}{30 min}\index{pain}\index{levure de boulanger}
\begin{ingredients}[2 baguettes]
\ingredient 375g de farine de blé T55
\ingredient 300mL d'eau 
\ingredient 12g de levure fraiche (6g de seche) avec temperature ambiante de 20°C
\ingredient 8g de sel 
\end{ingredients}


\begin{preparation}
\etape Dans un saladier, diluer la levure dans l'eau.
\etape Dans le bol du robot, mettre la farine et l'eau+levure
\etape Mélangez à la cuillère grossièrement puis pétrissez pendant 5 minutes jusqu'à enlever les grumeaux (c'est tellement liquide qu'il ne faut pas en attendre grand chose)
\etape Laissez la pâte lever à couvert pendant 1h30 
\etape Séparez votre pâte en deux puis versez dans un moule à baguette sans chercher à donner une forme
\end{preparation}


\begin{cuisson}
Préchauffez le four pendant 10 minutes à 240°C avec un lèche frite en bas.

Saupoudrez un peu de farine. Puis à l'aide d'un couteau tranchant, entaillez 
le pain en diagonale à plusieurs endroits, pas besoin de tailler trop profond. Des mouvements vifs du couteau permettent 
d'entailler plus 
facilement.

Juste après avoir mis les patons, mais avant de fermer la porte du four, versez un tiers (entre 1/3 et 1/2) de verre d'eau 
chaude sur le lèche 
frite.

Mettez alors le four à 240°C pendant 25 minutes à chaleur tournante.

Surveillez le pain et sortez-le du four quand la cuisson vous convient.


\end{cuisson}
\end{recette}

%recette de la campaillette grand siècle
\section{Pain 1 fournée}
\begin{recette}{Pain 1 fournée}{3}{2h30(+12h)+2h}{30 min}\index{pain}\index{levure de boulanger}
\begin{ingredients}[4 baguettes]
\ingredient 500g de farine de blé T65
\ingredient [facultatif] 5g de gluten pur
\ingredient 350g d'eau (30-35°C)
\ingredient 15g d'eau de bassinage
\ingredient 5g de levure sèche
\ingredient 9g de sel 
\end{ingredients}


\begin{preparation*}
\begin{tabular}{|c|M{6cm}|M{6cm}|}
\hline 
Etapes & Tout en une fois & nuit entre deux étapes \\ 
\hline 
 & \multicolumn{2}{M{12cm}|}{Mettez la farine et le sel dans un bol de robot.} \\ 
\hline 
 & \multicolumn{2}{M{12cm}|}{Dans un bol, délayez la levure dans l'eau tiède} \\ 
\hline 
 & \multicolumn{2}{M{12cm}|}{Versez alors l'eau dans la farine sans faire de puit} \\ 
\hline 
\textbf{Frasage}  & \multicolumn{2}{M{12cm}|}{Pétrissez pendant 5 minutes à vitesse 1} \\ 
\hline 
\textbf{Autolyse} & \multicolumn{2}{M{12cm}|}{Laissez reposer 1h} \\ 
\hline 
 & \multicolumn{2}{M{12cm}|}{Pétrissez pendant 10 minutes à vitesse 1} \\ 
\hline 
\textbf{Bassinage}  & \multicolumn{2}{M{12cm}|}{Ajoutez l'eau de bassinage, cuillère par cuillère (à café) en attendant que les bords du bol soient propres pour mettre la suivante, jusqu'à la fin du pétrissage au bout de 2 minutes à vitesse 2.} \\ 
\hline 
• & \multicolumn{2}{M{12cm}|}{(A chaque repos, je met une plaque en verre sur le bol du robot). Tous les rabattages sont séparés par 20 minutes, et le premier est fait au bout de 20 minutes après le début du pointage} \\
\hline
\textbf{Pointage} & 2h & (recommandé) 1h \\
\hline
\textbf{Rabattage} & 2-3 rabattages, jusqu'à ce que la pâte ait de la tenue & 1 rabattage. \\ 
\hline
\textbf{Blocage} & - & Mettez au frigo, couvert, pendant 12 à 14h à 4°C \\
\hline

\hline 
\textbf{Boulage} & \multicolumn{2}{M{12cm}|}{Découpez 3 morceaux égaux en volume et formez des boules sans les serrer} \\ 
\hline 
\textbf{Détente} & Laissez reposer 20 minutes & Laissez reposer 1h (au moins 13°C à coeur) \\ 
\hline 
• & \multicolumn{2}{M{12cm}|}{Déposez une couche de farine sur le moule à baguette à l'aide d'une passoire fine.
L'eau de bassinage rend la pâte beaucoup plus collante, il faut donc une petite couche.} \\  
\hline 
 \textbf{Façonnage}  & \multicolumn{2}{M{12cm}|}{Formez un boudin grossièrement sans trop le manipuler, puis saisissez une extrémité dans chaque main. Faites des 
mouvements de haut en bas pour que la pâte s'étire jusqu'à la longueur voulue, en prenant garde que la section reste à peu 
près constante partout} \\  
\hline 
\textbf{Apprêt} & \multicolumn{2}{M{12cm}|}{Disposez-les sur le moule fariné. 
Couvrez avec un torchon sec afin que ça ne sèche pas et que ça soit à l'abri de l'air. Laissez monter 30-45 minutes environ} \\  
\hline 
\end{tabular} 
\end{preparation*}

\begin{cuisson}
Préchauffez le four pendant 10 minutes à 240°C avec un lèche frite en bas.

Saupoudrez un peu de farine. Puis à l'aide d'un couteau tranchant, entaillez 
le pain en diagonale à plusieurs endroits, pas besoin de tailler trop profond. Des mouvements vifs du couteau permettent 
d'entailler plus 
facilement.

Juste après avoir mis les pâtons, mais avant de fermer la porte du four, versez un tiers (entre 1/3 et 1/2) de verre d'eau chaude sur le lèche frite.

Mettez alors le four à 250°C pendant 20 minutes avec la chaleur uniquement en dessous.
% le paton de 350g n'était pas assez cuit et de manière générale, le dessous était pas assez cuit et le dessus trop, en cuisant 22minutes à chaleur tournante à 240°C

Surveillez le pain et sortez-le du four quand la cuisson vous convient.

\textbf{Ressuage} Laissez le pain refroidir sur une grille (ou perpendiculairement au moule que vous avez utilisé) pendant 30 minutes pour bien évacuer l'humidité.
\end{cuisson}
\end{recette}

\section{Pain à hamburger}
\begin{recette}{Pain à hamburger}{3}{30 min+24h}{15 min}\index{Pain à hamburger}
\begin{ingredients}[10 pains à hamburger]
\ingredient 500g de farine T55
\ingredient 16g de levure boulangère sèche
\ingredient 10g de sel
\ingredient 25g de sucre
\ingredient 20cl d'eau tiède
\ingredient 8cl de lait tiède
\ingredient 1 oeuf
\ingredient 30g de beurre ramolli (pommade)
\end{ingredients}

\begin{preparation}
\etape Pesez la farine dans un bol de robot.
\etape Dans un bol, délayez la levure dans l'eau et le lait tiède, ajoutez deux cuillères à soupe de la farine pesée et laissez reposer 10 minutes. 
\etape Pendant ce temps, pesez le sel et le sucre et mélangez le à la farine. Puis faites un puit. 
\etape Versez l'eau dans le puit puis effondrez les bords du puits dans l'eau (sinon ça met trop de temps à se mélanger)
\etape Ajoutez dans la foulée l'oeuf entier préalablement battu
\etape Une fois l'oeuf bien incorporé ajoutez le beurre pommade
\etape Battez la pâte à vitesse 2 pendant 5 minutes (la pâte doit alors se détacher des parois de la cuve)
\etape Couvrez avec du film alimentaire et laissez poussez 2h à température ambiante
\etape Faites les pâtons (environ 90g par pâton)
\etape Disposez-les sur du papier cuisson et appuyez légèrement sur chaque boule pour en faire un disque de 1cm d'épais environ. 
\etape Laissez reposer 1h
\end{preparation}

\begin{cuisson}
Faites préchauffer le four à 200°C pendant 10 minutes avec un lèche frite tout en bas

Au moment d'enfourner (vous devrez sans doute faire 2 fournées de 5 pains), jetez 1/3 de verre d'eau chaude (la plus chaude du robinet) sur le lèche frite juste après avoir mis les patons au four et juste avant de fermer la porte. 

Faites cuire 10 minutes à 200°C (moi je fais avec chaleur tournante).
\end{cuisson}
\end{recette}

\section{Pâte brisée}\label{sec:pate_brisee}
\begin{recette}{Pâte brisée}{3}{30 min+24h}{15 min}\index{pâte brisée}
\begin{ingredients}[1 pâte normale]
\ingredient 250 g de farine de blé
\ingredient 125g de beurre
\ingredient 50g d'eau
\ingredient 1 pincée de sel
\ingredient 1 jaune d'œuf
\end{ingredients}

\begin{preparation}
\etape 30 minutes avant, sortez le beurre du frigo afin qu'il ramolisse. 
\etape Dans le bol du robot, mettez la farine, la pincée de sel, le beurre en dés et le jaune d'œuf. 
\etape Pétrissez à vitesse 1 avec la feuille (pas le crochet, l'autre). 
\etape Quand ça ne semble plus se mélanger, rajoutez l'eau. 
\etape Laisser reposer au frais 30 minutes minimum avant de l'abaisser. Pour cette étape, je met dans un sac congélation. Il 
est possible à ce moment là d'un congeler une partie pour avoir une pâte brisée toute prête plus tard. 
\end{preparation}
\end{recette}

\label{sec:pate_lasagne}
\section{Pâtes à Lasagnes}
\begin{recette}{Pâtes à Lasagnes}{4}{20min.+20min.}{}\index{lasagne}\index{pâte à lasagne}
\begin{ingredients}
\ingredient 400g de farine T55
\ingredient 4 œufs
\ingredient 2 cuillères à soupe d'huile d'olive
\ingredient 10g de sel
\end{ingredients}

\begin{preparation}
\etape Faites un puit avec la farine et le sel. Ajoutez les œufs et l'huile d'olive. 
\etape Faites effondrer les bords puis pétrissez jusqu'à ce que le mélange devienne homogène.
\etape Formez une boule et mettez-la dans un sac congélation. Laissez reposer 20 minutes.
\etape Etalez au maximum cette boule, puis coupez en lamelle quand votre plan de travail devient trop petit. 
\etape Arrêtez quand la pâte fait entre 1-2mm d'épaisseur. 
\end{preparation}
\end{recette}

\section{Pâte à Pizza}
\begin{recette}{Pâte à Pizza}{3}{30 min+48h}{15 min}\index{pizza}\index{pâte à pizza}
\begin{ingredients}[Pour deux petites pizza]
\ingredient 340 g de farine T55
\ingredient 10 g de farine complète % 3% de farine complete (dans la recette de l'atelier, c'est 15%)
\ingredient 210ml d'eau
\ingredient 1 cuillère à soupe d'huile d'olive
\ingredient 2g de levure sèche (de boulanger donc)
\ingredient 7g de sel
\end{ingredients}

\begin{preparation}
\etape Dans le bol du batteur, pesez la farine
\etape Mettez la levure dans l'eau puis ajoutez 2 cuillères à soupe de la farine précédemment pesée. Laissez reposer jusqu'à ce que ça fasse de la mousse en surface (pour une levure active, 10-20 minutes suffiront, sinon, 1 ou 2h pourraient être nécessaire)
\etape ajoutez alors le sel et la cuillère d'huile dans la farine restante
\etape Quand la levure est prête, versez le tout dans le bol, puis pétrissez 4 minutes à vitesse 1 (la bouillasse du centre va petit à petit incorporer la farine des cotés, d'abord très lentement, puis très rapidement à la fin. Faites manuellement une boule à la fin s'il reste de la chapelure sur les cotés à la fin du temps).
\etape Laissez alors reposer 20 minutes (autolyse)
\etape Pétrissez 10 minutes à vitesse 2. 
\etape Faites alors deux boules individuelles (de 280g normalement).
\etape sur un plan de travail propre et sans farine, effectuez des rabattages pour chacune des deux pâtes et mettez les dans un tupperware individuel.
\etape Laissez alors la pâte lever tranquillement au frigo pendant 48h.

\etape Sortez la pâte 30 minutes à 3 heures avant (en fonction de la saison).
\etape Faites préchauffer le four à 280°C (le four des pros est à plus de 500°C, moi c'est limité à 280).
\etape Je n'arrive pas à la faire à la main comme les pizzaiolo, elle se déchire, donc je l'étale au rouleau à patisserie.
%\etape Laissez reposer la pâte étalée une vingtaine de minutes avant de garnir afin que la pâte lève un peu, elle sera ainsi 
%plus aérée. (j'étalais un peu la pate, à peu près la moitié du diamètre voulu, je laisse reposer, puis je fini le travail après)

\etape Garnissez la pizza selon votre goût (évitez absolument tout les liquides, y compris le coulis de tomate et la mozarella imbibée sinon la pâte n'est pas bien) et enfournez à 280°C. 
\end{preparation}

\begin{cuisson}
Faites cuire la pizza environ 10 minutes à 280\degres C au four, chaleur tournante si vous avez. À ces températures, surveillez 
la cuisson, si le four est bien chaud ça peut être fait en 5 minutes à peine.

La pizza sera meilleure si vous faites cuire la pizza sur une pierre, chauffée au préalable. Utilisez une spatule à crêpe pour 
faire glisser la pâte depuis la pelle à pizza vers la pierre que vous aurez enlevé du four le temps d'y mettre la pizza. 

\begin{attention}
Ne faites pas cuire avec le four en mode grill !
\end{attention}

\end{cuisson}
\end{recette}



}% End of the ``group'' where section is deactivated


\chapter{Confitures}
\minitoc

% Beginning of group where section is deactivated
% This is only to get the good structure of the document 
% since ``section'' is in fact embedded in the 'recette' environment.
% This group allow us to deactivate sections ONLY in the given file and 
% not for the entire document.
{\renewcommand{\section}[1]{}

\section{Confiture de Fraise}
\begin{recette}{Confiture de Fraise}{0}{}{}\index{fraise}\index{confiture}
\begin{ingredients}
\ingredient 1 kg de fraises
\ingredient 1kg de sucre (1kg7 pour 2kg de fraises)
\end{ingredients}

\begin{preparation}
\etape Laver, équeuter et couper les fraises en morceaux
\etape Dans un chaudron assez grand pour contenir tous les ingredients, mettez les fraises puis versez le sucre dessus.
\etape Laissez mariner à couvert pendant 12h
\end{preparation}


\begin{cuisson}
\begin{enumerate}
\item Mixez les fraises
\item Lavez et stérilisez les pots et couvercles, puis retourner les pots sur un torchon propre (sans les essuyer, et surtout pas l'intérieur).
\item Chauffez sur feu moyen-vif le chaudron jusqu'à ce que le mélange entre en ébullition

\item Poursuivre la cuisson pendant 20 min environ en écumant la mousse rougeâtre qui se forme.
\item La confiture est cuite quand elle atteint 105°C (ou technique de la perle sur la cuillère, mais la température est beaucoup plus simple)
\item Verser dans des pots stérilisés (mettez beaucoup de confiture, fermez puis retournez les pots).
\end{enumerate}
\end{cuisson}
\end{recette}


\section{Confiture mangue/passion}
\begin{recette}{Confiture mangue/passion}{0}{}{}\index{fraise}\index{confiture}
\begin{ingredients}
\ingredient 2kg de mangue (juste la pulpe, sans peau ni noyau)
\ingredient 250ml de pulpe de fruit de la passion
\ingredient 1kg de sucre
\ingredient 1 citron vert (son jus)
\end{ingredients}

\begin{preparation}
\etape Vous pouvez nettoyer et peler les mangue trop mure sur plusieurs semaines, et les congeler au fur et à mesure, puis décongeler quand vous voulez faire la confiture
\etape Dans un chaudron assez grand pour contenir tous les ingredients, mettez les mangue puis versez le sucre dessus.
\etape Laissez mariner à couvert pendant au moins 1h
\end{preparation}


\begin{cuisson}
\begin{enumerate}
\item Ajoutez la pulpe de fruit de la passion et le citron dans les mangues
\item Lavez et stérilisez les pots et couvercles, puis retourner les pots sur un torchon propre (sans les essuyer, et surtout pas l'intérieur).
\item Chauffez sur feu moyen-vif le chaudron jusqu'à ce que le mélange entre en ébullition

\item Poursuivre la cuisson en écumant la mousse qui se forme (j'ai dû faire recuire donc je n'ai pas de temps de cuisson précis. Au moins une heure je dirais).
\begin{remarque}
J'ai filtré la confiture pour enlever les fibres de mangues qui sont désagréables en bouche (ce qui filtre les pépins aussi).
\end{remarque}
\item La confiture est cuite quand elle atteint 105°C (ou technique de la perle sur la cuillère, mais la température est beaucoup plus simple)
\item Verser dans des pots stérilisés (mettez beaucoup de confiture, fermez puis retournez les pots).
\end{enumerate}
\end{cuisson}
\end{recette}

\section{Confiture de pomme au citron}
\begin{recette}{Confiture de pomme au citron}{5}{}{}\index{confiture}\index{pommes}
\begin{ingredients}
\ingredient $3\unit{kg}$ de pommes vertes
\ingredient $3\unit{kg}$ de sucre
\ingredient $1\unit{L}$ d'eau
\ingredient $4$ citrons (zeste et jus)
\end{ingredients}

\begin{preparation}
\etape Épluchez et sortez les trognons des pommes. Mettez les épluchures et les trognons dans un sac de mousseline.
\etape Coupez les pommes et mettez les dans une casserole avec l'eau, le sac de mousseline, le zeste rapé et le jus du citron.
\etape Faire cuire jusqu'à ce que les pommes soient tendres.
\etape Laissez refroidir un peu, puis sortez le sac et tordez le pour extraire le jus et le rajouter dans la casserole.
\etape Ajoutez le sucre et remuez. 
\etape La confiture est cuite quand elle atteint 105°C (ou technique de la perle sur la cuillère, mais la température est beaucoup plus simple)
\end{preparation}
\end{recette}

\section{Confiture de poire}
\begin{recette}{Confiture de poire}{5}{}{}\index{confiture}\index{pommes}
\begin{ingredients}
\ingredient $2.5\unit{kg}$ de poires (pelées et épépinées
\ingredient $1.5\unit{kg}$ de sucre
\ingredient jus de 1 citron
\end{ingredients}

\begin{preparation}
\etape Pelez et épépinez les poires puis calculez la quantité de sucre à partir du poids de la chair restante.
\etape Mettez le sucre, le citron et les fruits à macérer quelques heures avant de cuire (pensez à mélanger)
\etape Faire cuire dans une marmite non couverte
\etape La confiture est cuite quand elle atteint 105°C (ou technique de la perle sur la cuillère, mais la température est beaucoup plus simple)
\end{preparation}
\end{recette}

\section{Marmelade d'orange à la moi}
\begin{recette}{Marmelade d'orange à la moi}{5}{}{}\index{confiture}\index{orange}
\begin{ingredients}[2 pots]
\ingredient \~ 500mL de sirop restant après fruits confits (voir \refsec{orange_confite})
\ingredient zeste d'un citron
\end{ingredients}

\begin{preparation}
\etape Faites chauffer le sirop et le jus de citron
\etape La confiture est cuite quand elle atteint 105°C (ou technique de la perle sur la cuillère, mais la température est beaucoup plus simple)
\etape Mettez dans les pots, fermez et retournez les pour qu'ils refroidissent.
\end{preparation}
\end{recette}

}% End of the ``group'' where section is deactivated

\chapter{Divers}
\minitoc

% Beginning of group where section is deactivated
% This is only to get the good structure of the document 
% since ``section'' is in fact embedded in the 'recette' environment.
% This group allow us to deactivate sections ONLY in the given file and 
% not for the entire document.
{\renewcommand{\section}[1]{}

\section{Chocolat chaud}
\begin{recette}{Chocolat chaud}{3}{20 min.}{}\index{chocolat chaud}
\begin{ingredients}[1 personne]
\ingredient 250mL de lait (ou 10cl crème liquide et le reste de lait)
\ingredient 15g de chocolat pur
\ingredient 24g de sucre
\end{ingredients}

\begin{preparation}
\etape Dans une casserole, mettez le lait et le chocolat à dessert en petits morceaux. Faites chauffer jusqu'à obtenir une 
préparation homogène, sans toutefois faire bouillir. 
\end{preparation}

\end{recette}


}% End of the ``group'' where section is deactivated


\chapter{Confiseries}
\minitoc

% Beginning of group where section is deactivated
% This is only to get the good structure of the document 
% since ``section'' is in fact embedded in the 'recette' environment.
% This group allow us to deactivate sections ONLY in the given file and 
% not for the entire document.
{\renewcommand{\section}[1]{}

\section{Calisson à la moi}
\begin{recette}{Calisson à la moi}{3}{2h30}{}\index{calisson}
\begin{ingredients}
\ingredient[orange confite]
\ingredient 2 orange
\ingredient 400g de sucre
\ingredient 500ml d'eau
\ingredient[calisson]
\ingredient 300g d'amande en poudre
\ingredient 180g de sucre glace
\ingredient 2 à 3 cuillères à soupe de fleur d'oranger
\ingredient 280g d'orange confite (ou moins)
\ingredient[glaçage]
\ingredient 150g de sucre glace
\ingredient 3 cuillère à soupe d'eau tiède. 
\end{ingredients}

\begin{preparation}
\etape Faites chauffer le sirop (eau + sucre) pour l'orange confite
\etape Peler et couper l'orange en tranches. 
\etape Une fois à ébullition, faites cuire les tranches d'orange à feu moyen (juste pour ébullition) pendant 1h environ (il faut que le sirop soit toujours liquide et l'orange pas trop dure, sinon impossible à mixer)
\etape Mixer le sucre au moulin à café et verser avec la fleur d'oranger dans la casserole. Faire chauffer un peu
\etape Mixer l'amande au moulin à café et rajouter dans la casserole quand le sucre et le liquide sont homogènes. Cuire à feu doux environ 8 minutes jusqu'à l'obtention d'une pâte qui ne colle plus aux doigts.B
\begin{remarque}
Rajouter un peu d'eau si la fleur d'oranger ne suffit pas
\end{remarque}
\etape Mixer l'orange confite
\etape Mélanger alors les deux (chauffer un peu si nécessaire), puis étaler sur un papier cuisson
\etape Mixez le sucre glace avec un peu d'eau tiède jusqu'à l'obtention d'une pâte lisse et homogène. 
\end{preparation}

\begin{cuisson}

\end{cuisson}
\end{recette}

\section{Crème de marron}
\begin{recette}{Crème de marron}{4}{2h}{}\index{chataîgne}\index{crème de marron}\index{marron}
\begin{ingredients}
\ingredient 500g de marron cuits à l'eau
\ingredient $400$ g de sucre
\ingredient 50cl d'eau
\end{ingredients}

\begin{preparation}
\etape Faites cuire les marrons à l'eau, puis pelez les.
\etape Préparez le sirop avec le sucre et l'eau et portez à ébullition
\etape Plongez les marrons dans le sirop, et faites frémir pendant 1h30 environ (sans couvrir).
\etape À l'issue de la cuisson, rajoutez de l'eau si c'est trop épais, puis mixez le tout
\end{preparation}
\end{recette}



\section{Ladoo à la noix de coco}
\begin{recette}{Ladoo à la noix de coco}{4}{30min.+1 nuit}{30min.}\index{ladoo}\index{noix de coco}
\begin{ingredients}[$\sim$ 20 boules]
\ingredient 120g (474mL) de noix de coco rapée + du rab' pour rouler les boules dedans
\ingredient 320g (avant 240) de lait concentré
\ingredient 1 cac d'essence de vanille (ou parfum divers, e.g. fruit de la passion)
\ingredient 1/2 cac de cardamome
\end{ingredients}
% J'ai fait avec poudre d'amande (230g) pour 240g de lait concentré. J'ai ensuite emrobé de sucre glace, poudre de cacao ou sucre en poudre (je sais pas encore lequel est le mieux). j'ai mis une cuillere a soupe d'alcool d'orange

\begin{remarque}
Ne pas faire ça avec de la noix de coco que vous rapez vous même. Ce sera trop humide et gras, et les boules se font soit très mal, soit pas du tout.
\end{remarque}



\begin{preparation}
\etape Mélangez tous les ingrédients dans le bol. 
\begin{remarque}
 Pour le lait concentré, commencez avec 120mL et ajoutez en plus si la consistance ne permet pas de faire des boules
\end{remarque}
\etape Préparez un bol d'eau (pour mouiller les mains et que ça n'accroche pas. En plus, ça permet aux ladoo d'être moins sec) et un bol de noix de coco rapée.
\etape Faites de boules de la taille qui vous convient, puis roulez-les dans la noix de coco râpée.
\end{preparation}
\end{recette}

\section{Meringues au chocolat}
\begin{recette}{Meringues au chocolat}{3}{20 min.}{1h}\label{sec:meringues}\index{meringues}
\begin{ingredients}
\ingredient 5 blancs d'œufs
\ingredient 250g de sucre
\ingredient une pincée de sel
\ingredient 600g de chocolat à dessert
\end{ingredients}

\begin{preparation}
\etape Dans un saladier, ajoutez les blancs d'œufs le sucre et le sel. Battez le tout à vitesse relativement importante jusqu'à ce que ce soit prêt
\etape Sur une plaque de cuisson, disposez du papier sulfurisé et préparez vos meringues à l'aide de deux cuillères à café. La préparation doit normalement se tenir. 
\end{preparation}

\begin{cuisson}
Le temps de cuisson est très variable. Comptez 1h30 à 120°C pour des meringues bien cuites. Moins si vous les 
souhaitez moelleuses.

Une fois cuit. Laissez reposer une à deux heures. Faites fondre le chocolat (sans ajouter d'eau ou quoi que ce soit)

Afin d'enrober les meringues, déposez les dans le chocolat fondu pour recouvrir le dessous, puis recouvrez la meringue de chocolat à l'aide d'une spatule. Sortez là du chocolat à l'aide de la spatule et d'une cuillère pour la faire glisser sur le papier sulfurisé, puis recommencez pour toutes les meringues. Il vous manquera du chocolat pour 4 ou 5 meringues normalement. Posez les ensuite sur un papier cuisson pour laisser refroidir. 
\end{cuisson}
\end{recette}

\section{Pruneaux à l'eau de vie}
\begin{recette}{Pruneaux à l'eau de vie}{3}{20 min.}{1h}
\begin{ingredients}
\ingredient 300g de pruneaux assez gros et bien moelleux
\ingredient 50cl d'eau de vie à 40°
\ingredient 1 clou de girofle ou un morceau de cannelle
\ingredient du thé fort (environ 1L et 4 sachets de thé)
\ingredient 200g de sucre cristallisé
\end{ingredients}

\begin{preparation}
\etape Faire une infusion de thé assez forte en y ajoutant le clou de girofle  qui peut être remplacé par un petit morceau d'écorce de cannelle si l'on en préfère le goût.
\etape Lorsque le thé est prêt, enlever le clou de girofle et verser l'infusion chaude mais non bouillante sur les pruneaux. Ils doivent être complètement recouverts de thé. Laisser tremper jusqu'au lendemain.
\etape Egoutter les pruneaux qui doivent être bien gonflés.
\etape Placer le sucre cristallisé dans une assiette creuse et y rouler les pruneaux qui en seront complètement enrobés; les laisser sécher à plat sans qu'ils se touchent quelques heures.
\etape Les piquer de quelques coups d'une grosse aiguille afin de faciliter la pénétration de l'alcool.
\etape Les placer avec précaution dans un bocal fermant bien. Les couvrir d'eau de vie. Boucher hermétiquement et laisser macérer au minimum 3 mois. 
\end{preparation}

\end{recette}

\section{Rose des Sables au chocolat}
\begin{recette}{Rose des Sables au chocolat}{0}{}{}\index{rose des sables}\index{chocolat}
\begin{ingredients}[6 pers.]
\ingredient 250g de chocolat patissier
\ingredient 250g de margarine
\ingredient 200g de sucre glace
\ingredient 500g de pétales de maïs soufflé (corn flakes)
\end{ingredients}

\begin{preparation}
\etape Faites fondre le chocolat et la margarine
\etape Incorporez le sucre glace hors du feu
\etape Mettez le tout dans un grand saladier et versez-y une bonne quantité de pétale de maïs
\etape Mélangez bien. Rajoutez-en s'il y a encore trop de chocolat liquide au fond du saladier
\etape Disposez en petits tas sur une feuille de papier aluminion et placez-les au frigo une demi-heure.
\end{preparation}

Les roses des sables se gardent 2 ou 3 jours dans une boite au réfrigérateur.
\end{recette}


\section{Tranches d'oranges confites}
\begin{recette}{Tranches d'oranges confites}{4}{}{}\index{orange confites}\index{oranges}\index{fruits confits}\label{orange_confite}
\begin{ingredients}
\ingredient 4 oranges
\ingredient $400$ g de sucre
\ingredient 35cl d'eau
\end{ingredients}

\begin{remarque}
Ceci est une recette à ma façon, qui ne nécessite pas d'avoir des oranges bios et qui plaira à ceux qui n'aiment pas le zeste. On doit pouvoir mettre plus de tranches d'oranges que je ne l'ai fait mais pas beaucoup. J'avais 12 tranches quand j'ai fait cuire dans la casserole.
\end{remarque}


\begin{preparation}
\etape Pelez les oranges sans les ouvrir en deux.
\begin{remarque}
Lors de la découpe de la tête de l'orange, tranchez le pourtour, décollez délicatement le tour avec le dos du couteau puis soulevez le couvercle. En faisant ainsi vous enlèverez une bonne partie de la tige blanche de l'intérieur de l'orange.
\end{remarque}
\etape Coupez les oranges en tranches assez épaisse, environ 5mm (trop fin et les tranches ne résisteront pas à la cuisson. Elles seront bonnes mais en lambeaux).
\etape Préparez le sirop avec le sucre et l'eau et portez à ébullition
\etape Plongez les tranches dans le sirop, et faites frémir (3/9), à couvert pendant 1h30 environ.
\etape À l'issue de la cuisson, sortez délicatement les tranches à l'aide d'une spatule (ça permet de ne pas les abîmer et de les égoutter en même temps) puis déposez-les sur du papier sulfurisé afin de les laisser sécher et refroidir
\begin{remarque}
Vous pouvez les passer légèrement sous l'eau pour enlever la couche de caramel collant si vous voulez les enrober plus tard (mais pas trop sinon elles sont délavées)
\end{remarque}
\end{preparation}

\begin{remarque}
\begin{itemize}
\item On peut enrober les tranches d'orange de chocolat. Il faut alors les laisser sécher sur une grille 20 minutes au four à 180°C puis laisser sécher dehors pendant 4 heures avant de les enrober.
\item Une fois les tranches cuites, on peut récupérer le sirop et le transformer en caramel pour ne pas le gâcher. Vous obtiendrez alors un caramel parfumé à l'orange. Ce caramel sera par contre peut-être un peu mou.
\end{itemize}
\end{remarque}
\end{recette}


\section{Truffes au chocolat}
\begin{recette}{Truffes au chocolat}{0}{20 min.}{2h}\index{truffes}\index{chocolat}

\begin{ingredients}[6 pers.]
\ingredient 200 g de chocolat à croquer
\ingredient 100 g de beurre
\ingredient 80 g de sucre
\ingredient 4cl de crème fraiche liquide
\ingredient 2 jaunes d'œufs
\ingredient 30g cacao en poudre
\end{ingredients}

\begin{preparation}
\etape Cassez le chocolat en petits morceaux et faites fondre dans un saladier au dessus d'une casserole d'eau, au bain marie. Ajoutez au chocolat le beurre et une cuillère à soupe de crème fraiche.
\etape Faites fondre le sucre avec la crème fraiche liquide dans une autre casserole, puis incorporez au chocolat quand les grains de sucre sont dissous. 
\etape Remuer soigneusement jusqu'à ce que ce soit homogène.
\etape Placez le saladier 2 heures au réfrigérateur.
\etape Faire des boules à l'aide de deux cuillères à café? Puis entourez de poudre de cacao, et roulez la truffe dans la paume de la main. Secouer les truffes pour les débarrasser du surplus et les remettre au frigo.
\end{preparation}
\end{recette}

}% End of the ``group'' where section is deactivated


\chapter{Chocolat}
\minitoc

\newpage
\section{Techniques du chocolat}
Le chocolat, en particulier le beurre de cacao, possède 6 formes de cristallisations différentes résumées dans \reffig{fig:cristallisation_chocolat}.

\begin{figure}[htb]
\centering
\includegraphics[width=0.6\linewidth]{figures/cristallisation_chocolat.pdf}
\caption{Différentes formes de cristallisation du chocolat. Seules les formes V et VI nous intéressent car elles sont plus compacte, le chocolat devient alors craquant et brillant.}\label{fig:cristallisation_chocolat}
\end{figure}

Le tempérage ou pré-cristallisation est une technique de travail du chocolat afin de sélectionner ses formes de cristallisation. Ceci va permettre d'obtenir un chocolat brillant et craquant.

\begin{enumerate}
 \item Faire fondre le chocolat à 45°C: On fait fondre et on prépare le chocolat
 \item Abaisser la température à 27°C: On commence la cristallisation
 \item Remonter la température à 30-31°C: On fait fondre tous les cristaux autre que les formes V et VI. 
 \begin{attention}
Notez le gap entre la forme IV qui fond à 28°C et la forme V qui ne fond qu'à partir de 32°C, vous avez donc un peu de marge mais il ne faut surtout pas faire fondre la forme V sous peine de devoir recommencer tout le cycle.
 \end{attention}

\end{enumerate}

\section{Tempérage par ensemencement}
%source: https://blog.cerfdellier.com/temperage-du-chocolat-avec-la-technique-de-lensemencement/

\begin{itemize}
\item Faites fondre 3/4 de votre paquet de chocolat noir de couverture à 50 / 55°C.
\item La technique de l'ensemencement consiste à incorporer le dernier quart des pistoles dans le chocolat fondu. Mélangez le chocolat jusqu'à une fonte complète des pistoles.
\item Une fois fondues, les pistoles auront refroidi le chocolat, qui atteindra une température de 32°C et mettra au point votre chocolat.
\end{itemize}

\begin{remarque}
\begin{itemize}
\item Si le chocolat a blanchi, c’est que vous êtes descendu en dessous des 28°C, ce qui cause une sur-cristallisation.
\item Si le chocolat ne se démoule pas, c'est qu'il y a sous-cristallisation, vous n'êtes pas descendu à 28°C.
\end{itemize}
\end{remarque}

\section{Tempérage chocolat avec tempéreuse Stadter}\label{sec:tempereuse}
% source: https://www.mercotte.fr/2008/04/18/et-si-on-parlait-chocolat-le-temperage-pourquoi-et-comment/

\begin{attention}
Ajoutez un peu d'eau entre le récipient plastique et celui en inox afin de mieux conduire la chaleur. Je n'ai pas encore testé et peut-être que la température sera plus proche de la valeur attendue (au lieu des 7 degrés de différences en moyenne)
\end{attention}


Coupez le chocolat en petit morceaux. Il existe normalement une règle des 2/3 pour l'ensemencement mais ça n'a pas fonctionné du tout pour moi donc je fais tout fondre à la tempéreuse.
\begin{remarque}
\begin{itemize}
\item J'utilise le chocolat lidl. Pour le noir, ne pas prendre le noir intense, mais le chocolat noir de base à 3.45 euros/kg. et pour le lait, les tablettes à emballage bleu.
\item Ne rien mettre dans le chocolat en train de fondre sinon ça perturbe un peu tout. il vaut mieux mélanger à la fin.
\item J'utilise une tempéreuse Stadter et j'ai dû calibrer les températures pour avoir la correspondance entre température du thermostat et température réelle.
\end{itemize}
\end{remarque}

\begin{table}[hb]
\centering
\begin{tabular}{|c|c|c|}
\hline 
Etape & Chocolat au lait & Chocolat noir \\\hline
Faire fondre au bain marie chaud & th 35°C (vrai 42°C)   & th 36°C (vrai 45°C) \\\hline
Refroidir                        & th 19°C (vrai 26°C)   & th 20°C (vrai 27°C)   \\\hline
Réchauffer au bain marie chaud   & th 23°C (vrai 30°C)   & th 24°C (vrai 31°C)   \\\hline
\end{tabular} 
\caption{Réglages thermostat pour la tempéreuse Stadter. Ces valeurs ne sont pas la température du chocolat mais juste le réglage de la machine. Les véritables températures attendues sont en parenthèse.}
\end{table}

% Correspondances mesurées:
% Th ; vrai T°C
% 40 ; 47.9
% 38 ; ~46
% 37 ; 45?
% 24 ; 31 (Chocolat solidifié)
% 25 ; ?

% \begin{tabular}{|c|c|c|}
% \hline 
% Etape & Chocolat au lait & Chocolat noir \\\hline
% Faire fondre au bain marie chaud & 42°C   & 45°C \\\hline
% Refroidir                        & 26°C   & 27°C   \\\hline
% Réchauffer au bain marie chaud   & 30°C   & 32°C   \\\hline
% \end{tabular} 
% \end{preparation}
% \end{recette}


% Beginning of group where section is deactivated
% This is only to get the good structure of the document 
% since ``section'' is in fact embedded in the 'recette' environment.
% This group allow us to deactivate sections ONLY in the given file and 
% not for the entire document.
{\renewcommand{\section}[1]{}

\section{Ganache}
\begin{recette}{Ganache}{4}{1h}{}\index{chocolat}
\begin{ingredients}
\ingredient 90g de chocolat noir
\ingredient 100g de crème liquide entière
\ingredient 9g de miel
\ingredient 25g de beurre
\end{ingredients}

\begin{preparation}
\etape Pesez et faites fondre le chocolat
\etape Faites bouillir la crème et le miel, puis versez progressivement le mélange sur le chocolat fondu
\etape Mélangez au fouet (vertical pour ne pas faire rentrer d'air) jusqu'à l'obtention d'une texture crémeuse et lisse
\etape Ajoutez le beurre en petits morceaux et mélangez délicatement pour faire fondre le beurre petit à petit
\etape Laissez tempérer un peu si c'est trop chaud avant de remplir vos chocolats avec.
\end{preparation}
\end{recette}

\section{Praliné}
\begin{recette}{Praliné}{4}{1h}{}\index{chocolat}
\begin{ingredients}
\ingredient 200g de poudre de noisettes
\ingredient 75g de sucre glace
\ingredient 125g de chocolat au lait
\ingredient 45g de beurre de cacao
\end{ingredients}

\begin{preparation}
\etape Torréfiez les fruits secs au four à 165°C pendant 20 minutes en mélangeant à mi-cuisson. Laissez ensuite refroidir
\etape Mixez les noisettes et le sucre glace afin d'obtenir une pâte onctueuse
\etape Mixez le chocolat fondu et le beurre de cacao
\etape Laissez tempérer un peu si c'est trop chaud avant de remplir vos chocolats avec.
\end{preparation}
\end{recette}

\section{Chocolat fourrés}
% Notes prises lors de l'atelier chocolat
%Pour la tempéreuse il faut pas faire la descente de température avec ça. Soit ensemencement, soit dehors, sans bain marie d'eau car humidité, et remuer sans arrêt mais pas trop fort
%
%Dans la tempéreuse, il faut aussi remuer. Elle met à 32 degrés.
%
%Pour les chocolats. Il faut mettre la première couche horizontale pour que l'épaisseur soit la même partout. Ensuite racler puis retourner
%
%La ganache il faut remuer au fouet, verticalement pour ne pas faire rentrer de bulles
%
%Pour mélanger le chocolat dans le lait, il faut pas verser tout le lait d'un coup mais en deux ou trois fois si beaucoup. 
%On peut faire infuser des trucs dans le lait, notamment du laurier sauce mais ensuite il faut passer au tamis et peser de nouveau. 
%
%Le praliné c'est sucre glace, noisette mixé (ou amande. Mais les noix sont plus grasses donc il faut faire attention), du beurre de cacao et du chocolat
%
%Il faut remettre au frais entre chaque couche. Pour la liaison, il faut faire fondre au décapeur thermique pour que la partie superficielle de la coque soit brillante. Puis verser une bonne couche sur la moitié la plus proche de nous. Pris repousser le chocolat avec la spatule à 45 degrés vers l'avant (pas vers nous)
%
%Pour le tempérage, il faut abaisser à 28 puis dès que c'est à 28 remonter sans attendre. Et ensuite remuer de temps en temps à 32 degrés. Toutes les 5 10 minutes. 
%Faut vraiment faire attention à l'humidité. Utiliser des gants pour la partie en contact avec le chocolat car nos mains sont humides. 
%
%Les doses de ganache de la recette sont pour 4 plaques je pense
\begin{recette}{Chocolat fourrés}{4}{3h}{}\index{chocolat}
\begin{ingredients}
\ingredient 200g de chocolat fondu (fondu, puis refroidi et enfin réchauffé)
\ingredient ganache ou praliné
\end{ingredients}

\begin{preparation}
\etape Moulez les coquilles du moule avec du chocolat noir et laisser refroidir à l'envers
\begin{itemize}
\item Avec un sopalin, essuyez chacun des moules (le creux) pour être sûr que c'est parfaitement propre et lisse
\item avec le moule bien à l'horizontale, Mettez une bonne couche de chocolat, puis renversez le moule pour vider le surplus
\item avec une spatule, raclez le surplus (toujours avec le moule à l'envers)
\item Déposez le moule sur un papier cuisson, à l'envers, pour le laisser solidifier un peu
\end{itemize}
\etape Remettre à l'endroit puis au frigo pendant que vous préparez la garniture
\etape Remplir les moules avec votre garniture à l'aide d'une poche à douille, en laissant un espace d'1 à 2mm sur le dessus
\begin{itemize}
\item Ne pas y aller trop doucement sinon ça fige sans avoir eu le temps de se mettre bien plat
\item enlever les bavures pour que ce soit bien net (sinon ça soude pas bien)
\end{itemize}
\etape Laissez refroidir pendant 10 minutes au frigo
\etape Déposez une fine couche de chocolat noir sur le moule pour fermer les coquilles
\begin{itemize}
\item Avec un décapeur thermique ou un sèche cheveux, faites fondre la partie superficielle de la coque jusqu'à ce qu'elle redevienne brillante (afin d'avoir une meilleure soudure)
\item Mettre une bonne couche de chocolat sur la moitié la plus proche de vous
\item Avec la spatule, repoussez le chocolat vers le fond, la spatule orientée de 45° vers l'avant
\item Si ce n'est pas parfaitement refermé et que c'est encore brillant, vous pouvez souder avec le dos d'une cuillère délicatement
\end{itemize}
\end{preparation}
\end{recette}

\section{Chocolat aux fruits secs}
% source: https://www.mercotte.fr/2008/04/18/et-si-on-parlait-chocolat-le-temperage-pourquoi-et-comment/
\begin{recette}{Chocolat aux fruits secs}{4}{1h}{}\index{chocolat}\index{raisin secs}\index{amandes entières}
\begin{ingredients}
\ingredient 300g de chocolat (noir ou au lait) ; C'est le minimum pour le tempérage, plus, c'est encore mieux
\ingredient 65g d'amande (ou noisette, ou autre)
\ingredient 55g de raisin (ou cranberries, ou autre)
\end{ingredients}

\begin{preparation}
\etape Pesez les fruits secs à l'avance
\etape Préparez le chocolat (tempéré) (voir \refsec{sec:tempereuse})
\etape Prendre un peu de chocolat et l'étaler dans le moule pour bien appuyer et imprimer sur le motif
\etape Ajoutez alors la garnitude dans le chocolat restant ou dans le moule selon préférence et versez
\begin{remarque}
Faute de mieux, utilisez du papier cuisson ou une plaque silicone, étalez alors une petite couche à la spatule de la même manière.
\end{remarque}
\end{preparation}
\end{recette}


\section{Crunch maison}
\begin{recette}{Crunch maison}{4}{10 min.}{}\index{crunch}\index{chocolat}
\begin{ingredients}
\ingredient 400g de chocolat au lait
\ingredient 70g de riz soufflé (à faire vous même ou achetez des céréales "rice crispies")
\end{ingredients}

\begin{preparation}
\etape Faites fondre le chocolat (voir \refsec{sec:tempereuse})
\etape Si vous utilisez un moule, étalez alors un peu de chocolat directement pour bien remplir les motifs du moule avant de mélanger le riz soufflé.
\etape Ajoutez le riz soufflé, puis étalez dans des moules à chocolat ou sur une plaque en silicone (ou papier sulfurisé).
\end{preparation}
\end{recette}

}% End of the ``group'' where section is deactivated


\chapter{Cocktails}
\minitoc

\newpage
\section{Vieillir alcool}
\subsection{Vieillir}
L'idée est de profiter de la torréfaction des tonneaux existant et de les récupérer au tarif le plus bas possible. Pour l'instant, je me tourne vers les copeaux de tonneaux de whisky vendus pour le barbecue ou le fumage pour donner du goût. Ces copeaux ont déjà perdu la plupart de leurs tanins et ça va nous éviter à ce que nos alcools aient un gout de bois. 

\begin{remarque}
Ne tentez pas de torréfier vous même du chêne, ça sera trop fort. A moins de savoir exactement ce que vous faites, cette méthode est plus sûre
\end{remarque}

La concentration de copeaux est une affaire de goût. Pour l'instant, j'ai fait 24g/L et au bout de deux ans ce n'était toujours pas top. J'ai donc augmenté la dose et je fais maintenant 37g/L (+50\%)

\bigskip

Comptez 37g de copeaux de tonneaux de whisky par litre. 
Mettez les copeaux dans l'alcool, notez la concentration, la date de début et la concentration en alcool. Je conseille de faire vieillir directement à 40\% vol pour pouvoir gouter au fur et à mesure. Pour l'avoir testé avec des alcools à 60\% vol, la dilution complique les choses et dilue le goût je trouve. 


\begin{tabular}{|c|c|c|p{7cm}|}
\hline 
Copeaux & Durée & Alcool & Résultat \\ 
\hline 
24g/L & 2 an & 62.5\% vol. & Manque de goût une fois dilué, mais prometteur \\ 
\hline 
37g/L & ? & 43\% vol. & TODO \\ 
\hline 
\end{tabular} 

\subsection{Dilution}
Pour diluer, achetez de l'eau minérale la moins minérale possible (regardez les résidus). 

source: Daniel HAESINGER et \\
\url{http://mapassionduverger.fr/transformation/la-reduction-des-eaux-de-vie/}


\begin{itemize}
\item Choisissez une eau minérale avec une dureté inférieure à 7\% en évitant le calcium et le magnésium qu'elle peut contenir, par exemple \textbf{Mont Roucous}
\begin{remarque}
1\% de dureté équivaut environ à 17.8mg/L de résidu. Il faut donc une eau avec moins de 125mg/L de résidus pour la dilution.
\end{remarque}
L'eau minérale que j'utilise a les caractéristiques suivantes (en mg/L):
\begin{itemize}
\item[\textbf{Calcium}] :  4.7
\item[\textbf{Magnésium}] :  1.8
\item[\textbf{Sodium}] : 5.9
\item[\textbf{Potassium}] : 2.8
\item[\textbf{Hydrogénocarbonates}] : 40.3
\item[\textbf{Chlorures}] : 1.2
\item[\textbf{Sulfates}] : 0.2
\item[\textbf{Nitrates}] : 0.5
\item[\textbf{Résidu sec à 180°C}] : 74
\item[\textbf{pH}] : 7.5
\end{itemize}
\item Faites bouillir l'eau, puis laissez refroidir. La porter à ébullition enlève l'oxygène et dépose un peu de calcaire sur la casserole (j'ai lu ça mais je n'en suis pas totalement convaincu).
\item Laissez refroidir au maximum, puis diluer en ajoutant l'eau, doucement, dans l'alcool (jamais l'inverse pour éviter qu'elle se trouble). % source: http://mapassionduverger.fr/transformation/la-reduction-des-eaux-de-vie/ et Daniel HAESINGER
\end{itemize}

Faites ensuite votre calcul de dilution. Gardez à l'esprit que la dilution sera fausse ; la table de Gay-Lussac donne les vraies propotions. Ceci est dû au fait que les molécules d'eau et d'alcool vont se lier légèrement et prendre moins de place que séparément. Dans la pratique, cette erreur représente quelques millilitres par litre, donc une erreur inférieur à 0.5\%. Personnellement, je néglige cette erreur.

% Beginning of group where section is deactivated
% This is only to get the good structure of the document 
% since ``section'' is in fact embedded in the 'recette' environment.
% This group allow us to deactivate sections ONLY in the given file and 
% not for the entire document.
{\renewcommand{\section}[1]{}

\section{Confiture de vieux garçon}
\begin{recette}{Confiture de vieux garçon}{0}{20 min.}{}

\begin{ingredients}[$\gtrsim$2L de liqueur]
\ingredient 1L de rhum blanc à 50\% vol (ou autre alcool équivalent)
\ingredient 250mL d'eau (dilue le rhum à 40\% vol)
\ingredient 300g de sucre de canne
\ingredient fruits divers: pêche, abricot, fraise, framboise, poire, cerises, raisins, reine claude.
\end{ingredients}

\begin{preparation}
\etape Ajoutez les fruits  et le sucre au rhum pur et à l'eau et laissez mariner 3 mois environ
\etape Si vous n'avez pas de problème de place, vous pouvez rajouter l'eau dedans aussi. Sinon, ajoutez l'eau plus tard
\end{preparation}
\end{recette}

\section{Limoncello}
\begin{recette}{Limoncello}{0}{20 min.}{}\index{limoncello}

\begin{ingredients}
\ingredient 4L d'alcool de fruit à 40°
\ingredient 20 citrons non traités
\ingredient[sirop]
\ingredient 2kg de sucre en poudre
\ingredient 2L d'eau
\end{ingredients}

\begin{preparation}
\etape Ne prendre que les zestes des citrons, les mettre dans l'alcool et laisser macérer une vingtaine de jours. Filtrez
\etape Filtrer
\etape Mélangez le sucre avec 2L d'eau
\etape Portez à ébullition, laissez frémir 10 minutes
\etape Lorsqu'il est froid, mélangez avec l'alcool. 
\etape Attendre 2 à 3 jours avant de mettre en bouteille
\end{preparation}
\end{recette}

\section{Mojito}
\begin{recette}{Mojito}{0}{20 min.}{}\index{mojito}

\begin{ingredients}
\ingredient 1/2 citron coupé en quatre
\ingredient 2 cc de sucre de canne
\ingredient 5 feuilles de menthe
\ingredient 6cl de rhum (blanc)
\ingredient glace pilée
\ingredient 6cl d'eau pétillante
\end{ingredients}

\begin{preparation}
\etape Ecraser au pilon le citron, le sucre et la menthe
\etape Rajouter le rhum
\etape Ajouter de la glace pilée et un peu d'eau pétillante
\end{preparation}
\end{recette}

\section{Planteur}
\begin{recette}{Planteur}{0}{20 min.}{}\index{Rhum}\index{planteur}

\begin{ingredients}[$\gtrsim$2L de rhum arrangé]
\ingredient 90cl de rhum blanc à 40\% vol (ou 70cl de rhum à 50\% vol)
\ingredient 1L de jus d'orange
\ingredient 1L de jus d'ananas
\ingredient 1L de jus de goyave
\ingredient 1L de jus de fruit de la passion
\ingredient gousse de vanille ou 2cas d'extrait de vanille maison
\ingredient 150g de sucre de canne
\ingredient bâton de cannelle
\ingredient 2 citron verts (entier si non traité, sinon juste le jus)
\end{ingredients}

\begin{preparation}
\etape Mélangez tous les ingrédients et laissez macérer 2 à 4 jours avant de déguster frais.
\end{preparation}
\end{recette}

\section{Porn star martini}
\begin{recette}{Porn star martini}{0}{20 min.}{}\index{vodka}\index{fruit de la passion}\index{maracudja}

\begin{ingredients}
\ingredient 5cl de pulpe de fruit de la passion (sans pépin)
\ingredient 2cl de vodka
\ingredient 1cl de sirop vanille
\ingredient 4 cuillère à café de sucre de canne
\end{ingredients}

\begin{preparation}
\etape Mélangez tous les ingrédients
\end{preparation}
\end{recette}

\section{Punch à la moi}
\begin{recette}{Punch à la moi}{0}{20 min.}{}\index{Rhum arrangé}\index{Punch}

\begin{ingredients}[$\gtrsim$2L de rhum arrangé]
\ingredient 1L de rhum blanc à 50\% vol
\ingredient 250mL d'eau (dilue le rhum à 40\% vol)
\ingredient 300g de sucre de canne
\ingredient 5g de citron
\ingredient 3 fruits de la passion
\ingredient 1 ananas victoria
\ingredient 1 mangue
\ingredient 1 gousse de vanille (ou une 1/2 si dose doublées)
\end{ingredients}

\begin{preparation}
\etape Ajoutez les fruits  et le sucre au rhum pur et laissez mariner 1 mois environ
\etape Si vous n'avez pas de problème de place, vous pouvez rajouter l'eau dedans aussi. Sinon, ajoutez l'eau plus tard
\end{preparation}
\end{recette}

\section{Punch coco}
\begin{recette}{Punch coco}{0}{20 min.}{}\index{Punch coco}

\begin{ingredients}[$\sim$ 3L]
\ingredient 80cl de crème de coco 
\ingredient 30cl de rhum 50\% 
\ingredient 60cL de lait de coco
\ingredient 500g de lait concentré sucré
\ingredient 1 gousse de vanille
\ingredient zeste d'un citron vert
\end{ingredients}

\begin{preparation}
\etape Mélanger tous les ingrédients sauf le rhum. 
\etape Ajoutez la moitié du rhum, goutez, et ajoutez du rhum à votre convenance. Il faut viser environ 18° d'alcool
\end{preparation}
\end{recette}

\section{Rhum à la mûre}
\begin{recette}{Rhum à la mûre}{0}{20 min.}{}\index{Rhum à la mûre}

\begin{ingredients}
\ingredient 160g de mûre
\ingredient 200g de rhum à 50\% vol
\ingredient 150g de sucre
\ingredient 5g de citron
\end{ingredients}

\begin{preparation}
\etape Ecraser les mûres
\etape Rajouter le rhum, le sucre et le citron
\etape Une semaine plus tard, filtrez pour enlever les pépins
\end{preparation}
\end{recette}

\section{Sangria}
\begin{recette}{Sangria}{4}{20 min.+ 24h}{}\index{Sangria}
% Pour alvaro, c'était: 3 bouteilles de 1.5L de vin, 1 bouteille de jus d'orange, pas toute la bouteille de limonade. du triple sec pour compenser je sais plus quoi. orange, citron et mangue car pas de peche. Pas de cannelle. 

\begin{ingredients}[2L]
\ingredient 75cl de vin rouge
\ingredient 25cl de limonade
\ingredient 20cl de jus d'orange
\ingredient 10cl de cointreau
\ingredient 2 oranges
\ingredient 1/2 citron jaune
\ingredient 2 peches
\ingredient 50g de sucre en poudre
\end{ingredients}

\begin{preparation}
\etape Lavez les fruits. Coupez les en morceaux
\etape Versez le vin rouge, le cointreau et le jus d'orange dans un grand récipient
\etape Ajoutez la cannelle, le sucre en poudre et les fruits
\etape Laissez macérer 24h au frais
\etape Au moment de servir, ajoutez la limonade fraiche et éventuellement des glaçons
\end{preparation}
\end{recette}

\section{Schrubb}
\begin{recette}{Schrubb}{4}{20 min.+ 6 jours}{45 min.}\index{Schrubb}\index{orange}

\begin{ingredients}[2L]
\ingredient[Rhum]
\ingredient 25g d'écorce d'oranges sèches non traitées (3 oranges sans le blanc)
\ingredient 1L de rhum blanc à 50°
\ingredient 253 ml d'eau minérale
\ingredient[Sirop]
\ingredient 670g de sucre
\ingredient 670ml d'eau
\ingredient Vanille, Cannelle, zeste de citron
\end{ingredients}

\begin{preparation}
\etape Pelez les oranges (si possible sans économe qui racle l'écorce et enlève le zeste) pour avoir les peaux les plus fines possibles (avec le moins de blanc possible --- Le blanc est amer).
\etape Faites les sécher pendant 3 jours environ. 
\etape Mettez à macérer les peaux d'oranges dans du rhum blanc pendant 3 jours au soleil (pas trop sinon l'amerture apparaitra). 
\etape Faites bouillir le sucre de canne avec l'eau, de la cannelle, de la vanille et des zestes de citron. 
\etape Faites cuire environ 45min à feu moyen. 
\etape Mélangez le sirop avec le rhum, puis complétez si besoin avec de l'eau pour que ça fasse 2L.
\begin{remarque}
En fonction de l'évaporation lors du sirop, il y aura plus ou moins de liquide, ça permet de rééquilibrer
\end{remarque}
\end{preparation}
\end{recette}

\section{Soupe de champagne}
\begin{recette}{Soupe de champagne}{4}{}{}\index{champagne}

\begin{ingredients}[6 personnes;33cl de préparation par bouteille]
\ingredient 75cl de mousseux
\ingredient 8cl de cointreau
\ingredient 2 citrons vert pressés (9cl)
\ingredient[16cl de Sirop]
\ingredient 135g de sucre de canne
\ingredient 135g d'eau
\end{ingredients}

\begin{preparation}
\etape Préparez le sirop de canne. 
\etape Rajoutez le cointreau et les deux citrons vert. En tout, vous devez avoir 33cl de préparation, rajoutez de l'eau au besoin
\etape Mettez au frais le mousseux et la préparation
\etape Au moment de servir, ouvrez le champagne et versez dans la préparation.
\end{preparation}
\end{recette}

\section{Vin chaud}
\begin{recette}{Vin chaud}{5}{20 min.}{}\index{Vin chaud}\index{orange}
\begin{ingredients}
\ingredient 75cl de vin
\ingredient 125 g de sucre
\ingredient 15cl de jus d'orange (jus d'une orange)
\ingredient 1/2 zeste d'orange
\ingredient 1/2 zeste de citron
\ingredient 2 bâton de cannelle
\ingredient 1 petit morceau de gingembre 
\ingredient 1 clou de girofle
\ingredient 1 Etoile d'anis (badiane)
\end{ingredients}

\begin{preparation}
\etape Mettez tout dans le vin
\etape Faites chauffer  la préparation à feu moyen jusqu'à frémissement, puis laissez frémir 5 minutes
\end{preparation}
\end{recette}

\section{Vin d'orange}
\begin{recette}{Vin d'orange}{4}{3 semaines}{}\index{apéritif}\index{orange}
\begin{ingredients}
\ingredient 5 bouteilles de vin blanc doux
\ingredient 1L d'eau de vie à 40°
\ingredient 3 ou 4 oranges non traitées
\ingredient 1kg de sucre en poudre
\end{ingredients}

\begin{preparation}
\etape Préparer une bouteille plastique d'eau minérale de 5 litres.

\etape Prendre les écorces de 3 ou 4 oranges non traitées et les introduire dans la bouteille pour les faire macérer dans 1 litre d'eau de vie.

\etape Laisser macérer pendant trois semaines, en agitant le mélange à peu près une fois par jour.

\etape Au bout de trois semaines ajouter à la préparation 5 bouteilles de 75 cl de vin blanc doux (12° de préférence) et du sucre en poudre (on peut aller jusqu'à 1 kg si ça rentre dans la bouteille).

\etape Il faut bien mélanger pour que le sucre soit bien absorbé par le liquide.

\etape Il ne reste plus qu'à transvaser la préparation dans les bouteilles de vin blanc. 
\end{preparation}
\end{recette}

}% End of the ``group'' where section is deactivated

\chapter{Techniques culinaires}
\minitoc
\newpage

\section{Astuces}
\subsection{Faire cuire de la viande}\index{viande!faire cuire}
\begin{itemize}
\item La poële doit être bien chaude avant de mettre la viande
\item Ne pas piquer la viande en cours de cuisson, ça laisse échapper le sang et ça fait durcir la viande
\item Ne jamais couvrir.
\end{itemize}

\subsection{Faire revenir des morceaux de viandes}\index{viande!faire revenir}
Quand on les fait revenir, c'est juste pour faire dorer l'extérieur, pas besoin que l'intérieur soit cuit, ça sera fait quand la sauce mijotera.

\subsection{Peler des tomates}\label{sec:peler_tomate}\index{tomate}
Il suffit de faire bouillir de l'eau dans une marmite. Préparez un récipient d'eau froide à coté.
Entaillez le bas de la tomate (pas le coté où il y a la tige verte) en faisant une croix au couteau (ça permet à la peau de commencer à se décoller par là. 

En mettant les tomates par deux ou trois, plongez les dans l'eau bouillante une dizaine de seconde, sortez les avec une écumoire et plongez les dans l'eau froide. 

\begin{remarque}
Comme les tomates restent très peu de temps dans l'eau bouillante, elle ne sont pas chaude une fois sorties du récipient d'eau froide, et la peau se décolle très facilement. Il ne faut donc pas les laisser trop longtemps dans l'eau bouillante, de sorte que vous ne vous bruliez pas en les pelant immédiatement après.
\end{remarque}

\subsection{Préparer du confit}
\begin{enumerate}
 \item Placer les conserves au bain marie pour faire fondre la graisse et pouvoir en extraire délicatement les cuisses
 \item Placer les cuisses dans une poële pour enlever le surplus de graisse et faire un peu dorer la peau, environ 5 minutes à feu vif
 \item Mettre les cuisses de confit dans un plat, coté peau vers le haut, au four pendant 10 minutes à 200°C
\end{enumerate}



\subsection{Épaissir une sauce}\index{sauce!épaissir}\index{épaissir}
Il suffit \emph{avant} de rajouter du liquide, de rajouter une cuillère à soupe rase de farine que vous mélangez avec les ingrédients (oignons, champignons, lardon, ou autre). Puis vous rajoutez le liquide en suivant la recette normale. Ça a le double avantage de permettre d'avoir plus de sauce et de l'épaissir (sinon, avoir plus de sauce implique qu'elle soit très liquide ce qui n'est pas forcément agréable).

\subsection{Ne pas rater les oignons}\index{oignon!faire revenir}
Le but est de chasser l'eau de l'oignon cru. Il faut donc faire revenir sans le couvercle. Il faut regarder les émanations de vapeurs pour avoir une idée du temps restant. Il ne faut pas que le feu soit trop fort par rapport au rythme avec lequel vous remuez les oignons, sinon ça crame. 

Si malgré tout vous avez un peu cramé les oignons, vous pouvez récupérer un peu en ajoutant un peu d'eau qui va un peu homogénéiser le tout. Mais si vous les avez complètement cramés, ça aura quand même le goût.

\subsection{Rétroengineering des recettes}
Exemple, vous avez une tablette de nougat et vous voulez en déduire les doses des ingrédients individuels à partir de la liste des ingrédients au dos de la boite et les informations nutritionnelles. 

Le site \url{https://informationsnutritionnelles.fr/amande} permet d'avoir les informations nutritionnelles des ingrédients individuels.

\begin{remarque}
Ne pas oublier que sans matière grasse, les oignons vont cramer de toute façon.
\end{remarque}

\section{Choisir les ingrédients}
\subsection{Noix de coco}
Les noix de coco fraiches (crémeux dedans) se distinguent des noix de coco sèches en regardant la coque de fibre. Si c'est bien lisse, l'intérieur sera crémeux, si c'est ondulé, alors ça a commencé à sécher.

\section{Les saveurs}
\subsection{Températures et saveurs}
En fonction des saveurs, voici à quelle température maximale elles peuvent être soumise sans perdre leurs qualités:

\begin{tabular}{|r|c|p{7cm}|}
\hline 
Cuisson & Température & Saveurs \\\hline 
On peut chauffer & >250°C & vanille, épices, thym, laurier \\\hline 
Pas trop (ne pas faire bouillir) & < 80°C & poireau, céleri \\\hline 
\multirow{2}*{On peut pas (fin de cuisson)} & < 45°C & Herbes aromatiques (coriandre, , aneth, estragon, ciboulette, basilic, cerfeuil) \\\cline{2-3} 
 & <48°C & Citron \\\hline 
\end{tabular} 

\subsection{Mariage des saveurs: Foodpairing}

\section{L'autocuiseur}\index{autocuiseur}
\begin{attention}
\begin{itemize}
\item Ne jamais utiliser l'autocuiseur avec une quantité de liquide inférieure à 25cl (un grand verre).
\item Ne jamais remplir l'autocuiseur au delà des $\sfrac{2}{3}$ de sa capacité.
\end{itemize}
\end{attention}

\reftab{tab:cuisson_autocuiseur} regroupe les temps de cuissons indicatifs de quelques aliments. Attention cependant. Pour les pommes de terre en particulier, c'est extrêmement dépendant de leur taille. Pensez à les faire cuire un peu plus longtemps (ou à les couper en morceaux plus petits) si elles sont grosses.

\begin{enumerate}
\item Introduire les aliments avec la quantité d'eau nécessaire à la cuisson et fermer l'autocuiseur
\item Placez sur feu vif jusqu'à l'apparition d'un échappement de vapeur par le régulateur.
\item Diminuez la puissance du feu de manière à ne maintenir qu'un léger échappement de vapeur par le régulateur 
\begin{remarque}
Le régulateur peut alors également ne fonctionner que par intermittence.

En plus d'économiser de l'énergie, ça permet de ne pas perdre l'eau trop rapidement, et donc de risquer l'évaporation totale du liquide.
\end{remarque}
\item Commencez alors à décompter le temps de cuisson.
\end{enumerate}

\begin{remarque}
Si vous utilisez du seul pour votre recette, faites le dissoudre immédiatement en remuant l'eau avec une cuillère en bois. Vous éviterez ainsi l'apparition de \og piqûres\fg qui pourraient altérer le fond de votre autocuiseur. La dissolution du sel est également plus facile et plus rapide en utilisant du seul fin et en salant l'eau chaude.
\end{remarque}

\begin{table}[htb]
\centering
\begin{tabular}{|l|r||l|r|}
\hline
\multicolumn{4}{|c|}{\textbf{LÉGUMES}}\\\hline
Artichauds & 12 min & Haricots secs & 20 min\\\hline
Asperges & 5 min & Haricots verts & 10 min\\\hline
Carottes & 15 min & Lentilles & 15 min\\\hline
Choux & 10 min & Petits-pois & 15 min\\\hline
Choux-fleurs & 4 min & Poireaux & 8 min\\\hline
Endives & 10 min & Pommes de terre & 16 min\\\hline
Épinards & 20 min & Salade & 20 min\\\hline
Haricots frais & 8 min & Salsifis & 25 min\\\hline
\end{tabular}

\bigskip

\begin{tabular}{|l|r||l|r|}
\hline
\multicolumn{4}{|c|}{\textbf{VIANDES}}\\\hline
\multicolumn{2}{|c||}{\textsc{Boeuf}} & \multicolumn{2}{c|}{\textsc{Porc}}\\\hline
Bourguignon & 40 min & Gras double & 35 min\\\hline
Pot au feu & 50 min & Potée & 40 min\\\hline
Rôti (1kg) & 10 min & Rôti (1kg) & 30 min\\\hline
\multicolumn{2}{|c||}{\textsc{Mouton}} & \multicolumn{2}{c|}{\textsc{Veau}}\\\hline
Blanquette & 20 min & Blanquette & 20 min\\\hline
Pieds & 20 min & Marengo & 20 min\\\hline
Ragoût & 25 min & Rôti (1kg) & 25 min\\\hline
\end{tabular}

\caption{Différents temps de cuisson pour les aliments dans un autocuiseur. Décomptez le temps dès que l'autocuiseur a atteint sa pression de fonctionnement (début de l'échappement de vapeur par le régulateur).}\label{tab:cuisson_autocuiseur}
\end{table}

\section{Kitchenaid Artisan}
Certaines sections sont valables quel que soit le robot, d'autres sont adaptés au Kitchenaid Artisan. Pour donner un ordre 
d'idée, les vitesses vont de 1 à 10.

Utilisez la vitesse 1 jusqu'à ce que les ingrédients soient correctement mélangés. Augmentez ensuite progressivement la vitesse 
jusqu'à la position désirée. Ajoutez les ingrédients aussi près des parois que possible et non pas directement sur la zone du 
batteur plat en mouvement. 

\subsection{Batteur}
\subsubsection{Pâtes levées}
Utiliser le crochet pétrisseur et la vitesse 2 (pas plus). 

\subsubsection{Mélanges liquides}
Elles doivent être mélangées à basse vitesse afin d'éviter les éclaboussures. N'augmenter la vitesse qu'une fois le mélange 
épaissi. 

\subsubsection{Ajout de noix, raisins secs ou fruits confits}
Doit être ajouté au cours des dernières secondes, à la vitesse 1. La pâte doit être suffisamment épaisse pouré viter que les 
fruits ne tombent au fond du moule durant la cuisson. Les fruits collants doivent être saupoudrés de farine pour une meilleure 
répartition dans la pâte.

\subsubsection{Blanc d'œufs et crème fouettée}
Graduellement jusqu'à une vitesse de 10 à 8 en fonction de la quantité (plus vous en mettez, moins vous avez besoin d'aller à 
fond).

\subsection{Hachage de la viande}
Il faut utiliser la vitesse 4. 

Ne pas hésiter à utiliser de la viande très froide ou partiellement surgelée (ça marche mieux). Il faut faire des petits cubes, 
idéalement 1cm$^3$ ou des lanières longues mais très fines (un carré de 0.5cm$^2$ de section).
Il faut hacher une première fois avec la grosse grille, puis si on le souhaite, utiliser la grille plus fine. La viande grasse 
ne doit être hachée qu'une seule fois.

\subsection{Passoire à fruits et légumes}
Coupez les fruits et légumes en morceaux adaptés aux dimensions de la trémie. 

Retirez les peaux dures et épaisses (peau d'orange), les noyaux (pêches, cerises), les queues (fraises, raisin, cerises).

Cuisez tous les fruits et légumes fermes ou durs avant de les passer à la machine (pommes).

\section{Couteaux de cuisine}
Il existe beaucoup de couteaux différents, avec chacun une fonction précise. Voici un récapitulatif de certains d'entre eux.

\begin{figure}[htb]
\centering
\includegraphics[width=0.6\linewidth]{figures/couteaux.pdf}
\caption{Nom et forme de quelques couteaux de cuisine. À noter que le couteau de chef est un couteau multi-usage aussi.}
\end{figure}

\begin{remarque}
Pour une utilisation optimale, il convient d'aiguiser le couteau avant chaque utilisation à l'aide d'un fusil. Pas besoin d'y passer des heures, quelques mouvements suffisent.

Il faut poser le couteau (près de la garde) sur le fusil, avec une inclinaison d'environ 30\degre, le coté tranchant vers l'avant. Poussez ainsi la lame pour que le fusil rencontre la totalité du fil du couteau. Le mouvement doit être régulier afin de ne pas changer l'angle. Faites ensuite de même en passant sur le dessous du fusil pour l'autre coté du couteau.

Il est important de faire un mouvement pour chaque tranchant du couteau à chaque fois afin que l'affûtage soit homogène.
\end{remarque}

% Pour ma pierre à aiguiser 1000 3000, le coté beige est le coté à 3000, et le coté rouge le coté à 1000. En gros, le coté gros grain est le coté le plus épais.

\section{Herbes aromatiques}
\begin{itemize}
\item Aneth : Son goût anisé est très utilisé pour les recettes à base de poissons et d'œufs. On l'utilise en petite quantité pour agrémenter le plat et on ne soumet pas l'aneth à une cuisson trop forte, pour ne pas en perdre les qualités gustatives.

\item Basilic : Riche en arômes s'accommode parfaitement avec les plats ensoleillés. Avec des légumes, des pâtes ou en salade, le basilic relève vos recettes d'un parfum à la fois doux et intense. On retiendra qu'il est l'élément principal de la fameuse sauce pistou.

\item Ciboulette : les feuilles de ciboulette sont souvent utilisées ciselées dans les sauces et les salades pour les agrémenter d'un goût proche de l'oignon doux. On l'appréciera notamment dans une sauce à la crème fraîche (légère de préférence !) et au poivre pour accompagner une pomme de terre au four.

\item Cumin : Sa saveur chaude et piquante accompagne les fromages, le pain, les carottes et l'agneau.

\item Estragon : les feuilles d'estragon, qu'elles soient utilisées fraîches, séchées ou en poudre s'accommodent particulièrement avec les viandes blanches, le poisson et les œufs.

\item Laurier : originaire d'Europe, la feuille de laurier entre dans la composition des potages, marinades, et civets. Elle accompagne également la composition des bouquets garnis.

\item Marjolaine : autrement nommée \og grand origan\fg, cette plante condimentaire relativement méconnue peut être incorporée à vos ragoûts ou accompagner vos tomates. Son goût étant assez prononcé, il est conseillé d'éviter de la cuire trop longtemps ou de la mélanger à d'autres herbes.

\item Menthe : herbe aromatique phare des cuisines méditerranéennes, la menthe accompagne de nombreuses préparations traditionnelles, avec sa saveur douce et fraîche. Dans le thé ou le taboulé, avec des nems ou dans des desserts, la menthe apporte cette petite touche de vitalité à vos plats.

\item Origan : indispensable sur la pizza napolitaine, très apprécié dans la sauce tomate, l'origan est reconnu pour son goût prononcé et ses vertus médicinales. Souvent confondu avec la marjolaine, dont le goût est plus délicat, les deux plantes ont cependant les mêmes usages.

\item Oseille : avec sa saveur piquante et acidulée, l'oseille qui se cuisine comme les épinards, relève les œufs et les poissons.

\item Persil : riches en vitamines A et C, les feuilles de persil sont utilisées dans les préparations de légumes, avec de la viande ou du poisson. Qu'elles soient plates ou frisées, leur odeur et leur goût sont prononcés.

\item Romarin : son nom signifie en latin rosée de mer, et il pousse à l'état sauvage sur le pourtour méditerranéen. Son odeur camphrée accompagne viandes rôties, pommes de terre ainsi que le vinaigre.

\item Thym : poussant en milieu aride et rocailleux, le thym possède une force olfactive méditerranéenne de caractère. Sa fraîcheur et sa profondeur parfument bouquets garnis, marinades, potages et grillades.
\end{itemize}



\section{Temps de cuisson au four/grill}\index{temps de cuisson}
\begin{table}[htb]
\centering
\begin{tabular}{|c|c|}\hline
Préparation & Temps de cuisson\\\hline\hline
\multicolumn{2}{|c|}{\bsc{Viandes}}\\\hline
B\oe uf & $15\unit{min}/500\unit{g}$\\\hline
Mouton (épaule, gigot) & $15$ à $20\unit{min}/500\unit{g}$\\\hline
Porc,veau & $30$ à $35\unit{min}/500\unit{g}$\\\hline\hline
\multicolumn{2}{|c|}{\bsc{Volailles}}\\\hline
Poulet & $25$ à $30\unit{min}/500\unit{g}$\\\hline
Caille & $20$ à $25\unit{min}$\\\hline
\end{tabular}
\caption{Cuisson Grill/rotissoire (Garder la porte entrouverte)}
\end{table}

\begin{table}[htb]
\centering
\begin{tabular}{|c|c|c|c|c|}\hline
Température à coeur & bleu & saignant & à point & bien cuit\\\hline
\multicolumn{5}{|c|}{\bsc{Bœuf}}\\\hline
Filet & 49 - 52 °C & 53 - 57 °C & 58 - 63 °C & + 65 °C\\\hline
rôti(rumsteak, faux-filet, tende) & 49 - 52 °C & 53 - 57 °C & 58 - 63 °C & + 65 °C\\\hline
côte & 49 - 52 °C & 53 - 57 °C & 58 - 63 °C & + 65 °C\\\hline
paleron basse température(60 à 75°C d'enceinte) & - & - & - & 58 - 63 °C\\\hline
paupiettes & - & - & - & 65 - 68 °C\\\hline\hline
\multicolumn{5}{|c|}{\bsc{agneau}}\\\hline
carré & - & - & 60 - 63 °C & -\\\hline
selle & - & 58 - 63 °C & 60 - 63 °C & -\\\hline
Gigot & - & 58 - 63 °C & 60 - 63 °C & 65 - 68 °C\\\hline
épaule & - & - & 60 - 62 °C & 65 - 68 °C\\\hline\hline
\multicolumn{5}{|c|}{\bsc{canard}}\\\hline
magret et filet & - & - & 58 - 60 °C & -\\\hline
cuisse & - & - & - & 70 - 72 °C\\\hline
cuisse confite & - & - & - & 78 - 80 °C\\\hline\hline
\multicolumn{5}{|c|}{\bsc{porc}}\\\hline
Filet mignon & - & - & 60 - 62 °C & -\\\hline
carré & - & - & - & 68 - 72 °C\\\hline
rôti (échine) & - & - & - & 68 - 72 °C\\\hline
Jambon & - & - & - & 68 - 72 °C\\\hline\hline
\multicolumn{5}{|c|}{\bsc{veau}}\\\hline
Filet mignon & - & 58 - 61 °C & 62 - 64 °C & -\\\hline
Quasi ou noix & - & - & 60 - 63 °C & -\\\hline
carré & - & 53 - 57 °C & 62 - 64 °C & 65 - 68 °C\\\hline
épaule & - & - & 62 - 64 °C & 65 - 68 °C\\\hline
paupiettes & - & - & - & 65 - 68 °C\\\hline\hline
\multicolumn{5}{|c|}{\bsc{poule \& dinde}}\\\hline
Filet blanc & - & - & - & 68 -70 °C\\\hline
cuisse & - & - & - & 70 - 72 °C\\\hline
Volaille farcie & - & - & 68 - 70 °C & 72 - 75 °C\\\hline
Volaille entière & - & - & 65 - 68 °C & 70 - 72 °C\\\hline\hline
\multicolumn{5}{|c|}{\bsc{pintade}}\\\hline
suprême & - & - & - & 69 - 72 °C\\\hline
cuisse & - & - & - & 73 - 75 °C\\\hline\hline
\multicolumn{5}{|c|}{\bsc{lapin}}\\\hline
 & - & - & - & 66 - 68 °C\\\hline\hline
\multicolumn{5}{|c|}{\bsc{gibier}}\\\hline
Noix de cerf/chevreuil & - & 58 - 63 °C & 60 - 63 °C & 70 - 72 °C\\\hline
cuisse de sanglier & - & - & 70 - 72 °C & 76 - 78 °C\\\hline\hline
\multicolumn{5}{|c|}{\bsc{foie gras}}\\\hline
Filet blanc & 58 °C & *63 °C & 72 °C & -\\\hline\hline
\multicolumn{5}{|c|}{\bsc{poisson}}\\\hline
cabillaud / Lieu & - & - & 51 °C & 54 °C\\\hline
Lotte & - & - & 52 °C & 54 °C\\\hline
sandre / Turbo & - & - & 51 °C & 54 °C\\\hline
saumon & 40 °C & 48 °C & 50 °C & 52 °C\\\hline
Thon & 42 °C & 48 °C & 50 °C & 52 °C\\\hline
Espadon / marlin & - & 48 °C & 50 °C & 52 °C\\\hline
poisson entier (avec arête) & - & - & + 48 °C & -\\\hline\hline
\end{tabular}
\caption{Cuisson viande et poisson : Température à coeur}
\end{table}

% If any, the index must be right after the bibliography, and before appendixes
\clearpage
\phantomsection
\printindex

\end{document}
