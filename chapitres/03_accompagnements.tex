% Beginning of group where section is deactivated
% This is only to get the good structure of the document 
% since ``section'' is in fact embedded in the 'recette' environment.
% This group allow us to deactivate sections ONLY in the given file and 
% not for the entire document.
{\renewcommand{\section}[1]{}

\section{Aubergines à la poële}
\begin{recette}{Aubergines à la poële}{4}{45 minutes}{}\index{aubergines}
\begin{ingredients}
\ingredient 1kg d'aubergines
\ingredient 2 cuillères à soupe d'huile d'olive
\end{ingredients}

\begin{preparation}
\etape Pelez et coupez en cube les aubergines (environ $1\unit{cm}^3$).
\etape Faites chauffer l'huile puis mettez les aubergines en remuant pour bien répartir l'huile. 
\etape Remuez régulièrement jusqu'à ce que les aubergines soit dorées et moelleuses.
\end{preparation}

Le goût peut faire penser à des champignons, c'est très fin et je n'ai pas encore trouvé comment utiliser ce goût.

\begin{remarque}
La quantité d'huile est un point crucial. Trop, et ça sera trop gras, mais pas assez, et ça cramera au lieu de dorer. Notez 
qu'au début, tant que les aubergines ne sont pas réduites, il n'est pas bon de rajouter de l'huile tant que ça absorbe ou tant 
qu'on voit que ça ne dore pas. En effet, ce qui compte c'est à la fin. Car quand les aubergines vont réduire, l'huile qu'elles 
ont absorbé va ressortir et ça ne sert donc à rien d'en mettre trop. 

C'est quelque chose qui se voit à l'œil à force de faire la recette.
\end{remarque}
\end{recette}

\section{Banane jaune (plantain)}
\begin{recette}{Banane jaune (plantain)}{3}{20 min}{1h}\index{banane}
\begin{ingredients}
\ingredient Comptez deux bananes par personne si c'est le seul accompagnement.
\end{ingredients}

\begin{preparation}
\etape Coupez les deux extrémités de la banane, puis fendez la peau sur toute la longueur sans l'enlever
\etape Faites cuire la banane avec la peau jusqu'à ce que la peau soit uniformément noire, puis continuez la cuisson jusqu'à ce que l'intérieur soit moelleux (contrôlez avec un couteau)
\end{preparation}
\end{recette}

\section{Carotte à la crème}
\begin{recette}{Carotte à la crème}{3}{20 min}{1h}\index{carottes}\index{crème}
\begin{ingredients}
\ingredient 1kg de carottes
\ingredient 2 échalotes
\ingredient 2 oignons
\ingredient 1 gousse d'ail
\ingredient 20g de beurre
\ingredient 20cl de bouillon (eau + bouillon cube de boeuf ou de volaille)
\ingredient 20 cl de crème fraîche épaisse
\ingredient sel, poivre, cerfeuil (frais ou pas)
\end{ingredients}

\begin{preparation}
\etape Éplucher et couper en rondelles les carottes. Peler et émincer les échalotes, les oignons et l’ail. Hacher le cerfeuil.

\etape Dans une cocotte, faire suer dans le beurre les oignons, les échalotes et l’ail pendant 4 minutes sur feu doux.

\etape Ajouter les rondelles de carottes et prolonger la cuisson 5 minutes en remuant de temps en temps. Il faut que les 
carottes commencent à rendre un peu de jus.

\etape Verser le bouillon, saler et poivrer. Couvrir et laisser cuire sur feu moyen pendant 25 minutes (ajouter de l'eau s'il 
n'en reste plus, les carottes doivent tout absorber).

\begin{remarque}
Les carottes doivent être cuites avant d'ajouter la crème. Une fois la crème ajoutée, ça cuit beaucoup moins vite.
\end{remarque}

\etape Ajouter la crème fraîche et le cerfeuil, bien mélanger. Goûter et rectifier l’assaisonnement si nécessaire. Laisser cuire 
encore 10 minutes sur feu doux.

\etape Servir aussitôt pour accompagner un rôti de porc ou des escalopes .
\end{preparation}
\end{recette}

\section{Carottes au four}
\begin{recette}{Carottes au four}{4}{}{}\index{carotte}
\begin{ingredients}
\ingredient 1.5kg de carottes
\ingredient 2 cas d'huile de tournesol
\ingredient 1 cac d'ail en poudre
\ingredient 1 cac d'herbes de provence
\ingredient sel 
\end{ingredients}

\begin{preparation}
\etape Préchauffez le four à 210°C
\etape Pelez et coupez les carottes en rondelles grossières d'un cm environ
\etape Mélangez avec l'huile, l'ail et les herbes de provence
\etape Etalez les carottes dans un lèche frite, puis saupoudrez de sel
\end{preparation}

\begin{cuisson}
Faites cuire environ 45 minutes à 210°C (j'ai ensuite laissé 20 minutes dans le four éteint, mais je sais pas si c'est crucial comme pour les patates).
\end{cuisson}
\end{recette}

\section{Fondue de poireaux}
\begin{recette}{Fondue de poireaux}{3}{20 min}{1h}\index{carottes}\index{crème}
\begin{ingredients}[3 personnes]
\ingredient 1kg de poireaux
\ingredient 20cl de crème liquide
\ingredient 1 cac de moutarde
\ingredient 2 cuillères à soupe de jus de citron
\ingredient sel
\end{ingredients}

\begin{preparation}
\etape émincez les poireaux
\etape Faites fondre 20g de beurre puis faites revenir les poireaux 2/3 minutes environ
\etape laissez cuire à feu doux (4/9) pendant 25 minutes
\etape Rajoutez la crème fraiche épaisse, le citron, la moutarde et le sel.
\etape Laissez cuire encore 10 minutes de plus à feux doux (4/9) et à couvert
\begin{remarque}
La fondue de poireaux accompagne très bien des poissons.
\end{remarque}
\end{preparation}
\end{recette}

\section{Frites}
\begin{recette}{Frites}{4}{}{}\index{pomme de terre}
\begin{ingredients}
\ingredient Pommes de terres (Environ 300-350g par personne)
\end{ingredients}

\begin{preparation}
\etape Pelez et coupez les pommes de terre en lamelles, toutes de taille similaire (5-7mm de coté par ex)
\etape Vous pouvez les laver si vous voulez, la communautée des fritophiles est divisée sur le sujet. Mais il faut que les frites soient sèches au moment de les plonger dans l'huile bouillante.
\end{preparation}

\begin{cuisson}
La taille maximum d'une fournée dépend de la quantité d'huile. Comptez 200g de frites fraiches ou 100g de frites surgelés par litre d'huile. 

Faites précuire les frites pendant 8 minutes à 160$\degres$ C. Épongez avec du papier absorbant pour enlever l'excès d'huile. 

Vous pouvez ensuite attendre une ou deux heures si vous le souhaitez avant la dernière cuisson, pratique pour timer le repas au mieux. 

Faites cuire 2m30 à 190$\degres$ C. Enlevez l'excès d'huile puis salez et servez. 
\end{cuisson}
\end{recette}


\section{Frites de patate douce}
\begin{recette}{Frites de patate douce}{}{30 min.}{45min.}\index{patate douce}
\begin{ingredients}
\ingredient 1.5kg de patate douce
\ingredient 50g de farine
\ingredient 9g de levure chimique
\ingredient 38g d'huile
\ingredient 5g de sel
\ingredient 2.5g d'épices (paprika, cumin)
\end{ingredients}
%\ingredient 1kg de patate douce
%\ingredient 35g de farine
%\ingredient 6g de levure chimique (1 cac)
%\ingredient 25g d'huile (2 cas)
%\ingredient 3g de sel
%\ingredient 1 cac d'épices (paprika, cumin)

\begin{preparation}
\etape Préchauffez le four à 220°C
\etape Pelez, lavez puis coupez les patates douces en frites.
\etape Dans un bol, mélangez la farine, le sel et les épices
\etape Mélangez les frites avec l'huile. Ajoutez alors la préparation farine/épice puis mélangez.
\etape Étalez les frites sur une seule couche sur une grande plaque.
\end{preparation}

\begin{cuisson}
Faites cuire 45 minutes à 200°C.
\end{cuisson}
\end{recette}

\section{Haricots verts}
\begin{recette}{Haricots verts}{}{}{}\index{haricots verts}
\begin{ingredients}
\ingredient 1kg de haricots verts surgelés
\ingredient bouillon cube
\ingredient ail en poudre, persil, sel
\end{ingredients}

\begin{preparation}
\etape Mettez les haricots dans une marmite, remplissez d'eau, salez, mettez le bouillon cube et faites bouillir
\etape Faites cuire 10 minutes à partir de l'ébullition
\etape Égouttez puis faites revenir 2 minutes dans la marmite avec un peu d'huile, l'ail et le persil afin d'assécher les haricots
\etape Versez les haricots ailleurs puis au besoin, rajoutez un peu d'eau en fin de cuisson pour récupérer ce qui a accroché plus facilement, mais pas trop d'eau
\end{preparation}
\end{recette}

\section{Poëlée forestière}
\begin{recette}{Poëlée forestière}{4}{}{}\index{pomme de terre}
\begin{ingredients}
\ingredient $500\unit{g}$ de pommes de terre coupées en dé
\ingredient $4$ ou $5$ oignons
\ingredient $200\unit{g}$ de lardons
\end{ingredients}

\begin{preparation}
\etape Faites cuire les lardons
\etape Une fois cuits, sortez les et faites revenir les oignons à feux doux dans la graisse des lardons en en rajoutant au 
besoin. Tournez les de temps en temps jusqu'à ce qu'ils soit dorés.
\etape sortez les et mettez les avec les lardons. Maintenant mettez les pommes de terre, surgelés ou coupées préalablement, à 
cuire à feux doux jusqu'à ce qu'elles soit cuites, et dorées. Il est important de les laisser cuire à feux doux, et de ne pas 
changer augmenter le feu pendant la cuisson.
\etape Une fois les pommes de terres cuites, ajoutez les oignons et les lardons, remuez de sorte à obtenir un mélange homogène 
et laissez le temps que les oignons et lardons se réchauffent, remuez et servez.
\end{preparation}
\end{recette}

\section{Pommes de terre au four}
\begin{recette}{Pommes de terre au four}{4}{}{}\label{sec:pomme-de-terre-four}\index{pomme de terre}
\begin{ingredients}
\ingredient environ 1-1.25 kg de pomme de terre pour 2 personnes
\ingredient 2 cac de sel rase pour 1kg de pomme de terre 
\ingredient huile d'olive
\ingredient un sac de congélation
\end{ingredients}

\begin{preparation}
\etape Quelques heures avant (pour que ce soit bien égoutté), pelez les pommes de terre. Coupez les en grosses frites de taille aussi uniforme que possible (pas trop épaisses (3-4 morceaux par demi pomme de terre)
\etape Lavez et laissez égoutter les pommes de terre jusqu'à ce qu'il soit temps de cuisiner, mais pendant 1 h environ. Essuyez au sopalin si vraiment c'est le rush et pour aller plus vite.
\etape Dans le sac de congélation mettez les pommes de terre et un peu d'huile d'olive, puis mélangez en la secouant.
\etape Disposez les pommes de terre dans un lèche frite, salez (sel et épices si vous voulez)
\end{preparation}

\begin{cuisson}
Faites préchauffer le four à 200\degres C 10 minutes environ, puis enfournez-les :
\begin{itemize}
\item 40 minutes pour des potatoes assez grosses
\item 35 minutes si elles sont taillées assez fines façon frites
\end{itemize}
en les disposant dans un grand plat à tarte. Puis laissez reposer 10-20 minutes four éteint. 
\begin{remarque}
Si vous ne pouvez pas faire reposer au four, laissez 20 minutes dehors couvert de papier aluminium et d'um torchon épais pour garder la chaleur, puis remettez au four éteint le temps de couper la viande qui était au four.
\end{remarque}

% Le four à 90°C est trop chaud et les patates trop cuites si vous faites ça au lieu de faire le four éteint. Il faut sans doute baisser encore pour que ça marche. 



%200°C pdt 40 minutes, c'est la version de tatie. Peut-être un peu plus de temps ou de température pour ajuster. 
\end{cuisson}
\end{recette}


\section{Pommes sarladaise au four}
\begin{recette}{Pommes sarladaise au four}{4}{}{}\index{pomme de terre}
\begin{ingredients}
\ingredient 2.5kg de pomme de terre
\ingredient sel, poivre, graisse de canard
\ingredient 1/2 oignon
\ingredient 1 gousse d'ail
\ingredient un sac de congélation
\end{ingredients}

\begin{preparation}
\etape Mixez l'ail et l'oignon et versez dans un grand saladier
\etape Ajoutez environ une louche de graisse de canard (pas plus)
\etape Coupez les pommes de terre en potatoes (grosses frites)
\etape Salez abondamment et poivrez puis mélangez
\etape Mélangez le tout
\end{preparation}

\begin{cuisson}
Faites préchauffer le four à 200\degres C 10 minutes environ, puis enfournez les 45 minutes à une heure en les disposant dans un grand plat 
à tarte.

Si les pommes de terre sont fermes ou farineuses, faites cuire un peu plus jusqu'à ce qu'elles soient moelleuse (1h puis laisser dans le four éteint pendant 30 minutes, c'est à peu près ce que j'ai fait la dernière fois)
\end{cuisson}
\end{recette}

\section{Pommes de terre vinaigrette}
\begin{recette}{Pommes de terre vinaigrette}{2}{}{}\index{pomme de terre}
\begin{ingredients}
\ingredient 2 grosses pommes de terre par personne
\ingredient 1 échalote hachée
\ingredient 2 cuillères à soupe de moutarde
\ingredient 4 cuillères à soupe d'huile d'olive
\ingredient 2 cuillères à soupe de vinaigre de vin
\ingredient ciboulette
\ingredient sel, poivre
\end{ingredients}

\begin{preparation}
\etape Faire bouillir une grande casserole d'eau. Eplucher les pommes de terre et les couper en morceaux. Jeter les pommes de 
terre dans l'eau bouillante et les faire pendant au moins 25mn (plus selon la taille des morceaux). Bien vérifier que les 
morceaux soient cuits au centre.

\etape Égoutter et faire refroidir les pommes de terre. Préparer la vinaigrette en mélangeant l'échalote hachée, la moutarde, 
l'huile d'olive et le vinaigre. Saler, poivrer.

\etape Dans un saladier, mélanger la sauce et les pommes de terre et rectifier l'assaisonnement si nécessaire.

\begin{remarque}
Mettre au frais si vous préférez la salade de pommes de terre froide que chaude.
\end{remarque}

\etape Au moment de servir parsemer de ciboulette.
\end{preparation}
\end{recette}

\section{Riz}
\begin{recette}{Riz}{2}{}{}\index{riz}\label{sec:riz}
\begin{ingredients}
\ingredient 600g de riz
\ingredient 720g d'eau (800g si à la casserole et non au cuiseur ; autre manière de faire, c'est même volume d'eau que de riz)
% historiquement je faisais 800g d'eau, puis en voyant une vidéo, j'ai tenté 750 et c'était mieux. Maintenant j'ai vu deux autres sources qui estiment à 720 donc il faut que j'essaie. Je soupçonne qu'il faut une marge fixe d'eau pour compter l'évaporation pendant cuisson et donc plus on fait de riz et moins on a besoin de mettre d'eau car toute la proportion d'eau n'est pas destinée au riz (en gros, c'est linéaire mais il y a une abscisse à l'origine
\ingredient 10g de sel
\end{ingredients}

\begin{preparation}
\etape Mesurez le volume de riz et faites le tremper dans de l'eau froide pendant 30 minutes
\etape égouttez le, puis rajoutez le sel et le volume d'eau pour la cuisson
\etape Portez à ébullition (sans forcément couvrir, ça prend entre 2 et 5 minutes en fonction de votre feu).
\etape Faites alors cuire pendant 15 minutes à couvert et à feu moyen (4/9)
\etape Eteignez le feu et laissez reposer hors du feu, toujours couvert sans ouvrir, pendant 10-15 minutes (pour absorption complete de l'eau)
\end{preparation}
\end{recette}

\section{Semoule}
\begin{recette}{Semoule}{2}{10 min}{}\index{semoule}
\begin{ingredients}
\ingredient semoule de blé dur (compter 60g par personne si accompagnement, sinon 100g)
\ingredient eau (un peu moins du volume de semoule. Par exemple pour 275ml de semoule, compter 250ml d'eau)
\ingredient huile d'olive, sel
\end{ingredients}

\begin{preparation}
\etape À l'aide d'un verre doseur, choisissez une quantité de semoule (60g par exemple)
\etape Ajoutez un filet d'huile à la semoule et mélangez bien afin que l'huile soit répartie autour des grains
\etape Faites bouillir une quantité d'eau égale au volume de la semoule que vous souhaitez faire cuire (par exemple, pour 100g 
de semoule, c'est environ $12.5\unit{cl}$. 
\begin{remarque}
Je fais chauffer l'eau au micro onde pour ma part. C'est rapide et même si je ne vois pas les bulles, l'eau est quand même bien 
chaude. 
\end{remarque}
\etape Ajoutez alors l'eau avec la semoule, et remuez bien avec une fourchette afin que ce soit homogène. Laissez à couvert 
(dans un bol avec une assiette par exemple) pendant 5 à 10 minutes.
\etape Égrénez enfin la semoule avec une fourchette afin de bien séparer les grains.
\end{preparation}
\end{recette}


}% End of the ``group'' where section is deactivated
