% Beginning of group where section is deactivated
% This is only to get the good structure of the document 
% since ``section'' is in fact embedded in the 'recette' environment.
% This group allow us to deactivate sections ONLY in the given file and 
% not for the entire document.
{\renewcommand{\section}[1]{}

\section{Banana bread}
\begin{recette}{Banana bread}{0}{30min.}{1h}\index{banane}
\begin{ingredients}
\ingredient 2 cuillères à soupe de lait
\ingredient 2 bananes mures (noires ou quasiment)
\ingredient 160g de sucre
\ingredient 250g de farine
\ingredient 85g de beurre fondu tiède (ou mou)
\ingredient 2 oeufs
\ingredient 4g de bicarbonate (1/2 cuillère à café)
\ingredient 11g de levure chimique (1 sachet)
\ingredient 1 pincée de sel
\ingredient une cuillère à soupe de rhum
\end{ingredients}

\begin{preparation}
\etape Préchauffez le four à 165°C
\etape Réduisez les bananes en purée avec une fourchette
\etape Faites blanchir les oeufs et le sucre
\etape Pendant ce temps, préparez un saladier avec levure, bicarbonate, sel et farine
\etape Ajoutez alors le lait, le beurre fondu tiédi aux oeufs
\etape Ajoutez enfin la farine, puis la purée de banane
\etape Disposez dans un moule à cake préalablement fariné
\end{preparation}

\begin{cuisson}
Faites cuire au four pendant 55 minutes à 165°C. Contrôlez 
que c'est cuit en plantant le couteau, qui doit ressortir sans pate sur la lame.
\end{cuisson}
\end{recette}

%https://tangerinezest.com/banane-flambee-au-rhum/
%https://www.youtube.com/watch?v=ddkCaxSQRo8
%j'essaie de refaire une recette de restaurant ou la banane avait une sorte de sauce onctueuse. J'ai testé plusieurs recettes, dont une avec de la crème liquide en pensant que c'était l'ingrédient manquant mais je trouvais qu'il y avait trop de beurre et la sauce était pas crémeuse. La photo de cette recette ressemble le plus à ce qu'on a mangé, je pense que la clé est de bien faire cuire les bananes pour que ce soient elles qui fassent la sauce. 
\section{Banane flambée}
\begin{recette}{Banane flambée}{0}{30min.}{1h}\index{banane}
\begin{ingredients}
\ingredient 4 bananes mures (noires ou quasiment)
\ingredient 25g de beurre
\ingredient 50g de sucre
\ingredient 2 cuillère à soupe de rhum
\end{ingredients}

\begin{preparation}
\etape Dans une poêle, faites fondre du beurre à feu moyen. Épluchez et coupez les bananes en 2 dans le sens de la longueur puis déposez-les dans la poêle.
\etape Faites dorer sur chaque face. Ne pas passer à l'étape suivante si les bananes n'ont pas fait une sorte de jus
\etape Ajoutez le sucre et retournez de nouveau les bananes après 2 minutes. Augmentez un peu votre feu et laisser légèrement caraméliser.
\etape Versez le rhum sur le dessus des bananes et faites flamber.
\end{preparation}

\begin{cuisson}
Faites cuire au four pendant 55 minutes à 165°C. Contrôlez 
que c'est cuit en plantant le couteau, qui doit ressortir sans pate sur la lame.
\end{cuisson}
\end{recette}

\section{Banoffee pie}
\begin{recette}{Banoffee pie}{0}{}{}\index{banane}\index{confiture de lait}\index{banoffee pie}
\begin{ingredients}
\ingredient $70$ g de beurre
\ingredient $250$ g de gâteaux types palets breton (2 paquets)
\ingredient $30$ g de sucre
\ingredient $400$ g de lait concentré sucré (une boite moyenne)
\ingredient un peu de lait
\end{ingredients}

\begin{remarque}
\og Banoffee\fg est la contraction de banana et toffee (caramel).
\end{remarque}

\begin{preparation}
\etape Piler les gâteaux pour faire de la chapelure avec des morceaux moyens.

\begin{remarque}
Pour ma part, je tape dans le paquet de gâteaux sans même l'ouvrir pour économiser un torchon et mettre directement le concassé 
dans le plat. Ceci marche très bien pour des palets bretons par exemple.
\end{remarque}

\etape Le mélanger au sucre et y ajouter le beurre fondu\footnote{on peut faire fondre le beurre au micro onde}. Rajoutez un peu 
de lait (un fond de verre) pour lier.

\etape Ensuite, on étale le mélange au fond d'un plat et on met au frigo une demi journée (que ce soit froid et que ça durcisse 
en fait).

\etape Pour le dessus, on met à cuire au bain marie pendant 3h une boite de lait concentré sucré qui va devenir un peu 
caramélisé.

\etape On dispose des fruits au dessus de ce caramel, tranche de bananes ou selon le gout.

\etape On recouvre de chantilly.
\end{preparation}
\end{recette}

\section{Brownie au Chocolat}
\begin{recette}{Brownie au Chocolat}{3}{15 min}{35 min}\index{brownie}\index{chocolat}
\begin{ingredients}
\ingredient 3 œufs
\ingredient 100 g de sucre
\ingredient 120 g de farine
\ingredient 350 g de chocolat à dessert
\ingredient 150 g de beurre
\ingredient 11g (1 paquet) de levure
\ingredient \textbf{Option: } morceaux de noix
\end{ingredients}

\begin{preparation}
\etape Faites préchauffer le four à 150°C
\etape Faire fondre le chocolat et le beurre au bain marie (je met un saladier au dessus d'une casserole remplie d'eau chaude).
\etape Battre au fouet les œufs et le sucre
\etape Incorporer la farine et la levure
\etape Ajouter ensuite le beurre et le chocolat fondus.
\begin{remarque}
On peut alors rajouter des éclats de noix juste avant de mettre dans le plat
\end{remarque}

\end{preparation}

\begin{cuisson}
%avant c'était 35 minutes a 150 sans préchauffage mais la derniere fois c'etait trop cuit
Faites cuire au four pendant 25 minutes à 150°C. Contrôlez 
que c'est cuit en plantant le couteau, qui doit ressortir sans pate sur la lame.
Laisser tiédir avant de consommer. 
\end{cuisson}
\end{recette}

\section{Broyé du Poitou à la confiture}
\begin{recette}{Broyé du Poitou à la confiture}{3}{15 min}{15 min}\index{broyé du poitou}\index{confiture}\index{gâteau basque}
\begin{ingredients}
\ingredient 1 œuf
\ingredient 125 g de sucre
\ingredient 250 g de farine
\ingredient 125 g de beurre
\ingredient 11g (1 paquet) de levure
\ingredient 8g (1 sachet) de sucre vanillé
\ingredient 300g de confiture
\end{ingredients}

\begin{preparation}
\etape Préchauffer le four à 210°C
\etape Mélangez la totalité des ingrédients jusqu'à obtenir une boule de pâte style pâte brisée
\etape Séparez en deux boules, l'une légèrement plus grande que l'autre (celle du dessus a besoin d'un peu plus de pâte pour ne 
pas s'embêter.
\begin{remarque}
Pour étaler les pâtes, utilisez deux papier cuisson pour que ça n'accroche pas au rouleau à patisserie.\end{remarque}
\etape  Étalez la première pâte (la plus petite) puis conservez uniquement le papier cuisson du bas. Posez la main à plat sur la pâte, puis renversez là pour la déposer sur le moule, puis enlevez délicatement le papier cuisson pour ne pas casser la pâte.
\etape Ajoutez alors la totalité de la confiture. Même si ça peut sembler beaucoup, la pâte va gonfler. Ne mettez pas de confiture sur environ 1cm au bord du plat pour pouvoir joindre les deux pâtes
\etape Préparez alors la deuxième pâte et déposez là sur le moule de la même manière que pour la première pâte
\etape Pincez les deux pâtes sur les bords pour les sceller

\end{preparation}

\begin{cuisson}
Enfournez alors la préparation pendant 15 minutes à 210°C, le temps que le dessus soit très légèrement roussi.
\end{cuisson}
\end{recette}

\section{Cake}
\begin{recette}{Cake}{3}{20 min}{40 min}\index{cake}
\begin{ingredients}
\ingredient 3 œufs
\ingredient 150g beurre mou
\ingredient 150g de sucre en poudre
\ingredient 200g de farine
\ingredient 100g de raisins secs
\ingredient 120g de fruits confits
\ingredient cerises confites
\ingredient 5 cl de rhum (3 cuillères à soupe)
\ingredient 0.5 sachet de levure chimique (avec 1 complet, c'est un peu trop gonglé aéré pour un cake)
\ingredient 1 pincée de sel
\end{ingredients}

\begin{preparation}
\etape Versez les raisins et le rhum dans une casserole, chauffez sur feu doux. Hors du feu, flamber, couvrir d'un couvercle et 
laisser tiédir.
\etape Coupez les fruits confits en petits dés, et fariner les légèrement.
\etape Préchauffez le four à 210°C. Beurrez et farinez le moule à cake.
\etape Séparez le blanc du jaune de deux œufs et conservez blanc ET jaune.
\etape Montez les blancs en neige avec une pincée de sel, puis réservez.
\etape Mélangez le beurre et le sucre. 
\etape Quand le mélange est bien crémeux, ajoutez 1 œuf entier et les 2 jaunes d'œufs, en 
mélangeant vivement.
\etape Incorporez la farine et la levure.
\etape Ajoutez les fruits confits, (sauf les cerises), et les raisins avec le rhum de macération.
\etape Incorporez les blancs en neige à la pâte, sans craindre de les faire retomber.
\etape Versez la moitié de la pâte dans le moule, disposer les cerises confites, et verser le reste de la pâte.

\end{preparation}

\begin{cuisson}
Enfourner pour 40 minutes de cuisson environ à 180°C. 
%Dans la rotissoire, j'ai mis 40 minutes à 190°C, puis 10 et 10 minutes, mais il faut retourner le moule car la cuisson n'était pas uniforme. 
\end{cuisson}
\end{recette}


\section{Cake au chocolat}
\begin{recette}{Cake au chocolat}{0}{20 min}{40 min}\index{cake}
\begin{ingredients}
\ingredient 60g cacao en poudre
\ingredient 200g beurre
\ingredient 200g sucre en poudre
\ingredient 200g farine
\ingredient 4 oeuf
\ingredient 5g levure chimique
\ingredient 1 pincée de sel
\end{ingredients}

\begin{preparation}
\etape Découpez les fruits confits en très petits cubes et faites-les macérer dans le cointreau.
\etape Allumez le four th.7 (210°C).
\etape Tamisez ensemble le cacao, la farine et la levure.
\etape Dans le robot mélangeur, travaillez le beurre et le sucre en une pommade blanche et mousseuse.
\etape Ajoutez les oeufs un à un en mélangeant soigneusement avant d'ajouter le suivant.
\etape Incorporez la farine/cacao au mélange et continuez de travailler pour obtenir une pâte bien liée.
\etape Ajoutez les fruits confits macérés puis mélangez de nouveau.
\etape Beurrez et farinez un moule à cake de 25 cm de long.
\etape Remplissez de pâte au 3/4 de la hauteur car le cake doit bomber à la cuisson.
\etape Laissez cuire 35 min.
\etape Piquez le centre avec la lame d'un couteau pointu qui doit ressortir sèche. Sinon, baissez la chaleur du four th.6 (180°C) et laissez cuire quelques minutes de plus.
\etape Sortez le gâteau du four et attendez 5 min avant de le démouler puis laissez-le refroidir complètement sur une grille.
\etape Vous pouvez le servir accompagné d'une crème anglaise.


\end{preparation}

\begin{cuisson}
https://www.cuisineaz.com/recettes/cake-au-cacao-et-fruits-confits-4306.aspx
\end{cuisson}
\end{recette}

\section{Carrot Cake}
\begin{recette}{Carrot Cake}{3}{20 min}{35 min}\index{carrot cake}\index{carotte}
\begin{ingredients}
\ingredient[Pour le gâteau]
\ingredient 200g de carotte
\ingredient 3 œufs
\ingredient 100g de farine
\ingredient 100g de sucre
\ingredient Un sachet de levure chimique
\ingredient 1 cuillère à café de cannelle moulue
\ingredient Extrait de vanille
\ingredient Quelques noix
\ingredient[Pour le glaçage]
\ingredient 1 blanc d'œuf (rajoutez le jaune dans le gâteau)
\ingredient 200g de sucre glace (le sucre normal ne fonctionne pas bien, trop gros grains)
\end{ingredients}

\begin{preparation}
\etape Préchauffer le four à 180°C (thermostat 6).
\etape Mixez les carottes et les noix afin d'obtenir de tout petits morceaux.
\etape Mélanger les oeufs, la farine, la levure, le sucre, les épices, finir avec les carottes et les noix.
\etape Bien mélanger.
\etape Versez dans un moule à gâteau.
\end{preparation}

\begin{cuisson}
Faites cuire 35 minutes à 180°C (thermostat 6). La lame d'un couteau plantée au centre du gâteau doit ressortir sèche.

Battez avec un batteur électrique le blanc d'œuf et le sucre du glaçage à vitesse maximum jusqu'à obtenir une pâte blanche (la 
pâte va 
sécher et durcir sur le gâteau, pas besoin de battre plus d'une minute, ça se forme très rapidement).

Mettre au frais quelques heures.
\end{cuisson}
\end{recette}

\section{Cannelés}
\begin{recette}{Cannelés}{0}{15 min+une nuit}{1h15}\index{cannelés}
\begin{ingredients}[30 gros (un peu moins de 2L de pâte)]
\ingredient 900g de lait 
\ingredient 44+10g de beurre salé ou [normal + pincée de sel](les 10g sont pour beurrer les moules)
\ingredient 410g de sucre
\ingredient 247g de farine T45
\ingredient 3 œufs + 2 jaunes + 0.5 blanc (la moitié d'un blanc d'un oeuf)
\ingredient 3 cuillère à soupe de rhum
\ingredient vanille
\end{ingredients}
% http://www.lacuisinedebernard.com/2010/01/voila-mes-declicieux-caneles-ils-vont.html
% 
% \begin{ingredients}[55 petits cannelés ou 17 gros (un peu plus d'1L de pâte)]
% \ingredient 50cl de lait (510g)
% \ingredient 25g de beurre (+15g pour beurrer les moules)
% \ingredient 232g de sucre
% \ingredient 140g de farine T45
% \ingredient 2 œufs + 1 jaune
% \ingredient 7.5cl de rhum (une cuillère à soupe)
% \ingredient vanille
% \end{ingredients}

\begin{preparation}
\etape Mettre à chauffer le lait la vanille et le beurre puis portez à ébullition (mélangez le tout au fouet quand c'est chaud. Il est possible que du beurre doux se mélange très mal par rapport au beurre salé)
puis faites refroidir 30 minutes avant de passer à la suite.
\etape Dans un saladier faire blanchir le sucre (au moins 5 minutes à vitesse 2) avec les œufs en utilisant le fouet (c'est mieux que la feuille), puis ajoutez la farine à la cuillère à soupe, petit à petit.
\etape Verser le lait tiédis à la louche pour éviter les grumeaux
\etape ajoutez enfin le rhum
\etape Laissez refroidir puis mettez au réfrigérateur une nuit.
\end{preparation}

\begin{cuisson}
Sortez la pâte du frigo au moins une heure avant pour qu'elle soit à température ambiante (cuisson ratée la dernière fois parce que la pâte était trop froide au bout d'une heure). 

\begin{remarque}
 On peut aussi mettre la pâte au bain marie dans de l'eau tiède dans l'évier. Une demi heure à ce traitement permet de réchauffer beaucoup plus vite. La cuisson était beaucoup mieux quand j'ai fait ça.
\end{remarque}


Ne pas trop secouer la pâte le jour J pour ne pas faire rentrer d'air dedans ((et éviter de trop faire gonfler les cannelés), remuez doucement avec une cuillère en bois.

Préchauffez le four à 275°C. 

Préparez les moules. Faire fondre un peu de beurre et tapissez les parois des moules de beurre. Le mieux c'est au 
doigt, sinon ça sera pas bien badigeonné et ça manquera de dorure à la cuisson (c'est très important pour bien les saisir).

Ré-homogénéisez la pâte avec une spatule (pas au fouet). Versez la dans les moules en les remplissant presque jusqu'au bord (ça gonfle pendant la cuisson, mais ça rediminue ensuite). 

Les temps de cuisson : 
\begin{itemize}
\item Pour des petits cannelés : 
Laissez cuire au max (300°C si vous avez, sinon 275°C à chaleur tournante) pendant 5 minutes. 

Puis baissez la température à 200°C sans la chaleur tournante pour 50 minutes de plus. 

\item Pour des grands cannelés : 
Laissez cuire au max (chaleur tournante) pendant 15 minutes, puis 36 minutes (ref=35) à 200°C (chaleur tournante aussi)
\end{itemize}

Laissez refroidir 30 minutes dans les moules avant de démouler (sinon ils se déforment et s'affaissent).
\end{cuisson}
\begin{remarque}
Les jours suivants, vous pouvez les conserver au frigo et les réchauffer 3 minutes à 220°C
\end{remarque}
\end{recette}



\section{Charlotte Aux Fraises}
\begin{recette}{Charlotte Aux Fraises}{0}{}{}\index{charlotte aux fraises}\index{fraise}
\begin{ingredients}[6 pers.]
\ingredient $500$ g de fraises (goûteuses et bien tendres), équeutées et coupées en morceaux
\ingredient $100$ g de fraises (les plus belles), pour la décoration
\ingredient $25$ cl de lait
\ingredient $35$ cl de crème fraîche liquide
\ingredient $1$ gousse de vanille
\ingredient 4 œufs
\ingredient $6$ cuil. à soupe de sucre
\ingredient $1$ cuil. à café de maïzena (ou de fécule)
\ingredient $4$ feuilles de gélatine (soit 8 g), mises à tremper dans de l'eau froide
\ingredient $1$ boîte de biscuits à la cuillère
\ingredient Chantilly en bombe pour la décoration
\end{ingredients}

\begin{preparation}
\etape[Commencez par confectionner la garniture de la charlotte]
\etape Portez à ébullition le lait et la gousse fendue, couvrez et laissez infuser.
\etape Fouettez les jaunes d'œufs avec 3 cuil. à soupe de sucre pour obtenir une mousse blanche.
\etape Incorporez la maïzena, puis le lait chaud.
\etape Remettez ensuite le tout dans la casserole et faites épaissir, sur feu doux, sans laisser bouillir.
\etape Retirez du feu, ajoutez la gélatine essorée, en mélangeant pour qu'elle se dissolve parfaitement, puis laissez légèrement 
tiédir.
\etape Fouettez la crème fraîche en chantilly, en ajoutant 3 cuil. à soupe de sucre en cours d'opération.
\etape Incorporez cette chantilly dans la crème à la vanille tièdie.

\etape[Enfin, procédez au montage de la charlotte]
\etape Tapissez un moule à charlotte de film plastique (ou d'aluminium).
\etape Mélangez les morceaux de fraises à la moitié de la crème de garniture.
\etape Versez la moitié de la garniture sans fraises dans le moule, couvrez d'une couche de biscuits, mettez la garniture aux 
fraises, puis des biscuits, le reste de garniture nature et terminez par une couche de biscuits.
\etape Posez une assiette sur la charlotte aux fraises et tassez légèrement.
\etape Laissez reposer la charlotte au réfrigérateur pendant au moins 4 heures.
\etape Au moment de servir, démoulez la charlotte aux fraises et décorez-la avec les fraises restantes et de la crème chantilly.
\end{preparation}
\end{recette}

\section{Christmas pudding à la moi}
\begin{recette}{Christmas pudding à la moi}{4}{1h+1 nuit}{8h}\index{christmas pudding}\index{noel}
\begin{ingredients}[4 personnes]
\ingredient 113g de beurre
\ingredient 55g de farine
\ingredient 4g de levure chimique
\ingredient 112g de biscotte broyée en chapelure
\ingredient 0.5g de noix de muscade
\ingredient 1g de cannelle
\ingredient 225g de sucre
\ingredient 350g de raisin sec (avant 400)
\ingredient 100g d'orange confite (en petits dés ou mixée, moi je mixe)
\ingredient 25g de poudre d'amande
\ingredient 60g de pomme mixée (~1/2)
\ingredient 2 oeufs
\ingredient 25g de rhum
\ingredient 75ml de vin rouge
\ingredient 75ml de guiness
\end{ingredients}

\begin{preparation}
\etape Mettez le beurre fondu, la farine, la levure, les biscottes mixées, le sucre et les épices (cannelle, noix de muscade) dans un bol et mélangez le tout à l'aide de la feuille. Ajoutez alors les fruits secs, la poudre d'amande, la pomme mixée et l'orange confite.

\etape Dans un autre bol, mélangez les oeufs, le rhum, le vin rouge et la guiness. Fouettez pour mélanger les oeufs puis ajoutez dans le bol du robot.

\etape Beurrez et farinez le récipient qui servira à cuire le gâteau. versez la préparation. Couvrez avec le couvercle du tupperware et laissez reposer pendant toute une nuit (ou 24h dans mon cas).
\end{preparation}

\begin{cuisson}
Le lendemain, ajoutez du papier cuisson et fermez avec une ficelle pour bien isoler. Coupez le surplus de papier une fois scellé.

Dans une grand marmite, mettez le tupperware et ajoutez de l'eau chaude jusqu'à la limite du tupperware (il ne faut pas le noyer). Couvrez avec un couvercle, idéalement en verre pour voir sans avoir à ouvrir. Portez à ébullition (environ 30 minutes à feu moyen/fort -- 5/9), puis baissez le feu jusqu'à juste maintenir l'ébullition mais pas plus (4/9). Faites alors cuire 8h en tout (en comptant les 30 premières minutes). Rajoutez de l'eau au besoin, mais je n'ai pas eu besoin d'en rajouter (avec le couvercle j'ai perdu très peu d'eau et j'en avais mis pas mal initialement).

Dégustez de préférence un peu tiède (20s au micro onde). Je n'ai gardé le gâteau qu'une semaine au frigo avant de le déguster mais c'est censé pouvoir se garder. De manière générale, ne pas manger si ça sens bizarre ou s'il y a de la moisissure mais sinon c'est bon, et ça se garde environ un an normalement, dans un endroit frais et sec.
\end{cuisson}
\end{recette}

\section{Cookies}
\begin{recette}{Cookies}{0}{20 min.}{10 min.}\index{cookies}
\begin{ingredients}
\ingredient 85 g de sucre de canne
\ingredient 85 g de beurre mou
\ingredient 1 œuf
\ingredient 150 g de farine
\ingredient 1 pincée de sel, extrait de vanille (ou 1 sachet de sucre vanillé)
\ingredient 1/2 sachet de levure
\ingredient 100g pépites de chocolat (je concasse une demi plaque de chocolat à dessert)
\end{ingredients}

\begin{preparation}
\etape Mélangez le beurre mou, le sucre, l'oeuf entier et la vanille
\etape Ajoutez petit à petit le mélange levure/farine
\etape Ajoutez enfin les pépites de chocolat. Finissez de pétrir à la main sinon vous ne vous en sortirez pas
\etape Préchauffer le four à 180°C
\etape Faire des petites boules que vous aplatirez ensuite avec la main
\end{preparation}

\begin{cuisson}
Faites cuire 10 minutes à 180°C . Sortez et laissez refroidir les cookies SANS LES TOUCHER. Ils ne semblent 
pas cuits mais c'est normal, ils vont finir de cuire hors du feu en refroidissant.

\begin{remarque}
Ne pas faire deux grilles en même temps, même avec chaleur tournante ça ne cuit pas bien. 
\end{remarque}

\end{cuisson}
\end{recette}

\section{Crème brûlée}
\begin{recette}{Crème brûlée}{3}{20 min}{2h}\index{crème brûlée}\index{crème catalane}
\begin{ingredients}[4 pers.]
\ingredient 5 jaunes d'œufs (vous pouvez réserver les blancs d'œufs pour faire des meringues \refsec{sec:meringues})
\ingredient 100g de sucre semoule
\ingredient 40cl de crème liquide entière (10cl par crème brûlée)
\ingredient 40g de sucre en poudre (il faut que les grains soient fins ; pour le caramel)
\ingredient extrait de vanille
\end{ingredients}

\begin{preparation}
\etape Mélanger dans un bol les 100g de sucre et les jaunes d'œufs au fouet.
\etape Verser la crème et l'extrait de vanille dans une casserole. Faites chauffer afin de bien mélanger et faire légèrement 
réduire. 
\etape Incorporez alors le sucre et les œufs dans la crème en prenant bien soin de faire refroidir un peu la crème (si la crème 
est à 100°C, les œufs vont cuire et vous préparerez une omelette bizarre.
\etape Répartir dans les plats de service (ramequin, assiette catalane).
\end{preparation}

\begin{cuisson}
Faites cuire au four à 100°C pendant 3 heures, sans préchauffage. La crème doit être un peu tremblotante 
(légèrement frémissante), mais pas plus. 

Laisser refroidir au frigo. 

Avant de servir, saupoudrez de sucre en poudre et le brûler avec un petit chalumeau de cuisine en décrivant des cercles avec le 
chalumeau jusqu'à former un caramel. 

\end{cuisson}
\end{recette}

\section{Crème chantilly}
\begin{recette}{Crème chantilly}{3}{20 min}{}\index{crème chantilly}\index{chantilly}
\begin{ingredients}[4 pers.]
\ingredient 50cl de crème liquide entière (C'est la matière grasse qui fait que la chantilly prend)
\ingredient extrait de vanille
\ingredient 50g de sucre
\end{ingredients}

\begin{preparation}
\etape Mélangez le sucre et la crème et l'extrait de vanille.
\etape Mettez au frais la crème, le saladier et le fouet du batteur (2h avant à peu près)
\etape 
Battre la crème à l'aide d'un fouet électrique. Changer de vitesse (du plus lent au plus rapide) progressivement, toutes les 30 
secondes environ.
\begin{remarque}
Ne pas fouetter trop vite (j'ai été jusqu'à 8/10), attendre que la crème prenne avant d'augmenter la vitesse
\end{remarque}
\end{preparation}
\end{recette}

\section{Crêpes}
\begin{recette}{Crêpes}{3}{10 min + 1h}{1h}\index{crêpes}
\begin{ingredients}[$\sim$ 24 grandes crêpes]
\ingredient 500g farine
\ingredient 6 œufs
\ingredient 1L de lait
\ingredient 30 g de sucre vanillé (4 sachets)
\ingredient 1 pincée sel
\ingredient Rhum et extrait d'orange pour parfumer.
\end{ingredients}


\begin{preparation}
\etape Dans un saladier, verser la farine et le sel.
\etape Y faire un puit et mettre les œufs et le 1/3 du lait. Mélanger le tout sans précautions
\etape Laissez reposer pendant 5 minutes (ça permet de bien humidifier les éventuels grumeaux)
\etape Mélangez de nouveau puis rajoutez le reste de lait
\etape Rajouter une cuillère à soupe de rhum et une cuillère à café d'extrait d'orange.
\etape Laisser reposer la pâte une heure au frais.
\end{preparation}

\begin{remarque}
Pour des crêpes plus légères, mettre moitié de lait et moitié d'eau
\end{remarque}

\begin{cuisson}
Faire cuire les crêpes dans une poêle très chaude légèrement huilée. 

\begin{enumerate}
 \item Faites chauffer la ou les poêles (idéalement deux) à feu moyen/vif, légèrement huilée
 \item Versez la pâte en faisant tourner la poêle pour étaler
 \item videz le surplus de pâtes dans le saladier (pour que ce soit plus fin)
 \item Retournez la crêpe quand le coté commence à roussir
 \item Surveillez et sortez la crêpe dès que des points roux apparaissent
 \begin{remarque}
  Si vous videz le surplus, la pâte sera très fine, elle craquèlera beaucoup plus vite quand vous la retournez
 \end{remarque}

\end{enumerate}
\end{cuisson}
\end{recette}

\section{Croustade aux pommes}
\begin{recette}{Croustade aux pommes}{3}{1h30}{15 min.}\index{croustade}\index{pomme}
\begin{ingredients}
\ingredient 12 feuilles de filo
\ingredient 4 pommes
\ingredient 100 g de beurre
\ingredient 50 g de sucre en poudre (+ du sucre pour saupoudrer dessus)
\ingredient 8 cuillères à soupe d'armagnac
\ingredient 1 demi cuillère à café de cannelle en poudre
\end{ingredients}


\begin{preparation}
\etape Clarifiez le beurre : faites-le fondre à feu doux dans une petite casserole, ôtez à l'aide d'une cuillère la fine 
pellicule blanche qui s'est formée à la surface, puis versez délicatement le beurre fondu dans un bol, pourque le petit lait 
déposé au fond reste dans la casserole. Le beurre ainsi obtenu ne contient plus du tout d'humidité, ce qui l'empêchera de 
noircir pendant la cuisson du feuilletage.
\etape Préparez les fruits : Épluchez les pommes, coupez-les en fines lamelles en éliminant le cœur et les pépins. 
\etape Faites-les 
sauter dans une poële à feu moyen dans 1 cuillère à soupe de beurre. Mettez les 50g de sucre et de la cannelle. 
Faites-les 
caraméliser 10 mn en secouant le manche de la poêle.
\etape Préchauffez le four à thermostat 6 (180°C).
\etape Avec un pinceau souple, badigeonnez de beurre clarifié le fond d'un moule à manqué. 
\etape Beurrez de la même façon 3 feuilles de filo. Disposez-les afin qu'une de leur extrémité dépasse à peine et l'autre 
dépasse beaucoup et faites des rotations afin que ces gros bouts qui dépassent finissent par dépasser de tous les cotés.  
Répartissez 1/3 des pommes, sucrez. Recommencez l'opération 2 fois. Rabattez les bords des feuilles de filo sur la dernière 
couche de fruits.
\begin{remarque}
On peut faire une couche de plus si on ne met pas de feuille sur le dessus (et le recouvrement sert de dessus en fait, surtout 
si on a bien respecté afin que ça dépasse pas mal de chaque coté.
\end{remarque}

\end{preparation}

\begin{cuisson}
Badigeonnez de beurre, poudrez de sucre et enfournez pour 15 mn.
Démoulez délicatement la croustade en la retournant sur une assiette, entaillez la surface avec un couteau et arrosez 
d'armagnac.
Servez aussitôt.
\end{cuisson}
\end{recette}

\section{Crumble aux pommes}
\begin{recette}{Crumble aux pommes}{0}{}{30 min}\index{crumble}\index{pommes}
\begin{ingredients}
\ingredient 5 pommes
\ingredient des framboises (ou du jus de citron, faute de framboises)
\ingredient $150$g de cassonnade
\ingredient $150$g de farine
\ingredient $125$g de beurre ramolli (pas fondu)
\ingredient une cuillère à soupe de cannelle
\end{ingredients}

\begin{preparation}
\etape Coupez les pommes en dés et disposez-les au fond du moule.
\etape Dans un saladier, mettez la farine, le beurre, le sucre et la canelle et malaxez le tout avec les mains. Mélangez jusqu'à 
obtenir quelque chose d'homogène et de friable.
\etape Répartissez le mélange sur les pommes sans tasser
\end{preparation}

\begin{cuisson}
Enfourner une demi-heure à $180$\degres C. Servir chaud ou tiède dans le plat de cuisson.
\end{cuisson}
\end{recette}

\section{Devil Food cake}
\begin{recette}{Devil Food cake}{3}{20 min}{40 min}\index{chocolat}
\begin{ingredients}
\ingredient[Gâteau]
\ingredient 170g de beurre mou
\ingredient 240g de sucre en poudre
\ingredient 80g de cacao amer en poudre
\ingredient 30cl de lait
\ingredient 3 œufs entiers
\ingredient 200g de farine 
\ingredient 11g de levure chimique
\ingredient 4g de sel
\ingredient[Glaçage]
\ingredient 120g de chocolat noir pâtissier
\ingredient 80g de chocolat lait pâtissier
\ingredient 30g de beurre doux
\ingredient 20cl de crème fraîche épaisse 30\% (\textbf{entière})
\end{ingredients}

\begin{preparation}
\etape Préchauffer le four à 180°c. 
\etape Mélangez les oeufs et le sucre jusqu'à obtenir une consistance crémeuse. 
\etape Dans une casserole, faites fondre le beurre puis ajoutez le chocolat en poudre
\etape Ajoutez la farine, la levure et le lait au mélange oeuf/sucre. 
\etape Ajoutez alors le mélange beurre/chocolat. 
\etape Versez la préparation dans un moule beurré et fariné. 
\end{preparation}

\begin{cuisson}
Faites cuire pendant 37 minutes à 180°C. Laisser tiédir dans les moules puis démouler sur une grille et attendre le refroidissement complet avant montage.

Glaçage chocolat : au bain-marie, faire fondre les deux chocolats, ajouter le beurre en parcelles. Faire chauffer dans une casserole la crème fraîche jusqu’à ébullition. Verser ensuite, hors du feu, la crème sur le chocolat et beurre fondus. Bien mélanger et laisser refroidir (sans toutefois que ce soit complètement froid).

Montage : répartir un peu moins de la moitié du glaçage sur le premier gâteau. Déposer le deuxième gâteau et y verser le reste du glaçage. Lisser à la spatule les côtés et le dessus du gâteau. Prêt à déguster…
\end{cuisson}
\end{recette}

\section{Far breton}
\begin{recette}{Far breton}{3}{15 min}{1h}\index{far breton}\index{flan patissier}\index{pommes}\index{pruneaux}
\begin{ingredients}
\ingredient 220g de farine
\ingredient 130g de sucre
\ingredient un sachet de sucre vanillé
\ingredient 75cl de lait
\ingredient 5 œufs
\ingredient 20g de beurre
\ingredient (Facultatif) 500g de pruneaux ou pomme, traditionnellement non
\end{ingredients}

\begin{preparation}
\etape Préchauffez le four à 180°C (thermostat 6)
\etape Dans un saladier, mélangez le sucre, la farine et le sucre vanillé
\etape Ajoutez les œufs en prenant soin de bien mélanger le tout à chaque fois
\etape Versez le lait et ajoutez le beurre au préalablement fondu puis mélangez jusqu'à obtenir une pâte homogène
\etape Ajoutez vos pruneaux si vous souhaitez obtenir un far aux pruneaux (pensez à les dénoyauter). Mais vous pouvez 
évidemment le déguster nature, ou avec des pommes.
\etape Beurrez le fond du moule et versez-y la pâte. Une autre manière de faire est de mouiller le plat, puis de saupoudrer de 
la farine.
\end{preparation}

\begin{cuisson}
Enfourner une heure environ à $180$\degres C
\end{cuisson}
\end{recette}

\section{Financier}
\begin{recette}{Financier}{3}{}{1h}\index{financier}\index{blanc d'œufs}
\begin{ingredients}
\ingredient 2 blancs d'oeufs
\ingredient 75g de sucre glace
\ingredient 70g de beurre fondu
\ingredient 25g de farine
\ingredient 40g de poudre d'amande
\ingredient vanille
\end{ingredients}

\begin{preparation}
\etape Faire préchauffer le four à $190\degres C$
\etape Battez les blancs et le sucre 
\etape Ajoutez le beurre fondu, la farine, la poudre d'amande et la vanille puis mélangez
\etape Répartissez  la pâte dans le moule à financiers posé sur la grille froide du four
\end{preparation}

\begin{cuisson}
Faites cuire 12 à 14 minutes dans le four préchauffé à 190°C
\end{cuisson}
\end{recette}

\section{Flan}
\begin{recette}{Flan}{3}{}{1h}\index{flan}
\begin{ingredients}[8 personnes]
\ingredient[Flan]
\ingredient 2 litre de lait
\ingredient 160g de sucre
\ingredient 10 œufs
\ingredient[Caramel]
\ingredient 200g de sucre % 100g initialement, j'ai testé 200 et c'était trop (partie solidifiée apres cuisson)
\end{ingredients}

\begin{preparation}
\etape Ne pas faire préchauffer le four.
\etape Dans une casserole, mettez le sucre pour le caramel. Vous pouvez ajouter une goutte d'eau pour que le caramel se fasse plus 
vite. Puis versez au fond du plat pour le flan.
\etape Dans la même casserole, faites chauffer le lait avec le sucre sans aller jusqu'à ébullition (pour ne pas cuire les oeufs ensuite), puis mélangez avec les oeufs (Optionnel: passez au tamis pour enlever les morceaux d'oeufs qui ne sont pas bien mélangés)
\etape Versez alors la préparation avec le lait.
\end{preparation}

\begin{cuisson}
Mettre le moule dans un autre récipient plus grand contenant de l'eau froide, et faites cuire au bain marie pendant 2h à 160°C, à four froid au départ (la surface doit 
être roussie).

\begin{remarque}
Il faut la croute sur le dessus, sinon c'est pas cuit. En fonction de la quantité il faudra plus ou moins longtemps, 2h c'est pour 2L de lait
\end{remarque}
\end{cuisson}
\end{recette}

\section{Flan coco}
\begin{recette}{Flan coco}{3}{20 min.}{1h}\index{flan}
\begin{ingredients}
\ingredient[Flan]
\ingredient 500ml de lait entier
\ingredient 400g de lait de coco
\ingredient 20cl de crème de coco
\ingredient 80g de sucre (si pas assez sucré, 200g la prochaine fois)
\ingredient 7 œufs
\ingredient[Caramel]
\ingredient 100g de sucre
\end{ingredients}

\begin{preparation}
\etape Faire préchauffer le four à $180\degres C$
\etape Dans une casserole, faites brunir 100g de sucre. Vous pouvez ajouter une goutte d'eau pour que le caramel se fasse plus 
vite.
\etape Pendant ce temps, mélangez les œufs, le lait et l sucre dans un récipient. Ajoutez une gousse de vanille fendue.
\etape Versez le caramel au fond du moule, puis ajoutez la préparation avec le lait.
\end{preparation}

\begin{cuisson}
Mettre le moule dans un autre récipient plus grand contenant de l'eau, et faites cuire au bain marie pendant 1h (la surface doit 
être roussie).
\end{cuisson}
\end{recette}

\section{Flan patissier}
\begin{recette}{Flan patissier}{3}{}{1h}\index{flan}
\begin{ingredients}[8 personnes]
\ingredient[Pâte]
\ingredient 430 grammes farine T45
\ingredient 200 grammes cassonade sucre blond ou roux de canne
\ingredient 200 grammes beurre doux froid
\ingredient 2 pincée levure chimique
\ingredient 2 pincée sel
\ingredient 2 oeuf
\ingredient[Flan]
\ingredient 5 oeufs
\ingredient 220 grammes sucre en poudre sucre blanc, fin ou extra fin
\ingredient 100 grammes maïzena
\ingredient 1 gousse de vanille
\ingredient 250 grammes crème liquide entière 25 cl
\ingredient 1 litre lait entier 1000g
\ingredient 1 cuillère à café arôme vanille ou extrait naturel de vanille
\end{ingredients}

\begin{preparation}
\etape Dans le bol du robot pâtissier, mettre la farine, le sucre blond ou roux, le beurre froid coupé en dés, la pincée de levure chimique et de sel et l'oeuf. Avec la feuille (fouet plat) en vitesse lente (vitesse 2), mélanger jusqu'à ce que la pâte s'agglomère autour de la feuille et se détache du bol.
\etape Prendre la pâte, faire une boule, l'aplatir et l'emballer dans du film alimentaire. Réserver au réfrigérateur minimum 30 minutes.
\etape Beurrer et fariner un cercle de 24 cm de diamètre et 6 cm de hauteur et le déposer sur une plaque perforée tapissée d'une feuille silicone (ou papier sulfurisé).
\etape Etaler la pâte sur un plan fariné en un cercle d'environ 35 cm de diamètre.
\etape Et foncer le cercle. Bien marquer les bords. Découper le surplus de pâte avec un petit couteau. Placer le tout a réfrigérateur.
\etape Préchauffer le four à 180°c, chaleur statique (zones de chauffe en haut et en bas).
\etape Dans une grande casserole en inox, faire bouillir le lait avec 120g de sucre et la gousse de vanille fendue et grattée.
\etape Pendant ce temps, dans un grand saladier fouetter les oeufs avec 100g de sucre restant, l'arôme vanille et la maïzena. Rajouter enfin la crème liquide et bien mélanger.
\etape Incorporer peu à peu le lait chaud et continuant de fouetter (retirer la gousse de vanille).
\etape Remettre le mélange dans la casserole, chauffer tout en fouettant jusqu'à ce que la crème soit bien épaissie (le fouet doit laisser des marques dans la crème). Retirer du feu.
\etape Verser l'appareil à flan dans le fond de tarte, lisser la surface à l'aide d'une spatule.
\end{preparation}

\begin{cuisson}
\begin{enumerate}
\item Enfourner au milieu du four (niveau 3) pour 50 minutes.
\item A la fin de cuisson, sortir la plaque du four. La poser sur une grille et laisser refroidir 15 minutes (le flan va s'affaisser, c'est normal).
\item Après 15 minutes de refroidissement, faire glisser la plaque de silicone avec le flan (en laissant le cercle) sur une volette. Puis retirer délicatement la feuille silicone (s'aider d'une longue spatule pour décoller le flan) et garder le flan sur la volette 1 heure (ainsi la base va sécher).
\item Faire glisser ensuite le flan (avec le cercle) sur le plat de service en s'aidant encore de la spatule pour décoller le flan. Retirer le cercle.
\item Laisser encore 30 minutes à température ambiante.
\item Placer au réfrigérateur minimum 3 heures (ou jusqu'au lendemain), sans couvrir.
\end{enumerate}

\end{cuisson}
\end{recette}

\section{Forêt noire à la moi}
\begin{recette}{Forêt noire à la moi}{3}{20 min.}{1h}\index{confiture}
%source: https://gabriellaboiteasucre.fr/2020/05/25/la-foret-noir-revisitee/
\begin{ingredients}
\ingredient[Moly cake]
\ingredient 193g de farine
\ingredient 193g de sucre
\ingredient 20cl de crème liquide
\ingredient 3 oeufs
\ingredient un sachet de levure chimique (11g)
\ingredient 30g de cacao non sucré (Van Houten ou autre)

\ingredient[crème chocolat (ou confiture)]
\ingredient 105g de chocolat
\ingredient 2 jaune d'oeufs
\ingredient 25g de sucre
\ingredient 125 de lait
\ingredient 125g de crème liquide

\ingredient[Chantilly Mascarpone]
\ingredient 20cl de crème liquide
\ingredient 120g de mascarpone
\ingredient 20g de sucre glace

\end{ingredients}

\begin{preparation}
\etape Faire préchauffer le four à $160\degres C$
\etape Beurrez et farinez le moule
\etape Fouettez les oeufs et le sucre jusqu'à ce que la préparation soit jaune clair et ait triplée de volume.
\etape Ajoutez la farine, le cacao et la levure chimique préalablement tamisés et remuez avec une spatule
\etape Faites monter la crème liquide en texture mousseuse et ajoutez la à la préparation
\etape Versez dans votre moule et au four pendant 45 minutes à 160°C
\etape [facultatif] Préparez la crème chocolat. 
\etape Faites fondre le chocolat
\etape Battre les oeufs et le sucre puis ajoutez le lait et la crème liquide
\etape Mettez le tout dans une casserole et laissez chauffer jusqu'à ce que la préparation épaississe (ou atteigne 82°C)
\etape Versez la crème en 2 fois sur le chocolat et mélangez. Laissez alors densifier au frais.
\etape Préparez alors la crème chantilly. Mélangez la crème liquide, le mascarpone et le sucre glace, fouettez comme pour la chantilly à vitesse pas trop importante pour éviter de faire du beurre.
\etape Pour le montage, coupez le gâteau en deux en comptant 2/3 de la hauteur pour le bas (parce que le haut a aussi l'épaisseur du bombé.
Imbibez l'intérieur avec le sirop, pochez des boules de chantilly sur le pourtour de la première couche puis étalez la confiture à l'intérieur.
\etape étalez de la chantilly sur le dessous du 2e étage, puis mettez sur le premier étage (ainsi, la confiture va un peu imbiber le gâteau du dessous sans détremper la crème chantilly)
\etape étalez enfin le reste de chantilly sur le dessus du gâteau et pochez quelques boules pour décorer si vous en avez le courage.

\end{preparation}

\begin{cuisson}

\end{cuisson}
\end{recette}

\section{Galette des rois frangipane}
\begin{recette}{Galette des rois frangipane}{4}{}{1h}\index{galette des rois frangipane}\index{frangipane}
\begin{ingredients}
\ingredient 2 pâtes feuilletées
\ingredient 1 jaune d’œuf
\ingredient[Pour la crème pâtissière]
\ingredient 220ml de lait
\ingredient 1 jaune d’œuf
\ingredient 1 pincée de sel
\ingredient 60 g de sucre
\ingredient 20 g de farine
\ingredient 20 g de maïzena
\ingredient[Pour la crème d’amande]
\ingredient 80 g de beurre mou
\ingredient 100 g de sucre
\ingredient 2 œufs
\ingredient 120 g de poudre d’amande
\ingredient arôme d’amande amère
\ingredient arôme vanille
\end{ingredients}

\begin{preparation}
\etape[Préparation de la crème pâtissière]
\etape Faites chauffer le lait à feu doux.
\etape Pendant ce temps, mélangez le jaune d’œuf avec le sucre. Ajoutez la farine et maïzena et mélangez bien de nouveau jusqu’à obtenir un appareil lisse.
\etape Versez une louche de lait chaud dans votre mélange œufs/sucre/farine pour commencer à le délayer en remuant doucement pour homogénéiser l’appareil.
\etape Puis versez le restant de lait sans cesser de remuer doucement.
\etape Versez le tout dans la casserole et remettez sur feu doux, sans cesser de remuer jusqu’à ce vous obteniez une crème épaisse. Cela peut prendre un petit moment, le tout est d’être patient et de ne pas cuire trop fort et de remuer sans cesse pour éviter que les œufs coagulent.
\etape Finissez en versant la crème pâtissière dans un cul de poule et laissez refroidir.
\etape[Réalisation de la crème d’amande]
\etape Préchauffez votre four à 200°C.
\etape Mélangez la poudre d’amande avec les œufs, ajoutez le sucre et le beurre mou.
\etape Mélangez la crème d’amande avec la crème pâtissière jusqu’à ce que le mélange soit homogène.
\etape Versez l’extrait d’amande amère et l’arome vanille (1 à 2 c. à café selon la force de l’arôme) mélangez bien
\etape[Montage de la galette à la frangipane]
\etape Étalez une pâte feuilletée sur une plaque à pâtisserie recouverte d’une feuille de papier cuisson. Versez en son centre la frangipane, laissez environ 2 à 3 cm d’espace entre la frangipane et les bords de la pâte feuilletée (n’oubliez pas de poser la fève).
\etape Soudez les bords des deux pâtes feuilletées en appuyant les dents d’une fourchette (on appelle cela « chiqueter »), cela évitera à la crème d’amande de tenter de sortir pendant la cuisson, et ça fait aussi très jolie.
\etape Faites un tout petit trou avec la pointe d’un couteau dans la pâte (au centre par exemple) afin d’éviter que la pâte ne gonfle de trop et badigeonnez la galette du jaune d’œuf avec un pinceau.

\end{preparation}

\begin{cuisson}
Baissez votre four à 180°C et enfournez sur une plaque à pâtisserie pendant 30 à 40 ou jusqu’à ce que votre galette soit bien dorée et gonflée.
\end{cuisson}
\end{recette}

\section{Gâteau À l'ananas}
\begin{recette}{Gâteau À l'ananas}{4}{30 min.}{1h}\index{gâteau à l'ananas}\index{ananas}
\begin{ingredients}
\ingredient[gâteau]
\ingredient $150$ g de sucre
\ingredient $150$ g de beurre fondu
\ingredient $150$ g de farine
\ingredient 1 pincée de sel
\ingredient $1$ paquet de levure (11g)
\ingredient $4$ œufs
\ingredient $1$ boite de tranches d'ananas (garder le jus pour le sirop)
\ingredient[caramel]
\ingredient 200g de sucre
\ingredient 5cl d'eau
\ingredient[sirop $\sim 25$ cl]
\ingredient jus de la boite d'ananas
\ingredient 3cl de rhum
\ingredient eau pour compléter jusqu'à 25cl
\end{ingredients}

\begin{preparation}
\etape Préchauffez le four à 150°C
\etape Égouttez les tranches et réservez le jus pour plus tard.
\etape Beurrez le moule. 
\etape Faites chauffer le sucre et l'eau du caramel à feu vif. Dès que la préparation bruni, versez la dans le moule. 
\etape Déposez alors les tranches d'ananas dans le caramel.
\etape Mélangez le sucre, les œufs, la farine, la levure, la pincée de sel et ajoutez en dernier le beurre.
\etape Versez et étalez ce mélange au dessus des ananas.
\end{preparation}

\begin{cuisson}
Faites cuire une heure à 150°C.

Démoulez chaud et ajoutez le jus d'ananas mélangé à du rhum (en tout 22cl environ) sur le dessus pour imbiber le gâteau.
\end{cuisson}
\end{recette}

\section{Gâteau à la broche à la moi}
\begin{recette}{Gâteau à la broche à la moi}{4}{45 min.}{1h}\index{gâteau à la broche}
\begin{ingredients}
\ingredient 300g de farine,
\ingredient 6 œufs
\ingredient 250g de beurre
\ingredient 250g de sucre
\ingredient 1 pincée de sel
\ingredient 2 cuillerées de pastis
\ingredient 1 cuillerée à café de fleur d’oranger
\end{ingredients}

\begin{preparation}
\etape Dans une casserole, versez 50g de sucre et le beurre et mettez à fondre à feu doux, arrêter quand le mélange est liquide et homogène
\etape Dans une saladier, versez 250g de sucre, les blancs d'œufs et une pincée de sel puis battez les blancs en neige
\etape Dans le saladier final, versez les jaunes d'oeufs. Incorporez le beurre + sucre, le pastis et la fleur d'oranger
\etape Incorporez la farine dans le saladier. 
\etape Ajoutez enfin les blancs en neige dans la patte sans les casser. 
\end{preparation}

\begin{cuisson}
Beurrez et farinez un plat à gratin. Mettez le four en mode grill avec la porte légèrement entrouverte. Positionnez le plat à gratin à mi hauteur. 

Versez une louche complète de pâte dans le plat, puis étalez là à l'aide d'une spatule en silicone. Disposez la pâte dans les 
4 coins car c'est le plus difficile d'accès. Faites cuire 5 minutes 
environ, jusqu'à ce que la pâte dore, puis recommencez jusqu'à épuisement de la pâte. 
\end{cuisson}
\end{recette}

% gateau de dominique (mais c'était une grosse quantité, j'ai donc divisé les doses par deux)
\section{Gâteau au beurre}
\begin{recette}{Gâteau au beurre}{4}{30 min}{30 min}\index{gâteau au beurre}

\begin{ingredients}
\ingredient 125g beurre 
\ingredient 200g de sucre
\ingredient 3 œufs
\ingredient 22.5 ml de jus de citron (1/2 citron)
\ingredient 300g farine
\ingredient Lait (à définir)
\ingredient 1 sachet de levure 
\ingredient Essences
\end{ingredients}

\begin{preparation}
\etape Mélangez œufs + sucre et faire mousser 
\etape Rajouter essences + citron pressé.
\etape Rajouter beurre ramolli
\etape Rajouter 150g farine + levure
\etape Encore 150g farine + 1 peu de lait pour aider à mélanger 
\end{preparation}

\begin{cuisson}
1h au four à 180°C
\end{cuisson}
\end{recette}

% la recette est pour un monsieur cuisine ou un truc du style, j'ai donc adapté, et il faut repréciser la recette une fois que je l'aurai testé une fois.
\section{Gâteau au chocolat (mélina)}
\begin{recette}{Gâteau au chocolat (mélina)}{0}{30 min}{30 min}

\begin{ingredients}
\ingredient 5 oeufs
\ingredient 1 pincée de sel
\ingredient 150g de beurre
\ingredient 200g de sucre
\ingredient 250g de chocolat 70%
\ingredient 60g de farine T45
\end{ingredients}

\begin{preparation}
\etape Préchauffer le four à 200°C. Beurrer le moule
\etape Séparez les blancs des jaunes d'oeufs
\etape montez les blancs en neige avec une pincée de sel
\etape Transférer les blancs en neige dans un récipient puis les placer au réfrigérateur
\etape Faites fondre le beurre et mélangez avec les jaunes et le sucre
\etape Faites fondre le chocolat puis mélangez avec le reste
\etape Ajoutez la farine
\etape Ajoutez enfin les blancs en neige et mélangez au batteur très rapidement (18s)
\etape Versez dans le moule
\end{preparation}

\begin{cuisson}
Faites cuire environ 20 minutes à 200°C

vérifiez la cuisson à l'aide d'un cure dent, prolonger la cuisson de quelques minutes si nécessaire. 
Retirer le gâteau du four et laisser refroidir quelques minutes dans le moule. Détacher le gâteau de la paroi du moule à l'aide d'un petit couteau, puis retirer le fond et sortir le gâteau. Servir chaud ou froid. 

\begin{remarque}
Plus le temps de cuisson est long, plus le gâteau est ferme. Pour obtenir un gâteau moelleux, ne pas la prolonger trop longtemps. 
\end{remarque}
\end{cuisson}
\end{recette}

\section{Gâteau Au Yahourt}
\begin{recette}{Gâteau Au Yahourt}{4}{30 min}{30 min}\index{gâteau au yahourt}\index{yahourt}

\begin{ingredients}
\ingredient 125g de yahourt nature (1 yahourt ; Faute de yahourt on peut remplacer avec 125g de creme liquide)
\ingredient 175g de farine (2 pots de yahourt)
\ingredient 250g de sucre (2 pots de yahourt)
\ingredient 50g de beurre
\ingredient $1$ paquet de levure chimique
\ingredient $2$ pommes
\ingredient $2$ œufs
\ingredient $1$ citron ou orange rapé
\ingredient ricard (pour parfumer)
\end{ingredients}

\begin{remarque}
On peut remplacer les pommes par des poires.
\end{remarque}

\begin{preparation}
\etape Préchauffer le four.
\etape Mélanger tous les ingrédients
\etape Ajouter les pommes coupées en tranches
\etape Beurrer le moule, fariner, puis verser la préparation.
\end{preparation}

\begin{cuisson}
Mettre au four 30 minutes, thermostat 5 (175\degres C)
\end{cuisson}
\end{recette}

\section{Gâteau Aux noix}
\begin{recette}{Gâteau Aux noix}{3}{15 min}{50 min}\index{gâteau aux noix}\index{noix}

\begin{ingredients}
\ingredient 100g de cerneaux de noix pilés (environ 250g de noix entières)
\ingredient 250g de sucre
\ingredient 250g de farine
\ingredient 4 œufs
\ingredient 100g de beurre (ou 100 mL d'huile de noix)
\ingredient 25cl de vin blanc
\ingredient un sachet de levure chimique
\ingredient une pincée de sel
\end{ingredients}

\begin{preparation}
\etape Préchauffez le four à 180°C
\etape Faites fondre le beurre
\etape Dans un saladier, délayez œufs et sucre
\etape Ajoutez le vin blanc et une pincée de sel, les cerneaux de noix passés au mixer, le beurre fondu, la levure et la 
farine. 
\etape Mélangez jusqu'à obtenir une mixture homogène. 
\end{preparation}

\begin{cuisson}
Mettre au four 50 minutes à 180°C.
\end{cuisson}
\end{recette}


\section{Gâteau basque à la moi}
\begin{recette}{Gâteau basque à la moi}{4}{1h}{15 min}\index{Gâteau basque à la moi}\index{crème patissière}\index{gâteau basque}
\begin{ingredients}
\ingredient[pâte]
\ingredient 1 œuf
\ingredient 125 g de sucre
\ingredient 250 g de farine
\ingredient 125 g de beurre fondu
\ingredient 1 paquet de levure
\ingredient 1 sachet de sucre vanillé
\ingredient[crème]

\ingredient 500mL (500g) de lait entier
\ingredient 50g de farine (ou 30g de farine et 50g de poudre d'amande)
\ingredient 120g de sucre
\ingredient 2 jaunes d'œufs et un oeuf entier
\ingredient 2 cuillère à soupe de rhum
\ingredient Vanille
\end{ingredients}

\begin{preparation}
\etape Préchauffer le four à 210°C
\etape Mélangez la totalité des ingrédients jusqu'à obtenir une boule de pâte style pâte brisée d'environ 570g
\etape Séparez en deux boules, l'une légèrement plus grande (55\%) que l'autre (45\%, soit environ 255g) (celle du dessus a besoin d'un peu plus de pâte pour ne 
pas s'embêter.
\begin{remarque}
Pour étaler les pâtes, utilisez deux papier cuisson pour que ça n'accroche pas au rouleau à patisserie.\end{remarque}
\etape  Étalez la première pâte (la plus petite) puis conservez uniquement le papier cuisson du bas. Posez la main à plat sur la pâte, puis renversez là pour la déposer sur le moule, puis enlevez délicatement le papier cuisson pour ne pas casser la pâte.
\etape Dans une casserole, mettre le lait à bouillir, avec la gousse de vanille. 
\begin{remarque}
versez un fond d'eau dans la casserole, puis videz l’excédent avant de faire chauffer le lait. Cela évite au lait d'accrocher.
\end{remarque}
\etape Dans un cul-de-poule, blanchir (battre ensemble) le sucre, les jaunes et l'oeuf entier jusqu'à obtenir un mélange mousseux.
\etape Incorporez la farine petit à petit sans cesser de remuer.
\etape Mettez le lait à bouillir, avec la gousse de vanille. Une fois que le lait bout, retirez la gousse de vanille et versez la moitié du lait progressivement dans votre mélange, tout en remuant.
\etape Puis versez ce mélange sur le reste de lait dans la casserole. Remettre la casserole à feu moyen, et remuez sans cesse, jusqu'à ce que la crème s'épaississe (Typiquement quand la mousse du dessus disparait complètement).
\etape Sortez la crème du feu et ajoutez le rhum ambré.
\etape Au besoin, réservez la crème dans un saladier, la filmer en attendant de finir la pâte.
\etape Étalez la crème (pas forcément tout) en faisant attention à ne pas en mettre sur 1cm environ au bord (ça va s'étaler en posant la pâte au dessus)
\etape Préparez alors la deuxième pâte et déposez là sur le moule de la même manière que pour la première pâte
\etape Pincez les deux pâtes sur les bords pour les sceller.

\end{preparation}

\begin{cuisson}
Enfournez alors la préparation pendant 15 minutes à 210°C, le temps que le dessus soit très légèrement roussi.
\end{cuisson}
\end{recette}


\section{Gâteau mangue-passion à la moi}
\begin{recette}{Gâteau mangue-passion à la moi}{4}{30 min.}{1h}\index{gâteau à l'ananas}\index{mangue}\index{fruit de la passion}\index{maracudja}
\begin{ingredients}
\ingredient[gâteau]
\ingredient $150$ g de sucre
\ingredient $150$ g de beurre fondu
\ingredient $150$ g de farine
\ingredient 1 pincée de sel
\ingredient $1$ paquet de levure (11g)
\ingredient $4$ œufs
\ingredient 4 fruits de la passion
\ingredient 2 grosses mangues pas trop mûres (un peu mou mais pas trop)
\ingredient[caramel]
\ingredient 200g de sucre (idéalement blanc pour mieux voir le caramel brunir)
\ingredient 5cl de jus de fruit de la passion (cf plus haut)
\ingredient[sirop $\sim 22$ cl]
\ingredient Reste de jus de fruit de la passion
\ingredient 3cl de rhum
\ingredient eau pour compléter jusqu'à 22cl
\end{ingredients}

\begin{preparation}
\etape Extraire la pulpe de maracudja, puis mettez là dans une passoire (au dessus d'un saladier), puis appuyer avec une spatule silicone ou cuillère pour casser la membrane autour des pépins, faire tomber la pulpe et filter les pépins. 
\etape Dans une casserole, mettez le sucre du caramel puis 50g du jus que vous venez d'extraire. Réservez le reste du jus de maracudja pour le sirop. 
\etape Coupez les mangues le long du noyau pour en faire 3 tranches dont le centre est le noyau. Enlevez la peau, et coupez les deux tranches externes en lamelles de 3-4mm.
\etape Préchauffez le four à 150°C
\etape Beurrez le moule. 
\etape Faites chauffer le sucre et le jus pour faire le caramel. Dès que la préparation bruni, versez la dans le moule. 
\etape Déposez alors les tranches de mangues sur le caramel, la partie épaisse à l'extérieur du moule, et la pointe fine au centre. 
\etape Mélangez le sucre, les œufs, la farine, la levure, la pincée de sel et ajoutez en dernier le beurre.
\etape Versez et étalez ce mélange au dessus des fruits.
\end{preparation}

\begin{cuisson}
Faites cuire une heure à 150°C.

Pendant que le gâteau cuit, ajoutez 3cl de rhum vieux au jus de maracudja. Ajoutez environ 50g de sucre, et enfin, complétez pour avoir 22cl de liquide. 

Démoulez chaud et ajoutez le sirop préparé précédemment sur le dessus pour imbiber le gâteau, à la cuillère à soupe pour y aller doucement. 
\end{cuisson}
\end{recette}

%source https://www.femina.fr/article/cyril-lignac-devoile-sa-recette-de-la-buche-de-noel-au-chocolat-et-creme-vanille
\section{Gateau Roulé}
\begin{recette}{Gateau Roulé}{4}{30 min.}{7 min.}\index{gâteau roulé}\index{roulé à la confiture}\index{confiture}
\begin{ingredients}[1 grande plaque 35x42cm]% ce n'est pas la plaque téfal, c'est celle des états-unis
\ingredient 6 œufs
\ingredient 85+135 g de sucre
\ingredient 85 g de farine
\ingredient confiture
\ingredient beurre ou huile pour beurrer le moule
\end{ingredients}

\begin{preparation}
\etape Préchauffer le four à 210°C et beurrer/huiler puis fariner le moule (un moule relativement grand, et rectangulaire de 
préférence.
\etape Séparez les blancs des jaunes de 3 oeufs. 
\etape Montez les  blancs avec 85g de sucre pour qu'ils forment un bec d'oiseau.
\etape Montez les jaunes avec les oeufs entier et les 135g de sucre (ça va monter un peu aussi, c'est bien)
\etape Mélanger la farine aux jaunes.
\etape Mélangez les deux appareils ensemble délicatement. \og Entourez\fg la pâte pour ne pas chasser l'air contenu dans les blancs\footnote{En gros, il faut faire le tour du 
saladier, le dessous, avec des mouvements amples, sans chercher à exploser l'aglomérat de blanc}.
\etape Versez dans le moule beurré puis étalez légèrement pour homogénéiser le biscuit à l'aide d'une spatule coudée
\end{preparation}

\begin{cuisson}
Cuisez pendant 7 minutes à 210°C.

À la sortie du four, retournez le moule sur un papier cuisson contre une table ou un plan de travail pour ne pas perdre d'humidité et laissez refroidir ainsi à l'envers.

Une fois froid, démoulez, puis badigeonnez-le de confiture et roulez.

\begin{remarque}
Le gâteau est meilleur au bout de 48h ry atteint son optimum au bout de 72h (la confiture imbibe alors la génoise.
\end{remarque}
\end{cuisson}
\end{recette}

%source: https://www.gastronomie-wallonne.be/gastro/desserts/gaufre_bruxelles.html
\section{Gauffre de Bruxelles}
\begin{recette}{Gauffre de Bruxelles}{0}{}{}\index{gauffre}
\begin{ingredients}[12 gauffres]
\ingredient 250 g de farine
\ingredient 375 ml de lait
\ingredient 10 g de sucre
\ingredient 100 g de beurre fondu
\ingredient 7.5g (1 sachet) de levure de boulanger sèche (ou 15 de fraiche)
\ingredient 3 oeufs
\ingredient 1 pincée de sel
\ingredient 1/4 de bâton vanille
\end{ingredients}

\begin{preparation}
\etape Délayez la levure dans un peu de lait tiède avec le sucre.
\etape Séparez les blancs des jaunes d'œufs. Battez les blancs en neige bien ferme à l'aide d'un batteur électrique ou d'un fouet.
\etape Dans un saladier, mettez la farine tamisée, faites-y un puits et versez le reste de lait tiède, battez énergiquement le tout. Ajoutez les jaunes, le beurre fondu, la levure puis incorporez délicatement les blancs en neige.
\etape Laissez reposer 1h à couvert (45 min si avec levure fraiche).
\etape Faites chauffer le gaufrier et passez à la cuisson de vos gaufres de Bruxelles. Comptez environ 1:50 de chaque coté
\end{preparation}
\end{recette}

\section{Madeleines}
\begin{recette}{Madeleines}{4}{30min.+1 nuit}{30min.}\index{madeleine}
\begin{ingredients}[36 madeleines]
\ingredient 200 g de beurre
\ingredient 200 g de farine
\ingredient 3 oeufs
\ingredient 130 g de sucre
\ingredient 2 cuillères à soupe de miel (40g)
\ingredient 60 mL de lait
\ingredient 10 g de levure chimique
\ingredient 1 sachet de sucre vanillé
\end{ingredients}

\begin{preparation}
\etape Fondre le beurre dans une petite casserole sur feu doux. Bouillonner, comptez encore 1 minute. Surveiller la cuisson. Prendre une couleur noisette. Réservez.
\etape Faites blanchir les oeufs et le sucre
\etape Ajoutez le miel, le lait et le sucre vanillé. Versez la levure chimique et la farine dans la préparation.
\etape Mélangez puis incorporez le beurre noisette tiède. Laissez refroidir cette pâte et conservez la pâte toute la nuit au réfrigérateur soit dans un bol, soit directement dans les moules (ce qui est mieux et plus pratique). Remplissez les moules aux trois quarts (ça gonfle)
\end{preparation}

\begin{cuisson}
Préchauffez le four à 220°C. 

Enfournez à 220°C pendant 4 minutes, puis à 200°C pendant 4min30 minutes. Surveillez la cuisson !

Démoulez dès la sortie du four.

\begin{remarque}
Vous pouvez aussi faire des coques en chocolat facilement en versant l'équivalent d'une cuillère à café de chocolat par moule, puis appuyez la madeleine et laissez là jusqu'à ce que le chocolat refroidisse. Comptez 100g de chocolat pour 18 madeleines environ.
\end{remarque}

\end{cuisson}
\end{recette}


\section{Marbré}
\begin{recette}{Marbré}{4}{30min.+1 nuit}{30min.}\index{madeleine}
\begin{ingredients}
\ingredient 150g de beurre
\ingredient 300g de sucre
\ingredient 2 oeufs
\ingredient 230g de farine
\ingredient 1 sachet de levure chimique
\ingredient 220g de crème liquide
\ingredient 20g de cacao en poudre non sucré
\end{ingredients}

\begin{preparation}
\etape Faire préchauffer le four à 150°C
\etape Mélanger le sucre et le beurre ramolli
\etape rajouter les deux oeufs
\etape rajouter la farine et la levure mélangé
\etape rajouter la crème liquide progressivement
\etape Mettez de coté la moitié de la pâte (environ 470g pour chacun)
\etape Dans le reste de pâte, rajouter les 20g de chocolat
\etape Dans un moule, versez la moitié de chacune des deux pâtes et avoir une couche de pâte blanche/noire, coupé dans la longueur.
\etape Faites alors de même avec la couche supérieure, de sorte que le blanc soit sur le noir, et le noir sur le blanc
\etape Avec le manche d'une cuillère à café, ou n'importe quoi de fin, faites des mouvements afin de faire les nervures du gâteau.
\end{preparation}

\begin{cuisson}
Faire cuire à 150°C pendant 1h10

\end{cuisson}
\end{recette}

\section{Moelleux au chocolat}
\begin{recette}{Moelleux au chocolat}{4}{30min.}{30min.}\index{Moelleux au chocolat}\index{chocolat}
\begin{ingredients}
\ingredient 200 g de chocolat
\ingredient 125 g de beurre
\ingredient 125 g de sucre
\ingredient 4 oeufs
\ingredient 125 g de farine
\ingredient 1 sachet de levure (11g)
\ingredient une pincée de sel
\end{ingredients}

\begin{preparation}
\etape Préchauffer le four à 180°C.
\etape Faire fondre le beurre et le chocolat.
\etape Séparer les blancs des jaunes d'oeufs dans deux saladiers.
\etape Montez les blancs en neige avec une pincée de sel puis réservez. 
\etape Mélanger les jaunes avec le sucre et un peu d'eau jusqu'à ce que le mélange soit mousseux. 
\etape Y ajouter le chocolat/beurre fondu et bien mélanger. Enfin, ajouter la farine et la levure.
\etape Monter les blancs en neige ferme. Les incorporer délicatement au précédent mélange. 
\end{preparation}

\begin{cuisson}
Mettre le tout dans un moule beurré et faire cuire au four 30 minutes. Vérifier la cuisson en piquant une lame de couteau qui doit ressortir sèche. 
\end{cuisson}
\end{recette}

\section{Mousse au chocolat}
\begin{recette}{Mousse au chocolat}{3}{30 min.}{2h}\index{mousse au chocolat}\index{chocolat}\index{blanc d'œufs}
\begin{ingredients}[4 personnes]
\ingredient 150g de chocolat à dessert
\ingredient 6 œufs
\ingredient 2 sachets de sucre vanillé
\ingredient une pincée de sel
\end{ingredients}

\begin{preparation}
\etape Séparez les blancs des jaunes dans deux récipients différents.
\etape Faire fondre le chocolat (au bain marie dans une casserole par exemple)
\etape Mélanger les jaunes d'œufs avec le sucre
\etape Ajoutez une pincée de sel aux blancs, puis battez les en neige (ferme).
\etape Ajoutez le chocolat fondu aux jaunes et sucre.
\etape Incorporez enfin les blancs en neige dans la préparation de chocolat fondu en aérant (faites des mouvements amples avec 
la cuillère pour casser les blancs le moins possible)
\etape Répartissez dans des récipients individuels par exemple, puis mettez au frigo pendant une à deux heures.
\end{preparation}
\end{recette}


\section{Palmier feuilletés}
\begin{recette}{Palmier feuilletés}{3}{30 min.+4h}{}\index{pâte feuilletée}
\begin{ingredients}
\ingredient 1 pâte feuilletée
\ingredient du sucre
\end{ingredients}

\begin{preparation}
\etape Étalez le sucre sur la pâte feuilletée, puis roulez là de part et d'autre. 
\etape Laissez là au congélateur quelques minutes afin de pouvoir couper sans écraser
\etape Préchauffez le four à 220°C
\etape Faites cuire environ 20 minutes
\end{preparation}
\end{recette}

\section{Panna Cotta aux framboises}
\begin{recette}{Panna Cotta aux framboises}{3}{30 min.+4h}{}\index{Panna Cotta}\index{fruits rouges}
\begin{ingredients}
\ingredient[Panna Cotta (5 pers.)]
\ingredient 50 cl de crème fraiche liquide
\ingredient 50g de sucre
\ingredient 2 feuilles de gélatine
\ingredient Vanille (en gousse ou de l'extrait)
\ingredient[Coulis]
\ingredient 500g de framboises (fraiches ou surgelées)
\ingredient 200g de sucre
\ingredient 20 cl d'eau
\ingredient 1 peu de jus de citron
\end{ingredients}

\begin{preparation}
\etape Faire tremper 3 feuilles de gélatine dans de l'eau froide
\etape Mettre la crème, la vanille, le sucre dans une casserole et faire chauffer jusqu'à frémissement.
\etape Quand le mélange commence tout juste à bouillir, retirer la casserole du feu et ajouter la gélatine. Bien remuer pour que 
la gélatine se dissolve complément.
\etape Verser dans des verres, coupelles…
\etape Laisser refroidir
\etape Porter l'eau et le sucre à ébullition
\etape Ajouter le sirop obtenu aux framboises
\etape Bien mixer le tout, et passer au tamis.
\etape Laisser refroidir
\end{preparation}
\end{recette}

\section{Pavlova}
\begin{recette}{Pavlova}{3}{30 min.+4h}{}
\begin{ingredients}
\ingredient[meringue]
\ingredient 4 blancs d'oeufs
\ingredient 200g de sucre
\ingredient[chantilly]
\ingredient 20cl de crème entière très froide
\ingredient 50g de sucre
\ingredient[garniture]
\ingredient 250g de fraises
\end{ingredients}

\begin{preparation}
\etape Montez le sucre et les blancs à haute vitesse au fouet pour faire une meringue
\end{preparation}
\begin{cuisson}
Faites cuire la meringue à 120°C pendant 1h15  pour environ 1.5 à 2cm d'épais (si elle fait 1cm d'épais 1h suffit)
\end{cuisson}
\end{recette}

\section{Poires pochées au vin rouge}
\begin{recette}{Poires pochées au vin rouge}{0}{1h30}{}\index{poires au vin}\index{poire}\index{vin}
\begin{ingredients}
\ingredient 4 moyennes poires assez fermes
\ingredient 40cl de vin rouge
\ingredient 200g de sucre
\ingredient 1 cuillère à soupe d'extrait de vanille
\ingredient 1 cuillère à soupe d'extrait d'orange (ou zeste non traité)
\ingredient un peu de canelle
\end{ingredients}

\begin{preparation}
\etape Faites chauffer le vin, le sucre et les arômes. Portez à ébullition
\etape rajoutez les poires coupées en morceaux grossiers (typiquement 8 morceaux par poire) et laissez cuire une heure et demi 
environ jusqu'à ce que le jus devienne un peu plus épais, et les poires moelleuses
\etape Laissez refroidir et dégustez les poires froides avec un peu de chantilly.
\end{preparation}
\end{recette}

\section{Riz au lait}
\begin{recette}{Riz au lait}{0}{1h}{}\index{riz}\index{riz au lait}
\begin{ingredients}[4 pers.]
\ingredient 120g de riz rond
\ingredient 650g de lait (demi-écrémé ça marche, entier aussi)%avec 625 c'était un tout petit peu crémeux, avec 635 ça n'a pas changé beaucoup
\ingredient 50g de sucre
\ingredient 1 sachet de sucre vanillé
\ingredient 70g de raisins sec
\ingredient 10g de maizena
\ingredient zeste de citron ou d'orange
\end{ingredients}

\begin{preparation}
\etape Portez de l'eau à ébullition et préparez une autre casserole avec le lait et le sucre en attendant.
\etape Préparez un bol avec les raisins et la maizena que vous mélangerez
\etape Une fois l'eau à ébullition, faites-y cuire le riz 3 minutes dedans. Commencez à faire chauffer le lait à feu moyen (5/9) pendant ce temps sans couvrir. 
\etape Au bout des 3 minutes, égouttez le riz
\etape Une fois que le lait bout, mettez-y le riz et les raisins, laissez l'ébullition reprendre puis couvrez et baissez le feu (2/9). Laissez cuire pendant 28 minutes. 
\begin{remarque}
Ça doit être très liquide à la fin, le riz va absorber pas mal de liquide en refroidissant. Il est aussi important de ne pas refroidir le riz de manière active (frigo ou bain d'eau froide)
\end{remarque}
\etape Versez alors dans un récipient et laissez à l'air libre refroidir doucement environ 4h (le riz va continuer de cuire et absorber le liquide durant cette période, il est important de ne pas mettre au frigo ou couvrir).
\end{preparation}
\end{recette}

\section{Semoule au lait}
\begin{recette}{Semoule au lait}{0}{1h}{}\index{riz}\index{riz au lait}
\begin{ingredients}[4 pers.]
\ingredient 100g de semoule fine
\ingredient 1L de lait
\ingredient 120g de sucre
\ingredient vanille et canelle
\ingredient zeste de citron ou d'orange
\end{ingredients}

\begin{preparation}
\etape Dans une casserole, porter à ébullition 1 litre de lait avec un peu de canelle, de vanille et le sucre. Une fois à ébullition, ajoutez la semoule en pluie.
\etape Remuer sans cesse pendant 5 minutes afin de laisser épaissir la semoule et éviter tout grumeau.
\etape Éteindre le feu et transférer la semoule au lait dans des ramequins ou petits bols. laisser tiédir et placer les ramequins au réfrigérateur.
\end{preparation}
\end{recette}

\section{Sorbet maison}
\begin{recette}{Sorbet maison}{0}{1h+24h+30min}{}\index{sorbet}
\begin{ingredients}
\ingredient[blanc en neige]
\ingredient 1 blanc d'oeuf
\ingredient une pincée de sel
\ingredient[sirop de sucre]
\ingredient 140g de sucre
\ingredient 80g d'eau
\ingredient[Pour les fruits]
\ingredient 250g de fruits
\ingredient 15cl de jus de fruit (ou le sirop des fruits au sirop à défaut)
\end{ingredients}

\begin{preparation}
\etape Si comme moi vous avez une sorbetière qui ne fait pas le froid, elle doit être placée 18h avant au congélateur. 
\etape La veille, préparez le sirop de sucre en mettant eau et sucre dans une casserole et en portant à ébulltion. 
\etape Mettez alors les fruits, le sirop de sucre et le jus de fruit au frigo. 
\etape Le lendemain, mélangez les fruits, le jus de fruit et le sirop de sucre, puis mixez. 
\etape Mettez le tout dans une bouteille au congélateur pendant 2h15
\begin{remarque}
Le but c'est d'avoir un jus le plus proche de zéro possible pour que la sorbetière n'ait pas à refroidir beaucoup le liquide. Sinon ça échouera.
\end{remarque}

\etape Montez les blancs en neige.
\etape Sortez alors la sorbetière, préparez là.
\etape Sortez le jus de fruit au tout dernier moment, incorporez le à la préparation quasi congelée.
\etape Laissez la sorbetière fonctionner 20 minutes environ.
\end{preparation}
\end{recette}

\section{Tarte chausson aux pommes}
\begin{recette}{Tarte chausson aux pommes}{4}{1h}{20min.}\index{tarte tatin}\index{pommes}\index{tarte}\index{calzone}\index{chausson aux pommes}
\begin{ingredients}
\ingredient 2 pâtes feuilletées
\ingredient 600g de compote de pomme ou poire
\ingredient 1/2 cac de cannelle, 1 cas de rhum
\ingredient un jaune d'oeuf ou un peu de lait pour la dorure
\end{ingredients}

\begin{preparation}
\etape Préchauffez le four à 220°C
\etape Piquez une des pâtes feuilletées puis étalez la au fond du plat
\etape Mélangez la compote, la cannelle et portez à ébullition (c'est pour que la compote ne soit pas trop liquide). Quand la compote commence à être quasiment sèche (à faire un bruit de succion quand vous tournez), arrêtez.
\etape une fois froid, rajoutez le rhum
\etape Versez et étalez la compote sur la pâte 
\etape Piquez la 2e pâte feuilletée et recouvrez la compote avec
\etape Soudez les deux pâtes ensemble en remontant les bords, puis mouillez un peu la soudure à l'aide d'un pinceau
\begin{figure}[htb]
\centering
\includegraphics[width=0.9\textwidth]{figures/tarte_chausson_pomme.pdf}
\caption{Comment sceller les deux pâtes entre elle}
\end{figure}
\etape Dorez enfin la pâte (jaune d'oeuf ou lait)
\end{preparation}



\begin{cuisson}
Mettez au four pendant 25 minutes environ à 220°C (surveillez la cuisson à partir de 20 minutes et arrêtez quand c'est doré un peu partout (et pas juste à quelques endroits).
\end{cuisson}
\end{recette}

\section{Tarte Tatin}
\begin{recette}{Tarte Tatin}{4}{1h+24h}{35min.}\index{tarte tatin}\index{pommes}
\begin{ingredients}
\ingredient 2kg de pommes (royal gala, pink lady, reine des reinettes, canada)
\ingredient 250g de pâte brisée (\refsec{sec:pate_brisee})
\ingredient 300+150g de sucre en poudre
\ingredient 300mL d'eau
\ingredient 40g de beurre
\ingredient cannelle, jus de citron, vanille
\end{ingredients}

\begin{preparation}
\etape Épluchez les pommes et citronnez-les.
\etape Coupez-les en quart dans la hauteur (et videz les)
\etape Faites un caramel avec 150g de sucre et un petit peu d'eau. Une fois coloré, déposez le au fond du plat à tarte. 
\etape dans un récipient relativement large, faites fondre 300g de sucre, 300 mL d'eau, 40g de beurre, la canelle et la 
vanille. 
\etape Une fois à ébullition, déposez la moitié des pommes et faites cuire 10 minutes. 
\etape Sortez les pommes à l'aide d'une écumoire puis faites 
cuire l'autre moitié de la même façon. Pendant la 2e cuisson, disposez les pommes de la première cuisson sur le caramel, le plus serré possible, en utilisant l'écumoire pour ne pas vous brûler
\etape Faites de même avec les pommes de la 2e cuisson
\etape Réservez le jus de cuisson des pommes à coté de la tarte, vous le ferez réduire pendant la cuisson pour en napper la 
tarte tatin à la fin.
\etape Laissez reposer toute une nuit afin que les pommes soient bien froide.
\end{preparation}

\begin{cuisson}
Faites préchauffer le four à 200°C. Faites réduire le jus de cuisson des pommes à feu moyen (pendant toute la cuisson). Si ça 
commence à être onctueux quand vous agitez la casserole (au lieu d'être simplement liquide) c'est que la gelée est prête, 
baissez le feu (ou éteignez s'il reste beaucoup de cuisson), quitte à rallumer juste avant de verser pour re-liquéfier la gelée.

Le lendemain matin, ajoutez la pâte feuilletée sur le dessus du plat. Enfoncez-la à l'intérieur du plat, profitez-en pour 
resserer les pommes au besoin. 

Presser avec la paume des mains pour bien faire adhérer la pâte aux pommes (sans la percer). 

Certains font un trou au milieu pour que la vapeur s'échappe.

Faites cuire à 200°C pendant 35 minutes environ. Recouvrez alors d'un plat de service puis retourner l'ensemble en 
un mouvement rapide mais contrôlé (attention aux projections, ça risque de couler).

Soulevez alors le plat de cuisson encore chaud. Versez alors la gelée de pomme que vous avez fait réduire. 

Vous pouvez le servir tiède avec une boule de glace à la vanille.


\end{cuisson}
\end{recette}

\section{Tarte Tatin à la banane}
\begin{recette}{Tarte Tatin à la banane}{5}{10 min.}{30 min.}\index{tarte tatin}\index{banane}
\begin{ingredients}
\ingredient 150g de sucre
\ingredient 50g de beurre
\ingredient 5cl d'eau
\ingredient 5 bananes
\ingredient 1 pâte feuilletée
\ingredient cannelle, vanille
\end{ingredients}

\begin{preparation}
\etape Préchauffez le four à 200°C
\etape Coupez les bananes en tranches d'1.5cm environ
\etape Dans une casserole, versez le sucre et laissez cuire jusqu'à l'obtention de la couleur caramel. 
\etape Ajoutez alors le beurre et laissez-le fondre. 
\etape Ajoutez les bananes, la vanille et la canelle
\etape Décuire le caramel avec l'eau puis laissez cuire à feu doux jusqu'à ce que le caramel ait bien fondu dans l'eau. Retirez 
alors du feu. 
\etape Disposez la préparation dans un moule à tarte. 
\etape Déposez la pâte feuilletée sur le dessus et enfoncez le surplus de pâte feuilletée sur les cotés afin de constituer un 
bord (en repoussant les bananes par la même occasion).
\etape Faites trois trous au couteau sur le milieu de la pâte. Il faut que le trou soit suffisamment important pour que l'air 
passe. À la fourchette par exemple, il faut étirer le trou avec la fourchette, piquer n'est pas suffisant.
\end{preparation}

\begin{cuisson}
Enfournez 25 minutes à 200°C
\end{cuisson}
\end{recette}

\section{Tiramisu}
\begin{recette}{Tiramisu}{4}{1h+24h}{}\index{tiramisu}\index{café}\index{thé}
\begin{ingredients}
\ingredient 250g de mascarpone
\ingredient 500 g de boudoirs (\~ 24)
\ingredient 3 œufs
\ingredient 50cl de café fort
\ingredient 100g de sucre en poudre
\ingredient 30g de cacao amer/pur (Van Houten)
\ingredient Rhum
\end{ingredients}

\begin{preparation}
\etape Séparez le blanc du jaune d'œufs
\etape Battez les blancs en neige\footnote{Les 
blancs sont prêts quand ils ne tombent pas en retournant le plat.} (avec une pincée de sel).
\etape Réservez les blancs puis à la place, mélangez les jaunes avec le sucre et faites blanchir
\etape Ajoutez le mascarpone au fouet
\etape Mélangez ensuite la préparation du mascarpone avec les blancs en neige délicatement. Enveloppez le tout de mouvement 
circulaires, on longeant les bords et le dessous du récipient avec de ne pas casser les blancs en neige.
\etape Prendre un moule (un plat à gratin ou quelque chose du genre) et saupoudrez le fond de Van Houten
\etape Trempez les boudoirs dans le café fort et le rhum (le mélange doit être froid) puis étalez-les sur le plat.
\begin{remarque}
Les boudoirs ne doivent pas être totalement imbibés, juste l'extérieur, donc ne les attardez pas trop dans le café.
\end{remarque}
\etape Étalez de la crème sur les boudoirs puis saupoudrez de Van Houten
\etape Répétez les deux dernières opérations jusqu'à épuisement des ingrédients (typiquement 2 couches)
\end{preparation}

\begin{remarque}
Préparez le Tiramisu la veille afin de le faire reposer au frigo au moins quelques heures.
\end{remarque}
\end{recette}

\section{Tourte des Pyrénées à la moi}
\begin{recette}{Tourte des Pyrénées à la moi}{3}{20 min}{1h}\index{Tourte des Pyrénées}
\begin{ingredients}
\ingredient 250 g de farine
\ingredient 175 g de beurre fondu
\ingredient 1 sachet de levure chimique
\ingredient 4 oeufs
\ingredient 175 g de sucre
\ingredient 3 cuillère à soupe de rhum vieux
\end{ingredients}
%source: https://www.saintpedebigorre-tourisme.com/tourte-pyrenees/
% modifié pour avoir les doses de tatie

\begin{remarque}
Les transvasages bizarres sont dûs au fait que j'ai un robot et donc je peux pas mélanger tout d'un seul coup dans le même bol.
\end{remarque}


\begin{preparation}
\etape Préchauffez le four à 170°C
\etape Montez les blancs en neige avec une pincée de sel. 
\etape Pendant ce temps, beurrez et farinez le moule à l'aide de la farine que vous utiliserez pour la suite
\etape Une fois les blancs montés, transvasez dans un autre saladier.
\etape Battre les jaunes d'œufs avec le sucre et le sucre vanillé
\etape Ajoutez la levure, le surplus de farine utilisé pour le moule (complétez pour avoir la bonne quantité), le beurre fondu et les parfums puis mélangez bien le tout.
\etape Incorporez les blancs montés en neige.
\etape puis versez la préparation.
\end{preparation}

\begin{cuisson}
Faites cuire 45 minutes à 170°C.
% 50 minutes c'était cuit, mais un peu sec
\end{cuisson}
\end{recette}


}% End of the ``group'' where section is deactivated
