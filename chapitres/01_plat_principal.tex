% Beginning of group where section is deactivated
% This is only to get the good structure of the document 
% since ``section'' is in fact embedded in the 'recette' environment.
% This group allow us to deactivate sections ONLY in the given file and 
% not for the entire document.
{\renewcommand{\section}[1]{}

\section{Bibimbap}
\begin{recette}{Bibimbap}{4}{1h30}{}\index{bibimbap}\index{coréen}
\begin{ingredients}
\ingredient[Marinade]
\ingredient 300g de viande (boeuf ou canard, sans os c'est plus simple)
\ingredient 12.5cl de sauce soja
\ingredient 2 cuillères à soupe d'huile de tournesol
\ingredient 1 cuillère à soupe de vinaigre
\ingredient 1 cuillère à soupe de miel
\ingredient 1 gousse d'ail
\ingredient 20g de sucre en poudre
\ingredient[Plat]
\ingredient 500g de riz % 4 doses du cuiseur, c'est 650g)
\ingredient 1 courgette (je ne me souviens pas si les légumes sont crus ou revenus avant, je crois crus)
\ingredient 1 carotte
\ingredient 4 oeufs
\ingredient 1 oignon
\ingredient 8 feuilles de salade
\ingredient 200g de champignons (4 champignons)
\end{ingredients}

\begin{preparation}
\etape Emincez la viande en lamelles très fines, 5mm d'épais maximum et 1cm de large environ, pour une longueur de 3cm environ
\etape Faites mariner la viande finement émincée la nuit précédente (sinon au moins 30 minutes)
\etape Egouttez la viande de la marinade
\etape Emincez les légumes, normalement rapés (courgette, carotte, oignon, champignons)
\etape Faites cuire le riz (cuiseur à riz, ou dans 75cl d'eau salée jusqu'à évaporation totale, peut-être moins d'ailleurs, pour que le riz soit un peu sec)
\etape Faites revenir les légumes 
\etape Dans la cocotte, rajoutez alors le riz, la viande cru, les œufs entiers et crus ainsi que le reste de marinade. 
\etape Faites revenir le tout à feu vif. Les oeufs doivent colorer les autres ingrédients. Faites cuire jusqu'à ce que ce soit homogène, environ 2/3 minutes
\end{preparation}
\end{recette}

\section{Blanquette de Veau}
\begin{recette}{Blanquette de Veau}{5}{1h}{4h}\index{blanquette de veau}\index{veau}
\begin{ingredients}
\ingredient 1,2 kg d'épaule ou de tendron de veau coupé en morceaux
\ingredient 100g de lardons
\ingredient 2 carottes
\ingredient 4 oignons
\ingredient 1 branche de celeri (nouveau)
\ingredient 2 gousse d'ail
\ingredient 25cl de vin blanc sec
\ingredient 25cl de bouillon de volaille (cube + eau)
\ingredient 20cl de crème fraîche
\ingredient 1 bouquet de persil, céleri, sel, poivre, farine, quelques gouttes de citron
% ajouter poireau et navet ntait pas une bnne idee.
\end{ingredients}

\begin{preparation}
\etape Pelez carotte, ail, oignons. Coupez les carottes en petits dés, émincez l'oignon et mettez les dans un récipient commun;

\etape Mettez l'ail en bouilli à l'aide d'une fourchette et mettez le dans un bol avec 25cl d'eau et le cube de bouillon. 
Faites chauffez 1 minute au micro onde.

\etape Dans de l'eau bouillante non salée, plongez les morceaux de viande une minutes afin de les faire blanchir, puis égouttez et réservez-les. (cette étape permet en particulier de bien ``cautériser'' la viande pour que le jus n'en parte pas.

\etape Faites revenir les lardons, réservez les à part de la viande. 

\etape faites revenir les légumes (sauf céleri) tout en profitant de l'eau qu'ils rendent pour déglacer les sucs de la viande

\etape Ajoutez les lardons, puis saupoudrez une à deux cuillères à soupe de farine. Mélanger, puis versez le bouillon et le vin 
blanc afin d'obtenir une sauce épaisse et un peu blanche. 

\etape Ajoutez alors le céleri coupé en 3 ou 4 morceaux, sel et poivre et la viande. Ne pas rajouter d'eau, même si ça ne recouvre pas complètement, la sauce va augmenter au cours de la cuisson. Laissez mijoter (85-96°C) 4h à feu très doux (Si la viande est 
très dure, c'est 
que vous ne l'avez pas fait assez cuire). 

\etape Juste avant de servir, rajoutez la crème fraîche, un peu de persil et quelques gouttes de citron
\end{preparation}
\end{recette}

\section{Bœuf Bourguignon}
\begin{recette}{Bœuf Bourguignon}{4}{15 min.+12h+45min.}{6h}\index{bœuf bourguignon}\index{bœuf}\index{daube}
\begin{ingredients}
\ingredient 1.5kg de viande de bœuf (de préférence un peu grasse)
\ingredient 100g de lardons fumés (plutôt lardon qu'allumette vu que ça va cuire longtemps)
\ingredient 75cl de vin rouge corsé
\ingredient 5cl de cognac
\ingredient 3 gousses d'ail
\ingredient 2 oignons
\ingredient 4 échalotes
\ingredient 4 carottes
\ingredient 100g d'olives noires
\ingredient 20cl de bouillon de volaille (adapter pour que la viande soit recouverte quand ça mijote)
\ingredient 1 cuillère à soupe de coulis de tomate
\ingredient une cuillère à soupe de farine
\ingredient huile d'olive, sel, sucre, poivre, un clou de girofle, céleri, herbe de provence, bouquet garni
\end{ingredients}

\begin{preparation}
\etape Disposez la viande dans un plat ou saladier (je met dans une marmite pour ma part), en étalant en une couche la moins 
haute possible (afin que tout baigne dans le vin). 
\etape Émincez oignons et échalotes. Coupez les carottes en petits cubes (de 0.5cm de coté environ). Hachez l'ail. 
\etape Ajoutez sur la viande oignons, échalotes, carotte et bouquet garni. 
\etape Dans un autre saladier, versez le vin et le cognac. Ajoutez-y ail, épices (clou de girofle, céleri et herbe de 
Provence. Salez et poivrez. Mélangez puis ajoutez cette préparation sur la viande.
\etape Faites ensuite mariner cette préparation environ 12h au réfrigérateur (en couvrant le récipient par un couvercle ou du
film étirable). 
\etape Sortez la marinade du frigo. Égouttez la viande puis les légumes de la marinade dans des récipients séparés
\etape Pendant ce temps, faites revenir les lardons dans le récipient qui servira à mijoter (une cocotte ou marmite).
Réservez-les dès qu'ils sont un peu blanchis, il ne faut pas qu'ils soient cuits, juste qu'il rendent le gras
\etape Complétez le gras des lardons avec de l'huile d'olive si besoin et saisissez la viande à feu vif, juste pour la colorer
un peu, très légèrement. Réservez la viande. 
\etape Faites alors revenir les légumes de la marinade, afin qu'ils rendent l'alcool qu'ils ont absorbé, et qu'ils rissolent un
peu. 
\etape Saupoudrez une cuillère à soupe de farine sur les légumes, puis mélangez la bien. Ajoutez alors le bouillon de bœuf,
mélangez. 
\etape Incorporez ensuite la cuillère à soupe de tomate, une pincée de sucre, le vin de la marinade et les olives. 
\etape Mélangez la préparation et ajoutez enfin les lardons et la viande. 
\end{preparation}

\begin{cuisson}
%2/9 n'est pas assez fort j'ai l'impression, mais j'attends de voir ce que ça donne au long terme avant de changer la recette et mettre 3/9
% avec de la viande pas grasse, j'avais fait cuire 6h 2/9, puis 3h à 3/9 puis 3h à 30/9 plusieurs jours après. La viande pas grasse initialement sèche était mieux au bout de tout ce temps. Donc si c'est sec et dur, faut continuer à cuire.
Couvrez la marmite et faites mijoter (85-96°C) à feu doux (2-3/9) pendant environ 6h. Si la viande n'est pas grasse, il faudra peut-être plus longtemps. Gouttez la viande. Il faut qu'elle commence à perde sa tenue et à s'émietter.
\end{cuisson}
\end{recette}

\section{Brandade de poisson}
\begin{recette}{Brandade de poisson}{2}{10 minutes+1 nuit+1h30}{}\index{brandade de poisson}\index{morue}
\begin{ingredients}
\ingredient 1kg de poisson surgelés (filets de colin ou autre, le moins cher)
\ingredient gros sel (quantité à définir
\ingredient 2 gousses d'ail
\ingredient 50g de persil frais
\ingredient 50cl de crème fraiche épaisse
\ingredient un peu de lait (ou l'eau de cuisson du poisson à défaut)
\ingredient 2.5kg de pomme terre
\ingredient laurier, persil, beurre, huile d'olive, jus de citron
\end{ingredients}

\begin{remarque}
C'est bien sur la brandade de morue à la base, mais la lotte coûtant moins cher que la morue, j'ai adapté la recette. Le 
principe est de faire avec n'importe quel poisson.
\end{remarque}


\begin{preparation}
\etape La veille, mettez le poisson au sel (saumure avec 150g de sel / L) (15h au sel au frigo dans mon cas)
\etape Le lendemain, émincez l'ail très finement 
\etape Faire cuire le poisson dans une casserole d'eau froide au départ, avec deux feuilles de laurier. Portez à ébullition et 
laisser 
cuire à petite ébullition pendant 10 minutes.
\etape En parallèle, faites cuire les pommes de terre dans une casserole d'eau pendant 20 minutes (ou à l'auto-cuiseur 16 
minutes).
\etape Égouttez le poisson, enlevez l'arrête centrale et émiettez-le dans une casserole. Ajoutez de l'huile puis laissez alors rissoler à feu moyen en 
remuant de temps en temps.
\etape Hors du feu, ajoutez la crème fraîche, l'ail et le persil et tourner à nouveau pour bien mélanger les ingrédients.
\etape Écrasez les pommes de terre à la fourchette et mettre la purée ainsi formée dans la casserole.
 Si le mélange est trop sec et ne forme 
pas une purée, ajouter un peu de lait
\etape Placez dans un plat à gratin, ajouter quelques fines lamelles de beurre dessus et faire gratiner pendant 5-10 minutes à 
200°C (j'ai fait préchauffer 10 minutes, puis mis en grill porte fermée pendant 10 minutes)
\end{preparation}
\end{recette}

\section{Butter Chicken}
\begin{recette}{Butter Chicken}{}{1 nuit+45 min.}{1h}\index{poulet au beurre}\index{butter chicken}
\begin{ingredients}
\ingredient 1kg de haut de cuisse de poulet

\ingredient[Pour la marinade]
\ingredient 2 gousses d'ail (ou 1 cas de pâte d'ail)
\ingredient 2cm de gingembre
\ingredient 1/2 cuillère à café de paprika
\ingredient 10cl de crème fraiche
\ingredient 1 cuillère à soupe de garam masala
\ingredient 1/2 cuillère à café de curcuma (ou pas, c'est le truc qui donne un goût de terre)
\ingredient 1/2 cas de sel

\ingredient[Pour la sauce]
\ingredient 40g de beurre
\ingredient 24cl de crème fraiche (le reste du pot)
\ingredient 1 cuillère à soupe de jus de citron
\ingredient 2 gousses d'ail (ou 1 cas de pâte d'ail et dans ce cas pas de citron)
\ingredient 2cm de gingembre
\ingredient un peu de cardamone (normalement c'est un truc de graine un peu écrasé). Peut-être 1/2 cuillère à café)
\ingredient 2 cuillère à soupe de coriandre
\ingredient 1 cuillère à café de garam masala
\ingredient 1 cuillère à café de paprika
\ingredient 50g de cajou moulue
\ingredient 70g de concentré de tomate
\end{ingredients}

\begin{preparation}
\etape Mixez les ingrédients pour la marinade
\etape Enlevez la peau du poulet et coupez les haut de cuisse en deux le long de l'os puis mélangez à la marinade
\etape Laissez mijoter pendant au moins une nuit.
\etape Enlevez le poulet de la marinade en secouant sans chercher à tout enlever. Gardez la marinade pour plus tard
\etape Faites revenir les morceaux de poulet dans un peu d'huile jusqu'à ce que ça commence à dorer/accrocher (je déglace et réserve la marinade qui accroche pour pouvoir continuer à saisir le poulet) puis réservez
\etape Faites chauffer les épices, ail et gingembre dans le beurre pendant 1 à 2 minutes, puis ajoutez les noix de cajou, le concentré de tomate, le reste de marinade et un peu de bouillon
\etape Faites mijoter 20 minutes
\etape Mixez la sauce
\etape Ajoutez le poulet déjà saisi et faites mijoter 10 minutes
\etape Ajoutez la crème, laissez mijoter 1-2 minutes
\end{preparation}

\begin{cuisson}
Laisser mijoter 30 minutes max à feu doux, parfois moins si le poulet a cuit 20 minutes saisi dans la poële, 10 minutes dans la sauce suffisent. 

Du riz en accompagnement.
\end{cuisson}
\end{recette}

\section{Canard madère}
\begin{recette}{Canard madère}{4}{45min}{4h}\index{canard madère}\index{madère}\index{canard}
% (Excellent)

\begin{ingredients}
\ingredient 4 cuisses de canard
\ingredient 20g de beurre
\ingredient $2$ oignons
\ingredient $1$ carotte
\ingredient $2$ gousses d'ail
\ingredient 150g d'olives noires
\ingredient 50 cl de madère
\ingredient 2 cuillère à soupe rase de farine
\ingredient sel, poivre du moulin
\end{ingredients}

\begin{preparation}
\etape Écrasez les gousses d'ails, émincez l'oignon et coupez la carotte en petits cubes (en fine lamelle que vous coupez en 
tranche)
\etape Mettez l'ail dans le récipient qui recevra la viande saisie
\etape Faites fondre le beurre dans une sauteuse et faites revenir les cuisses à feu vif. Pas besoin que la viande soit cuite à l'intérieur, c'est juste pour faire dorer.
\etape Réservez les morceaux puis faites revenir l'oignon et les dés de carotte.
\etape Ajoutez alors la farine, remuez jusqu'à l'incorporer autour des légumes. Ajoutez enfin le madère et les olives noires. Ajoutez un peu de poivre, mélangez et rajoutez enfin les morceaux de canard.
\end{preparation}

\begin{cuisson}
Faites cuire pendant 4h à feu doux (3/10).
\end{cuisson}


\begin{remarque}
Mon avis est que cette sauce va très bien avec du riz.
\end{remarque}

\end{recette}

\section{Canard aux pruneaux}
\begin{recette}{Canard aux pruneaux}{3}{}{}\index{canard aux pruneaux}\index{pruneaux}\index{canard}
% (recette que j'ai inventé)
\begin{ingredients}
\ingredient morceaux de canards (8 manchons par exemple)
\ingredient 3 échalotes
\ingredient 150g de champignons
\ingredient 25cl de bouillon de volaille
\ingredient une cuillère à café de fond de veau
\ingredient 10cl de cognac
\ingredient 20cl de vin blanc
\ingredient pruneaux
\ingredient huile, beurre, sel, poivre
\end{ingredients}

\begin{preparation}
\etape Faites revenir les morceaux de canard à feu vif dans une sauteuse avec moitié beurre moitié huile d'olive. Une fois bien 
doré, réservez les.
\etape déglacez avec le cognac, et mettez les échalotes et les champignons dans la sauteuse. Couvrez et laissez mijoter jusqu'à 
ce que ce soit cuit (en remuant de temps en temps)
\etape rajoutez le bouillon de volaille, le vin blanc, le fond de veau et les pruneaux. Remuez, puis rajoutez les morceaux de 
canard.
\etape Laissez mijoter 30 minutes environ (ou plus longtemps si les morceaux sont plus gros et plus longs à cuire).
\end{preparation}

\end{recette}

\section{Canard laqué}
\begin{recette}{Canard laqué}{3}{10 min + 6h}{2h}\index{canard laqué}\index{canard}\index{sauce soja}\index{miel}

\begin{ingredients}
\ingredient 4 cuisses de canard (ou morceaux de canard)
\ingredient 6 pincées de sel
\ingredient 5g de poudre aux cinq-épices (ou 5 baies)
\ingredient 50g de miel
\ingredient 15cl de sauce soja
\ingredient 5cl de vinaigre blanc
\ingredient 40g de fécule de maïs (ou farine)
\ingredient 2 gousses d'ail écrasées et finement hachées
\ingredient 10 g de levure de boulanger (ou 20g de levure fraiche)
\end{ingredients}

\begin{preparation}
\etape Plongez les morceaux de canard dans de l'eau bouillante 30 secondes puis lavez et essuyez l'intérieur et l'extérieur avec 
des serviettes en papier.
\etape En utilisant un poinçon, faites de multiples trous dans la peau et les muscles des morceaux.
\etape Dans un bol, déposez un sac de congélation de taille moyenne. Versez tous les ingrédients de la laque, puis entortillez 
la poche pour la fermer et remuez jusqu'à obtenir un mélange homogène (c'est le miel le plus difficile à mélanger)
\etape Ajoutez alors les morceaux de canard, et fermez le sac de congélation pour de bon.
\etape Laissez mariner le canard au moins 6 heures, au réfrigérateur de préférence, en le retournant et l'arrosant de laque de 
temps en temps.
\end{preparation}

\begin{cuisson}
Passez la laque quelques minutes dans une casserole afin de l'épaissir. pas trop longtemps sinon ça fera de la gelée. 

Ajoutez un peu d'eau dans le lèche frite afin que le jus de crame pas. 

Faites ensuite cuire les morceaux de canard au four, à 200°C pendant une heure. En cours de cuisson, arrosez les morceaux de 
laque de temps en temps, quand la viande a un peu perdu son enrobage. 

(Sur une autre recette, c'est 200°C pendant une heure enrobé de papier aluminium, puis 40 minutes à 150°C sans le papier, et en arrosant toutes les 10 minutes de la laque

Servez chaud ou froid.
\end{cuisson}
\end{recette}

\section{Canard à l'orange}
\begin{recette}{Canard à l'orange}{3}{30 min}{}\index{canard à l'orange}\index{orange}\index{vinaigre}

\begin{ingredients}
\ingredient 40 à 50cl de jus d'orange
\ingredient 2 cuillères à soupe de gelée de groseille
\ingredient 2 cuillères à soupe de fécule de pomme de terre
\ingredient 15cl de bouillon de volaille
\ingredient 30g de sucre
\ingredient 15cl de vinaigre de xérès
\end{ingredients}

\begin{preparation}
\etape Préparez le jus d'orange
\etape Faites chauffer le sucre et le vinaigre et laissez réduire jusqu'à la formation d'un caramel et la dissipation des odeurs 
de vinaigre.
\begin{attention}
Lors de la disparition complète des odeurs de vinaigre, le caramel va commencer à prendre, il faut donc que ça aille vite à ce 
moment là, afin de ne pas se retrouver avec un vrai caramel très épais. 
\end{attention}

\etape Une fois pris, rajoutez de suite le jus d'orange, les morceaux de carotte et le citron. Laissez cuire 10 à 15 minutes
\etape Pendant ce temps, préparez à peu près 3 cuillères à soupe de fécule de pomme de terre dans 15cl de bouillon.
\etape Mélangez les deux et laissez cuire environ 10 minutes en remuant tout le temps. 
\etape Ajoutez 2 cuillères à soupe de gelée de groseille, et en rajouter si c'est trop acide.
\end{preparation}
\end{recette}

\section{Cassoulet à la moi}
\begin{recette}{Cassoulet à la moi}{0}{10h}{5h}\index{cassoulet}
\begin{ingredients}[7 pers.]
\item 1 kg de mouton
\item 500 g de haricots blancs secs
\item 2 tomates
\item 1 carotte
\item 3 gousses d'ail
\item 1 oignon piqué de 2 clous de girofle
\item 2 oignons
\item 1 cuillère à soupe de farine
\item 50 cl de bouillon de légumes
\item sel, poivre, thym, laurier, huile
\end{ingredients}

\begin{preparation}
\etape La veille, faites tremper les haricots dans un grand volume d'eau froide.
\etape Le lendemain, égouttez les haricots. Mettez-les dans une grande casserole avec les tomates coupées en quartiers, 
l'oignon, 1 branche de thym et une feuille de laurier. Couvrez d'eau froide et portez à ébullition.
\etape Baissez le feu et laissez cuire 1 h à feu moyen, salez et poivrez à mi-cuisson.
\etape Pendant ce temps, découpez la viande en morceaux. Faites chauffer une cocotte avec le beurre et l'huile. Faites-y dorer 
les morceaux de viande sur toutes les faces.
\etape Retirez-les de la cocotte et réservez à part.
\etape Émincez l'oignon et coupez la carotte en tout petits cubes
\etape Faites revenir oignon et carotte jusqu'à ce qu'ils soient légèrement dorés. 
\etape Saupoudrez de farine, remuez à la cuillère en bois et laissez blondir.
\etape Versez le bouillon, salez, poivrez et mélangez. Remettez la viande dans la cocotte.
\etape Pelez et coupez la carotte en rondelles et l'ail en morceaux. Ajoutez-les dans la cocotte ainsi que le thym et le 
laurier. Portez à ébullition, puis couvrez et laissez mijoter 1 h à feu moyen.
\etape Égouttez les haricots, ils doivent être encore un peu fermes. Ajoutez-les dans la cocotte avec la viande.
\etape Poursuivez la cuisson pendant 1 h.
\etape Servez bien chaud, avec la sauce de cuisson réduite à part.
\end{preparation}
\end{recette}


\section{Chatrou (poulpe)}
\begin{recette}{Chatrou (poulpe)}{4}{1h30}{}\index{chatrou}\index{poulpe}
\begin{ingredients}
\ingredient 5-10cl d'huile tournesol (il y avait un fond de 2-3mm dans l'autocuiseur de jacqueline, quantité à déterminer)
\ingredient 2kg de chatrou
\ingredient 4-5 cives
\ingredient 8 gousses d'ail
\ingredient 1 oignon
\ingredient 1 piment végétarien
\ingredient poivre, 1 clou de girofle
\ingredient 20ml de jus de citron (jus de 1/2 citron)
\ingredient 200ml de coulis de tomate (100ml par kg de chatrou)
\end{ingredients}

\begin{preparation}
\etape Égouttez et émincez le poulpe dans la marmite (jetez cette eau là, vous n'en aurez pas besoin)
\etape Préparez un bol (pour la fin de cuisson) et deux saladiers (un pour le bouillon et un pour le chatrou). 
\etape Émincez les cives, le piment végétarien et mettez dans le bol du futur bouillon avec le clou de girofle
\etape Emincez l'oignon dans le 2e saladier pour l'instant vide.
\etape Hachez 8 gousses d'ail. Mettez les dans le bol avec le jus de citron. 
\etape Faites revenir le poulpe à feu vif (7/9) jusqu'à ce que le jus rendu soit à ébullition.
\etape Une fois à ébullition, couvrez, baissez à feu moyen (5/9) et laissez cuire pendant 15 minutes. 
\etape Avec une passoire, récupérez le chatrou et mettez dans le saladier avec l'oignon émincé. Avec une louche, prélevez deux louches de jus et mettez dans le bol d'ail. Versez alors le jus restant dans un saladier préparé tout à l'heure avec le clou de girofle.
\etape Faites chauffer l'huile (il en faut pas mal)
\etape Faites revenir le chatrou avec l'oignon pendant 10 minutes environ. Quand ça commence à accrocher (7-8 minutes) écourtez la cuisson plutôt que de cramer le fond
\etape Ajoutez le jus du saladier et le coulis de tomate, puis grattez le fond avec une spatule (ne pas chercher à épaissir la sauce)
\etape Faites mijoter 1h à feu moyen (4/9) et à couvert
\etape Ajoutez alors le jus du bol (+ail) et laissez à couvert et hors du feu pendant 10 minutes
\end{preparation}
\end{recette}


\section{Cochon roussi à la moi}
\begin{recette}{Cochon roussi à la moi}{}{30 min.+1 nuit+1h}{2h}
\begin{ingredients}
\ingredient[Marinade]
\ingredient 3 gousses d'ail
\ingredient 2 oignon
\ingredient 200ml d'eau
\ingredient 8g d'huile
\ingredient 2 cac de colombo (10g)
\ingredient 1/3 de cac de muscade (0.5g)
\ingredient 1/3 de cac de canelle (1g)
\ingredient 1g de poivre
\ingredient 2g de thym
\ingredient[cuisson]
\ingredient 1 roti dans l'échine de 1.5kg
\ingredient 1 patate
\ingredient 1 cive (ou les tiges d'une botte de 5 oignons)
\ingredient 10g de jus de citron (normalement 25, i.e 1/2 citron)
\ingredient 450ml d'eau
\ingredient 9g de sel
\end{ingredients}
% j'ai fait une version sans faire la marinade parce que j'avais pas le temps, et qui était très bonne aussi. C'était même plus simple à cuisiner parce que je n'avais pas à enlever la marinade des morceaux.

\begin{preparation}
\etape Coupez le roti en cube grossier (des 1/4 de cylindres, puis des tranches de 4-5cm de large)
\etape Mixez l'oignon et l'ail
\etape Ajoutez l'eau, l'huile et le jus de citron et les épices
\etape Mettez la viande à mariner pendant la nuit
\etape Le lendemain, séparez la viande de la marinade
\etape Déposez le porc dans la marmite/cocotte avec la pomme de terre coupée en morceaux grossiers et saisissez les morceaux (avec une cocotte en fonte le bon moment pour les tourner c'est quand ils ne collent plus au fond).
\etape Réservez la viande. Faites alors rissoler les cives
\etape Ajoutez deux cuillères à soupe de farine, puis ajoutez la marinade, l'eau, le sel et le jus de citron
\end{preparation}

\begin{cuisson}
Faites mijoter pendant 4h environ à feu doux (3/9) et à couvert.
\end{cuisson}
\end{recette}

\section{Colombo de poulet}
\begin{recette}{Colombo de poulet}{4}{1h}{1h}\index{poulet}\index{colombo}
\begin{ingredients}
\ingredient 1.5kg de poulet
\ingredient 4 oignons
\ingredient [optionnel] 4 piments végétariens
\ingredient 5 gousses d'ail
\ingredient 15ml de jus de citron (1/2 citron)
\ingredient 50cl de bouillon de poulet
\ingredient 5g de sel
\ingredient 40g de poudre de colombo
\ingredient 2 cuillère à soupe de maizena
\end{ingredients}


\begin{preparation}
\etape Emincez l'oignon
\etape Faites chauffer l'eau avec le bouillon cube dans un bol 2 minutes au micro onde puis mélangez. 
\etape Emincez finement le piment végétarien, mixez l'ail et mettez le tout à infuser dans le bouillon
\etape Pesez la poudre de colombo et le sel dans un bol
\etape Saisissez dans du beurre ou de l'huile les morceaux de poulet à feu vif puis réservez-les.
\etape Faites revenir l'oignon.
\etape Une fois revenus, rajoutez la poudre de colombo et le sel et mélangez. 
\etape Ajoutez enfin le bouillon et le jus de citron puis mélangez. Ajoutez alors le poulet.
\end{preparation}

\begin{cuisson}
Laissez alors mijoter à couvert et à feu doux (3/9) 45 minutes environ.
\end{cuisson}
\end{recette}

\section{Court-bouillon de poisson}
\begin{recette}{Court-bouillon de poisson}{}{1h}{2h}
\begin{ingredients}
\ingredient morceaux de poisson
\ingredient 1 cive (200g)
\ingredient 1 oignon
\ingredient huile de tournesol
\ingredient graines de roucous (environ 15)
\ingredient 4cl de jus de citron (1 citron)
\ingredient 5 gousses d'ail
\ingredient 20cl d'eau
\ingredient un peu de mélange 4 épices, persil, thym, clou de girofle
\ingredient sel, poivre
\end{ingredients}
% la dernière fois, 50g de jus de citron jaune en bouteille, c'était trop acide

\begin{preparation}
\etape [facultatif] Mettez les têtes de poissons dans 50cl d'eau, avec 5g de sel, laurier et faites cuire pendant 30 minutes puis filtrez pour avoir du bouillon au lieu de l'eau pour la suite de la cuisson
\etape Faites chauffer l'huile avec les graines de roucous dans la marmite. Une fois coloré (\~5 minutes), jetez les graines
\etape Faites revenir le poisson pendant 5 minutes à feu vif (7/9)
\etape Réservez le poisson et faites revenir oignons et cives émincées
\etape Ajoutez une seule cuillère à soupe de farine. Ajoutez le jus de citron, l'eau et l'ail écrasé puis mélangez. 
\etape Portez à ébullition
\etape Ajoutez le poisson, mettez à feu doux (3/9) et à couvert pendant 10 minutes
\end{preparation}
% Jacqueline  prépare un mélange avec un demi citron, deux gousses d'ail et un peu d'huile à verser en fin de cuisson mais j'ai remarqué qu'elle ne versait pas toute la marinade. Donc j'ai un peu modifié la recette
% les temps de cuissons ne sont pas sûrs, car elle fait au pif et une fois couvert, elle n'a pas vraiment fait cuire, elle a coupé le feu, c'est tout
\end{recette}

\section{Couscous à la moi}
\begin{recette}{Couscous à la moi}{4}{}{}\index{couscous à la moi}\index{poulet}\index{agneau}
\begin{ingredients}[8 pers.]
\ingredient 800g de semoule fine
\ingredient 1kg d'agneau
\ingredient (facultatif) 1kg poulet / 4 merguez
\ingredient $2$ oignons (130g)
\ingredient $1$ carotte (130g)
\ingredient $1$ navet (130g)
\ingredient $1$ aubergine (130g) (plus petite possible)
\ingredient 1 poireau (130g)  (plus petit possible)
\ingredient 1 branche de céleri (130g)
\ingredient[falcultatif] 20g de piquillos ou piment végétarien antillais
\ingredient $4$ gousses d'ail
\ingredient $50\unit{g}$ de concentré de tomate en boite % 50g de tomates séchées la dernière fois
\ingredient 1 cuillère à soupe rase de sucre
\ingredient $100\unit{g}$ de raisins secs
\ingredient 1 bouillon cube
\ingredient 1 cuillère à soupe de jus de citron (pour remplacer le citron confit)
\ingredient épices : \begin{itemize}
		\item une cuillère à soupe bombée de Ras-el-hanout
		\item une demi cuillerée à café de curry
		\item 1 bouquet garni
		\end{itemize}
\ingredient sel
\end{ingredients}

\begin{preparation}
\etape Coupez en petits cubes aubergine, carotte, poireau, navets et céleri et réservez dans un saladier.
\etape Dans un bol, faites chauffer le bouillon cube et de l'eau 2 minutes au micro-onde. Ajoutez le concentré de tomate, les épices, le jus de citron et l'ail mixé puis mélangez à la fourchette. 
\etape Dans la marmite, faites revenir l'agneau coupé en gros morceaux (4x4cm) dans un peu d'huile d'olive d'un seul coté.
\etape Mouillez avec les 2L d'eau puis portez à ébullition pendant 30 minutes tout en écumant de temps en temps. 
\etape Ajoutez alors les raisins, le contenu du saladier et du bol puis laissez mijoter pendant 2h30 environ à feu moyen 4/9
\etape Si vous voulez mettre du poulet ou des merguez, rajoutez-les 1h avant la fin de cuisson sinon ça sera trop cuit
\etape Dans le saladier déjà sale des légumes, versez 30g d'huile, une demi cuillère à café de ras el-hanout et 5g de sel. Mélangez le tout avant de mettre la semoule (sinon ça va faire des boulettes d'épices).
\etape Ajoutez la semoule et mélangez bien le tout. 
\etape Ajoutez alors de l'eau jusqu'à recouvrir la semoule, secouez le saladier pour égaliser la surface et laissez reposer 30 minutes avant d'égrener à la fourchette.
\end{preparation}
\end{recette}



\section{Curry de crevettes}
\begin{recette}{Curry de crevettes}{1h}{30min.}{}
\begin{ingredients}
\ingredient 1.5kg de crevettes
\ingredient 2 oignons
\ingredient 5 gousses d'ail
\ingredient 1 cas de pâte de curry
\ingredient 40cl de lait de coco
\ingredient 20cl de crème liquide % normalement 35cl et 4cac de crème fraiche, mais je remplace la liquide par la fraiche pour avoir un truc plus épais et plus concentré. 
\ingredient 20cl de crème fraiche
\end{ingredients}

\begin{preparation}
\etape Pelez les crevettes
\etape Émincez l'oignon
\etape Faites dorer l'oignon dans un peu d'huile
\etape Réservez, puis saisissez les crevettes 2-3 minutes à feu vif avec le curry (juste pour colorer, c'est déjà cuit)
\etape Ajoutez la crème liquide, crème fraiche et lait de coco
\etape Laissez réduire un peu à feu doux
\end{preparation}
\end{recette}

\section{Curry de poisson}
\begin{recette}{Curry de poisson}{}{30min.}{}
\begin{ingredients}
\ingredient 1kg de morceaux de poisson
\ingredient 1 oignon
\ingredient 2cac de curry
\ingredient 45ml de jus de citron (1 citron)
\ingredient 1 morceau de gingembre (50g)
\ingredient 30cl d'eau + 1 bouillon cube
\ingredient 20cl de crème fraiche
\ingredient sel, poivre
\end{ingredients}

\begin{preparation}
\etape Emincez l'oignon, mixez le gingembre et réservez dans un récipient
\etape Coupez le poisson en cube grossiers (j'ai fait des tranches d'environ 1cm dans des filets de 4-5cm de large
\etape Faites revenir l'oignon et le gingembre sans chercher à faire bien dorer
\etape Ajoutez le curry, le bouillon et le sel puis faites cuire 5 minutes à feu doux
\etape Ajoutez la crème fraiche, portez à ébullition puis rajoutez le poisson
\etape Faites alors cuire 7 minutes à couvert et à feu moyen-vif (7/9)
\etape En fin de cuisson, ajoutez le jus de citron et éventuellement de la coriandre. 
\end{preparation}
\end{recette}

\section{Croque monsieur}
\begin{recette}{Croque monsieur}{3}{}{}\index{croque monsieur}
\begin{ingredients}
\ingredient 24 tranches de pain de mie
\ingredient 120g de beurre (5g par tranche)
\ingredient fromage en tranche
\begin{remarque}
J'achète un morceau d'emmental de 500g, et je fais des tranches. Avec deux tranches sur la largeur je fais une surface de pain 
de mie.
\end{remarque}

\ingredient 4 tranches de jambon blanc
\begin{remarque}
C'est aussi très bon si on remplace le jambon par du saumon ou de la charcuterie diverse.
\end{remarque}

\ingredient poivre
\end{ingredients}

\begin{preparation}
\etape Faites fondre la moitié du beurre et mélangez ensuite le reste du beurre en petit cube pour obtenir du beurre très mou. Si c'est trop fondu, mettez le bol de beurre dans un bain marie d'eau froide pendant 15 minutes.
\etape Beurrez un coté du pain de mie
\etape disposez le coté beurré à l'extérieur (il sera en contact avec la partie chaude)
\etape disposez une couche de fromage, une couche de jambon, une pincée de poivre, puis une autre couche de fromage et enfin une 
tranche de pain de mie, coté beurré à l'extérieur
\etape faites cuire dans un appareil pour les croque-monsieurs (ou au four le cas échéant)
\end{preparation}
\end{recette}

\section{Escalopes à la milanaise}
\begin{recette}{Escalopes à la milanaise}{3}{}{}\index{escalopes à la milanaises}\index{escalopes}\index{veau}
\begin{ingredients}
\ingredient 2 escalopes de veau
\ingredient 30g de parmesan râpé
\ingredient 30g de chapelure
\ingredient 1 œuf
\end{ingredients}

\begin{preparation}
\etape Prenez deux assiettes. Dans l'une d'elle, on mélange chapelure et parmesan. Dans l'autre on bat l'œuf
\etape On trempe les deux faces des escalopes d'abord dans l'œuf puis dans le mélange chapelure/parmesan
\etape Faites ensuite cuire les escalopes dans un peu de matière grasse.
\end{preparation}
\end{recette}


\section{Garbure}
\begin{recette}{Garbure}{3}{30 min}{5 heures}\index{garbure}\index{confit}
\begin{ingredients}
\ingredient 1 chou vert, coupé en fines lanières
\ingredient 400 g de poitrine nature coupée en gros dés
\begin{remarque}
Une carcasse, des cous, talon de jambon coupé en dés et couenne conviennent très bien
\end{remarque}

\ingredient 200 g de haricots lingots (à faire tremper la veille dans l’eau) 
\ingredient 6 cuisses de confit de canard
\ingredient 4 pommes de terre, épluchées et coupées en quartier
\ingredient 4 gousses d’ail, entières
\ingredient 2 poireaux, taillés en rondelles
\ingredient 2 navets, coupés en quartier
\ingredient 2 carottes, coupées en rondelles
\ingredient 2 oignons, émincés
\ingredient 12 grains de poivre, thym, laurier
\end{ingredients}


\begin{preparation}
\etape Mettez la viande, le poivre, thym laurier.
\etape Rajoutez les légumes et les clous de girofle piqué dans un oignon. 
\etape si vous utilisez du confit, rajoutez le.
\etape recouvrir la viande puis portez à ébullition
\end{preparation}

\begin{cuisson}
Faites mijoter 4h et à couvert. Rajoutez le chou petit à petit (il prend trop de place cru, il faut attende qu'il cuise un peu pour pouvoir en rajouter). 

J'ai fait cuire 7h parce que j'ai dû rajouter le chou petit à petit, j'avais pas la place de tout mettre d'un coup.
\end{cuisson}
\end{recette}


\section{Gigot de 7h}
\begin{recette}{Gigot de 7h}{3}{45 min}{7 heures}\index{gigot de 7h}\index{gigot}\index{agneau}
% source: https://lescolisduboucher.com/recettes/agneau/gigot-de-7-heures-a-la-cuillere-d-alain-ducasse
\begin{ingredients}
\ingredient[Pate morte]
\ingredient 300g de farine
\ingredient 20cl d'eau
\ingredient un pincée de sel
\ingredient[Gigot]
\ingredient un gigot ou épaule d'agneau
\ingredient 4 ou 5 gousses d'ail
\ingredient 2 gros oignons
\ingredient 1 carotte
\ingredient 20cl de vin blanc sec
\ingredient 25cl de bouillon
\ingredient sel, poivre, thym, laurier
\end{ingredients}


\begin{preparation}
\etape Préparez la pâte morte et laissez reposer.
\etape Emincez carottes et oignons.
\etape Salez et poivre abondamment le gigot des deux cotés.
\etape Dans la cocotte préalablement huilée, saisissez le gigot de tous les cotés jusqu'à avoir une jolie croûte dorée.
\begin{remarque}
Suivant la taille de votre cocotte, il vous faudra peut-être couper l'os situé à l'extrémité, vers la souris du gigot, ne jetez 
pas ce bout d'os, mettez-le au fond de la cocotte, cela apportera encore plus de goût.
\end{remarque}
\etape Réservez le gigot puis faites fondre l'oignon et la carotte dans les sucs pendant 5 minute environ
\etape Déglacez avec le vin blanc. Ajoutez le bouillon et les herbes. 
\etape Ajoutez le gigot préalablement salé et poivré sur les légumes. Ajoutez enfin les gousses d'ail
\etape Roulez la pâte morte en un boudin de 2cm de diamètre environ, puis posez le sur le tour de la cocotte. Posez ensuite le couvercle par dessus pour le sceller (je déconseille de mettre sur boudin sur le couvercle d'abord, il risque de tomber en retournant le couvercle).
\end{preparation}

\begin{cuisson}
Faites cuire le gigot dans la cocotte fermée pendant 7 heures à 120°C. 

Je conseille de servir ce plat avec des pommes de terre au four (\refsec{sec:pomme-de-terre-four}). Il est aussi possible 
d'épaissir un peu la sauce à la fin de la cuisson du gigot. 
\end{cuisson}
\end{recette}

\section{Gratin Dauphinois}
\begin{recette}{Gratin Dauphinois}{3}{}{}\index{gratin dauphinois}\index{pomme de terre}
\begin{ingredients}
\ingredient $800$ g de pommes de terre
\ingredient $25$ cl de lait entier
\ingredient $30$ cl de crème fraîche
\ingredient sel
\ingredient poivre
\ingredient noix de muscade
\ingredient $1$ grosse noix de beurre
\ingredient $3$ gousses d'ail
\end{ingredients}

\begin{preparation}
\etape Laver, éplucher et émincer les pommes de terre en tranches de $3$ mm environ.\footnote{Ne pas les laver après la coupe.}
\etape Les disposer dans une casserole avec $25$ cl de lait (entier si possible), une grosse noix de beurre, sel, poivre et 
muscade.
\etape Porter à ébullition puis baisser le feu légèrement et poursuivre la cuisson une dizaine de minutes.\footnote{Remuer de 
temps en temps avec une spatule pour éviter que la préparation attache.}
\etape Quand les pommes de terres s'enrobent d'une sorte de crème, verser à ce moment $30$ cl de crème.
\etape Laisser cuire à petit feu pendant une dizaine de minutes environ.
\etape Retirer du feu, ajouter l'ail.
\etape Disposer délicatement les pommes de terre dans un plat à gratin.
\etape Aplanir la surface et laisser refroidir pour que les goûts se mélangent.
\end{preparation}

\begin{cuisson}
Enfourner à $180\degres$ et laisser cuire entre $20$ et $30$ minutes. Servir dans le plat de cuisson.
\end{cuisson}
\end{recette}

\section{Hachis Parmentier}
\begin{recette}{Hachis Parmentier}{3}{2h}{30 min.}\index{hachis parmentier}\index{pomme de terre}
\begin{ingredients}[1 plat à gratin]
\ingredient 2.5kg de pommes de terre
\ingredient 600g de steak hachés
\ingredient 2 oignons
\ingredient 1 carotte
\ingredient 60cl de lait
\ingredient 100g de fromage rapé
\ingredient sel
\ingredient noix de muscade
\ingredient 40g de beurre
\end{ingredients}

\begin{preparation}
\etape Emincez les oignons et carottes.
\etape Épluchez les pommes de terre et coupez-les en gros cubes. 
\etape Faites chauffer une grande casserole remplie d’eau salée. Aux premiers bouillons, plongez-y les morceaux de pommes de terre et faites-les cuire pendant 25 min jusqu’à ce qu’elles soient fondantes.
\etape Égouttez les pommes de terre puis passez-les encore chaudes au moulin à légumes. 
\etape Mélangez-les avec le lait préalablement chauffé et 30 g de beurre. Salez et poivrez puis ajoutez la noix de muscade moulue et mélangez bien.
\etape Faites revenir le steak hachés puis réservez
\etape Dans la graisse du steak hachés, faites revenir les légumes
\etape Mélangez alors le tout avec la purée. Ajoutez enfin du fromage rapé
\end{preparation}

\begin{cuisson}
Préchauffez le four à 210°C. Déposez dans un plat à gratin et faites gratiner le hachis parmentier de bœuf au four pendant 30 min.
\end{cuisson}
\end{recette}

\section{Lapin confit}
\begin{recette}{Lapin confit}{4}{}{}\index{lapin}\index{confit}
\begin{ingredients}
\ingredient morceaux de lapin
\ingredient 2L d'huile de tournesol
\ingredient 200g de sel
\ingredient poivre, thym, laurier, romarin
\end{ingredients}

\begin{preparation}
\etape Préparez une assiette avec un peu de gros sel. Posez les morceaux de chaque coté dans le sel pour que quelques grains restent collés (ne pas en mettre plus pour les petits morceaux et ne pas chercher à saler dans les petits interstices). Déposez alors les morceaux dans un plat, puis mettez un peu d'épices et une pincée de sucre
\etape recommencez pour chaque morceaux. Laissez ainsi entre une nuit et 24h maxi. 
\etape Le lendemain, rincez et essuyez les morceaux
\end{preparation}

\begin{cuisson}
Faites chauffer l'huile dans une cocotte en fonte (ou équivalent) au four à 170°C. Mettez le thym, romarin et laurier dans l'huile pour la parfumer si vous en avez mis dans le sel la veille (rincez les aromates à l'eau avant). Faites cuire les morceaux pendant 5 minutes environ. 

Baissez alors le four à 130°C et faites cuire une à deux heures. Le lapin sera toujours blanc. Il faut le passer au four environ 20 minutes à 180°C pour qu'il devienne croustillant et consommable.
\end{cuisson}
\end{recette}

\section{Lapin à la tomate}
\begin{recette}{Lapin à la tomate}{5}{1h}{1h}\index{lapin à la tomate}\index{lapin}\index{poulet}
\begin{ingredients}[4 pers.]
\ingredient un lapin
\ingredient un oignon
\ingredient 1 ou 2 carottes
\ingredient 100 ou 200g de lardons fumés
\ingredient 150 à 200g de champignons
\ingredient un cube de volaille et 20cl d'eau
\ingredient 20cl de vin blanc (un verre)
\ingredient une cuillère à soupe rase de farine
\ingredient une boîte de coulis de tomate (entre 200 et 500g, la quantité exacte importe peu)
\ingredient sel, poivre, herbes de Provence
\end{ingredients}

\begin{preparation}
\etape Faites bien dorer les morceaux de lapin dans du beurre (et un peu d'huile) ; en plusieurs fois s'il n'y a pas de place 
dans la cocotte (attention, ça éclabousse!).
\etape Réservez les morceaux de lapin dans une assiette.
\etape Faites revenir l'oignon émincé et les carottes coupés en petits morceaux (rajoutez un peu d'huile si besoin). Réservez.
\begin{remarque}
Pendant ce temps, je met le bouillon cube et l'eau dans un bol que je fais chauffer au micro-onde, puis je mélange avec une 
fourchette quand c'est chaud.
\end{remarque}
\etape Rajoutez ensuite les lardons, faites revenir. Réservez les lardons et conservez la graisse. 
\etape Ajoutez alors les champignons, faites les revenir, puis ajoutez les lardons, oignon et carotte.
\etape Saupoudrez alors le tout avec la farine et mélangez. 
\etape Mouillez ensuite avec un verre de vin blanc, le coulis de tomate et le bouillon préalablement préparé. Ajoutez les herbes 
de Provence et mélangez.
\etape Rajoutez les morceaux de lapin dans la cocotte et remuez-les un peu dans la sauce.
\end{preparation}

\begin{cuisson}
Couvrez et laissez cuire à feu très doux 1h en mélangeant de temps en temps. Ajoutez sel et poivre en fin de cuisson.
\begin{remarque}
Les lardons salent déjà pas mal la sauce, je ne la resale quasiment jamais. Par contre je poivre avant de faire mijoter une 
heure.
\end{remarque}
\end{cuisson}
\end{recette}

\section{Lapin en gibelotte}
\begin{recette}{Lapin en gibelotte}{4}{}{}\index{paupiettes}\index{lapin}\index{gibelote}\index{poulet}
\begin{ingredients}
\ingredient un lapin
\ingredient 100 g de champignons
\ingredient deux ou trois oignons
\ingredient 25 cl de vin blanc sec
\ingredient 25 cl de bouillon (1 bouillon cube de volaille)
\ingredient $100\unit{g}$ de lardons
\ingredient 1 cuillère à soupe rase de farine
\ingredient 2 cuillères à café de fond de veau
\ingredient sel, poivre (un sachet d'arômes).
\end{ingredients}

\begin{preparation}
\etape Faire revenir les lardons (réservez), puis les champignons (réservez), et enfin les oignons (réservez).
\etape Découper le lapin et faire dorer les morceaux dans de l'huile d'olive (penser à laisser un peu plus longtemps les cuisses 
qui ont plus de viande)
\etape Réserver les morceaux
\etape Dans les sucs, mettez une cuillère à soupe rase de farine. Laissez roussir, puis diluez avec un peu du vin blanc.
\etape Ajoutez alors le reste de vin blanc, le bouillon, 2 cuillères à soupe de fond de veau, les lardons, oignons et 
champignons. Remuez pour diluer le fond de veau.
\etape Arômatisez selon votre gout.
\end{preparation}

\begin{cuisson}
Faire cuire à feu doux pendant 1h30.

\begin{remarque}
C'est aussi excellent avec des paupiettes de veau.

Dans ce cas, à la fin de la cuisson, stockez séparément les paupiettes et la sauce, pour pouvoir dégraisser la sauce une fois 
froide.
\end{remarque}
\end{cuisson}
\end{recette}

\section{Lasagnes}
\begin{recette}{Lasagnes}{4}{1h}{5h+24h+1h}\index{lasagne}\index{bœuf}
\begin{ingredients}
\ingredient[bolognaise]
\ingredient voir page~\pageref{sec:bolognaise}
\ingredient[Béchamel]
\ingredient 500ml de lait
\ingredient 30g de beurre
\ingredient 30g de farine 
\ingredient Noix de muscade
\ingredient[lasagnes]
\ingredient 400g de mozarella
\ingredient 200g de gruyère rapé
\ingredient 600g ($\sim 2$ paquets) de pâtes fraiches pour lasagne
\end{ingredients}

\begin{remarque}
La recette donne chez moi 6 couches de pâtes (donc 5 couches de bolognaise) avec 2 paquets de 250g de pâtes fraiches (chaque paquet a 6 pâtes, et je prend 2 pâtes par couche). 
\end{remarque}

\begin{preparation}
\etape La veille, préparez la bolognaise qui mijotera 3h environ

\etape Le lendemain, préparez un roux. Faites fondre le beurre puis incorporez-y la farine. Laissez épaissir à feux doux sans colorer (roux blanc).
\etape Réservez puis portez du lait à ébullition.
\etape Hors du feu, incorporez le lait au roux petit à petit pour ne pas former de grumeaux. 
\etape Faites alors chauffer puis tournez jusqu'à ce que ça épaississe (il ne faut plus que ça ressemble à du lait ou de la crème). Je le met assez fort et je remue rapidement jusqu'à ce que ce soit bon. Il faut 2-3 minutes à peine. 
\etape Couvrez la béchamel d'un film plastique en attendant de vous en servir pour éviter qu'une croute se forme.
\etape Préchauffez le four à 180°C
\etape Préparez un plat à gratin pour le montage. Déposez au fond une couche de pâte, puis une couche de sauce par dessus ($\sim$ 2 louches). 
Disposez de la mozarella. Continuez jusqu'à épuisement de la garniture. 
\etape Finissez par une couche de pâte, la béchamel, puis le fromage râpé. 
\end{preparation}

\begin{cuisson}
Enfournez 30 minutes à 180°C. Vous pouvez terminer par 5 minutes en grill afin de faire dorer le dessus.

\end{cuisson}
\end{recette}

\section{Lasagnes au saumon}
\begin{recette}{Lasagnes au saumon}{4}{}{}\index{lasagne}\index{saumon}
\begin{ingredients}
\ingredient 500g de pavé de saumon
\ingredient 3 oignons
\ingredient 50cl de crème fraiche
\ingredient 15cl de crème fraiche liquide
\ingredient 400g de mozarella
\ingredient 200g de gruyère râpé
\ingredient 450g ($\sim 2$ paquets) de pâtes fraiches pour lasagne
\end{ingredients}

\begin{preparation}
\etape Couper le saumon en petits cubes
\etape Émincez l'oignon
\etape Saisir les morceaux de saumon dans un peu d'huile, puis réservez.
\etape Faites revenir l'oignon (rajoutez un peu d'huile au besoin)
\etape Remettez alors le saumon. Salez, poivrez et ajoutez la crème fraiche
\etape Préchauffez le four à 180°C
\etape Préparez un plat à gratin pour le montage. Déposez au fond une couche de pâte, puis une couche de sauce par dessus. 
Disposez des tranches de mozarella et du gruyère râpé. Continuez jusqu'à épuisement de la garniture. 
\etape Finissez par une couche de pâte, déposez les derniers morceaux de mozarella, la crème fraiche liquide, puis le fromage 
râpé. 
\end{preparation}

\begin{cuisson}
Enfournez 30 minutes à 180°C. Vous pouvez terminer par 5 minutes en grill afin de faire dorer le dessus.
\end{cuisson}
\end{recette}

\section{Légumes farcis}
\begin{recette}{Légumes farcis}{4}{}{}\index{aubergine}\index{farce}\index{poivron}
\begin{ingredients}
\ingredient Aubergines ou autres légumes à farcir (comptez 6 demi aubergines farcies pour 1kg de saucisse et un plat à gratin normal.
\ingredient[Farce]
\ingredient 750g de boule de campagne (pour récupérer la mie)
\ingredient 1kg de chair à saucisse
\ingredient 3 oignons
\ingredient 2 gousses d'ail
\ingredient persil
\ingredient sel, poivre, noix de muscade
\ingredient 5cl de cognac
\ingredient [option] 1 oeuf (je ne le met pas)
\end{ingredients}

\begin{preparation}
\etape Préchauffez le four à 200°C
\etape Creusez les légumes
\etape mixez oignon, ail et persil
\etape Mixez la mie du pain
\etape Mélangez tous les ingrédients et farcissez les légumes avec
\end{preparation}

\begin{cuisson}
Enfournez 45 minutes à 200°C (les patates il faut compter 1h, les champignons en théorie 30 minutes)
\end{cuisson}
\end{recette}

\section{Lentilles}
\begin{recette}{Lentilles}{3}{}{}\index{lentilles}
\begin{ingredients}
\ingredient 500g de lentilles
\ingredient un oignon
\ingredient une demi carotte
\ingredient deux gousses d'ail
\ingredient 6 saucisses / saucisse de morteaux, autre morceaux fumés
\ingredient poitrine demi-sel
\ingredient cube de bouillon de volaille
\ingredient laurier sauce, herbes de Provence, sel, poivre
\end{ingredients}

\begin{preparation}
\etape Découpez finement la carotte et oignons en mirepoix (petits cubes) et mixez l'ail.
\etape Mettez le bouillon cube dans un bol d'eau au micro onde 2 minutes puis mélangez. Ajoutez-y l'ail.
\etape Faites des entailles verticales dans la couenne pour la séparer en 4 demi cylindres (afin que le gras s'échappe mieux)
\etape Ajoutez l'huile dans une sauteuse et passez brièvement le petit salé du coté de la couenne.
\etape Réservez le petit salé et colorer les saucisses sur toutes les faces.
\etape Retirez les saucisses de la cocotte.
\etape Ajoutez les légumes dans la cocotte et les faire revenir doucement à l'huile.
\etape Ajoutez les lentilles. Mouillez à hauteur avec de l'eau et le cube de bouillon de volaille, rajoutez les viandes et 
laissez cuire 1h à feu doux.
\end{preparation}
\end{recette}

\section{Makis saumon/avocat}
\begin{recette}{Makis saumon/avocat}{}{1h30}{}
\begin{ingredients}[5 rouleaux, pour 2 personnes]
\ingredient 315g de riz % (315g de riz la fois où j'ai fait, i.e 2 dose du cuiseur à riz)
\ingredient 50g de vinaigre de riz (4 cuillères à soupe)
\ingredient 30g de sucre (2 cuillères à soupe)
\ingredient 8g de sel (1/2 cuillère à soupe)
\ingredient 1 avocat
\ingredient 100g de saumon
\ingredient 5-6 feuilles de Nori (ou feuille de riz)
\end{ingredients}

\begin{preparation}
\etape Faites cuire le riz
\etape Dans une casserole, mettre le vinaigre, le sucre et le sel. Chauffez doucement jusqu'à dilution du sucre.
\etape Ensuite versez dans le riz cuit, et mélangez
\etape Laissez refroidir pendant 1h minimum.
\etape Préparez le saumon et l'avocat. 1cm de large maximum, et entre 3 et 5mm d'épaisseur pour les lamelles. 
\etape Etalez du riz sur environ 9cm de large et la plus petite épaisseur possible (environ 5mm en pratique), un peu au milieu de la feuille de nori. 
\begin{center}
\includegraphics[width=7cm]{figures/maki.pdf}
\end{center}
\etape Humidifiez les deux derniers centimètres de la feuille afin de faire la jointure.
\etape Réservez au frais jusqu'au moment du repas
\etape Préparez un bol d'eau froid avec un sopalin imbibé. Découpez délicatement sans appuyer afin de ne pas écraser. Mouillez/nettoyez la lame périodiquement afin qu'elle n'accroche pas pendant la découpe des makis.  
\end{preparation}
\end{recette}

\section{Nouilles japonaises}
\begin{recette}{Nouilles japonaises}{5}{4h}{1h}\index{nouille}\index{poulet}\index{Champignon noir}
\begin{ingredients}
\ingredient 450g de nouilles somen (fines)
\ingredient 2 cuisses entières de poulets (ou 4 pilons ou 4 cuisses)
\ingredient 15g de champignons noirs déshydratés
\ingredient 2 oignons
\ingredient 12.5cl ou 150g de sauce pour sauté (attention à la teneur en sel de la sauce 
\ingredient 1 demi poivron rouge
\ingredient [optionnel] 1 cac d'huile de sésame
\end{ingredients}

\begin{preparation}
\item Faites tremper les champignons noirs dans de l'eau la veille (ou dans de l'eau chaude $\sim$ 30 minutes si vous n'avez pas eu le 
temps)
\begin{remarque}
Enlevez le pieds des champignon noir si présent, cette partie est un peu dure
\end{remarque}

\item Émincez le plus finement possible l'oignon et le poivron
\item Émincez le champignon noir en coupant grossièrement en petits dés (entre 0.5 et 0.75cm de coté)
\item Enlevez le maximum de viande des morceaux de poulets et coupez les en dés grossiers. 
\item Faites chauffer l'eau pour les pâtes et saler un peu.
\item Faites revenir les os, puis mettez les dans l'eau pour les nouilles. Réutilisez l'eau des champignons si vous l'avez gardé
\item Faites revenir les morceaux de poulet et laissez le cuire (il faut 
qu'il soit cuit pour la suite) puis réservez le
\item Faites revenir oignon, poivron en rajoutant un peu d'huile. Rajoutez les champignons à mi cuisson
\item Laissez bien dorer. Rajoutez un peu d'eau en fin de cuisson au besoin pour déglacer les sucs
\item Plongez les pates dans l'eau bouillante 2min30s, égouttez puis faites revenir pendant 2-3 minutes les pâtes avec les 
légumes et le poulet en rajoutant la sauce.
\end{preparation}
\end{recette}

\section{Paëlla (à la moi)}
\begin{recette}{Paëlla (à la moi)}{2}{4h}{30 min}\index{paëlla}\index{fruits de mer}\index{chorizo}
\begin{ingredients}
\ingredient[Paella]
\ingredient 1kg de riz étuvé
\ingredient 1800g de bouillon
\ingredient 20g de sel (attention, le sel doit être mis dans le bouillon, mais la quantité dépend de la quantité de riz)
\ingredient 1kg de crevettes cuites
\ingredient 1kg d'anneaux de calmars décongelés (le moins cher en fait)
\ingredient 500g-1kg de moules
\ingredient 500g de coques (facultatif mais mieux)
\ingredient 200g de chorizo doux
\ingredient 1kg de pilons de poulet
\ingredient 140g de petits pois (les petites boites)
\ingredient 5g de paprika (2.5 cac)
\ingredient 1 dose de safran
\ingredient[fumet de poisson]
\ingredient les têtes et écailles des crevettes et autres coquilles
\ingredient 20g de beurre
\ingredient 1 oignon
\ingredient 1 petit poireau
\ingredient Thym, laurier
\ingredient sel (voir la quantité de riz)
\ingredient 75cl de vin blanc
\end{ingredients}

\begin{preparation}
\etape Durant la recettes, vous aurez besoin des récipients suivants:
\begin{itemize}
\item un tupperware si vous préparez la veille, sinon un saladier, dans lequel stocker chair de crevettes, moule et fruits à coque
\item un saladier dans lequel stocker les déchets de crevettes, moule et fruits à coque (une fois le bouillon fait, vous n'en aurez plus besoin)
\item un grand récipient dans lequel stocker le bouillon et tous les liquides au fur et à mesure
\item Une assiette ou n'importe quoi pour stocker le chorizo
\item un saladier pour stocker le poulet cuit
\end{itemize}
\etape Ces étapes peuvent être faites la veille pour gagner du temps
\etape Coupez grossièrement l'oignon et le poireau et réservez dans un grand saladier.
\etape Pelez les crevettes et réservez d'une part la chair dans un tupperware et d'autre part les déchets dans le saladier avec poireau et oignon.
\etape Faites cuire les fruits à coques et les moules dans du vin blanc et sortez au fur et à mesure quand ils sont ouverts.
\etape Réservez le jus de cuisson pour plus tard. 
\etape Réservez la chair des moules et des fruits à coques avec la chair des crevettes, et les coquilles avec les déchets des crevettes.
\etape Faites cuire les déchets de crustacés 3 minutes dans le beurre avec l'oignon et le poireau
\etape Ajoutez alors thym, laurier, le sel. Ajoutez le jus de cuisson des fruits à coque et compléter pour avoir environ 1L de bouillon
\etape Portez à ébullition. Une fois que ça bout, baissez le feu, couvrez et maintenez à petite ébullittion pendant 30 minutes environ. 
\etape Filtrez et réservez le bouillon pour plus tard
\etape Ensuite, ou le lendemain:
\etape Faites cuire la moitié du chorizo, réservez, puis faites cuire le poulet dans le gras du chorizo. Rajoutez un peu d'eau (20cl maxi), couvrez et faites cuire 30 minutes à feu doux et réservez. Récupérez l'eau et mettez dans le bouillon
\etape Pendant ce temps Emincez les anneaux d'encornets: je coupe les petits anneaux en deux, et les gros anneaux en 4. Puis mettez dans une passoire pour les égoutter (pensez à récupérer le jus)
\etape Faites chauffer la marmite à feu très vif. Faites cuire les encornets pendant 5 minutes en remuant. Réservez avec les fruits de mers
\etape Faites cuire le reste de chorizo émincé puis réservez avec la première moitié que vous avez déjà fait cuire
\etape Faites alors cuire le riz, le paprika et le safran dans le gras du chorizo pendant 5 à 10 minutes
\etape Ajoutez alors le bouillon et complétez avec de l'eau pour avoir la quantité nécessaire (adaptez en fonction de la quantité de riz, voir plus haut). 
\etape Faites bouillir, couvrez, et laissez cuire à feu moyen pendant 20 minutes. 
\etape Eteignez sans ouvrir et laisser reposer pendant 20 minutes
\etape Ajoutez alors le poulet et les fruits de mers et réchauffez 5 minutes puis servez.
\end{preparation}
\end{recette}

\section{Pâtes à l'ail}
\begin{recette}{Pâtes à l'ail}{3}{30 min.}{}\index{pâtes}\index{ail}
\begin{ingredients}[Pour 500g de pâtes]
\ingredient 20g d'ail en poudre (pas le même  goût avec l'ail frais
\ingredient 60g de parmesan
\ingredient 20 cl de crème liquide légère 
\ingredient sel, poivre
\end{ingredients}


\begin{preparation}
\etape Faites chauffer le parmesan et l'ail en poudre dans la crème pour mélanger puis éteignez le feu, salez, poivrez et laissez reposer à couvert.
\etape Une fois les pâtes cuites, rajoutez dans le plat le contenu de la casserole. Remuez jusqu'à ce que ça ait la consistance qui vous convienne (avec la chaleur des pâtes, ça va épaissir un peu).
\end{preparation}
\end{recette}

\section{Pâtes à la bolognaise (à la moi)}
\begin{recette}{Pâtes à la bolognaise (à la moi)}{4}{1h}{3h}\index{pâtes}\index{pâtes à la bolognaise}\index{bolognaise}
\begin{ingredients}
\ingredient 500g de viande hachée
\ingredient 250g de chair à saucisse
\ingredient 100g de lardons fumés
\ingredient deux oignons
\ingredient 1 carotte
\ingredient 1 petite branche de céleri (ne pas en mettre trop sinon c'est pas bon)
\ingredient 1kg de purée de tomate
\ingredient 25cl de vin rouge
\ingredient 25cl de lait
\ingredient sel, poivre, céleri, laurier (deux feuilles), sucre ou lait (pour adoucir le coulis)
\end{ingredients}
% La dernière fois, j'ai fait 1.5kg de coulis, 40cl de vin, 40cl de lait, et 800g de haché (en conservant les autres proportions)

\begin{preparation}
\etape Coupez la carotte en petits dés et émincez l'oignon.
\etape Faites revenir la viande (steak haché, lardon et chair à saucisse)
\etape Réservez la viande dans le récipient qui servira à mijoter. 
\etape Retirer le surplus de gras et faire revenir les légumes
\etape Ajoutez la viande aux légumes, une cuillère à soupe de farine.
\etape Déglacez les sucs à l'aide du vin dans la poêle qui a servi à saisir. 
\etape Versez le vin, le lait, le coulis de tomate puis mélangez. Ajoutez alors la branche de céleri
\end{preparation}


\begin{cuisson}
Couvrir et laisser cuire à feu très doux (3/9, pas moins, il faut une petite ébullition quand même) pendant 3h. Ajoutez sel et poivre en fin de cuisson.
\end{cuisson}
\end{recette}\label{sec:bolognaise}

\section{Pâtes à la carbonara}
\begin{recette}{Pâtes à la carbonara}{4}{}{}\index{pâtes}\index{pâtes à la carbonara}\index{carbonara}
\begin{ingredients}[Pour 500g de pâtes]
\ingredient 100g de lardons
\ingredient 2 oignons
\ingredient 50 à 100g de parmesan
\ingredient 20 cl de crème liquide légère 
\ingredient herbes aromatiques (herbes de Provence, romarin,\dots)
\ingredient sel (1/2 cac), poivre
\end{ingredients}
% la recette italienne originale:
% Pour 500g de pâtes
% 300g de guanciale/parmesan (essayer avec 200?)
% 300g de pancetta (essayer avec 150g?)
% 1 oeuf entier
% 4 jaunes d'oeufs

\begin{preparation}
\etape Faites cuire les lardons
\etape Une fois cuits, réservez-les dans un saladier puis faites rissoler l'oignon afin qu'il soit bien doré.
\etape Dans le saladier, rajoutez l'oignon. Déglacez alors la poêle avec un peu d'eau et versez dans le saladier.
\etape Ajoutez la crème liquide, les herbes aromatique, le sel et le poivre, puis laissez infuser jusqu'à ce que les pâtes soient cuite
\etape Une fois les pâtes cuites, rajoutez dans le plat le contenu de la casserole et le parmesan. Remuez jusqu'à ce que ça ait la consistance qui vous convienne (avec la chaleur des pâtes, ça va épaissir un peu).
\begin{remarque}
S'il y a vraiment trop de liquide, rallumez un peu le feu sous la marmite tout en remuant jusqu'à ce que la consistance vous convienne.
\end{remarque}
\end{preparation}

\end{recette}

\section{Pâtes au boudin}
\begin{recette}{Pâtes au boudin}{2}{10 min}{}\index{pâtes}\index{boudin}
\begin{ingredients}
\ingredient 200 g Boudin à l'oignon
\ingredient 500g de pâtes
\end{ingredients}

\begin{preparation}
\etape Enlevez la peau au boudin, et coupez le en morceaux grossiers
\etape Mettez ces morceaux dans une poêle à feu moyen
\etape Avec une spatule en bois, remuez et écrasez un peu les morceaux pour qu'ils se désolidarisent et finissent par former une 
sorte de bouilli peu agréable à l'œil.
\etape Ajoutez ça aux pâtes et remuez pour que ça soit homogène.
\end{preparation}
\end{recette}

\section{Pâtes au hareng}
\begin{recette}{Pâtes au hareng}{4}{1h}{}\index{pâtes}\index{hareng fumé}
\begin{ingredients}
\ingredient 25cl de crème fraîche liquide + 5cl d'eau (sinon c'est trop épais)
\ingredient 1 fenouil
\ingredient 200g de hareng fumé
\ingredient 100g de parmesan
\ingredient sel, poivre, aneth, un peu de jus de citron
\end{ingredients}

\begin{remarque}
Il est possible de préparer la sauce avant, puis faire cuire les pâtes plus tard et d'ajouter la sauce et le hareng fumé sur les pâtes chaudes
\end{remarque}


\begin{preparation}
\etape Émincez le fenouil (comme un oignon). Coupez le hareng fumé en lamelles puis en morceaux
\etape Faites revenir le fenouil émincé dans un peu d'huile pendant 10 minutes environ.
\etape Rajoutez alors un peu d'eau (5cl environ), couvrez et laissez mijoter à feu doux (3/9) pendant 10 minutes (c'est pour attendrir le fenouil qui a tendance à être encore dur à ce stade).
\etape Enlevez le couvercle, et laissez évaporer l'eau résiduelle s'il en reste encore beaucoup
\etape Réservez dans un saladier. 
\etape Ajoutez alors un peu d'eau pour déglacer (5cl) dans la poële et versez dans le saladier (faites réduire et ne versez qu'un peu avant que ça n'accroche de nouveau). 
\etape Ajoutez alors la crème liquide, un peu de citron, du poivre et le hareng dans le saladier. 
\etape Laissez infuser jusqu'à ce que les pâtes soient cuites (possibilité de préparer la suite de la recette plus tard, pratique pour prévoir le repas un peu à l'avance)
\etape Égouttez les pâtes, ajoutez alors le contenu du saladier et le parmesan dans les pâtes. 
\etape Remuez jusqu'à avoir la consistance voulue puis servez les.
\end{preparation}
\end{recette}

\section{Pâtes au pesto}
\begin{recette}{Pâtes au pesto}{2}{10 min}{}\index{pâtes}\index{pesto}\index{basilic}\index{pignon de pin}
\begin{ingredients}
\ingredient 2 gousses d'ail
\ingredient 50g de parmesan râpé
\ingredient 25g de pignon de pin
\ingredient 50g de feuilles de basilic frais
\ingredient 5cl d'huile d'olive
\ingredient 500g de pâtes
\end{ingredients}

\begin{preparation}
\etape Faites griller les pignons de pin à la poêle sur feu moyen. Surveillez les afin de ne pas les faire cramer. Dès que 
c'est roussi, arrêtez.
\etape dans un mixer, ajoutez l'ail, le basilic, les pignons et le parmesan. Mixez jusqu'à obtenir une pâte lisse et homogène 
(des coups de spatules seront nécessaires pour mixer les morceaux récalcitrants). 
\etape Dans un petit saladier, incorporez l'huile d'olive à la pâte ainsi obtenue. 
\begin{remarque}
Je rajoute l'huile dans le mixer, je remixe, et je met le tout dans un saladier. Ensuite, pour nettoyer le mixer, je met un peu 
de savon et d'eau, ça facilite le lavage
\end{remarque}
\etape incorporez cette sauce dans les pâtes une fois qu'elles seront cuites. 

\end{preparation}
\end{recette}

\section{Pâtes au roquefort}
\begin{recette}{Pâtes au roquefort}{4}{}{}\index{pâtes}\index{pâtes au roquefort}\index{roquefort}\index{champignon}
\begin{ingredients}[Pour 500g de pâtes]
\ingredient 250g de champignons
\ingredient 20cl de crème liquide
\ingredient 100g de roquefort
\end{ingredients}

\begin{preparation}
\etape Ecrasez le roquefort à la fourchette pour qu'il fonde plus facilement
\etape Dans une casserole, faites chauffer la crème liquide et le roquefort à feu doux (3/9). Eteignez dès que le roquefort est fondu (il ne faut pas trop qu'il cuise).
\etape Faites revenir les champignons émincés et faites les bien dorer et versez dans la crème de roquefort
\etape Ajoutez un peu d'eau dans la poële où vous avez fait revenir les champignons pour récupérer les sucs qui ont attaché, puis faites réduire et ajoutez au récipient de crème de roquefort. Couvrez et réservez jusqu'à la cuisson des pâtes. 
\etape une fois les pâtes cuites et égouttées, versez la préparation et remuez. 
\etape Servez
\end{preparation}
\end{recette}

\section{Pâtes au saumon (à la moi)}
\begin{recette}{Pâtes au saumon (à la moi)}{4}{1h}{}\index{pâtes}\index{saumon}
\begin{ingredients}
\ingredient 25cl de crème fraîche liquide + 5cl d'eau (sinon c'est trop épais)
\ingredient 1 fenouil
\ingredient 200g de saumon fumé
\ingredient 100g de parmesan
\ingredient sel, poivre, aneth, un peu de jus de citron
\end{ingredients}

\begin{remarque}
Il est possible de préparer la sauce avant, puis faire cuire les pâtes plus tard et d'ajouter la sauce et le saumon fumé sur les pâtes chaudes
\end{remarque}


\begin{preparation}
\etape Émincez le fenouil (comme un oignon). Coupez le saumon fumé en lamelles puis en morceaux
\etape Faites revenir le fenouil émincé dans un peu d'huile pendant 10 minutes environ.
\etape Réservez dans un saladier. 
\etape Ajoutez alors un peu d'eau pour déglacer (5cl) dans la poële et versez dans le saladier. 
\etape Ajoutez alors la crème liquide, un peu de citron, du poivre et le saumon dans le saladier. 
\etape Laissez infuser jusqu'à ce que les pâtes soient cuites (possibilité de préparer la suite de la recette plus tard, pratique pour prévoir le repas un peu à l'avance)
\etape Égouttez les pâtes, ajoutez alors le contenu du saladier et le parmesan dans les pâtes. 
\etape Remuez jusqu'à avoir la consistance voulue puis servez les.
\end{preparation}
\end{recette}


\section{Paupiettes de veau à la Normande}
\begin{recette}{Paupiettes de veau à la Normande}{4}{1h30}{}\index{paupiettes}\index{veau}\index{pommeau de normandie}
\begin{ingredients}
\ingredient 8 paupiettes de veau
\ingredient 200g de champignons
\ingredient 4 oignons
\ingredient 1 pomme assez ferme
\ingredient 200g de lardons
\ingredient 25cl de pommeau de Normandie
\ingredient 25cl de bouillon de Volaille
\ingredient sel, poivre, sucre, fond de veau, farine
\end{ingredients}

\begin{preparation}
\etape Faites revenir les paupiettes de veau à feu vif, d'abord sur la partie gras, puis sur les autres faces. 
\etape Réservez les paupiettes et mettez les champignons.
\etape Une fois les champignons revenus, réservez les dans un récipient différent des paupiettes
\etape Faites cuire les lardons pour rajouter un peu de gras, réservez-les avec les champignons
\etape mettez les oignons à revenir avec une pincée de sucre. 
\etape Pendant ce temps, coupez les pommes en petits cubes de $5\unit{mm^3}$ environ. 
\etape Une fois les oignons cuits, rajoutez les champignons et les lardons, puis ajouter en saupoudrant, une cuillère à soupe 
rase de farine. Remuez jusqu'à ce que ce soit homogène puis ajoutez le bouillon, une cuillère à soupe de fond de veau et le 
pommeau de Normandie. 
\etape Rajoutez alors les pommes et les paupiettes.
\end{preparation}

\begin{cuisson}
Faites alors cuire pendant une heure environ à feu doux et à couvert en remuant de temps en temps.
\end{cuisson}


\end{recette}

\section{Petits pois}
\begin{recette}{Petits pois}{3}{30 minutes}{2h}\index{petits pois}
\begin{ingredients}
\ingredient 1kg de petits pois surgelés (pas en conserve)
\ingredient $4$ ou $5$ oignons
\ingredient 2 carottes
\ingredient un peu de salade
\ingredient 200g de lardons fumés
\ingredient 6 saucisses
\ingredient (facultatif) un jarret demi sel
\ingredient 10g de beurre
\ingredient une cuillère à soupe de farine
\ingredient 30cl de bouillon de volaille
\end{ingredients}

\begin{preparation}
\etape Émincez les oignons, coupez les carottes en petits cubes (d'abord en julienne, puis en petits morceaux)
\etape Préparez le bouillon en mettant au micro-onde un bouillon cube dans un bol d'eau.
\etape Faites revenir les lardons puis réservez-les. Faites dorer les saucisses puis réservez-les aussi.
\etape Faites revenir les oignons et les carottes. Rajoutez un peu d'huile au besoin.
\etape Rajoutez alors les lardons, une cuillère à soupe de farine, et mélangez.
\etape Rajoutez le bouillon, les petits pois, les saucisses, la salade, mélangez et faites mijoter à feu moyen (5/9) pendant 2h.
\end{preparation}

\end{recette}


\section{Pizza ventrèche-roquefort}
\begin{recette}{Pizza ventrèche-roquefort}{3}{10 min + 20 min}{20 min}\index{pizza}\index{crème}
\begin{ingredients}
\ingredient[Pour la sauce tomate]
\ingredient Petite boite de concentré de tomate
\ingredient huile d'olive
\ingredient Origan, Poivre

\ingredient[Pour la pizza]
\ingredient Pâte à pizza
\ingredient 50g de ventrèche fine
\ingredient 50g de roquefort
\ingredient 150g de fromage râpé
\end{ingredients}

\begin{preparation}
\etape Préchauffez le four à la température maximale (moi c'est 275°C)
\etape Sortez la pâte du frigo 10 minutes avant afin qu'elle soit à température ambiante ou faites une pâte vous même.
\etape Étalez là dans un plat pour aller au four directement avec le papier sulfurisé fourni.
\etape Mélangez le concentré de tomate, de l'origan, un peu de poivre et de l'huile d'olive.
\etape Étalez le avec une cuillère à café. Il ne dois pas y en avoir beaucoup (pas besoin que la couche de coulis soit 
uniforme).
\etape saupoudrez abondamment de gruyère râpé. Il faut généralement au moins un paquet de 200g.
\etape Alignez les tranches fines de ventrèche, quadrillez si vous en avez assez. Émiettez le roquefort par dessus et enfournez.
\end{preparation}

\begin{cuisson}
Faites cuire la pizza environ 10 minutes à 275\degres C. Surveillez bien entendu, ça peut être aussi court que 4-5 minutes si 
le four est extrêmement chaud.
\end{cuisson}
\end{recette}

\section{Pizza 4 fromages}
\begin{recette}{Pizza 4 fromages}{3}{}{20 min}\index{pizza}

\begin{ingredients}
\ingredient[Pour la sauce tomate]
\ingredient Petite boite de concentré de tomate
\ingredient huile d'olive
\ingredient Origan, Poivre

\ingredient[Pour la pizza]
\ingredient Pâte à pizza
\ingredient 5 tranches de fromage de chèvre (buche)
\ingredient 4 tranches de mozarella
\ingredient 75g de roquefort
\ingredient Fromage râpé (environ 200g)
\end{ingredients}

\begin{preparation}
\etape Faites préchauffer le four à 280\degres C.
\etape Sortez la pâte du frigo et étalez là dans un plat pour aller au four.
\etape Mélangez le concentré de tomate, de l'origan, un peu de poivre et de l'huile d'olive.
\etape Étalez le avec une cuillère à café. Il ne doit pas y en avoir beaucoup (pas besoin que la couche de coulis soit 
uniforme).
\etape saupoudrez abondamment de gruyère râpé. Il faut généralement au moins un paquet de 200g. 
\etape Répartissez les tranches de chèvre. 
\begin{figure}[htb]
\centering
\includegraphics[width=0.7\textwidth]{figures/pizza_4_fromages.pdf}
\caption{Comment répartir roquefort, chèvre et mozzarella (le rapé est uniformément réparti sur la base tomate avant d'y déposer les trois autres fromages).}
\end{figure}
\end{preparation}

\begin{cuisson}
Faites cuire la pizza environ 10 minutes à 280\degres C.
\end{cuisson}
\end{recette}

\section{Pizza chèvre/miel}
\begin{recette}{Pizza chèvre/miel}{3}{}{20 min}\index{pizza}\index{miel}

\begin{ingredients}
\ingredient[Pour la sauce tomate]
\ingredient Petite boite de concentré de tomate
\ingredient huile d'olive
\ingredient Origan, Poivre

\ingredient[Pour la pizza]
\ingredient Pâte à pizza
\ingredient Fromage de chèvre (buche)
\ingredient Fromage râpé (environ 200g)
\ingredient Miel
\end{ingredients}

\begin{preparation}
\etape Faites préchauffer le four à 280\degres C.
\etape Sortez la pâte du frigo et étalez là dans un plat pour aller au four.
\etape Mélangez le concentré de tomate, de l'origan, un peu de poivre et de l'huile d'olive.
\etape Étalez le avec une cuillère à café. Il ne dois pas y en avoir beaucoup (pas besoin que la couche de coulis soit 
uniforme).
\etape saupoudrez abondamment de gruyère râpé. Il faut généralement au moins un paquet de 200g. 
\etape Répartissez les tranches de chèvre. 
\etape Trempez le manche d'une cuillère à café dans le pot de miel et déposez un mince filet de miel un peu partout.
\end{preparation}

\begin{cuisson}
Faites cuire la pizza environ 10 minutes à 280\degres C.
\end{cuisson}
\end{recette}

\section{Pizza au saumon}
\begin{recette}{Pizza au saumon}{3}{10 min + 20 min}{20 min}\index{pizza}\index{crème}\index{saumon}
\begin{ingredients}
\ingredient Pâte à pizza
\ingredient 10cl de crème fraîche liquide (ça peut marcher avec 20 aussi puisqu'on fait réduire)
\ingredient 100g de saumon fumé
\ingredient \~ 150-200g de fromage râpé
\ingredient Un minuscule fenouil
\ingredient poivre, herbes
\end{ingredients}

\begin{preparation}
\etape Dans une casserole à feu moyen-vif, mettez la crème liquide, sel, poivre, et épices (pour ma part, céleri et mélange 
pour poisson). Une pincée de farine. Remuez sans arrêt jusqu'à ce que ça commence à accrocher (typiquement, quand vous remuez, 
il y a une pellicule qui reste contre le fond. Éteignez le feu et laissez refroidir.
\etape Pendant ce temps là, coupez les tranches de saumon fumé en carré grossier de 5cm de coté environ (c'est uniquement pour pouvoir mieux répartir sur la pizza, c'est pas obligatoire).
\etape Émincez le plus finement possible le fenouil. Pour ma part, sur un petit fenouil, j'émince uniquement les tiges et je 
n'utilise pas le bulbe. 
\etape Sur la pâte à pizza (elle même sur un papier cuisson), étalez la crème. Mettez alors la pâte sur la grille du four.
\etape Mettez alors le fromage rapé. Puis saupoudrez le fenouil sur toute la surface.
\end{preparation}

\begin{cuisson}
Faites cuire la pizza environ 10-15 minutes à 280\degres C (oui, 280). Une fois sortie du four, rajouter le saumon fumé. 
\end{cuisson}
\end{recette}

\section{Poisson pané}
\begin{recette}{Poisson pané}{4}{30 minutes}{15 min.}\index{poisson}
\begin{ingredients}
\ingredient 4 filets de poisson au choix (merlu, cabillaud, dorade...)
\ingredient 3 œufs et 3 cas de lait
\ingredient 8 c. à soupe de farine
\ingredient 8 c. à soupe de chapelure
\ingredient 2 noix de beurre
\ingredient sel, poivre du moulin
\end{ingredients}

\begin{preparation}
\etape Dans une assiette plate, versez la farine et dans une autre la chapelure. Dans une assiette creuse, battez les œufs avec le lait pour en faire une omelette.
\etape Salez et poivrez à votre convenance les filets de poisson des deux côtés. Coupez-les en gros morceaux. Farinez-les puis imprégnez-les des deux côtés tour à tour d'œufs battus puis de chapelure.
\etape [optionnel] Pour un peu plus de panure, refaire la procédure oeuf / chapelure pour une deuxième couche
\etape Faites fondre du beurre dans une poêle et faites frire votre poisson pané jusqu'à ce que la panure soit bien dorée (environ 3-4 minutes de chaque coté). Servez immédiatement.
\end{preparation}
\end{recette}

\section{Poisson sauce au vin blanc}
% https://afcmlacuisine.fr/cabillaud-sauce-vin-blanc
\begin{recette}{Poisson sauce au vin blanc}{4}{45 minutes}{15 min.}\index{poisson}
\begin{ingredients}
\ingredient 1kg de poisson (j'ai fait avec deux maquereaux la dernière fois)
\ingredient 4 échalottes
\ingredient 20cl de crème fraiche
\ingredient 25cl de vin blanc
\ingredient 20cl de bouillon
\ingredient thym, laurier
\end{ingredients}

\begin{preparation}
\etape Préparez de la saumure (comptez 200g de sel pour 1L d'eau)
\etape Mettez-y le poisson environ 20 minutes
\etape Pendant ce temps, faites revenir les échalottes émincés dans un peu d'huile
\etape Ajoutez une cuillère à soupe de Maizena, puis ajoutez le vin blanc
\etape Faites réduire 2 minutes à ébullition
\etape Ajoutez la crème et le bouillon
\etape Rincez le poisson, ajoutez le dans la sauce, puis faites cuire à couvert pendant 15 minutes à feu moyen (4/9)
\end{preparation}
\end{recette}

\section{Porc au caramel}
\begin{recette}{Porc au caramel}{3}{1h30}{40min}\index{porc}\index{caramel}
\begin{ingredients}
\ingredient[Pour la sauce]
\ingredient un roti dans l'échine (1.5kg environ)
\ingredient 2 oignons
\ingredient 2 gousses d'ail
\ingredient Un morceau de gingembre frais ($\sim 50$ g)
\ingredient 12.5cl de sauce soja

\ingredient[Pour le caramel]
\ingredient 150g de sucre
\ingredient 5cl d'eau
\end{ingredients}

\begin{preparation}
\etape couper le porc en fins morceaux (je fais des tranches d'un demi centimètre environ, puis des lamelles d'1cm de large, et
je coupe ces lamelles en morceaux d'1 cm de long).
\etape pelez puis mixez l'ail, le gingembre et l'oignon que vous réservez dans un grand saladier (qui contiendra aussi le porc)
\etape Faire revenir les morceaux de porc à feu très vif pendant 3 minutes environ (pour 2kg je fais environ 4 fournées). Il
faut juste faire blanchir la viande et légèrement dorer. Réservez les morceaux cuits dans le saladier contenant gingembre et
oignon
\etape Dans la sauteuse, ajoutez enfin la totalité du porc et des légumes mixés. 
\etape Saupoudrez une cuillère à soupe de farine puis mélangez. 
\etape Ajoutez la sauce soja et laissez le tout réchauffer à feu doux pendant qu'on s'occupe du caramel
\etape Dans une casserole, versez le sucre et le fond d'eau. Mettez à feu vif et attendez que le caramel prenne une coloration
brune.
\etape Nappez alors le caramel obtenu sur le porc dans la sauteuse qui va ainsi durcir. Ne mélangez pas, laissez ainsi.
\end{preparation}

\begin{cuisson}
Couvrez et laissez mijoter à feux doux pendant 40 minutes environ.
\end{cuisson}
\end{recette}

\section{Poulet à la moutarde}
\begin{recette}{Poulet à la moutarde}{5}{30 min}{50 min}\index{poulet à la moutarde}\index{moutarde}\index{poulet}
\begin{ingredients}[6 pers.]
\ingredient 1kg de morceaux de poulet
\ingredient 8 échalotes
\ingredient 25cl de bouillon de volaille
\ingredient 4 cuillères à soupe \textbf{bombées} de moutarde à l'ancienne (il n'en faut pas moins, sinon il y a trop de crème)
\ingredient 50 cl de crème fraiche épaisse
\ingredient sel, poivre, romarin
\end{ingredients}

\begin{preparation}
\etape Verser un fond d'huile dans une marmite, puis saisir à feu vif les morceaux de poulet, d'abord coté peau, avant de les réserver.
\etape Faites revenir dans le gras les échalotes pelées et émincées, en remuant jusqu'à ce qu'elles soient dorées.
\etape Mouillez avec le bouillon, saler, poivrer. Mettez le romarin. Remettre les morceaux de viande, la peau contre le fond, couvrir et laissez mijoter (4/10) pendant 30 min.
\etape Enlevez les morceaux de viande afin de pouvoir bien remuer puis ajoutez la moutarde et la crème. Mélangez 
jusqu'à ce que la crème et la moutarde fassent une mixture homogène (plus de grumeau de moutarde ou de crème). 
\etape Rectifiez l'assaisonnement, remettez les morceaux de viande et laissez mijoter à couvert et à feu un peu plus fort (5/10) pendant 20 minutes environ.
\end{preparation}
\end{recette}

\section{Poulet au curry}
\begin{recette}{Poulet au curry}{4}{45min}{1h}\index{poulet au curry}\index{poulet}\index{curry}
% C'était excellent, mais cette fois là j'avais utilisé moyennement de l'huile (et graisse de canard). peut-être que c'est la quantité de graisse qui a fait que c'était super bon (pas sûr)

\begin{ingredients}
\ingredient 4 cuisses de poulet
\ingredient une pomme mixée
\ingredient 2 oignons
\ingredient une gousse d'ail
\ingredient une boite (500g) de coulis de tomate
\ingredient 25cl de bouillon de volaille
\ingredient 20cl de crème fraiche
\ingredient 2 cuillères à soupe bombées de curry
\ingredient un peu de jus de citron
\ingredient un peu de muscade râpée, un peu de cannelle, 1/4 de cac de sel, poivre
\end{ingredients}

\begin{preparation}
\etape Émincez les oignons, et mixez la pomme.
\etape Saisissez les morceaux de poulet dans une sauteuse
\etape Réservez les morceaux avec la pomme mixée puis faites blondir l'oignon dans les sucs.
\etape Pendant ce temps, préparez le bouillon de volaille, et écrasez la gousse d'ail (à la fourchette) pour l'incorporer au 
bouillon.
\etape Ajoutez alors le curry, la cannelle, la muscade avec les oignons. Mélangez, puis ajoutez le bouillon et la tomate. Ajoutez enfin les morceaux de poulet et la pomme mixée.
\end{preparation}

\begin{cuisson}
Faites cuire pendant 45 minutes environ, à couvert et à feu doux (3/10). Ajoutez alors la crème fraiche et un peu de jus 
de citron, puis laissez mijoter encore 15 minutes environ. 

Servez avec un riz bazmati.
\end{cuisson}
\end{recette}


\section{Poulet Chasseur}
\begin{recette}{Poulet Chasseur}{4}{1h}{1h}\index{poulet}\index{poulet chasseur}
\begin{ingredients}
\ingredient 8 morceaux de poulet
\ingredient $250\unit{g}$ de champignons
\ingredient 3 échalotes
\ingredient $4\unit{cl}$ de Cognac
\ingredient $4\unit{cl}$ de vin blanc
\ingredient Un bol de bouillon de volaille
\ingredient farine, beurre, huile, sel, poivre
\ingredient estragon, cerfeuil
\end{ingredients}


\begin{preparation}
\etape Découpez et dégraissez les morceaux de poulet. 
\etape Épluchez, lavez et émincez les champignons. Épluchez et ciselez les échalotes.
\etape Saisissez dans du beurre ou de l'huile les morceaux de poulet à feu vif puis réservez-les.
\etape Faites revenir les échalotes, réservez-les dans une assiette (pas avec les morceaux de poulet).
\etape Faites revenir les champignons dans la sauteuse. 
\etape Rajoutez alors les échalotes. Ajoutez le cognac, sortez la sauteuse de feu et faites flamber. 
\etape Ajoutez une cuillère à soupe rase de farine, mélangez.
\etape Ajoutez le vin blanc, le bouillon de volaille, le cerfeuil et l'estragon. 
\end{preparation}

\begin{cuisson}
Laissez alors mijoter à couvert et à feu doux une heure environ. À la fin, contrôlez l'assaisonnement et la liaison.
\end{cuisson}
\end{recette}

\section{Poulet gascon}
\begin{recette}{Poulet gascon}{4}{45min}{1h}\index{poulet gascon}\index{poulet}\index{floc de gascogne}
% (Excellent)

\begin{ingredients}
\ingredient 4 cuisses de poulet
\ingredient $200$ g de champignons
\ingredient 200g de lardons
\ingredient $2$ fenouils
\ingredient $2$ gousses d'ail
\ingredient $15$ cl de floc de Gascogne
\ingredient 15cl de bouillon de volaille
\ingredient 1 cuillère à soupe rase de farine
\ingredient sel, poivre du moulin
\end{ingredients}

\begin{preparation}
\etape Émincez finement le fenouil et les champignons et placez les dans des récipients séparés.
\etape Faites chauffer le bouillon cube et l'eau au micro-onde. Écrasez les gousses d'ail à l'aide d'une fourchette et ajoutez 
les dans le bouillon.
\etape Faites fondre le beurre dans une sauteuse et faites revenir les cuisses à feu vif. Pas besoin que la viande soit cuite à 
l'intérieur, c'est juste pour faire dorer.
\etape Réservez les morceaux puis faites revenir le fenouil et réservez le quand il est revenu.
\etape Faites revenir les lardons puis réservez les
\etape Faites enfin revenir les champignons.
\etape Remettez alors dans la sauteuse fenouil et lardon puis mélangez.
\etape Ajoutez alors la farine, remuez jusqu'à l'incorporer autour des légumes.
\etape Ajoutez le bouillon de volaille afin d'homogénéiser la farine entourant les légumes et le bouillon.
\etape Ajoutez enfin le floc de Gascogne, l'ail écrasé et les morceaux de poulets.
\end{preparation}

\begin{cuisson}
Faites cuire pendant une heure environ à feu doux et à couvert.
\end{cuisson}
\end{recette}

%https://bistroguru.com/recette/recette-poulet-korma/
%https://www.mesinspirationsculinaires.com/article-poulet-korma.html
% ingredient for the korma sauce at safeway: Water, Sugar, Desiccated Coconut, Cream, Coconut Paste, Onion, Canola Oil, Food Starch Modified, Contains 2% or Less of Tomato Paste, Heavy Cream, Ginger, Garlic, Spices (Including Turmeric), Salt, Lactic Acid, Lemon, Juice Concentrate, Dried Cilantro Leaf. 
\section{Poulet Korma}
\begin{recette}{Poulet Korma}{3}{1h30}{}
\begin{ingredients}
\ingredient[etape 1]
\ingredient 4 oignons moyens hachés
\ingredient 2 c-a-c bombées de garam massala % aux us j'ai mis 2 cac)
\ingredient 2 c-a-c bombées coriandre en poudre
\ingredient 1 cac rase de paprika
\ingredient 2 c-a-soupe huile végétale
\ingredient 30g de beurre
\ingredient[etape 2]
\ingredient 1kg de poulet coupé en morceau
\ingredient 70g de concentré de tomate (2 cas)
\ingredient 4 gousses ail écrasées (ou 40g de pâte d'ail)
\ingredient 50g de gingembre frais râpé % aux us j'ai mis l'équivalent de 2-3 gousses d'ail
\ingredient 400 ml bouillon de poulet
\ingredient[etape 3]
%\ingredient 400 ml lait de coco
\ingredient 1 yaourt nature (ou 20cl de creme fraiche) % aux us j'ai mis 450g de creme fraiche
\ingredient 60g d'amande moulu
\ingredient 5g de sel
\ingredient poivre
\end{ingredients}

\begin{preparation}
\etape Emincez les oignons, mixez l'air et le gingembre et réservez dans un bol.
\etape Faites revenir l'oignon dans l'huile, le beurre et les épices pendant 10 minutes. 
\etape Ajoutez le poulet, le bouillon, le concentré de tomate, l'ail et le gingembre mixé et faites cuire à couvert et à feux moyen (5/9) pendant 30 minutes.
\etape Ajoutez alors le lait de coco et les amandes moulues. Faites alors cuire 30 minutes de plus à feu très doux (3/9) et à couvert (attention à ne pas faire bouillir la sauce) Rectifier l'assaisonnement.
\etape Servir chaud accompagné de riz.
\end{preparation}
\end{recette}

\section{Purée de pois cassé}
\begin{recette}{Purée de pois cassé}{3}{45min}{}\index{purée de pois cassé}\index{pois cassé}\index{saucisse de morteau}
% (Excellent)

\begin{ingredients}
\ingredient 500g de pois cassé
\ingredient 3 pommes de terre
\ingredient 1 carotte
\ingredient 1 oignon
\ingredient 1 saucisse de morteau
\ingredient 1 bouillon cube
\ingredient sel, poivre, céleri
\end{ingredients}

\begin{preparation}
\etape Mettez les pois cassé, les légumes et la saucisse dans une marmite. Ajoutez de l'eau froide (attention, les pois cassés gonflent, il faut donc un peu d'eau sinon ça va accrocher). Salez, poivrez, mettez les 
épices (céleri par exemple), ajoutez le bouillon cube. 
\etape Faites chauffer à feu vif jusqu'à ébullition
\etape faites cuire 45 minutes à feux doux.
\etape réservez la saucisse et mixez (si vous avez beaucoup d'eau, enlevez en, puis rajoutez en pour avoir la consistance 
voulue.
\etape Piquez la saucisse tant qu'elle est chaude et avant de la servir pour enlever le gras qu'elle contient. 
\end{preparation}

\end{recette}


\section{Raclette}
\begin{recette}{Raclette}{4}{20 minutes}{}\index{raclette}
\begin{ingredients}[Par personne]
\ingredient 250g de fromage
\ingredient 600g de pommes de terre (570g)
\ingredient 200g de charcuterie
\ingredient petits oignons au vinaigre
\begin{remarque}
 Ces doses sont pour des gros mangeurs, mais en fonction des familles, il faut compter ça direct pour tout le monde.
\end{remarque}

\end{ingredients}

\begin{preparation}
\etape Je n'enlève pas la croute du fromage
\etape Mettez à tremper la plaque en marbre dès que la raclette est terminée pour que ça se nettoie plus facilement
\end{preparation}
\end{recette}

\section{Ratatouille confite et son roti de porc}
\begin{recette}{Ratatouille confite et son roti de porc}{4}{2h}{10h}\index{ratatouille}\index{roti}\index{porc}
\begin{ingredients}
\ingredient 200g de lardons % indispensable pour le gout
\ingredient 1.5kg de rôti de porc (échine)
\ingredient 4 gros oignons
\ingredient 2 belles aubergines
\ingredient 2 poivrons verts
\ingredient 1 poivron rouge
\ingredient 4 courgettes
\ingredient 10 tomates
\ingredient 140g de concentré de tomates
\ingredient 5 gousses d'ail
\ingredient 4 morceaux de sucre
\ingredient huile d'olive, sel, poivre, herbes de provence, laurier
\end{ingredients}

\begin{preparation}
\etape Peler les tomates (voir \refsec{sec:peler_tomate}), puis les écraser à la main dans une marmite. Ajoutez les 5 gousses 
d'ail écrasées, le sucre, le laurier et les herbes. 
\etape Faire réduire jusqu'à la consistance d'une purée, il faut que le liquide ait quasiment disparu. Normalement, en faisant 
réduire à feu moyen à moyen-vif pendant que vous faites revenir les autres légumes ça sera prêt. 
\begin{remarque}
Les autres légumes rendront de l'eau, ce n'est pas grave si la purée est bien réduite.
\end{remarque}
\etape Pendant ce temps, préparez les légumes : 
\etape Émincer les oignons épluchés et les faire fondre à feu doux dans une poêle avec du poivre. Réservez dans un saladier
\etape Émincer les poivrons et les faire fondre avec du sel jusqu'à ce qu'ils soient mous. Les mettre dans le saladier
\etape Laver les courgettes, les couper en petits cubes et les faire dorer à la poêle. Les mettre dans le saladier.
\etape Laver les aubergines, les couper en petits cubes et les faire dorer avec du poivre. Les réserver avec oignons, poivrons 
et courgettes.
\etape Dans la purée de tomate, rajouter le concentré de tomate, les lardons crus et les légumes revenus (oignon, poivron, 
courgette et aubergine).
\end{preparation}

\begin{cuisson}
Laisser mijoter à couvert pendant 7 à 10h environ à feu très doux (au minimum). Si la ratatouille commence à accrocher parce 
qu'il n'y a pas assez de jus, rajoutez un bol d'eau environ, remuez et remettez à mijoter.

2h avant de servir, rajoutez le rôti dans la ratatouille confite déjà chaude, et laissez mijoter au minimum jusqu'au début du 
repas. Tournez le rôti à mi-cuisson.

Le confit de ratatouille est prêt quand il change de couleur et devient foncé. 

%TODO pour cuire le roti, il faut environ 4h dans la ratatouille, après (ou à ce moment là) il a commencé à cramer. Il faut 
laisser la ficelle sinon il risque de se désagréger.
\end{cuisson}
\end{recette}

\section{Risotto aux champignons}
\begin{recette}{Risotto aux champignons}{3}{1h}{}\index{risotto}
\begin{ingredients}
\ingredient 500g de riz (cuisson longue 20 min)
\ingredient 200g de champignons
\ingredient 2 oignons
\ingredient 10cl de vin blanc (pas trop, ou pas du tout de vin blanc, sinon ce n'est pas évaporé au bout des 20 minutes de 
cuisson)
\ingredient 25cl de crème fraiche liquide
\ingredient 65cl de bouillon de volaille (à 500g de riz correspond 1L de liquide (vin + crème fraiche liquide + bouillon)
\ingredient sel, poivre, herbes de provence
\end{ingredients}

\begin{preparation}
\etape Faites blondir l'oignon émincé dans une sauteuse
\etape Réservez-le et faites revenir les champignons
\etape Réservez les champignons, ajoutez de l'huile et faites-y rissolez le riz jusqu'à ce qu'il devienne translucide.
\etape Ajoutez l'oignon, les champignons et mélangez.
\etape Ajoutez le vin blanc et remuez quelques instants 
\etape Ajoutez alors le bouillon de volaille et la crème fraiche liquide. 
\etape Ajoutez les herbes et rectifiez l'assaisonnement.
\etape Laissez cuire 25 minutes à feu moyen et à couvert, le temps que le riz soit cuit. 
\begin{remarque}
Il faut plus de temps que le temps normal indiqué pour le riz
\end{remarque}
\end{preparation}
\end{recette}


\section{Rougail saucisse}
\begin{recette}{Rougail saucisse}{3}{30 min.}{}
\begin{ingredients}
\ingredient 1kg de saucisses fumées
\ingredient 1kg de coulis de tomates
\ingredient 25cl de bouillon
\ingredient 8 oignons
\ingredient 10 gousses d'ail
\ingredient 15g de gingembre frais % normalement c'est 1/2 cac de gingembre moulu
% ne pas mettre de rhum vieux. 
\ingredient 1 cac de curcuma en poudre (1.5g)
\ingredient 1 cac de garam massala (1g)
\ingredient 8g de sel
\end{ingredients}


\begin{preparation}
\etape Emincez l'oignon
\etape Coupez les saucisses en rondelle
\etape Faites chauffer le bouillon cube dans 50cl d'eau au micro onde pendant 2 minutes, puis mélangez à la fourchette. 
\etape Mixez l'ail et le gingembre et mettez à macérer dans le bouillon
\etape Faites revenir les saucisses puis réservez les.
\etape Faites revenir l'oignon émincé. Ajoutez une cuillère à soupe de maïzena, le sel et les épices, mélangez.
\etape Ajoutez le bouillon avec ail et gingembre, déglacez les sucs et mélangez. 
\etape Ajoutez enfin le coulis de tomate et les saucisses.
\etape Portez à ébullition puis faites mijoter à feu doux (3/9) pendant 1h environ. 
\end{preparation}
\end{recette}


\section{Salade bigourdane}
\begin{recette}{Salade bigourdane}{5}{30 min}{}\index{salade bigourdane}\index{salade landaise}
\begin{ingredients}
\ingredient salade verte
\ingredient 10 à 15 noix
\ingredient 100g de lardons
\ingredient 100g de gésiers de canard confits
\begin{remarque}
Le plus pratique, ce sont les paquets de gésiers au rayon lardons ou canard. Les gésiers de volailes sont émincés et absolument 
pas gras. Beaucoup moins embêtant que des gésiers entiers, conservés dans la graisse.
\end{remarque}
\ingredient 50g de gruyère non râpé
\ingredient 1/4 de baguette de pain frais
\ingredient une gousse d'ail (ou ail semoule)
\ingredient vinaigrette (voir \refsec{sec:vinaigrette} ou \refsec{sec:vinaigrette-moutarde})

\end{ingredients}

\begin{preparation}
\etape Préparez les noix, que vous laissez dans un saladier. Ajoutez la salade, la vinaigrette et remuez.
\etape Si vous avez une gousse d'ail fraiche, frottez le pain avec. Puis coupez le pain en deux dans le sens de la longueur, et 
faites 3 à 4 lamelles dans chacun des deux morceaux. Coupez ensuite ces lamettes en petits cube d'environ un centimètre de long.
\etape Coupez le gruyère en cubes d'un demi centimètre de coté environ.
\etape Recoupez les gésiers pour faire des cubes d'un peu moins d'1 cm de coté.
\etape Faites revenir les lardons à la poêle. 
\etape Réservez-les puis faites revenir les gésiers, et réservez-les avec les lardons.
\etape Faites alors dorer le pain dans la graisse ainsi rendue. Si vous avez de l'ail semoule à la place de la gousse d'ail, 
ajoutez le à ce moment là dans la poêle afin de mélanger au pain. Il faut que le pain soit légèrement croustillant au bord, mais 
moelleux à l'intérieur. Il faut donc le surveiller, le tourner de temps en temps, et ne pas mettre à feu trop vif. Le pain ne va 
pas forcément dorer, il se peut que vous vous retrouviez avec des biscottes si vous cherchez absolument à ce qu'il colore. 
\etape Une fois le pain presque prêt, ajoutez les lardons et gésiers, remuez avec de faire réchauffer le tout, et de re-graisser 
le pain afin qu'il finisse de dorer. 
\etape Une fois chaud et prêt, éteignez le feu. Ajoutez alors le fromage en cube dans la poêle en dehors du feu, puis versez 
immédiatement sur la salade, et mangez de suite. Ainsi, le fromage sera légèrement fondant, sans faire de filaments pour autant.
\end{preparation}
\end{recette}

\section{Salade de pomme de terre aux harengs}
\begin{recette}{Salade de pomme de terre aux harengs}{5}{30 min}{}\index{hareng}
\begin{ingredients}
\ingredient 2.5kg de pomme de terre
\ingredient 300-500 de harengs
\ingredient 3 oignons
\ingredient 20 cornichons
\ingredient 8 oeufs (à faire dur)
\ingredient 240g de vinaigrette
\end{ingredients}

\begin{preparation}
\etape Faites mariner harengs, oignons et vinaigrette pendant 2 jours
\etape Faites cuire les pommes de terre dans l'eau. Mettez les dans l'eau froide salée. Une fois à ébullition, comptez 15 minutes (piquez au couteau pour vérifier)
\etape récupérez de l'eau bouillante pour faire les oeufs dur (9 minutes dans de l'eau à ébullition, puis plongez dans l'eau froide)
\etape Plongez les pommes de terre dans l'eau froide puis pelez les.
\etape coupez les cornichons et les pommes de terre. Mélangez au reste.
\end{preparation}
\end{recette}

\section{Saumon au champagne}
\begin{recette}{Saumon au champagne}{4}{20 min.}{45 min.}\index{saumon}\index{poisson}\index{champagne}
\begin{ingredients}
\ingredient 1kg de saumon
\ingredient 1 oignon
\ingredient 1 carotte
\ingredient 40g de beurre (ou huile d'olive)
\ingredient 50cl de crème fraîche
\ingredient 3 brins de thym frais
\ingredient 75 cl de mousseux brut premier prix
\ingredient sel et poivre du moulin 
\end{ingredients}

\begin{preparation}
\etape Préchauffez le four à 180°C
\etape Mixez oignon et carotte avec un peu de champagne.
\etape Dans une grande casserole, mélangez au reste de champagne, le beurre, thym, sel et poivre puis portez à ébullition (pour que le temps de cuisson du poisson soit plus facile à calculer)
\etape Dans un plat allant au four, mettez le saumon, versez la préparation au champagne bouillante. 
\etape Couvrez et faites cuire au four pendant 20 minutes environ
\etape Réservez alors le saumon au chaud
\etape Dans une casserole, mettez deux cuillères à soupe de farine et 20g de beurre
\etape Une fois homogène, ajoutez la sauce et faites réduire à 1/4 du volume initial
\etape Ajoutez alors la crème hors du feu puis servez
\end{preparation}
\end{recette}

\section{Saumon laqué miel/soja}
\begin{recette}{Saumon laqué miel/soja}{4}{20 min.}{45 min.}\index{saumon}\index{poisson}\index{sauce soja}
\begin{ingredients}
\ingredient 1kg de saumon
\ingredient 15cl de sauce soja (10 cas, ou 150g)
\ingredient 80g de miel (4 cas)
\ingredient 2 gousses d'ail (en poudre ou frais)
% enlevé la farine, parce que ça fait deux fois que la sauce est forte et je sais pas si c'est la sauce soja ou la farine. 
%\ingredient 1 cuillère à soupe de farine (testé et sauce était forte, mais sait pas si c'est dû à la farine)
\end{ingredients}

\begin{preparation}
\etape Mixez toute la préparation afin que ce soit homogène
\etape Laissez le saumon mariner au moins une heure dans un plat allant au four, recouvert de papier alu (sinon, la veille au soir)
\end{preparation}

\begin{cuisson}
Préchauffez le four à 180°C. Faites cuire dans le plat recouvert de papier alu pendant 16 minutes environ à chaleur tournante. (le saumon est froid puisqu'il était au frigo).
\begin{remarque}
Attention, sans chaleur tournante, j'ai dû faire 26 minutes à 200°C.
\end{remarque}

\end{cuisson}
\end{recette}

\section{Soupe à l'oignon}
\begin{recette}{Soupe à l'oignon}{3}{1h}{1h}\index{oignon}
\begin{ingredients}
\ingredient 2 kg d'oignons
\ingredient 25cl de vin blanc
\ingredient 2L d'eau
\ingredient 50g de beurre et 2 cas d'huile
\ingredient 1 cube de bouillon de volaille
\ingredient 2 feuilles de laurier et une branche de thym, sel, poivre
\end{ingredients}

\begin{preparation}
\etape Pelez les oignons et émincez les (au robot de préférence vu la quantité)
\etape Faites les revenir dans la marmite avec le beurre et l'huile pendant 30 minutes environ pour qu'ils soient bien dorés
\etape Ajoutez 2 cuillères à soupe de farine bombées, mélangez bien puis ajoutez l'eau, le vin blanc et le bouillon de volaille. Salez, ajoutez le thym et le laurier
\end{preparation}

\begin{cuisson}
Laissez cuire 1h à feu moyen
\end{cuisson}
\end{recette}

\section{Tagliatelles aux Noix de St Jacques}
\begin{recette}{Tagliatelles aux Noix de St Jacques}{4}{}{}\index{pâtes}\index{tagliatelle}\index{noix st Jacques}
\begin{ingredients}
\ingredient 500g de noix St Jacques
\ingredient 5 gousses d'ail
\ingredient 6 champignons
\ingredient 15cl de vin blanc
\ingredient 25cl de crème liquide
\ingredient 4 cuillères à soupe rase de sauce tomate (ne surtout pas en mettre plus)
\ingredient sel, poivre, persil
\end{ingredients}

\begin{preparation}
\etape Faire revenir les noix St Jacques, les gousses d'ail et le persil finement hachés avec une noix de beurre pendant 2 à 3 
minutes.
\etape Rajouter les champignons et laisser cuire quelques minutes. N'attendez pas que les champignons soient cuits, laissez 
simplement fondre un peu puis ajoutez le vin blanc et laissez réduire jusqu'à ce que l'odeur de vin disparaisse presque 
complètement.
\etape Ajoutez alors la crème liquide et la sauce tomate.
\begin{remarque}
Il ne faut surtout pas mettre plus de tomate que les 4 cuillères à soupe. On peut rajouter un soupçon de sucre pour corriger un 
peu la tomate ou le vin blanc.
\end{remarque}
\etape Laissez mijoter 5 minutes puis servir sur une assiette les tagliatelles cuites puis disposez les noix en sauce par dessus
\end{preparation}
\end{recette}

\section{Tajine d'agneau aux pruneaux}
\begin{recette}{Tajine d'agneau aux pruneaux}{5}{2h}{5h}\index{tajine}\index{agneau}\index{pruneaux}
\begin{ingredients}
\ingredient 250g de pruneaux
\ingredient 1kg d’oignon (à peu près)
\ingredient 1,5kg d’épaule d’agneau (je prends un gigot)
\ingredient 2 gousses d'ail
\ingredient 75cl de bouillon de volaille (ou d'eau)
\ingredient une cuillère à café de cannelle
\ingredient un petit morceau de gingembre frais (équivalent en volume des gousses d'ail)
\ingredient une dosette de safran (0.1g)
\ingredient quelques grains de coriandre ($\sim 10$)
\ingredient 1 cac rase de sel, poivre
\end{ingredients}

\begin{preparation}
\etape Saisissez les morceaux de viande dans un peu d'huile ou de graisse puis réservez-les dans une cocotte. 
\etape Faites alors revenir les oignons émincés dans l'huile d'olive et profitez-en pour récupérer les sucs de la viande. 
Quand ils sont dorés, mettez-les dans la cocotte.
\etape Salez et poivrez. Ajoutez l'ail et le gingembre écrasé, 
la cannelle, le safran et les grains de coriandre.
\etape Rajoutez le bouillon  et les pruneaux.
\begin{remarque}
Servez accompagné de semoule de blé.
\end{remarque}
\end{preparation}

\begin{cuisson}
Portez à ébullition, puis faites mijoter (85-96°C) à feux doux (2-3/9) et à couvert pendant 5 heures environ.
\end{cuisson}
\end{recette}

\section{Tartiflette}
\begin{recette}{Tartiflette}{3}{1h}{30 min}\index{tartiflette}\index{roblochon}

\begin{ingredients}
\ingredient $1,5\unit{kg}$ de pommes de terre à chair ferme
\ingredient $200\unit{g}$ de lardons
\ingredient 2 oignons
\ingredient $1$ reblochon fermier
\ingredient 20-25cl de crème fraiche
\ingredient 5cl de vin blanc sec (facultatif)
\end{ingredients}

\begin{preparation}
\etape Éplucher les pommes de terre. Faites les cuire à l'autocuiseur 15 minutes (à partir du moment où ça siffle).
\etape Au terme de la cuisson, égoutter et laisser tiédir. (ne pas rafraichir !!!)
\etape Faites revenir les lardons quelques minutes puis réservez les
\etape Émincez l'oignon et faites le suer à la poêle dans la graisse des lardons.
\etape Coupez en cubes grossiers les pomme de terre. Mélangez ces cubes avec les oignons, les lardons, la crème et le petit 
verre de vin blanc sec et étalez ça dans le plat à gratin. 
\etape Découpez le reblochon en deux dans le sens de l'épaisseur (pour plus de facilité, on peut le découper alors qu'il est 
encore dans l'emballage, le fromage se tient mieux) et le déposer sur vos pommes de terre, croûte vers le bas.
\end{preparation}

\begin{cuisson}
Enfourner à four très chaud ($220-250\degres C$). Jusqu'à ce que le reblochon fonde et gratine en surface.
\end{cuisson}
\end{recette}

\section{Tourin à l'ail}
\begin{recette}{Tourin à l'ail}{3}{}{}\index{tourin à l'ail}\index{ail}
\begin{ingredients}
\ingredient 20 gousses d'ail
\ingredient 2 gros oignons
\ingredient 2 cuil. à soupe de farine
\ingredient 3 oeufs
\ingredient 1 cuil. à soupe de vinaigre
\ingredient huile d'olive, sel, poivre
\end{ingredients}

\begin{preparation}
\etape Faire bouillir 2 l d'eau avec 20 gousses d'ail épluchées.
\etape Dans un faitout, faire revenir deux gros oignons émincés dans de l'huile d'olive jusqu'à ce qu'ils deviennent 
translucides, sans les faire brunir.
\etape Ajouter deux cuil. à soupe de farine, mélanger et mouiller avec les 2 l d'eau et l'ail.
\etape Faire bouillir, ajouter une bonne pincée de sel et deux pincées de poivre.
\etape Couvrir et laisser mijoter à feux doux pendant une petite heure.
\etape Pendant ce temps, casser trois oeufs en séparant les blancs des jaunes.
\etape Ajouter dans les jaunes une cuil. à soupe de vinaigre de vin rouge, et diluer avec un peu de bouillon. Réserver.
\etape Au bout de 1 h de cuisson, ajouter les blancs d'oeuf au bouillon en agitant continuellement avec une cuillère en bois : 
ils formeront de longs filaments blancs.
\begin{remarque}
Pour cela, il faut commencer à remuer le bouillon pour lui donner une rotation relativement importante, puis verser lentement le 
blanc d'œuf tout en continuant de remuer.
\end{remarque}
\etape Hors du feu, ajouter les jaunes en les mélangeant d'un mouvement large et ferme.
\etape Remettre à feu très doux, sans laisser bouillir, une dizaine de minutes.
\end{preparation}

\begin{remarque}
Au moment de servir le tourin, vous pouvez disposer dans chaque assiette une tartine de pain de campagne arrosée d'huile 
d'olive, et assaisonner de poivre suivant votre goût : pour le tourin, soyez avare de sel et prodigue de poivre !
\end{remarque}
\end{recette}

\section{Velouté de champignons}
\begin{recette}{Velouté de champignons}{0}{1h}{}\index{champignons}\index{velouté de champignons}
\begin{ingredients}
\ingredient 500g de champignons
\ingredient 20cl de crème fraiche
\ingredient 1L de bouillon de volaille
\ingredient 3 échalottes
\ingredient 2 gousses d'ail
\ingredient 1 cuillère à soupe rase de farine
\ingredient sel, poivre, céleri, persil
\end{ingredients}

\begin{preparation}
\etape Faire suer l'échalotte dans un peu de graisse
\etape Ajoutez les champignons émincés et faites les revenir un peu
\etape Ajoutez alors une cuillère à soupe de farine puis mélangez
\etape Ajoutez alors le bouillon, l'ail émincé, sel, poivre, céleri et persil
\etape Couvrez et laissez cuire à feu doux pendant 45 minutes
\etape Ajoutez la crème fraiche, puis mixez le tout
\end{preparation}
\end{recette}

\section{Velouté de butternut aux noix}
\begin{recette}{Velouté de butternut aux noix}{4}{1h}{}\index{velouté}\index{courgette}\index{velouté de courgettes}
\begin{ingredients}
\ingredient Une courge
\ingredient bouillon de volaille (un cube + un bol de bouillon)
\ingredient 1 échalotte (adapter en fonction de la quantité de courge)
\ingredient 2 noix
\ingredient sel, poivre, curry
\end{ingredients}

\begin{preparation}
\etape Pelez et épépinez la courge, puis coupez la en petits cubes. Émincez finement l'échalotte
\etape Faites suer la courge et l'échalotte dans un peu d'huile d'olive
\etape Ajoutez alors le bouillon, sel et poivre. Rajoutez de l'eau jusqu'à recouvrir les morceaux de courge. 
\etape Laissez cuire 
une vingtaine de minutes (feu moyen, avec un couvercle) jusqu'à ce que la courge soit moelleuse.
\etape Séparez les légumes du bouillon. 
\etape Ajoutez une cuillère à café rase de curry deux noix aux légumes. Puis mixez le tout. Durant le processus, rajoutez du 
bouillon jusqu'à avoir la fluidité voulue.
\end{preparation}
\end{recette}

\section{Velouté de potimarron}
\begin{recette}{Velouté de potimarron}{4}{30 min.}{45 min.}\index{velouté}\index{potimarron}\index{velouté de potimarron}
\begin{ingredients}
\ingredient Un potimarron (pas obligé de l'éplucher)
\ingredient bouillon de volaille (un cube + un bol de bouillon)
\ingredient 2-4 oignons
\ingredient 20cl de crème liquide
\ingredient 1 clou de girofle (piqué dans un oignon)
\ingredient 1 cuillère à soupe de curry
\ingredient 1 cac rase de sel, poivre, noix de muscade
\end{ingredients}

\begin{preparation}
\etape Pelez et épépinez la courge, puis coupez la en petits cubes. Coupez l'oignon en morceaux grossiers, gardez un oignon à part pour les clous de girofles.
\etape Faites suer la courge et l'oignon (sauf celui pour le clou de girofle) dans un peu d'huile d'olive avec le curry et la noix de muscade
\etape Ajoutez alors le bouillon, sel et poivre. Rajoutez l'eau jusqu'à recouvrir les morceaux de courge. Ajoutez l'oignon piqué du clou de girofle.
\etape Laissez cuire 45 minutes (feu moyen 6/9, avec un couvercle) jusqu'à ce que la courge soit moelleuse.
\etape Enlevez la girofle de l'oignon (et jetez là), rajoutez la crème liquide puis mixez le tout
\end{preparation}
\end{recette}

\section{Velouté de courgette}
\begin{recette}{Velouté de courgette}{4}{1h}{}\index{velouté}\index{courgette}\index{velouté de courgettes}
\begin{ingredients}
\ingredient 1kg de courgettes
\ingredient bouillon de volaille
\ingredient 125g de boursin (ou équivalent ail et fines herbes)
\ingredient sel, poivre
\end{ingredients}

\begin{preparation}
\etape Coupez les courgettes en morceaux 
\etape Dans la marmite, ajoutez de l'eau jusqu'à couvrir les courgettes et faites bouillir pendant 45 minutes dans le bouillon
de volaille. 
\etape Égouttez alors mixez les courgettes ainsi cuites avec le fromage
\etape Ajoutez sel, poivre. Vous pouvez aussi rajouter un soupçon de céleri.
\end{preparation}
\end{recette}

}% End of the ``group'' where section is deactivated
