% Beginning of group where section is deactivated
% This is only to get the good structure of the document 
% since ``section'' is in fact embedded in the 'recette' environment.
% This group allow us to deactivate sections ONLY in the given file and 
% not for the entire document.
{\renewcommand{\section}[1]{}

\section{Optimiser une marinade}
\begin{itemize}
\item Couper les aromates et légumes en tout petits morceaux pour augmenter les surfaces d'échanges. Pour certains, on peut même mixer. 
\item Chauffer un peu la marinade en amont afin d'aider à diffuser les arômes
\item Eviter de trop saler, c'est l'eau du produit qui risque de sortir et non les arômes qui vont rentrer. 
\item L'alcool déshydrate, mieux vaut le flamber rapidement pour éliminer une grande partie de l'éthanol
\item 1/4 de cuillère à café  de bicarbonate de sodium par litre dans la marinade attendrit les viandes et favorise les échanges. 
\end{itemize}

\section{Langoustes au barbecue}
\begin{recette}{Langoustes au barbecue}{0}{10 min + 6h}{}\index{langouste}
\begin{ingredients}
\ingredient langoustes
\end{ingredients}

\begin{preparation}
\etape Coupez les langoustes dans le sens de la longueur. (j'ai utilisé un hachoir de boucher et un marteau)
\etape Faites cuire 2 minutes coté chair, Laissez plus longtemps si la chair n'est pas encore blanchie
\etape Retournez et faites cuire 5 minutes coté carapace.
\end{preparation}
\end{recette}

\section{Cuisse de poulet bi-cuisson}
\begin{recette}{Cuisse de poulet bi-cuisson}{0}{10 min + 6h}{}\index{langouste}
\begin{ingredients}
\ingredient Cuisses ou pilons de poulet
\end{ingredients}

\begin{preparation}
\etape Faites cuire le poulet 30 minutes au four à 180°C. Ainsi il est presque cuit, et beaucoup plus facile et rapide à faire au barbecue. C'est notamment pratique quand il y en a une grande quantité à faire. 
\etape Réservez le jus de cuisson du poulet pour vous en servir de sauce au moment du repas.
\etape Faites alors cuire au dernier moment au barbecue à feu vif pour saisir et faire dorer, pas besoin de laisser longtemps, il est déjà quasiment cuit
\etape 
\end{preparation}
\end{recette}

\section{Marinade Aigre-douce}
\begin{recette}{Marinade Aigre-douce}{0}{10 min + 6h}{}\index{marinade}\index{miel}\index{vinaigre}
\begin{ingredients}
\ingredient 2 cuillères à soupe d'huile d'olive
\ingredient 2 cuillères à soupe de vinaigre (de vin ou balsamique, au choix)
\ingredient 2 cuillères à soupe de miel liquide
\ingredient 1 cuillère à soupe de moutarde
\ingredient 1 gousse d'ail hachée
\ingredient une pincée de sel, et un peu de poivre
\end{ingredients}

\begin{preparation}
\etape Mettez les ingrédients dans une poche de congélation
\etape Fermez la poche de manière grossière (en entortillant l'ouverture par exemple) puis secouez jusqu'à ce que la marinade 
soit homogène
\etape mettez la viande dans la poche, de préférence du porc qui va très bien avec, et laissez reposer au frigo quelques heures, 
une nuit typiquement
\etape Il ne reste plus qu'à ouvrir la poche et faire griller les morceaux marinés comme de la viande normale.
\end{preparation}
\end{recette}

\section{Marinades au vin blanc}
\begin{recette}{Marinades au vin blanc}{0}{5 min + 6h}{}\index{marinade}\index{moutarde}
\begin{ingredients}
\ingredient $\sfrac{1}{2}$ litre de vin blanc sec
\ingredient $1$ verre d'eau
\ingredient $1$ grosse cuillère à soupe de moutarde forte
\ingredient $1$ grosse cuillère à soupe de moutarde à l'ancienne
\ingredient $1$ grosse cuillère à soupe de thym
\end{ingredients}

\begin{preparation}
\etape Mettez les ingrédients dans une poche de congélation
\etape Fermez la poche de manière grossière (en entortillant l'ouverture par exemple) puis secouez jusqu'à ce que la marinade 
soit homogène
\etape mettez la viande dans la poche, de préférence du porc qui va très bien avec, et laissez reposer au frigo quelques heures, 
une nuit typiquement
\etape Il ne reste plus qu'à ouvrir la poche et faire griller les morceaux marinés comme de la viande normale.
\end{preparation}

\begin{remarque}
On peut ajouter du poivre et des oignons à cette marinade \dots  et aussi d'autres herbes\dots
\end{remarque}
\end{recette}

\section{Marinade de Porc au paprika}
\begin{recette}{Marinade de Porc au paprika}{3}{5 min + 6h}{}\index{marinade}\index{paprika}\index{porc}
\begin{ingredients}
\ingredient 2 cuillères à soupe d'huile d'olive
\ingredient 1 cuillère à café de paprika
\ingredient herbe de provence, poivre
\end{ingredients}

\begin{preparation}
\etape Mettez les ingrédients dans une poche de congélation
\etape Fermez la poche de manière grossière (en entortillant l'ouverture par exemple) puis secouez jusqu'à ce que la marinade 
soit homogène
\etape mettez la viande dans la poche, de préférence du porc qui va très bien avec, et laissez reposer au frigo quelques heures, 
une nuit typiquement
\etape Il ne reste plus qu'à ouvrir la poche et faire griller les morceaux marinés comme de la viande normale.
\end{preparation}

\end{recette}

\section{Marinade Tandoori}
\begin{recette}{Marinade Tandoori}{3}{5 min + 6h}{}\index{marinade}\index{tandoori}
\begin{ingredients}
\ingredient $2$ cuillères à soupe de Tandoori
\ingredient $2$ cuillères à soupe de jus de citron (ou de vinaigre)
\ingredient $2$ cuillères à soupe d'huile d'olive
\ingredient $1$ ou $2$ yahourt nature
\end{ingredients}

\begin{preparation}
\etape Mettez les ingrédients dans une poche de congélation
\etape Fermez la poche de manière grossière (en entortillant l'ouverture par exemple) puis secouez jusqu'à ce que la marinade 
soit homogène
\etape mettez la viande dans la poche, de préférence du porc qui va très bien avec, et laissez reposer au frigo quelques heures, 
une nuit typiquement
\etape Il ne reste plus qu'à ouvrir la poche et faire griller les morceaux marinés comme de la viande normale.
\end{preparation}

\begin{remarque}
Vous pouvez également mariner le poulet dans un peu de yaourt additionné de citron vert, de l'ail écrasé, et du \textbf{curry}. 
Ces marinades doivent imprégner assez longtemps le poulet.
\end{remarque}

\end{recette}

\section{Marinade texane}\label{sec:travers_texane}
\begin{recette}{Marinade texane}{4}{15 min + 6h}{}\index{marinade}\index{sauce soja}
\begin{remarque}
Normalement, c'est fait avec des coustilles (travers de porc). La viande doit mariner au moins 6 heures, l'idéal étant plus de 24h (j'ai fait 36h la dernière fois).
\end{remarque}
\begin{ingredients}
\ingredient 1 oignon
\ingredient 3 gousses d'ail
\ingredient 2 cuillères à soupe de miel
\ingredient 3 cuillères à soupe de sauce soja
\ingredient 1 cuillère à soupe d'huile d'olive
\ingredient 1 cuillère à soupe de Ketchup
\ingredient 2 cuillères à soupe de vinaigre
\ingredient 1 cuillère à café de sel
\ingredient Thym, laurier (ne pas mixer, mettre à part)
\end{ingredients}

\begin{preparation}
\item Je met l'oignon et les gousses d'ails coupées grossièrement dans un mixeur avec le vinaigre et l'huile (ceci permet de 
mieux couper les morceaux)
\item Je prépare dans un bol le reste de la marinade, puis j'inclus le contenu du mixeur
\item Versez un peu d'eau dans le mixeur pour récupérer un maximum de marinade.
\item Je mélange puis étale la mixture sur la viande que je laisse mariner quelques heures (environ une nuit).
\item Ajoutez alors thym et laurier sur la viande avant de fermer avec du papier aluminium
\end{preparation}

\begin{remarque}
La marinade est un peu épaisse, et il faut l'étaler et non la verser (vu que j'en fais pas beaucoup dans ces cas là, je met la 
viande dans un plat et étale à la cuillère sur chaque face).
\end{remarque}
\end{recette}


}% End of the ``group'' where section is deactivated
