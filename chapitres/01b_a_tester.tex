% Beginning of group where section is deactivated
% This is only to get the good structure of the document 
% since ``section'' is in fact embedded in the 'recette' environment.
% This group allow us to deactivate sections ONLY in the given file and 
% not for the entire document.
{\renewcommand{\section}[1]{}

\section{Gratin pays (Guadeloupe)}
\begin{recette}{Gratin pays (Guadeloupe)}{0}{1h30}{}\index{patate douce}\index{igname}\index{banane jaune}\index{banane plantain}
\begin{ingredients}
\ingredient 2 bananes jaunes (plantain)
\ingredient 750g de patate douce (blanche) (la orange est trop sucrée pour ce gratin)
\ingredient 750g d'igname (1 igname)
\ingredient 2 gousses d'ail
\ingredient 25cl de crème fraiche %(au resto, on m'a dit "du lait")
\ingredient 100g de fromage rapé
\ingredient sel, poivre, noix de muscade
\end{ingredients}

\begin{preparation}
\etape Mettez la patate douce et l'igname dans l'eau froide salée et portez à ébullition
\etape Une fois à ébullition, rajoutez les bananes entières avec la peau. Faites cuire les bananes 10 minutes et le reste 20 minutes. Les morceaux doivent être fondant, continuer la cuisson au besoin
\etape Ne mettez pas toute la banane car il ne faut pas que ce soit trop sucré. 
\etape Préchauffez le four à 180°C
\etape Ecrasez le tout, ajoutez de la muscade, l'ail écrasé, la crème fraiche et étalez dans un plat à gratin
\etape Saupoudrez de fromage rapé puis enfournez pendant 30 minutes.
\end{preparation}
\end{recette}

\section{Pulled Pork}
\begin{recette}{Pulled Pork}{0}{1h30}{}
\begin{ingredients}
\ingredient roti de porc dans l'échine (ou autre morceau de porc
\ingredient sauce barbecue
\ingredient 75cl de cidre
\end{ingredients}

\begin{preparation}
\etape badigeonnez la viande de sauce barbecue
\etape Faites cuire 20 minutes au four à 275°C pour saisir
\etape Ajoutez le cidre
\etape Faites alors cuire 6-7h à 130°C dans un récipient hermétiquement fermé.
\end{preparation}
\end{recette}

\section{Riz indien}
\begin{recette}{Riz indien}{0}{1h30}{}
\begin{ingredients}
\ingredient 500g de riz
\ingredient 1 carotte à raper
\ingredient petite boite de petit pois
\ingredient 75g d'oignon frit
\end{ingredients}

\begin{preparation}
\etape Rapez la carotte
\etape Faites cuire le riz au cuiseur à riz en mettant la carotte rapée et les petits pois dès le début en plus de l'eau et du sel
\etape Une fois cuit, rajoutez l'oignon frit
\end{preparation}
\end{recette}

%https://bistroguru.com/recette/recette-poulet-korma/
%https://www.mesinspirationsculinaires.com/article-poulet-korma.html
% ingredient for the korma sauce at safeway: Water, Sugar, Desiccated Coconut, Cream, Coconut Paste, Onion, Canola Oil, Food Starch Modified, Contains 2% or Less of Tomato Paste, Heavy Cream, Ginger, Garlic, Spices (Including Turmeric), Salt, Lactic Acid, Lemon, Juice Concentrate, Dried Cilantro Leaf. 
\section{Poulet Korma}
\begin{recette}{Poulet Korma}{0}{1h30}{}
\begin{ingredients}
\ingredient 1 yaourt nature (ou 20cl de creme fraiche) % aux us j'ai mis 450g de creme fraiche
\ingredient 4 gousses ail écrasées
\ingredient 50g de gingembre frais râpé % aux us j'ai mis l'équivalent de 2-3 gousses d'ail
\ingredient 2 c-a-c garam massala % aux us j'ai mis 2 cac)
\ingredient 1kg de poulet coupé en morceau
\ingredient 70g de concentré de tomate (2 cas)
\ingredient 4 oignons moyens hachés
\ingredient 2 c-a-soupe huile végétale
\ingredient 30g de beurre
\ingredient 2 c-a-c coriandre en poudre
\ingredient coriandre ciselée
\ingredient 400 ml lait de coco
\ingredient 400 ml bouillon de poulet
\ingredient 60g d'amande moulu
\ingredient sel
\ingredient poivre
\end{ingredients}

\begin{preparation}
\etape Emincez les oignons, mixez l'air et le gingembre et réservez dans un bol.
\etape Faites revenir l'oignon dans l'huile, le beurre et les épices pendant 10 minutes. 
\etape Ajoutez le poulet, le bouillon, le concentré de tomate, l'ail et le gingembre mixé et faites cuire à couvert pendant 30 minutes.
\etape Ajoutez alors le lait de coco et les amandes moulues. Faites alors cuire 30 minutes de plus à feu très doux et à couvert (attention à ne pas faire bouillir la sauce) Rectifier l'assaisonnement.
\etape Servir chaud accompagné de riz.
\end{preparation}
\end{recette}

\section{Pâtes à l'ail}
\begin{recette}{Pâtes à l'ail}{3}{30 min.}{}\index{pâtes}\index{ail}
\begin{ingredients}[Pour 500g de pâtes]
\ingredient 125g d'ail frais
\ingredient 60g de parmesan
\ingredient 20 cl de crème liquide légère 
\ingredient sel
\end{ingredients}


\begin{preparation}
\etape Faites cuire les gousses d'ail en chemise (sans enlever la peau) pendant 20 minutes au four à 200°C sans préchauffage. 
\etape Ecrasez l'ail en purée et mélangez à la crème liquide et au sel, puis laissez infuser jusqu'à ce que les pâtes soient cuite
\etape Une fois les pâtes cuites, rajoutez dans le plat le contenu de la casserole et le parmesan. Remuez jusqu'à ce que ça ait la consistance qui vous convienne (avec la chaleur des pâtes, ça va épaissir un peu).
\begin{remarque}
S'il y a vraiment trop de liquide, rallumez un peu le feu sous la marmite tout en remuant jusqu'à ce que la consistance vous convienne.
\end{remarque}
\end{preparation}
\end{recette}


}% End of the ``group'' where section is deactivated
